\section{Algebraic Space}
Recall the notion of (covering) geometric 0-stacks, which we call (covering) Algebraic Spaces. it is the smallest pair of classes that satisfies the following
\begin{itemize}
	%	\item Stability under $\sum$ \ref{thm:stabSums} 
	\item (covering) affines are (covering) algebraic spaces. %If $\bT \ni \Spec A \to X$ is a $\bT$-cover and $X$ is a $\bT \Set$, then $X$ is a covering algebraic space
	\item stable under covering quotients: If $X$ is an algebraic space, $Y$ modal 0-type and $X \to Y$ is fibered in covering algebraic spaces, then $Y$ is an algebraic space. Additionally, if $X$ is covering, then $Y$ is covering.
\end{itemize}
\subsection{Equivalence relations vs Surjections}
%\subsection{For groupoids}
%\begin{lemma}
%	If $R \twoheadrightarrow X \to X$ is a $\bT$-htpy-coequalizer diagram of two $\bT$-covers between affines, then $X$ is a  1-stack.
%\end{lemma}

%\subsection{For sets}
\begin{lemma}{\label{quotient-by-equivalence-relation}}
	Denote $\bT Set$ for the sets that are $\bT$-sheaves. Assume given a $\bT$set  $X$ then the following maps are mutually inverse
	\begin{align*}
		\sum_{R:X\to X\to \bT\Prop} R\ \mathrm{equivalence\ relation} &\simeq \sum_{Y:\bT \mathrm{\Set}} \sum_{p:X\to Y} p\ \bT\mathrm{surjective} \\
		R &\mapsto (L_\bT \| X//R \|_0,[\_]) \\
		\lambda x,y.  (p(x)=p(y)) &\mapsfrom (Y,p) 
	\end{align*}
	%where $X / R$ is defined by applying $L_T \| \_ \|_0 $ at the higher inductive type $X // R$.
\end{lemma}

\begin{proof}
	\begin{itemize}
		\item Well-definedness: The map $[\_] : X \to \|X // R\|_0 \to L_T \|X // R\|_0$ is the composition of a surjective with a $\bT$-surjective map \todocite, hence its $\bT$-surjective. \\
		Conversely given $(Y,p)$ as $Y$ is a sheaf, we have for all $x,y : X$ that $p(x) =_Y p(y)$ is a sheaf.
		\item If $x,y : X$ then we have a chain of equivalences 
		\[
		R(x,y) \simeq (\bar x =_{\|X//R\|_0} \bar y) \overset{\mathsf{ap}_\eta}{\to} ([x] =_{L_T\|X//R\|_0} [y])
		\]
		where the first map is plain HoTT, meaning that $\|X//R\|_0$ is seperated. The second map is an equivalence by \ref{lemma:sep}. %, i.e. the unit of the modality \ref{lemma:idTypesOfSheafification}, but as the $\bar x =_{\|X//R\|_0} \bar y$ is already a sheaf, it is an isomorphism as well. \\
		\item Let $(Y,p)$ be in the RHS. Let $R(x,y) = (p(x)=p(y)) : \bT \Prop$. By plain HoTT, There is a map $\eta :  X // R  \to Y$ ( defined by the universal property of the set truncation and by induction on the higher inductive type $ X // R$ on canonical terms through the map $p : X \to Y$). I claim $\eta$ exhibits $Y$ as the localization for $\bT \Set$-modality of $X // R$. Let $T$ be another $\bT \Set$ equipped with a map $X // R  \to T$. By precomposition we obtain a map $X \to T$. 
		Claim: it factors uniquely through $p : X \to Y$. 
		% https://q.uiver.app/#q=WzAsNCxbMCwwLCJYIl0sWzEsMCwiXFx8WCAvIFJcXHwiXSxbMiwwLCJUIl0sWzEsMSwiWSJdLFswLDFdLFsxLDJdLFswLDNdLFszLDIsIlxcZXhpc3RzISIsMix7InN0eWxlIjp7ImJvZHkiOnsibmFtZSI6ImRhc2hlZCJ9fX1dXQ==
		\[\begin{tikzcd}
			X & {X // R} & T \\
			& Y
			\arrow[from=1-1, to=1-2]
			\arrow[from=1-1, to=2-2]
			\arrow[from=1-2, to=1-3]
			\arrow["{\exists!}"', dashed, from=2-2, to=1-3]
		\end{tikzcd}\]
		Proof: \\
		Existence: We want to define a map $Y \to T$. Let $y : Y$. As $p$ is $\bT$-surjective and $T$ is a sheaf, we may assume we merely have some element in the fiber of $p$ over $y$. Now push this element through     
		\[\|\fib_p y\| \to \|X // R\|_0 \to T\]
		where the first map is by Plain HoTT and the second one is induced from $X // R \to T$ by assumption and the fact that $T$ is a set.. One can easily check this makes the diagram commute.
		Uniqueness follows from $X \to Y$ beeing $\bT$-surjective and the following
		Fact: Two parellel maps $Y \rightrightarrows T$ into a $\bT \Set$ $T$ are already equal if the become equal after precomposition with a $\bT$-surjection $X \to Y$.  \\
		Proof of the fact : Let $y : Y$. The goal is an identity type of a $\bT \Set$, hence a $\bT \Prop$. Hence As the fiber over $y$ in $X$ is $\bT$-merely inhabited, we may assume an actual term in the fiber. 	As $X \to Y$ equalizes the arrows, this term allows us to conclude. \qed (fact)	\qed(Claim) \\
		We apply the fact to the ($\bT$-)surjectivity of $X \to X // R $ to get a unique factorization 
		% https://q.uiver.app/#q=WzAsNCxbMCwwLCJYIl0sWzEsMCwiXFx8WCAvIFJcXHwiXSxbMiwwLCJUIl0sWzEsMSwiWSJdLFswLDEsIiIsMCx7InN0eWxlIjp7ImhlYWQiOnsibmFtZSI6ImVwaSJ9fX1dLFsxLDJdLFswLDNdLFszLDIsIlxcZXhpc3RzISIsMix7InN0eWxlIjp7ImJvZHkiOnsibmFtZSI6ImRhc2hlZCJ9fX1dLFsxLDNdXQ==
		\[\begin{tikzcd}
			X & {X // R} & T \\
			& Y
			\arrow[two heads, from=1-1, to=1-2]
			\arrow[from=1-1, to=2-2]
			\arrow[from=1-2, to=1-3]
			\arrow[from=1-2, to=2-2]
			\arrow["{\exists!}"', dashed, from=2-2, to=1-3]
		\end{tikzcd}\]
		making the right triangle commute. This is what we wanted to show.
	\end{itemize}
\end{proof}



	\begin{definition}{\label{def:coveringEqRel}}
		%A modal equivalence relation $R : U^2 \to \bT \Prop$ on a set $U$ is covering if the fibers $R_s \equiv \sum_{t: S} R(s,t)$  are covering 0-stacks.
		An equivalence relation $R$ on an affine $S$ is called covering, if all the propositions $R(s,t)$ are sheaves and one of the following conditions is satisfied
		
		\begin{itemize}
			\item  every fiber $R_s \equiv \sum_{t: S} R(s,t)$ merely admits a $\bT$-catlas.
			\item  every fiber $R_s \equiv \sum_{t: S} R(s,t)$ is a covering 0-stack.
		\end{itemize}
	\end{definition}

\begin{proof}
	Every sheaf admitting a $\bT$-catlas is a covering 0-stack. 
	Conversely: if the fibers are covering 0-stacks. Let us first observe, that for all $s , t : S$, $R(s,t)$ is a geometric proposition: $R(s,t)$ is the fiber of the projection $\sum_{t : S} R(s,t) \to S$ between geometric stacks, which are stable under finite limits. \\
	
	For all $t : S$ we can choose a geometric atlas $\Spec A_t \to R(s,t)$ by \ref{def:algprop}. Then 
	\[
	\sum_{t:S} \Spec A_t \to \sum_{t : S} R(s,t)
	\]
	is a $\bT$-atlas. As $\sum_{t : S} R(s,t)$ is a covering 0-stack by assumption, the map has to be a $\bT$-catlas by \ref{lemma:atlasIsCatlas}. 
\end{proof}
\begin{think}
	It may be useful to define covering equivalence relations also for general modal sets and not only affines
\end{think}

\begin{lemma}{\label{lemma:fundamental-property-algebraic-spaces}}
	%Assume that $\bT$ satisfies descent for propositions and for sets 
	Assume that the topology has descent.
	Given an affine ($\bT \Set$ would be enough, cmp prev remark) $X$, the following types are equivalent:
	\begin{itemize}
		\item The type of covering equivalence relations on $X$.
		\item The type of $\bT$sets $Y$ equipped with a map $X \to Y$ fibered in types admitting a $\bT$-catlas.
	\end{itemize}
\end{lemma}

\begin{proof}
	By the equivalence in \ref{quotient-by-equivalence-relation} it is enough to check that
%	\begin{itemize}
%		\item The identity types in $X/R$ are 
%		(-1)-stacks if and only if the relation $R$ is redundant . For any $x,y:X$ we know that:
%		\[R(x,y) \simeq [x] =_{X/R}[y]\]
%		so the direct direction is immediate. For the converse we use the assumption that a modal proposition being a  (-1)-stack is a sheaf and that the map $[\_]:X\to X/R$ is $\bT$-surjective.
		%\item 
		The fibers of: 
		\[[\_]:X\to L_\bT \| X//R \|_0\] 
		merely admit a $\bT$-catlas if and only if the relation $R$ is covering. For any $y:X$ we have that:
		\[\sum_{x:X} R(x,y) \simeq \mathrm{fib}_{[\_]}([y])\]
		so the direct direction is immediate. The converse follows from $\bT$-surjectivity of $[\_]$ and from \ref{cor:DescentCatlas}.
%	\end{itemize}
\end{proof}
\begin{think}{\label{think:getRidOfLocalChoice}}
	Is it maybe useful to also say that a map between $\bT \Set$s $X \to Y$ is a geometric atlas iff its fibered in types that merely admit a $\bT$-catlas? \\
	If we say this, we can get rid of the local choice chapter, because we dont need to prove descent for types admitting a $\bT$-catlas. We would only need descent for covering 0-stacks which we already have.
\end{think}
