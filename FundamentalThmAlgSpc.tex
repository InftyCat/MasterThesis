\section{Fundamental Theorem of algebraic spaces}
%\subsection{For groupoids}
%\begin{lemma}
%	If $R \twoheadrightarrow X \to X$ is a $\bT$-htpy-coequalizer diagram of two $\bT$-covers between affines, then $X$ is a  1-stack.
%\end{lemma}

%\subsection{For sets}
\begin{lemma}{\label{quotient-by-equivalence-relation}}
	Denote $\bT Set$ for the sets that are $\bT$-sheaves. Assume given a $\bT$set  $X$ then the following maps are mutually inverse
	\begin{align*}
		\sum_{R:X\to X\to \bT\Prop} R\ \mathrm{equivalence\ relation} &\simeq \sum_{Y:\bT \mathrm{\Set}} \sum_{p:X\to Y} p\ \bT\mathrm{surjective} \\
		R &\mapsto (L_\bT \| X//R \|_0,[\_]) \\
		\lambda x,y.  (p(x)=p(y)) &\mapsfrom (Y,p) 
	\end{align*}
	%where $X / R$ is defined by applying $L_T \| \_ \|_0 $ at the higher inductive type $X // R$.
\end{lemma}

\begin{proof}
	\begin{itemize}
		\item Well-definedness: The map $[\_] : X \to \|X // R\|_0 \to L_T \|X // R\|_0$ is the composition of a surjective with a $\bT$-surjective map \todocite, hence its $\bT$-surjective. \\
		Conversely given $(Y,p)$ as $Y$ is a sheaf, we have for all $x,y : X$ that $p(x) =_Y p(y)$ is a sheaf.
		\item If $x,y : X$ then we have a chain of equivalences 
		\[
		R(x,y) \simeq (\bar x =_{\|X//R\|_0} \bar y) \overset{\mathsf{ap}_\eta}{\to} ([x] =_{L_T\|X//R\|_0} [y])
		\]
		where the first map is plain HoTT, meaning that $\|X//R\|_0$ is seperated. The second map is an equivalence by \ref{lemma:sep}. %, i.e. the unit of the modality \ref{lemma:idTypesOfSheafification}, but as the $\bar x =_{\|X//R\|_0} \bar y$ is already a sheaf, it is an isomorphism as well. \\
		\item Let $(Y,p)$ be in the RHS. Let $R(x,y) = (p(x)=p(y)) : \bT \Prop$. By plain HoTT, There is a map $\eta :  X // R  \to Y$ ( defined by the universal property of the set truncation and by induction on the higher inductive type $ X // R$ on canonical terms through the map $p : X \to Y$). I claim $\eta$ exhibits $Y$ as the localization for $\bT \Set$-modality of $X // R$. Let $T$ be another $\bT \Set$ equipped with a map $X // R  \to T$. By precomposition we obtain a map $X \to T$. 
		Claim: it factors uniquely through $p : X \to Y$. 
		% https://q.uiver.app/#q=WzAsNCxbMCwwLCJYIl0sWzEsMCwiXFx8WCAvIFJcXHwiXSxbMiwwLCJUIl0sWzEsMSwiWSJdLFswLDFdLFsxLDJdLFswLDNdLFszLDIsIlxcZXhpc3RzISIsMix7InN0eWxlIjp7ImJvZHkiOnsibmFtZSI6ImRhc2hlZCJ9fX1dXQ==
		\[\begin{tikzcd}
			X & {X // R} & T \\
			& Y
			\arrow[from=1-1, to=1-2]
			\arrow[from=1-1, to=2-2]
			\arrow[from=1-2, to=1-3]
			\arrow["{\exists!}"', dashed, from=2-2, to=1-3]
		\end{tikzcd}\]
		Proof: \\
		Existence: We want to define a map $Y \to T$. Let $y : Y$. As $p$ is $\bT$-surjective and $T$ is a sheaf, we may assume we merely have some element in the fiber of $p$ over $y$. Now push this element through     
		\[\|\fib_p y\| \to \|X // R\|_0 \to T\]
		where the first map is by Plain HoTT and the second one is induced from $X // R \to T$ by assumption and the fact that $T$ is a set.. One can easily check this makes the diagram commute.
		Uniqueness follows from $X \to Y$ beeing $\bT$-surjective and the following
		Fact: Two parellel maps $Y \rightrightarrows T$ into a $\bT \Set$ $T$ are already equal if the become equal after precomposition with a $\bT$-surjection $X \to Y$.  \\
		Proof of the fact : Let $y : Y$. The goal is an identity type of a $\bT \Set$, hence a $\bT \Prop$. Hence As the fiber over $y$ in $X$ is $\bT$-merely inhabited, we may assume an actual term in the fiber. 	As $X \to Y$ equalizes the arrows, this term allows us to conclude. \qed (fact)	\qed(Claim) \\
		We apply the fact to the ($\bT$-)surjectivity of $X \to X // R $ to get a unique factorization 
		% https://q.uiver.app/#q=WzAsNCxbMCwwLCJYIl0sWzEsMCwiXFx8WCAvIFJcXHwiXSxbMiwwLCJUIl0sWzEsMSwiWSJdLFswLDEsIiIsMCx7InN0eWxlIjp7ImhlYWQiOnsibmFtZSI6ImVwaSJ9fX1dLFsxLDJdLFswLDNdLFszLDIsIlxcZXhpc3RzISIsMix7InN0eWxlIjp7ImJvZHkiOnsibmFtZSI6ImRhc2hlZCJ9fX1dLFsxLDNdXQ==
		\[\begin{tikzcd}
			X & {X // R} & T \\
			& Y
			\arrow[two heads, from=1-1, to=1-2]
			\arrow[from=1-1, to=2-2]
			\arrow[from=1-2, to=1-3]
			\arrow[from=1-2, to=2-2]
			\arrow["{\exists!}"', dashed, from=2-2, to=1-3]
		\end{tikzcd}\]
		making the right triangle commute. This is what we wanted to show.
	\end{itemize}
\end{proof}

\begin{definition}
	A modal equivalence relation $R$ on a type $X$ is called covering, if for any $y:X$ the fibers
		\[R_y :\equiv \sum_{x:X} R(x,y)\]
		merely admits a $\bT$-catlas.

\end{definition}

\begin{lemma}{\label{lemma:fundamental-property-algebraic-spaces}}
	%Assume that $\bT$ satisfies descent for propositions and for sets 
	Assume that the topology has descent.
	Given a $\bT$set $X$, the following types are equivalent:
	\begin{itemize}
		\item The type of covering equivalence relations on $X$.
		\item The type of $\bT$sets $Y$ equipped with a map $X \to Y$ fibered in types admitting a $\bT$-catlas.
	\end{itemize}
\end{lemma}

\begin{proof}
	By the equivalence in \ref{quotient-by-equivalence-relation} it is enough to check that
%	\begin{itemize}
%		\item The identity types in $X/R$ are 
%		(-1)-stacks if and only if the relation $R$ is redundant . For any $x,y:X$ we know that:
%		\[R(x,y) \simeq [x] =_{X/R}[y]\]
%		so the direct direction is immediate. For the converse we use the assumption that a modal proposition being a  (-1)-stack is a sheaf and that the map $[\_]:X\to X/R$ is $\bT$-surjective.
		%\item 
		The fibers of: 
		\[[\_]:X\to L_\bT \| X//R \|_0\] 
		merely admit a $\bT$-catlas if and only if the relation $R$ is covering. For any $y:X$ we have that:
		\[\sum_{x:X} R(x,y) \simeq \mathrm{fib}_{[\_]}([y])\]
		so the direct direction is immediate. The converse follows from $\bT$-surjectivity of $[\_]$ and from \ref{cor:DescentCatlas}.
%	\end{itemize}
\end{proof}
%\begin{corollary}
%	Assume $\bT$ satisfies descent for propositions and for sets.
%	A type is a  0-stack iff its merely the $\bT$-quotient of an affine scheme by a covering equivalence relation.
%\end{corollary}
%\begin{theorem}{\label{thm:QuotientOfAlgebraicSpace}}
%	Assume $\bT$ satisfies descent for propositions. 
%	The quotient of a  $0$-stack $X \in \bT \Set$ by an $0$-covering equivalence relation $R$ is a  $0$-stack. TODO
%\end{theorem}
%
%\begin{proof}
%	The identity types in $X / R$ are propositional  0-stacks, hence $(-1)$-\truncation s of  -1-stacks by \ref{lemma:prop0stacks} as desired. \\
%	How to find an atlas: todo. How to proceed, if we could choose all atlasses we want at the same time?
	% Motivation why the choice of atlasse should work: Let $T = X / R$. 
	%  If we could choose -1-atlasses $\tilde X_t$ for the covering 0-stacks $\fib_{[]}(t)$ for all $t : T$ at the same time, then $\sum_{t : T} \tilde X_t \to \sum_{t : T} \fib_{[]}(t) \to T$ has as domain a is fibered in covering $-1$-stacks, as the fiber over $t$ would be $\tilde X_t$ which is an affine scheme in the topology. Moreover, This is enough as \\ %, hence by definition a covering -1 stack. \\
	
	%     Given $p_1 , p_2 : R \rightrightarrows X$ fibered in covering $0$-stacks, hence the fibers merely have $-1$-atlasses.
	%     %Claim: There exists a $-1$-atlas $R' \to R$
	%     As $X$ has Local choice with respect to $-1$-atlasses, we find a $-1$ atlas $f : X' \to X$ with 
	%     \[
	%     \prod_{x' : X'} \text{-1-atlas}(\sum_{x : X} R(x,f(x'))
	%     \]
	
%\end{proof}
%\begin{rmk}
%	This is equivalent to saying that  $1$-stacks that are $0$-types are geomeric $0$-stacks: One direction we prove later. If $R$ is a 0-covering equivalence relation on a  0-stack $X$, then $ X/ R$ is a  1-stack by observing that any -1-atlas $X' \to X$ gives a 0-atlas $X' \to X \to X/ R$. Moreover, $ X/ R$ is a 0-type, hence by assumption a  0-stack.
%\end{rmk}


% \begin{example}
%     The Zariski topology does not descent along $\bT$-covers between affines
% \end{example}
% \begin{proof}
% Assume it would hold.
% By the previous example pick such an open affine subset $U \subset \Spec A$ and pick a Zariski atlas $V \to U$ such that $V$ is mereley of the form $D(a_1) + \hdots + D(a_n)$ for some $a_i \in A$. Let $x : \Spec A$. Then pulling pack the Zariski atlas along $U(x) \to U$ gives us a Zariski atlas of the open proposition $V' \to U(x)$. Now $V' + 1 \to U(x) + 1$ is a Zariski atlas with total space in the Zariski topology. By assumption, $U(x) + 1$ is in the topology, hence $U(x)$ would be a sum of principal opens. As it is a propososition, it would be a principal open subset of $1$. 
% This is not a contradiction, because an open subset can be non principal although all the fibers are principal open props...
% This is a contradiction by the assumption on $U \subset \Spec A$ beeing not principal open.

% \end{proof}


% \begin{lemma}
%     A morphism $f : X \to Y$ of  $n$-stacks is fibered in covering $n$-stacks if there exists a covering $n$-atlas of $f$.
% \end{lemma}
