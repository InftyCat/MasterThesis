\documentclass{article}
\usepackage{graphicx} % Required for inserting images


%%%%%%%%%%%%%%%%%%%%%%%%%%%%%%%%%%%%%%%%%%%%%%%%
% load packages 

\usepackage[a4paper,nohead,left=3.5cm,right=3.5cm,top=4cm,bottom=3cm]{geometry}
%\usepackage{german}    % only for German articles
%\usepackage{a4wide}   % longer lines
\usepackage[intlimits,tbtags]{amsmath}   % for more basic mathematical symbols
\usepackage{amssymb}   % for more mathematical symbols
\usepackage{amsfonts}
\usepackage[utf8]{inputenc}
\usepackage{textcomp}
\usepackage{mathtools}
% \usepackage{stmaryrd}  % for more mathematical symbols
\usepackage{latexsym}  % for more mathematical symbols,
% already contained in amsmath-package
% \usepackage{accents}  % for more dots etc on symbols
\usepackage{amsxtra}   % for accents as superscripts
\usepackage{amstext}   % for text in formulas, accents, etc.
\usepackage{bm}        % boldface for non-latin letters
\usepackage{amsthm}    % for theorem-environments
\usepackage{amscd}     % for commutative diagrams
%\usepackage{MnSymbol}
%\usepackage[ansinew]{inputenc}
\usepackage{enumitem}  % Better enumerations.

\usepackage{graphicx}
\usepackage[dvipsnames]{xcolor}
\usepackage[arrow, matrix, curve]{xy}
\usepackage[colorlinks=true, citecolor=Blue, linkcolor=Blue, urlcolor=Blue]{hyperref}
\usepackage{xfrac}
\usepackage[utf8]{inputenc}
\usepackage{scalerel}
\newcommand{\tA}{{\hspace{1pt}\sim}_{A}\hspace{1pt}}
\newcommand{\tR}{\hspace{1pt}{\sim}_{R}\hspace{1pt}}
\newcommand{\bA}{\mathbb{A}}
\newcommand{\bB}{\mathbb{B}}
\newcommand{\bC}{\mathbb{C}}
\newcommand{\bD}{\mathbb{D}}
\newcommand{\bE}{\mathbb{E}}
\newcommand{\bF}{\mathbb{F}}
\newcommand{\bG}{\mathbb{G}}
\newcommand{\bH}{\mathbb{H}}
\newcommand{\bI}{\mathbb{I}}
\newcommand{\bJ}{\mathbb{J}}
\newcommand{\bK}{\mathbb{K}}
\newcommand{\bL}{\mathbb{L}}
\newcommand{\bM}{\mathbb{M}}
\newcommand{\bN}{\mathbb{N}}
\newcommand{\bO}{\mathbb{O}}
\newcommand{\bP}{\mathbb{P}}
\newcommand{\bQ}{\mathbb{Q}}
\newcommand{\bR}{\mathbb{R}}
\newcommand{\bS}{\mathbb{S}}
\newcommand{\bT}{\mathbb{T}}
\newcommand{\bU}{\mathbb{U}}
\newcommand{\bV}{\mathbb{V}}
\newcommand{\bW}{\mathbb{W}}
\newcommand{\bX}{\mathbb{X}}
\newcommand{\bY}{\mathbb{Y}}
\newcommand{\bZ}{\mathbb{Z}}
\DeclareMathOperator{\Sh}{Sh}

%%%%%%%%% calligraphic %%%%%%%%%%%%%%%%%%%%%%%
\newcommand{\mc}[1]{\mathcal{#1}}
\newcommand{\R}{\Rightarrow}
\newcommand{\cA}{\mathcal{A}}
\newcommand{\cB}{\mathcal{B}}
\newcommand{\cC}{\mathcal{C}}
\newcommand{\cD}{\mathcal{D}}
\newcommand{\cE}{\mathcal{E}}
\newcommand{\cF}{\mathcal{F}}
\newcommand{\cG}{\mathcal{G}}
\newcommand{\cH}{\mathcal{H}}
\newcommand{\cI}{\mathcal{I}}
\newcommand{\cJ}{\mathcal{J}}
\newcommand{\cK}{\mathcal{K}}
\newcommand{\cL}{\mathcal{L}}
\newcommand{\cM}{\mathcal{M}}
\newcommand{\cN}{\mathcal{N}}
\newcommand{\cO}{\mathcal{O}}
\newcommand{\cP}{\mathcal{P}}
\newcommand{\cQ}{\mathcal{Q}}
\newcommand{\cR}{\mathcal{R}}
\newcommand{\cS}{\mathcal{S}}
\newcommand{\cT}{\mathcal{T}}
\newcommand{\cU}{\mathcal{U}}
\newcommand{\cV}{\mathcal{V}}
\newcommand{\cW}{\mathcal{W}}
\newcommand{\cX}{\mathcal{X}}
\newcommand{\cY}{\mathcal{Y}}
\newcommand{\cZ}{\mathcal{Z}}
\DeclareMathOperator{\chains}{Chains}

%%%%%%%%%%%%% mathematical fraktur  %%%%%%%%%%%%%%%%%%%%%
\newcommand{\mf}[1]{\mathfrak{#1}}
\newcommand{\senk}{\ \big \vert \ }
\newcommand{\fA}{\mathfrak{A}}
\newcommand{\fB}{\mathfrak{B}}
\newcommand{\fC}{\mathfrak{C}}
\newcommand{\fD}{\mathfrak{D}}
\newcommand{\fE}{\mathfrak{E}}
\newcommand{\fF}{\mathfrak{F}}
\newcommand{\fG}{\mathfrak{G}}
\newcommand{\fH}{\mathfrak{H}}
\newcommand{\fI}{\mathfrak{I}}
\newcommand{\fJ}{\mathfrak{J}}
\newcommand{\fK}{\mathfrak{K}}
\newcommand{\fL}{\mathfrak{L}}
\newcommand{\fM}{\mathfrak{M}}
\newcommand{\fN}{\mathfrak{N}}
\newcommand{\fO}{\mathfrak{O}}
\newcommand{\fP}{\mathfrak{P}}
\newcommand{\fQ}{\mathfrak{Q}}
\newcommand{\fR}{\mathfrak{R}}
\newcommand{\fS}{\mathfrak{S}}
\newcommand{\fT}{\mathfrak{T}}
\newcommand{\fU}{\mathfrak{U}}
\newcommand{\fV}{\mathfrak{V}}
\newcommand{\fW}{\mathfrak{W}}
\newcommand{\fX}{\mathfrak{X}}
\newcommand{\fY}{\mathfrak{Y}}
\newcommand{\fZ}{\mathfrak{Z}}
\newcommand{\lp}{_\flat} %{\boldsymbol{\cdot}}
\newcommand{\hp}{^\sharp}
\newcommand{\cp}{\boldsymbol{\cdot}}

\newtheorem{theorem}{Theorem}[section]
\newtheorem{satz}[theorem]{Satz}
\newtheorem{lemma}[theorem]{Lemma}
\newtheorem{korollar}[theorem]{Korollar}
\newtheorem{example}[theorem]{Example}
\newtheorem{prop}[theorem]{Proposition}
\DeclareMathOperator{\Spec}{Spec}
\newtheorem{corollary}[theorem]{Corollary}
\theoremstyle{definition}

\newtheorem{definition}[theorem]{Definition}
\newtheorem{ziel}[theorem]{Ziel}
\newtheorem{frage}[theorem]{Frage}
\newtheorem*{notation}{Notation}
\newtheorem*{slogan}{Slogan}
\newtheorem*{construction}{Construction}
\newtheorem*{bemerkung}{Bemerkung}
\newtheorem*{exercise}{Exercise}

\newtheorem*{note*}{Note}

\newtheorem{rmk}{Remark}
\newtheorem{bsp}[theorem]{Beispiel}
\newtheorem{aufgabe}[theorem]{Aufgabe}
\newtheorem*{beweis}{\it Beweis}
\newcommand{\gray}[1]{{\color{gray} #1}}
%%%%%%%%%%    Math operators    %%%%%%%%%%%%%%%%%%%%%%%%%%%

\DeclareMathOperator{\id}{id}             % identity morphism
% \DeclareMathOperator{\ker}{ker}           % kernel
\DeclareMathOperator{\im}{im}             % image
\DeclareMathOperator{\Hom}{Hom}           % homomorphisms
\DeclareMathOperator{\End}{End}           % endomorphisms
\DeclareMathOperator{\Span}{Span}         % linear span

\usepackage{tikz-cd}
\DeclareMathOperator{\pr}{pr}
\usepackage{quiver}
\renewcommand{\:}{\colon}
\DeclareMathOperator{\isContr}{isContr}

\newcommand{\type}{\ \mathrm{Type}}
\usepackage{stmaryrd}



\newcommand{\op}{^{op}}
%\renewcommand{\subset}{\subseteq}
%\newcommand{\colim}[1]{\mathrm{colim} \limits_{#1}}
\newcommand{\colim}[1]{\underset{#1}{\mathrm{colim} \ }}
\DeclareMathOperator{\sSet}{\mathsf {sSet}}
\DeclareMathOperator{\Pos}{\mathsf {Pos}}
\DeclareMathOperator{\Set}{\mathsf {Set}}
\DeclareMathOperator{\Fun}{Fun}
\DeclareMathOperator{\Cat}{\mathsf {Cat}}
\DeclareMathOperator{\const}{const}
\DeclareMathOperator{\Vect}{\mathsf{Vect}}
\DeclareMathOperator{\Top}{\mathsf{Top}}
\DeclareMathOperator{\Ring}{\mathsf{Ring}}
\DeclareMathOperator{\Field}{\mathsf{Field}}

\DeclareMathOperator{\Ab}{\mathsf{Ab}}
\DeclareMathOperator{\GL}{GL}
\DeclareMathOperator{\Ch}{\mathsf{Ch}}
\DeclareMathOperator{\Grp}{\mathsf{Grp}}

\DeclareMathOperator{\HomC}{\Hom_{\cC}}
\DeclareMathOperator{\HomD}{\Hom_{\cD}}
\usepackage{ascii}
\DeclareMathOperator{\Ob}{Ob}
\DeclareMathOperator{\FinVect}{FinVect}
\setlength\parindent{0pt} % Keine Einrueckung von Absaetzen
\newcommand{\etale}{\' etale }
\newcommand{\Etale}{\' Etale }
\DeclareMathOperator{\fib}{fib}
\newcommand{\todo}{{\color{Red} Todo}}
\newcommand{\todocite}{[ref?]}
\newcommand{\el}{\in}
\usepackage{wasysym}
\newcommand{\ci}{\fullmoon}
\DeclareMathOperator{\isProp}{isProp}
\DeclareMathOperator{\Prop}{Prop}
%\renewcommand{\in}{\colon}

\newcommand{\details}{[...]}
\DeclareMathOperator{\tp}{tp}
\DeclareMathOperator{\Nat}{Nat}
\renewcommand{\contentsname}{Inhalt}
\font\maljapanese=dmjhira at 2.5ex
\newcommand{\yo}{\textrm{\!\maljapanese\char"48}}
\DeclareMathOperator{\Aut}{Aut}
\DeclareMathOperator{\Mod}{\mathsf{Mod}}
\DeclareMathOperator{\Mat}{Mat}

\DeclareMathOperator{\isInv}{isInv}
\DeclareMathOperator{\Alg}{Alg}
\newtheorem{axiom}{Axiom}
\newtheorem{question}{Question}
\renewcommand{\mid}{\ | \ }

\newcommand{\fun}[4]{
	\begin{align*} 
		#1 &\to #2 \\ 
		#3 &\mapsto #4 
\end{align*}}
\newcommand{\funn}[5]{
	\begin{align*} 
		#1 \colon #2 &\to #3 \\ 
		#4 &\mapsto #5
\end{align*}}

\newcommand{\RHom}{R \cH om}
\newcommand{\Ltimes}{\overset{\mathrm{L}}{\otimes}}
\title{Geometric stacks in Synthetic Algebraic Geometry }
\newcommand{\ignore}[1]{}
\usepackage[style=alphabetic]{biblatex}
\addbibresource{form.bib}
\author{Tim Lichtnau, Hugo Moeneclay}
\date{Oktober 2024}

\begin{document}
	\maketitle
	\begin{abstract}
		\noindent This is meant as a short summary of the progress of the Master Thesis of Tim Lichtnau so far. We work in Synthetic Algebraic Geometry, i.e. Homotopy Type Theory + 3 Axioms. A type gets interpreted as a Zariski-sheaf on the site given by the opposite category of finitely presented algebras over a fixed ring \cite{cherubini2023foundationsyntheticalgebraicgeometry}. Affine types get interpreted as the representable sheaves.
	\end{abstract}
	\begin{definition}
		A Grothendieck topology $\bT$ is a subclass of affine schemes, such that 
		\begin{itemize}
			\item $1 \in \bT$
			\item $\bT$ is $\sum$-stable, i.e. if $X : \bT, B : X \to \bT$, then $\sum_{x :X} B x$ belongs to $\bT$.
		\end{itemize}
	\end{definition}
	A map is a $\bT$-cover iff its fibered in $\bT$.
	Relating to the classical definition, we could call a family of maps of affines $\{U_i \to U\}_{i=1}^n$ covering iff 
	\[
	\sum_{i=1}^n U_i \to U
	\]
	is a $\bT$-cover. \\
	We fix a topology $\bT$, for which we want to define the notion of (geometric) stack. \\
	\begin{definition}
		A type $X$ is a (higher) stack iff its $\| \Spec A \|$-local for any $\Spec A \in \bT$. \\
		An $n$-stack is a stack that is an $n$-type.
	\end{definition}
	This exactly captures the expected classical notion of a stack for a Grothendieck topology \cite{SAGsheaves}, e.g. a 0-type $X$ is a stack, if for any $\bT$-cover $A \to B$, 
	\[X^B \to X^A \rightrightarrows X^{A \times_B A}\]
	is an equalizer diagram.
	%Stacks are stable under $\sum$ and identity types. \\
	%We may assume on our Topology, that $\bT$ is subcanonical, i.e. affine schemes are stacks. \\
	% \begin{itemize}
	%     \item 
	%     \ignore{\item WLOG (*) $Zar \subset \bT$ with
	%     \[
	%     Zar \equiv \left \{ \sum_{i=1}^n D(f_i) \ \mid \ (f_1,\hdots,f_n) = R \right\} 
	%     \]}
	%     %\item ... Later WLOG (*) $\bT$ saturated
	% \end{itemize}
	\begin{example}[\cite{SAGmoeneclay}]
		\ \begin{itemize}
			\item 
			The fppf topology is given by the faithfully flat affine schemes. 
			\item The \etale topology is given by formaly-\etale + faithfully flat affine schemes. \\
			\item The smooth topology is given by smooth + faithfully flat affine schemes.
		\end{itemize}
	\end{example}
	
	\newpage
	We start by defining the relative setting (this corresponds to fibers of smooth morphism of geometric stacks in \cite{simpson1996algebraicgeometricnstacks}). There is a short inductive definition:
	\begin{definition}
		A stack $X$ is covering, whenever inductively
		\begin{itemize}
			\item $X \in \bT$ or
			\item $X$ is equipped with a map $\bT \ni \Spec A \to X$ fibered in covering stacks.
		\end{itemize}
		
	\end{definition}
	We call a map $X \to Y$ fibered in covering stacks a geometric cover (That corresponds to smooth morphisms in \cite{simpson1996algebraicgeometricnstacks}) 
	%If $X$ is affine we call it a geometric atlas.
	\begin{definition}
		A stack $X$ is geometric iff it merely admits a geometric cover $\Spec A \to X$.
	\end{definition}
	We have the following classical labels associated to geometric stacks depending on the topologies:
	\begin{tabular}{c|c}
		$\bT$ & Geometric stacks for $\bT$ \\
		\hline 
		\etale & (Higher) Deligne Mumford Stacks  \\
		smooth & (Higher) Artin stacks \\    
		fppf & Something similar to Artin stacks ?     
	\end{tabular}
	
	%Observe that every covering stack is geometric.
	\begin{example}
		Every stack that is a scheme is geometric. 
		%Every affine in the topology is covering.
	\end{example}
	
	\begin{theorem}[Stability Results]
		\ \begin{itemize}
			\item The class of covering / geometric stacks is $\sum$-stable. %In particular geometric covers are stable under composition
			\item The class of covering / geometric stacks is closed under quotients: If $X \to Y$ is a geometric cover with $X$ covering / geometric stack, then $Y$ is covering / geometric
			\item   Geometric stacks are closed under taking identity types.
			\item Every geometric stack is a geometric $n$-stack for some $n$
			\item Covering / Geometric stacks have descent: Both types $\mathsf{GeometricStack}$ and $\mathsf{CoveringStack}$ are a stack.
		\end{itemize}
	\end{theorem}
	It is maybe worth mentioning, that proving descent was surprisingly easy. \\ \\
	%A $\bT$-atlas is a $\bT$-cover with affine domain.
	Under some very mild condition on the topology \footnote{We can always enforce this condition without changing the notion of (covering / geometric) stack}  (e.g. satisfied by \etale or fppf), which is equivalent to saying that every geometric cover between affines is a $\bT$-cover we have the following explicit description depending on the truncation level $n$:
	\begin{theorem}
		An $n$-stack $X$ is geometric if and only if 
		\begin{itemize}
			\item [$(n=0)$:]
			there merely exists a map $\Spec A \to X$ whose fibers $F$ merely admit a $\bT$-cover $\bT \ni \Spec B \to F$. \footnote{One can reformulate also as taking a quotient of $\Spec A$ by an equivalence relation satisfying a certain property.}
			\item [$(n \ge 1)$:]
			there merely exist a map $\Spec A \to X$ whose fibers are covering $(n-1)$-stacks. Additionally $X$ is covering iff we can choose $\Spec A$ to lie in $\bT$.
		\end{itemize}
	\end{theorem}
	
	For the \etale topology we have the following notable results 
	\begin{theorem}
		\
		\begin{itemize}
			\item Every Deligne-Mumford stack is a 1-gerbe, i.e. $X \to \|X\|_1^\bT$ is a geometric cover, where the latter means the $\bT$-sheafification of the $1$-truncation of $X$.
			\item 
			A Deligne-Mumford stack $X$ is covering iff $\pi_0^\bT X := \|X\|_0^\bT$ and all higher homotopy groups 
			\[
			\pi_i^\bT(X,x) = \|\Omega^i (X , x)\|_0^\bT , i \ge 1
			\]
			are covering algebraic spaces for the \etale topology.  
		\end{itemize}
	\end{theorem}
	\section{Examples}
	We can reproduce examples from Stacks project:
	Let $\mu_\ell = \Spec R[T] / (T^\ell - 1)$ denote the group of $\ell$.th roots of unity.
	\begin{example}[Non-example]
		If $2 \neq 0$, the sheaf quotient of $\bA^1$ by the $\mu_2$ action is not an algebraic space.    
	\end{example}
	
	
	\begin{example}[Not locally-separated examples]
		Assume $\ell \neq 0$ prime. Let $\mu_\ell$ act on $\Spec B$ in one of the following ways:
		\begin{enumerate}
			\item Let $\mu_\ell$ act on $\Spec B = \bA^1$. 
			\item Put $\ell= 2$. Let $\mu_2$ act on the cross
			\[
			\Spec B \equiv \sum_{x , y : R} x y = 0
			\]
			via the swap.
		\end{enumerate}
		Define an equivalence relation
		\[
		R_{\mu_\ell}(x,y) = (x = y) + (x \neq 0) \times \sum_{g : \mu_\ell \setminus \{1\}} g x = y
		\]
		Then $\Spec B / R_{\mu_\ell}$ is an algebraic space that is not a scheme.
	\end{example}
	There is a general way to produce examples of algebraic spaces:
	\begin{lemma}
		Quotients of geometric stacks $X$ by groups that are covering stacks are geometric stacks, if the isotropy stacks are covering. This happens for example if $X$ has $\bT$-flat identity types or if the group action is free.    
	\end{lemma}
	\begin{theorem}
		Schemes do not have descent, i.e. even if Schemes are stacks, the type of Schemes is not a stack!
	\end{theorem}
	\begin{proof}
	Given $a : R$, e.g. $a = 1$, the idea is to consider the twisted line with double origin:
		\[\sum_{x : R} \left(\Spec R[T]/(T^2+a)\right)^{x = 0},\]
		which is etale-merely a scheme, namely the line with double origin. Then one needs to show, that if it would be a scheme then $T^2 + a$ has a root.		
	\end{proof}	
	\begin{theorem}
		geometric covers for the etale topology are formally \etale
	\end{theorem}
	\begin{proof}
		Surprisingly involved. Here is some detail missing ala formally \etale implies flat
	\end{proof}
	\begin{prop}
		For the etale or the smooth topology, geometric stacks are stable under taking tangent spaces.
	\end{prop}
	\section{A notion of flat for any topology}
	
	\begin{definition}
		Denote $\Top$ the topologies containing $\mathsf{Bool}$, e.g. finer than the Zariski-topology.
		Let $\mathsf{FLAT}$ consists of all the classes of affines $\bP$ containing $1, \bot$ stable under $\sum$. \\
		Given $\bP: \mathsf{FLAT} , \bT: \Top$ we say $\bP$ flattens $\bT$ iff ($\bT \subset \bP$ and)
		\[
		\bT= \{X : \bP\ | \ \|X\|_\bT \}
		\]
	\end{definition}
	\begin{theorem}{\label{thm:Flat}}
		\
		\begin{enumerate}
			\item There is at most one $\bP$ that flattens a topology. Then we say, the topology is flatten.
			\item A topology can be idempotently flattened without changing the stacks
			\item For any $\bP: \mathsf{FLAT}$ and any Lavwere Tierney Operator $j$, $\{ X: \bP \ | \ \|X\|_j \}$ is flattened by $\bP$.
		\end{enumerate}
	\end{theorem}
	\begin{tabular}{c|c}
		Topology $\bT$ & $\bT$-flat \\
		\hline
		fppf & flat affines \\
		\etale & formally \etale + flat affines \\
		Zariski & finite sum of principal opens	
	\end{tabular}
	\\ The notion of $\bT$-flat turns out to be incredibly useful in the case of the \etale-topology (because there $\bT$-flat affines are stable under identity types), for example for showing that Deligne-Mumford stacks are 1-gerbes or that geometric covers are formally \etale.
	
	Question:
	When is an Artin stack Deligne Mumford?
	
	\printbibliography
\end{document}