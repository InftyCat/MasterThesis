\documentclass{article}
\newcommand{\Cov}{\emph{Cov} }
\newcommand{\affineA}{(affine ?)}
\newcommand{\Cover}{\emph{Cov}er }
\newcommand{\truncation}{$\bT$runcation}
%\newcommand{\Atl}{\emph{Atlas}}
\usepackage{graphicx} % Required for inserting images
\newcommand{\red}[1]{{\color{red} #1}}
\newcommand{\St}{\mathsf{St}}
\newcommand{\CS}{\mathsf{CS}}
\newcommand{\Zar}{\mathsf{Zar}}
\newcommand{\refl}{\mathsf{refl}}
\newcommand{\ap}{\mathsf{ap}}
\newcommand{\Aff}{\mathsf{Aff}}

%%%%%%%%%%%%%%%%%%%%%%%%%%%%%%%%%%%%%%%%%%%%%%%%
% load packages 

\usepackage[a4paper,nohead,left=3.5cm,right=3.5cm,top=4cm,bottom=3cm]{geometry}
%\usepackage{german}    % only for German articles
%\usepackage{a4wide}   % longer lines
\usepackage[intlimits,tbtags]{amsmath}   % for more basic mathematical symbols
\usepackage{amssymb}   % for more mathematical symbols
\usepackage{amsfonts}
\usepackage[utf8]{inputenc}
\usepackage{textcomp}
\usepackage{mathtools}
% \usepackage{stmaryrd}  % for more mathematical symbols
\usepackage{latexsym}  % for more mathematical symbols,
% already contained in amsmath-package
% \usepackage{accents}  % for more dots etc on symbols
\usepackage{amsxtra}   % for accents as superscripts
\usepackage{amstext}   % for text in formulas, accents, etc.
\usepackage{bm}        % boldface for non-latin letters
\usepackage{amsthm}    % for theorem-environments
\usepackage{amscd}     % for commutative diagrams
%\usepackage{MnSymbol}
%\usepackage[ansinew]{inputenc}
\usepackage{enumitem}  % Better enumerations.

\usepackage{graphicx}
\usepackage[dvipsnames]{xcolor}
\usepackage[arrow, matrix, curve]{xy}
\usepackage[colorlinks=true, citecolor=Blue, linkcolor=Blue, urlcolor=Blue]{hyperref}
\usepackage{xfrac}
\usepackage[utf8]{inputenc}
\usepackage{scalerel}
\newcommand{\tA}{{\hspace{1pt}\sim}_{A}\hspace{1pt}}
\newcommand{\tR}{\hspace{1pt}{\sim}_{R}\hspace{1pt}}
\newcommand{\bA}{\mathbb{A}}
\newcommand{\bB}{\mathbb{B}}
\newcommand{\bC}{\mathbb{C}}
\newcommand{\bD}{\mathbb{D}}
\newcommand{\bE}{\mathbb{E}}
\newcommand{\bF}{\mathbb{F}}
\newcommand{\bG}{\mathbb{G}}
\newcommand{\bH}{\mathbb{H}}
\newcommand{\bI}{\mathbb{I}}
\newcommand{\bJ}{\mathbb{J}}
\newcommand{\bK}{\mathbb{K}}
\newcommand{\bL}{\mathbb{L}}
\newcommand{\bM}{\mathbb{M}}
\newcommand{\bN}{\mathbb{N}}
\newcommand{\bO}{\mathbb{O}}
\newcommand{\bP}{\mathbb{P}}
\newcommand{\bQ}{\mathbb{Q}}
\newcommand{\bR}{\mathbb{R}}
\newcommand{\bS}{\mathbb{S}}
\newcommand{\bT}{\mathbb{T}}
\newcommand{\bU}{\mathbb{U}}
\newcommand{\bV}{\mathbb{V}}
\newcommand{\bW}{\mathbb{W}}
\newcommand{\bX}{\mathbb{X}}
\newcommand{\bY}{\mathbb{Y}}
\newcommand{\bZ}{\mathbb{Z}}
\DeclareMathOperator{\Sh}{Sh}

%%%%%%%%% calligraphic %%%%%%%%%%%%%%%%%%%%%%%
\newcommand{\mc}[1]{\mathcal{#1}}
\newcommand{\R}{\Rightarrow}
\newcommand{\cA}{\mathcal{A}}
\newcommand{\cB}{\mathcal{B}}
\newcommand{\cC}{\mathcal{C}}
\newcommand{\cD}{\mathcal{D}}
\newcommand{\cE}{\mathcal{E}}
\newcommand{\cF}{\mathcal{F}}
\newcommand{\cG}{\mathcal{G}}
\newcommand{\cH}{\mathcal{H}}
\newcommand{\cI}{\mathcal{I}}
\newcommand{\cJ}{\mathcal{J}}
\newcommand{\cK}{\mathcal{K}}
\newcommand{\cL}{\mathcal{L}}
\newcommand{\cM}{\mathcal{M}}
\newcommand{\cN}{\mathcal{N}}
\newcommand{\cO}{\mathcal{O}}
\newcommand{\cP}{\mathcal{P}}
\newcommand{\cQ}{\mathcal{Q}}
\newcommand{\cR}{\mathcal{R}}
\newcommand{\cS}{\mathcal{S}}
\newcommand{\cT}{\mathcal{T}}
\newcommand{\cU}{\mathcal{U}}
\newcommand{\cV}{\mathcal{V}}
\newcommand{\cW}{\mathcal{W}}
\newcommand{\cX}{\mathcal{X}}
\newcommand{\cY}{\mathcal{Y}}
\newcommand{\cZ}{\mathcal{Z}}
\DeclareMathOperator{\chains}{Chains}

%%%%%%%%%%%%% mathematical fraktur  %%%%%%%%%%%%%%%%%%%%%
\newcommand{\mf}[1]{\mathfrak{#1}}
\newcommand{\senk}{\ \big \vert \ }
\newcommand{\fA}{\mathfrak{A}}
\newcommand{\fB}{\mathfrak{B}}
\newcommand{\fC}{\mathfrak{C}}
\newcommand{\fD}{\mathfrak{D}}
\newcommand{\fE}{\mathfrak{E}}
\newcommand{\fF}{\mathfrak{F}}
\newcommand{\fG}{\mathfrak{G}}
\newcommand{\fH}{\mathfrak{H}}
\newcommand{\fI}{\mathfrak{I}}
\newcommand{\fJ}{\mathfrak{J}}
\newcommand{\fK}{\mathfrak{K}}
\newcommand{\fL}{\mathfrak{L}}
\newcommand{\fM}{\mathfrak{M}}
\newcommand{\fN}{\mathfrak{N}}
\newcommand{\fO}{\mathfrak{O}}
\newcommand{\fP}{\mathfrak{P}}
\newcommand{\fp}{\mathfrak{p}}
\newcommand{\fQ}{\mathfrak{Q}}
\newcommand{\fR}{\mathfrak{R}}
\newcommand{\fS}{\mathfrak{S}}
\newcommand{\fT}{\mathfrak{T}}
\newcommand{\fU}{\mathfrak{U}}
\newcommand{\fV}{\mathfrak{V}}
\newcommand{\fW}{\mathfrak{W}}
\newcommand{\fX}{\mathfrak{X}}
\newcommand{\fY}{\mathfrak{Y}}
\newcommand{\fZ}{\mathfrak{Z}}
\newcommand{\lp}{_\flat} %{\boldsymbol{\cdot}}
\newcommand{\hp}{^\sharp}
\newcommand{\cp}{\boldsymbol{\cdot}}

\newtheorem{theorem}{Theorem}[section]
\newtheorem{satz}[theorem]{Satz}
\newtheorem{lemma}[theorem]{Lemma}
\newtheorem{korollar}[theorem]{Korollar}
\newtheorem{example}[theorem]{Example}
\newtheorem{prop}[theorem]{Proposition}
\DeclareMathOperator{\Spec}{Spec}
\newtheorem{corollary}[theorem]{Corollary}
\theoremstyle{definition}

\newtheorem{definition}[theorem]{Definition}
\newtheorem{ziel}[theorem]{Ziel}
\newtheorem{frage}[theorem]{Frage}
\newtheorem*{notation}{Notation}
\newtheorem*{slogan}{Slogan}
\newtheorem*{construction}{Construction}
\newtheorem*{bemerkung}{Bemerkung}
\newtheorem*{exercise}{Exercise}

\newtheorem*{note*}{Note}

\newtheorem{rmk}{Remark}
\newtheorem{bsp}[theorem]{Beispiel}
\newtheorem{aufgabe}[theorem]{Aufgabe}
\newtheorem*{beweis}{\it Beweis}
\newcommand{\gray}[1]{{\color{gray} #1}}
%%%%%%%%%%    Math operators    %%%%%%%%%%%%%%%%%%%%%%%%%%%

\DeclareMathOperator{\id}{id}             % identity morphism
% \DeclareMathOperator{\ker}{ker}           % kernel
\DeclareMathOperator{\im}{im}             % image
\DeclareMathOperator{\Hom}{Hom}           % homomorphisms
\DeclareMathOperator{\End}{End}           % endomorphisms
\DeclareMathOperator{\Span}{Span}         % linear span

\usepackage{tikz-cd}
\DeclareMathOperator{\pr}{pr}
\usepackage{quiver}
\renewcommand{\:}{\colon}
\DeclareMathOperator{\isContr}{isContr}

\newcommand{\type}{\ \mathrm{Type}}
\usepackage{stmaryrd}



\newcommand{\op}{^{op}}
%\renewcommand{\subset}{\subseteq}
%\newcommand{\colim}[1]{\mathrm{colim} \limits_{#1}}
\newcommand{\colim}[1]{\underset{#1}{\mathrm{colim} \ }}
\DeclareMathOperator{\sSet}{\mathsf {sSet}}
\DeclareMathOperator{\Pos}{\mathsf {Pos}}
\DeclareMathOperator{\Set}{\mathsf {Set}}
\DeclareMathOperator{\Fun}{Fun}
\DeclareMathOperator{\Cat}{\mathsf {Cat}}
\DeclareMathOperator{\const}{const}
\DeclareMathOperator{\Vect}{\mathsf{Vect}}
\DeclareMathOperator{\Top}{\mathsf{Top}}
\DeclareMathOperator{\Ring}{\mathsf{Ring}}
\DeclareMathOperator{\Field}{\mathsf{Field}}

\DeclareMathOperator{\Ab}{\mathsf{Ab}}
\DeclareMathOperator{\GL}{GL}
\DeclareMathOperator{\Ch}{\mathsf{Ch}}
\DeclareMathOperator{\Grp}{\mathsf{Grp}}

\DeclareMathOperator{\HomC}{\Hom_{\cC}}
\DeclareMathOperator{\HomD}{\Hom_{\cD}}
\usepackage{ascii}
\DeclareMathOperator{\Ob}{Ob}
\DeclareMathOperator{\FinVect}{FinVect}
\setlength\parindent{0pt} % Keine Einrueckung von Absaetzen
\newcommand{\etale}{\' etale }
\newcommand{\Etale}{\' Etale }
\DeclareMathOperator{\fib}{fib}
\newcommand{\todo}{{\color{Red} Todo}}
\newcommand{\todocite}{[ref?]}
\newcommand{\el}{\in}
\usepackage{wasysym}
\newcommand{\ci}{\fullmoon}
\DeclareMathOperator{\isProp}{isProp}
\DeclareMathOperator{\Prop}{Prop}
%\renewcommand{\in}{\colon}

\newcommand{\details}{[...]}
\DeclareMathOperator{\tp}{tp}
\DeclareMathOperator{\Nat}{Nat}
\renewcommand{\contentsname}{Inhalt}
\font\maljapanese=dmjhira at 2.5ex
\newcommand{\yo}{\textrm{\!\maljapanese\char"48}}
\DeclareMathOperator{\Aut}{Aut}
\DeclareMathOperator{\Mod}{\mathsf{Mod}}
\DeclareMathOperator{\Mat}{Mat}

\DeclareMathOperator{\isInv}{isInv}
\DeclareMathOperator{\Alg}{Alg}
\newtheorem{axiom}{Axiom}
\newtheorem{question}{Question}
\renewcommand{\mid}{\ | \ }

\newcommand{\fun}[4]{
	\begin{align*} 
		#1 &\to #2 \\ 
		#3 &\mapsto #4 
\end{align*}}
\newcommand{\funn}[5]{
	\begin{align*} 
		#1 \colon #2 &\to #3 \\ 
		#4 &\mapsto #5
\end{align*}}

\newcommand{\RHom}{R \cH om}
\newcommand{\Ltimes}{\overset{\mathrm{L}}{\otimes}}

\renewcommand{\GS}{\mathsf{GS}}
\title{Thesis}
\author{Tim Lichtnau }
\date{May 2024}

\begin{document}
\newtheorem*{warning}{Warning}
\newtheorem*{why}{Why I did it this way}
\newtheorem*{think}{Think about}
\maketitle
\section{Atlas}
\begin{definition}{\label{def:TAtlas}}
	% Let $T \subset \cU$ be any subtype of the universe. 
	% A $\bT$-cover 
	Given $\cV \subset \cU$ a subclass stable under $\sum$, a $\cV$-cover is a map fibered in $\cV$.
	A $\cV$-atlas of $X$ is a $\bT$-cover $\Spec A \to X$ out of an affine scheme. \\
	In the context of a topology $\bT$, We call a $\cV$-atlas $\Spec A \to X$ a $\cV$-catlas, if the domain $\Spec A$ belongs to $\bT$.
	% of a type $\bT$ is an affine Scheme $\Spec A$ with a formally \etale \etale-surjective map
	% \[
	% \Spec A \to T.
	% \]    
\end{definition}

% \begin{lemma}
%     $\bT$-atlasses are stable under composition.
% \end{lemma}
%\begin{rmk}{\label{rmk:defatlas}}
%	Any good enough TODO scheme has a Zariski atlas. If $\bT$ is finer than the Zariski-topology then in the definition we may replace affine scheme by good enough scheme, if its just about the question whether a type admits an atlas.
%\end{rmk}

\begin{example}
	Let $X$ be a (1-)type. $X$ has a $\Zar$-atlas, iff there exists some $f : \Spec A \to X$ fibered in types of the form $\Spec (R_{f_1} \times \hdots \times R_{f_n})$ for $(f_1,\hdots,f_n) \in Um(R)$. 
\end{example}
\begin{rmk}{\label{ZLCGivesZariski}}
	If one applies ZLC to an affine scheme $\Spec A$ the resulting principal open cover $D(f_i), f_i \in A$ will induce indeed a zariski atlas $\bigsqcup D(f_i) \to \Spec A$, because the fiber over $x : \Spec A$ is $\bigsqcup D(f_i(x))$.
\end{rmk}
Question: Does every zariski atlas of $\Spec A$ have this form? \nameref{ex:weirdZarAtlasses}
%Let $\cZ$ be the class of types which have a Zariski cover.

\begin{example}{\label{ex:PnIsStack}}
	$\bP^n$ has a zariski atlas given by the standart homogeneous principal opens $\sum_{i=0}^n D_+(x_i)$. The fiber over a point $[y_0 : \hdots . y_n]$ is $D(y_0) + \hdots D(y_n)$ where $(y_1,\hdots,y_n) \in Um(R)$. % Indeed the standart principal opens $D_+(x_0) , \hdots , D_+(x_n)$ form a presentable open cover.    
\end{example}

\begin{definition}
	A Zariski sheaf $X$ is a scheme if there merely exists some affine $S$  map $S \to X$ whose fibers are Zariski-merely inhabited finite sums of open propositions 
\end{definition}
\begin{lemma}{\label{lemma:IsScheme}}
	Every $\Zar$-sheaf that admits a $\Zar$-atlas is a scheme. 
\end{lemma}
\begin{proof}
	Obvious.
\end{proof}
% Warning: Let $X$ be a . Then $X$ is already affine if it has a Zariski cover, i.e.  there exists some $f : \Spec A \to X$ fibered in types of the form $\Spec (R_{f_1} \times \hdots \times R_{f_n})$ for $(f_1,\hdots,f_n) \in Um(R)$. More generally: We can facto

\section{Preparation}


\begin{lemma}{\label{lemma:havingAbstractAtlasClosedUnderId}}
	Let $C$ be a class of types stable under $\sum$. %HOPE WE DONT NEED (because we want to apply it to \cV = covering stacks) finite limits, i.e. containing 1, stable under dependent sums and finite limits
	The class $\mathsf{HasAtlas}_C$ of types $Y$ which admit a map $\Aff \ni S \to Y$ fibered in $C$ is stable under identity types. %.  If it contains $C$ and is closed under dependent sums, then it is closed under taking identity types.
\end{lemma}
\begin{proof}
%	Obviously 1 has an atlas, and the class of types admitting an atlas is stable by $\sum$ by \ref{thm:atlasStableSum}.
%	It remains to show, that identity types in $Y$ have an atlas provided that $Y$ has an atlas.
	
	%This is a special case of stability under dependent sums. But lets prove it anyway.
	By assumption we can choose a map $\Aff \ni V \overset{p}{\to} Y$ fibered in $C$. Let $y,y' : Y$.  Then we have the map
	\begin{align*}
		(\fib_p y) \times_V (\fib_p y') &\to y = y' \\
		(v , q : y = p v) , (v', q' : y' = p v') , (h : v = v') &\mapsto q \cdot h \cdot q'^{-1}
	\end{align*}
	
	The fiber over $j : y = y'$ looks like  %because $y$ and $y'$ are free we may only show the statement for the fiber over the path  $\mathsf{refl}_y : y = y$. 
	\[
	\sum_{v}  ( \underbrace{\sum_{v'} (h : v = v')}_{\mathsf{isContr}}) \times (q : y = p v) \times (q'  : y' = p v') \times (q \cdot h \cdot q'^{-1} = j) \simeq \sum_v (v = py) \simeq \fib_p y
	\]
	Hence the map is fibered in $C$. It suffices to show, that	$(\fib_p y) \times_V (\fib_p y')$ has an atlas, because then we can compose such an atlas with the above map to obtain an atlas of $y = y'$.
	By assumption the fibers of $p$ have an atlas, so we can choose $q : W \to \fib_p y, q' : W' \to \fib_p y'$ atlasses. Then $W \times_V W' \to (\fib_p y) \times_V (\fib_p y')$ is an atlas: The domain is a fiber product of types in $\Aff$, hence it belongs to $\Aff$. The fiber over $(x,x')$ is equivalent to the product of fibers $(\fib_q x) \times (\fib_{q'} x')$ which is in $C$ by stability under dependent sums (so in particular under finite products).
	
\end{proof}
\begin{lemma}{\label{lemma:AtlasSum}}
	Let $\cU' \subset \cU$ be stable under dependent sums.
	Let $X$ be a type with a  map $p : U \to X$ fibered in $\cU'$.  For any $x : X$, let $Y_x$ be a type and moreover for any $u : U$, we are given a map $q_u : V_u \to Y_{p(u)}$ fibered in $\cU'$. Then the induced map
	\[
	p : \sum_{u : U} V_u \to \sum_{x : X} Y_{x}
	\]
	is fibered in $\cU'$
\end{lemma}
\begin{proof}
	The fiber of $p$ over some $(x,y) \in \sum_{x :X} Y_x$ is given by
	\[
	\sum_{u : \fib_p x} \fib_{q_u} (y') 
	\]
	where $y' : Y_{p(u)}$ (depending on $u$) is the transport of $y : Y_x$ along $x = p(u)$. As $\cU'$ is stable under dependent sum %(\ref{lemma:LexSumStable}, \ref{lemma:LexStability}), 
	those fibers are again in $\cU'$. This shows the result.
\end{proof}

\section{Lex Modalities}
\begin{lemma}[Stability resuls]{\label{lemma:LexStability}}
	Lex Modalities are stable under 
	\begin{enumerate}
		\item Conjunction
		\item Composition
	\end{enumerate}
	
\end{lemma}
\begin{lemma}{\label{lemma:LexSumStable}}
	Let $\ci$ be a lex-modality. Let $X$ be $\ci$-modal and $B : X \to \cU_{\ci}$ be a family of modal types. Then $\sum_{x : X} B_x$ is $\ci$-modal
\end{lemma}
\begin{lemma}{\label{lemma:mod_comm_sum}}
	Let $B  : \bullet X \to \cU$. Then $\bullet (\sum_{x : X} B (\eta x)) = \sum_{x : \bullet X} \bullet B x$
\end{lemma}
\begin{proof}
	Observe that 
	\[
	\sum_{x : X} B \eta x \to \sum_{x : \bullet X} B x
	\]
	is a $\bullet$-equivalence, because for all modal types $T$, the type $B x \to T$ is modal for any $x : \bullet X$. \\
	Then it follows by \todocite.
\end{proof}

\begin{lemma}{\label{lemma:sep}}
	For a type $X$ the following are equivalent:
	\begin{itemize}
		\item the identity types of $X$ are sheaves
		\item the unit $X \to \bullet X$ is a monomorphism
	\end{itemize}
In this case we call $X$ seperated
\end{lemma}

\section{Local Choice}
In this section let $\bT$ denote a topology finer than the zariski topology.
\begin{definition}
	Let \Cov be a class of morphisms (which we think of $n$-atlasses of some $n$), containing $\bT$-atlas, (stable under pullback NECESSARY TODO?)
	A type $S$ has \emph{local choice} wrt \Cov if for any $\bT$ -surjective map $X \to Y$ and any map $f : S \to Y$ there exists a map  $p' : S' \to S$ in \Cov and a commutative diagram
	% https://q.uiver.app/#q=WzAsNCxbMCwwLCJUJyJdLFswLDEsIlQiXSxbMSwwLCJYIl0sWzEsMSwiWSJdLFsxLDNdLFsyLDNdLFswLDIsIiIsMix7InN0eWxlIjp7ImJvZHkiOnsibmFtZSI6ImRhc2hlZCJ9fX1dLFswLDFdXQ==
	% https://q.uiver.app/#q=WzAsNCxbMCwwLCJUJyJdLFswLDEsIlQiXSxbMSwwLCJYIl0sWzEsMSwiWSJdLFsxLDNdLFsyLDMsInAiLDJdLFswLDIsIiIsMix7InN0eWxlIjp7ImJvZHkiOnsibmFtZSI6ImRhc2hlZCJ9fX1dLFswLDFdXQ==
	\[\begin{tikzcd}
		{S'} & X \\
		S & Y
		\arrow[dashed, from=1-1, to=1-2]
		\arrow[from=1-1, to=2-1]
		\arrow["p"', from=1-2, to=2-2]
		\arrow["f",from=2-1, to=2-2]
	\end{tikzcd}\]
	%We say $S$ has affine local choice, if one can arrange $S'$ to be affine.
\end{definition}
\begin{prop}{\label{prop:LocalChoice}}
	%Let $\bT$ be a finer topology than the zariski topology.
	Assume that \Cov is stable under composition and that Zariski-covers are in \Cov.
	$S$ has  $\bT$-local choice wrt \Cov if it has a projective \Cover, i.e. there exists a projective (or, assuming ZLC, affine scheme resp.)  $\hat{S}$ with a map $g : \hat{S} \to S$ in \Cov. %, satisfying local choice wrt \Cov
\end{prop}
\begin{proof}
	%We may assume that $f = \id_S$.
	By stability under composition of \Cov, We may assume that $g : \hat{S} \to S$ is the identity.
	As $p$ is $\bT$-surjective, for any $x : S$ there merely is a $\Spec B_x \in T$  and a map $\Spec B_x \to \| \fib_p (x) \| $. 
	Claim: No matter on the assumptions (on $S = \hat{S}$), there exists a Zariski cover $S' \overset{p'}{\to} S$ with $S'$ projective (affine resp.) and a term in
	\[\prod_{x : S'} \sum_{\Spec B_x \in T} \Spec B_x \to \| \fib_p (fp' x)\| \]
	Proof: In the case of projectivity, just use $p' = \id_S$ and in the case of having ZLC and $S$ beeing affine, use ZLC (\ref{ZLCGivesZariski}). \qed(Claim)\\    
	By setting 
	\[(S'' := \sum_{x : S'} \Spec B_x) \overset{\pi}{\longrightarrow} S' \]
	
	the projection, we are now in the situation that for any $t : S''$ we merely have a point in $\fib_p((p''(t)))$ and $S'' \to S'$ is a $\bT$-cover, thus it is in \Cov. Moreover, $S''$ is a projective type (affine), as it is a dependent sum of projectives (affines). Hence again we now can find a lift $S'' \to X$. %By replacing $S''$ again with a Zariski cover we find a lift $S'' \to X$     
	making
	% https://q.uiver.app/#q=WzAsNSxbMCwwLCJUJyciXSxbMCwxLCJUJyJdLFswLDIsIlQiXSxbMSwyLCJYIl0sWzEsMCwiWSJdLFswLDFdLFsyLDMsImYiXSxbNCwzLCJwIiwyXSxbMCw0XSxbMSwyLCJwJyJdXQ==
	\[\begin{tikzcd}
		{S''} & X \\
		{S'} \\
		S & S
		\arrow[from=1-1, to=1-2]
		\arrow[from=1-1, to=2-1]
		\arrow["p"', from=1-2, to=3-2]
		\arrow["{p'}", from=2-1, to=3-1]
		\arrow["\id", from=3-1, to=3-2]
	\end{tikzcd}\]
	commute. Now $S'' \to S' \to S$ as the composition of Zariski-covers and \Cover is a \Cover \details as desired.
	%For the general case, the previous proof is enough. \todo
\end{proof}
The next lemma shows, that the class of types equipped with a $\bT$-atlas is stable under dependent sums.
\begin{lemma}{\label{lemma:AtlasSum}}
	Let $\cU' \subset \cU$ be stable under dependent sums (e.g. $\bT$-inhabited types)
	Let $X$ be a type with a  map $p : U \to X$ fibered in $\cU'$.  For any $x : X$, let $Y_x$ be a type and moreover for any $u : U$, we are given a map $q_u : V_u \to Y_{p(u)}$ fibered in $\cU'$. Then the induced map
	\[
	p : \sum_{u : U} V_u \to \sum_{x : X} Y_{x}
	\]
	is fibered in $\cU'$
\end{lemma}
\begin{proof}
	The fiber of $p$ over some $(x,y) \in \sum_{x :X} Y_x$ is given by
	\[
	\sum_{u : \fib_p x} \fib_{q_u} (y') 
	\]
	where $y' : Y_{p(u)}$ (depending on $u$) is the transport of $y : Y_x$ along $x = p(u)$. As $\cU'$ is stable under dependent sum %(\ref{lemma:LexSumStable}, \ref{lemma:LexStability}), 
	those fibers are again in $\cU'$. This shows the result.
\end{proof}


\section{Covering stacks}
Fix $\bT$ a topology, which we call the covering-affines.
\begin{definition}
	covering stacks are the smallest class containing contractible Types such that: If $Y$ is a stack and $\bT \ni S \to Y$ is fibered in covering stacks, then $Y$ is a covering stack.	
\end{definition}
We call such map $X \to Y$ whose fibers are covering stacks a geometric cover. If $X$ is affine we call it a geometric atlas. If $X$ is in $\bT$ we call it a geometric catlas. 
\begin{definition}
	We call $X$ a geometric stack if it merely has a geometric atlas, i.e some $\Spec A \to X$ fibered in covering stacks.
\end{definition}
\begin{prop}[Recursion principle for covering / geometric stacks]
	Let $P$ be a property of covering / geometric stacks. Assume
	\begin{itemize}
		\item contractibles have $P$
		\item If $S$ is (covering) affine and $S \to Y$ is fibered in covering stacks having $P$ then $Y$ has $P$
	\end{itemize}
	Then every covering / geometric stack has $P$.
\end{prop}
%\begin{proof}
%	Replace $P$ by $P \land \mathsf{is-covering-stack}$. Then usual induction
%	
%\end{proof}
\begin{why}
	Should $P$ be defined more generally for all sheaves?
	No, because we want for the recursion principle for geometric stacks, that the fibers are covering stacks (proof of truncatedness).
%	If $P$ is defined only for covering stacks, do we need to talk about $P$-covers between covering stacks without knowing that the fibers are covering stacks as well?
\end{why}
\begin{prop}{\label{prop:csHasAtlas}}
Every covering stack $X$ merely admits a geometric catlas. %, i.e. a geometric cover $Y \to X$ with $Y \in \bT$. 
\end{prop}

\begin{proof}
%We apply the recursion principle of covering / geometric stacks 
\begin{itemize}
	\item If 	$X$ is covering affine, then $X \to X$ is a geometric catlas. 
	\item If $X$ is obtained as a quotient then it already is equipped with a catlas. %, i.e. if its equipped with a cover $Y \to X$ with $Y$ a covering stack, then by induction $Y$ admits a $\cV$-catlas $S \to Y$. Then $S \to Y \to X$ is a $\cV$-catlas by  \ref{lemma:coversstableundercomp}. \\
	%\item If $X$ is obtained as a sum, i.e. we have a $\cV$-cover $f : X \to Y$, then by induction $Y$ admits an $\cV$-catlas $g : S \to Y$ and the fibers merely have $\cV$-catlasses $S_y \to \fib_f y$ s. By choice of $S$, we can choose such catlasses $S_{g s} \to \fib_f (g s)$ for all $s : S$. By \ref{lemma:AtlasSum} the map 
	%\[
	%\sum_{s : S} S_{gs} \to (\sum_{y: Y} \fib_f y ) \simeq X
	%\]
	%is a $\cV$-cover. Its domain is a covering affine as $\bT$ is $\sum$-stable. Hence $X$ admits a $\cV$-catlas .
	
\end{itemize}
\end{proof}

\subsection{Needing finitely many steps}
In this subsection we want to prove that one could equivalently define covering stacks just by induction over the natural numbers.

\begin{lemma}{\label{lemma:cstinh}}
	Every covering stack $X$ is $\bT$-merely inhabited.
\end{lemma}
\begin{proof} 
	\begin{itemize}
		\item If $X$ is in $\bT$ then its clear.
		\item  If $X$ is obtained by a quotient, we have a map $\Spec A \to X$ with domain in $\bT$. Now use that we get a map on $\bT$-propositional-truncations and that Spec A is T-merely inhabited.
		%		\item if $X$ is obtained by  $X = \sum_{y: Y} B y$ for $Y$ a covering $\cV$-stack and $B y$ covering $\cV$-stacks, by induction all the $B y$ are $\bT$-merely inhabited. Hence, for all $y : Y$, we can conclude $\| X\|_\bT$. As $Y$ is $\bT$-merely inhabited by induction and the goal is a stack, we can conclude. 
	\end{itemize}
\end{proof}
\begin{prop}{\label{prop:FindCommonN}}
	Given a geometric stack $Y$ and a family $M : Y \to (\bN \to \Prop_{\ci})$  be a family of upwards closed merely inhabited subsets of $\bN$. Then there exists some $n$, such that $M y n$ for all $y : Y$.
\end{prop}
\begin{proof}
	Write $M_n = \{y : Y \ | \ M y n\}$.
	Choose a geometric atlas $f : S \to Y$.
	For any $x : S$, $M(f x) n$ for some $n$. By foundations Prop 3.3.5, we merely find some $n$ such that $f(x) \in M_n$ for all $x : S$. Let us show that for general $y : Y$ we have $y \in M_n$. Using that $y \in M_n$ is modal , we can conclude by $\bT$-surjectivity of $f$, which follows from \ref{lemma:cstinh}
	
\end{proof}
\begin{prop}{\label{prop:OneToRuleThemAll}}
	Let $W : \GS \to (\bN \to \Prop_{\ci})$ be a family of upwards closed subsets of $\bN$. Assume
	\begin{itemize}
		\item $W 1$ is merely inhabited
		\item whenever there is some $n : \bN$ and a geometric atlas $S \to X$ fibered in covering stacks $F$ satisfying $W F n \equiv: W_n F$, then $W_{n+1} X$.
	\end{itemize}  %(or more generally $W X$ is merely inhabited). 
	Then for any $X \in \GS$, $W X$ is merely inhabited.
\end{prop}

\begin{proof}
		We apply the recursion principle for geometric stacks.
	\begin{itemize}
		\item If $Y$ is contractible its clear by assumption
		\item Assume $Y$ is equipped with a geometric atlas $f : S \to Y$, such that every fiber has $W_n$ for some $n$. Apply \ref{prop:FindCommonN} to $M y n = W_n (\fib_f y)$ to find some $n$ such that $W_n (\fib_f y)$ for all $y : Y$.
		Then we can conclude by applying the assumption.
	
		%	\item Let $X$ be an $n$-truncated covering stack. By \ref{prop:csHasAtlas} we find a geometric catlas $S \to X$. All the fibers are (at most) $n$-truncated. 	
	\end{itemize}
\end{proof}


\begin{definition}
	Define \begin{align*}
		W_0 &\equiv \bT \\
		W_{n+1} &\equiv \{X \ stack \ | \  X \text{ has a } W_n-catlas \}
	\end{align*}
\end{definition}
\begin{why}
	$W 0 $ is not defined as $\isContr$, because for $\sum$ stability later, we want to apply \ref{thm:atlasStableSum}, so we need that Zariski covers are allowed covers.
\end{why}
\begin{lemma}
	$W$  is monotone
\end{lemma}
\begin{proof}
	We prove $\forall n . W n \subset W (n+1)$.
	Induction. $n = 0$. For any $X :\bT$ , $ X \to X$ is a $W_0$-catlas, as $1 \in \bT = W_0$. If $X \in W_n$, it admits a $W_{n-1}$ catlas. By induction this is a $W_n$ catlas. So $X \in W_{n+1}$.
\end{proof}
\begin{lemma}{\label{prop:WnSigma}}
	For all $n : \bN$, $W_n$ covering stacks are $\sum$-stable. \\
\end{lemma}
\begin{proof}	
	Induction over $n$. If $n = 0$, then this is the stability under $\sum$ of $\bT$ \\
	If we wish to prove the statement for $n+1$, we may assume that $W_n$ covering stacks are $\sum$-stable. We have $\Zar \subset \bT \subset W_n$. So we can apply \ref{thm:atlasStableSum}. \\
\end{proof}	
\begin{prop}{\label{prop:WTrunc}}
	Every covering stack has $W_n$ for some $n$.
\end{prop}
\begin{proof}
	The idea is to apply \ref{prop:OneToRuleThemAll}.
	We need that $X \in W_n$ is a sheaf for $X$ a stack. \\
	Let $T \in \bT$ such that $T \to \exists (\bT \ni S \to X \ W_n \text{-atlas})$. We want to construct a $W_n$-catlas of $X$. By Zariski local choice we find a Zariski atlas $T' \to T$ with a term in 
	\[\prod_{t : T'} \sum_{S_t : \bT} W_n \mathsf{atlas} (S_t , X) \]
	 From this we obtain a map 
	 \[\sum_{t : T'} S_t \to T' \times X \to X\].  As $T' \in \bT \subset W_n$ by $\sum$-stability of $\bT$, both maps are $W_n$-covers. By $\ref{prop:WnSigma}$ the composite is a $W_n$-cover. Its domain is in $\bT$ by $\sum$-stability of $\bT$. This is what we wanted to show.
\end{proof}

\subsection{Stability}
\begin{theorem}{\label{thm:CSSum}}
	The class of covering / geometric stacks is $\sum$-stable.
\end{theorem}
\begin{proof}
	The geometric case follows from the covering case by \ref{thm:atlasStableSum}.
	Let $X$ be a covering stack and $B : X \to \CS$ a family of covering stacks.
	We apply \ref{prop:FindCommonN} to the predicate ' $X$ belongs to $W n$ for some $n$', which holds definitely for some $n$ by \ref{prop:WTrunc}.
	So we merely find an $n : \bN$ such that $B x \in W_n $ for all $x : X$. By making $n$ larger, we may assume $X$ has $W n$ for some $n$. Conclude by \ref{prop:WnSigma}\\


\end{proof}
\begin{lemma}{\label{lemma:coversstableundercomp}}
	geometric covers are stable under composition.
\end{lemma}
\begin{proof}
	covering stacks are stable under $\sum$.
\end{proof}


\begin{prop}{\label{prop:stackQuot}}
	The class of covering / geometric stacks is stable under quotients: If $X \to Y$ is fibered in covering stacks and $X$ is a (covering) stack and $Y$ is a stack then $Y$ is a covering / geometric stack.
\end{prop}
\begin{proof}
	Choose a geometric (c)atlas of $X$. Then the composition with the map $X \to Y$ is a cover by \ref{lemma:coversstableundercomp}. As the domain is (covering) affine, its a geometric (c)atlas.
\end{proof}
Now we want to show that the clash of terminology regarding 'covering' is reasonable:


\begin{prop}{\label{prop:affineCoveringStack}}
	Let $\bT$ be saturated.
	A covering stack $X$ is affine iff its a covering affine.
\end{prop}
\begin{proof}
	The converse is clear. The direct direction follows by the recursion principle. choosing a geometric catlas  $S \to X$. As both $S$ and $X$ are affine the fibers are affine. By induction the fibers are covering affines. By saturatedness of the topology $X$ is covering affine.
\end{proof}
\begin{lemma}{\label{lemma:atlasIsCatlas}}
	Let $\bT$ be saturated. Let $X$ be a covering stack. Let $f : \Spec A \to X$ be a geometric atlas. Then $\Spec A \in \bT$
\end{lemma}
\begin{proof}
	As $\Spec A \simeq \sum_{x : X} \fib_f x$ is a dependent sum of covering stacks, it is a covering stack again by \ref{thm:CSSum}. We conclude by \ref{prop:affineCoveringStack}.
\end{proof}	
\begin{lemma}{\label{lemma:geomStackPlusStable}}
	geometric stacks are stable under finite sums.
	If $\bT$ is finer than the zariski topology, then this holds for covering stacks as well
\end{lemma}
\begin{proof}
	We have to show that finite sums of geometric (c)atlasses are geometric (c)atlasses.
	For the geometric case just use that affines are stable under finite sums. For the covering case use that $1 + \hdots + 1 \in \Zar \subset \bT$, hence the topology is stable under finite sums.
\end{proof}
%\subsection{Geometric stacks}

\begin{lemma}{\label{lemma:geometricStacksClosedUnderId}}
	geometric stacks are closed under $\id$-types.
\end{lemma}
\begin{proof}
	
	This is \ref{lemma:havingAbstractAtlasClosedUnderId}, using that covering stacks are closed under $\sum$ (\ref{thm:CSSum})
\end{proof}

\begin{warning}
	The previous lemma does not hold for covering stacks: Identity types of things in $\bT$ could be empty.
\end{warning}

\subsection{About the covering stacks in a subuniverse}
\begin{definition}
	Let $\cV \supset \mathsf{Aff}$ be a superclass stable under $\sum$. covering $\cV$ stacks are the smallest intermediate class $\bT \subset \CS_\cV \subset \cV$ such that: If $X : \bT$ ,  $Y : \cV$ and $X \to Y$ is fibered in $\CS_\cV$, then $Y \in \CS_\cV$. \\
	$X$ is a geometric $\cV$-stack if its in $\cV$ and it merely admits a map $\Spec A \to X$ fibered in $\CS_\cV$.
\end{definition}
%We call such map $X \to Y$ whose fibers are covering $\cV$-stacks a geometric-$\cV$-cover. If $X$ is affine we call it an geometric-$\cV$ atlas. If $X$ is in $\bT$ we call it a geometric-$\cV$-catlas. 
\begin{definition}
	We define the saturation of $\bT$ as the class of covering Aff-stacks. We call a topology $\bT$ saturated if it coincides with its saturation, or more concretely: Every affine schemes that has a catlas lies itself in $\bT$. \\ 
\end{definition}
In a further chapter we will develop this theory further.



\begin{prop}{\label{prop:coveringVstackDescr}}
	Let $\cV$ be stable under finite limits and containing (covering) affines. $X$ is a (covering) $\cV$-stack iff it is in $\cV$ and a covering / geometric stack.
\end{prop}
\begin{proof} 	
	The direct direction is clear. For the converse we apply the recursion principle to the property '$X \in \cV$ implies $X$ is a (covering) $\cV$-stack'. If $X$ is contractible, its clear. Otherwise its equipped with a geometric (c)atlas. The fibers are in $\cV$, as they can be written as a fiberproduct of $S, X, 1 \in \cV$.  By induction all fibers are covering $\cV$-stacks (we may show the covering part of the proposition first). %We are left to show that $F$ is a covering $\cV$-stack. \\
	%	We can choose $S \to F$ a $\cV$-atlas, so in particular a geometric atlas of $F$, which was assumed to be a covering stack. Then $S \in \bT$ by \ref{lemma:atlasIsCatlas}. So we actually have a $\cV$-catlas.
\end{proof}
\begin{prop}{\label{prop:CSVSum}}
	(covering) $\cV$-stacks are stable under dependent sums. In particular the saturation of a topology defines a topology.
\end{prop}
\begin{proof}
	Both the classes $\cV$ and (covering) stacks are stable under dependent sums. Hence the intersection of them is $\sum$-stable as well. \\
	The saturation is a class of affines, that in particular contains $1 \in \bT$. We have argued its stable under $\sum$.
\end{proof}
\begin{prop}{\label{prop:V'stacks}}
	A stack $X$ merely admits some map $S \to X$ out of a (covering) affine fibered in covering $\cV$-stacks, iff its a covering / geometric stack whose identity types are in $\cV$. 
\end{prop}
\begin{proof}
	The direct direction: By \ref{lemma:havingAbstractAtlasClosedUnderId} the identity types are geometric $\cV$-stacks. \\
	The converse direction: Choose a geometric (c)atlas $f : S \to X$. As each fiber $\sum_{s : S} f s =_X x$ is in $V$ by $\sum$-stability of $\cV$ and is a covering stack, its a covering $\cV$-stack by \ref{prop:coveringVstackDescr}.
\end{proof}
\begin{definition}
	Let $n \ge -2$. A covering / geometric $n$-stack is a covering / geometric stack that is an $n$-type.
\end{definition}
\begin{prop}{\label{prop:nstack}}
	Let $X$ be a stack. For all $n \ge 0$, the following are equivalent:
	\begin{enumerate}
		\item $X$ is a covering / geometric $n+1$-stack
		\item $X$ merely admits some map $S \to X$ out of a (covering) affine fibered in covering $n$-stacks
		\item $X$ merely admits some covering / geometric $n$-stack $Y$ and a map $Y \to X$ fibered in covering $n$-stacks.
	\end{enumerate}
\end{prop}
\begin{proof}
	\
	\begin{enumerate}
		\item[1 . $\Leftrightarrow$ 2.]
		$X$ is a covering / geometric $n+1$ stack iff its a covering / geometric stack whose identity types are $n$-types. But this is equivalent to 2. by \ref{prop:V'stacks}.
%		\begin{align*}
%			& \text{$X$ is a covering / geometric $n+1$ stack} \\
%			&\overset{ \ref{lemma:geometricStacksClosedUnderId}} {\Leftrightarrow} \text{$X$ is a covering / geometric stack whose identity types are $n$-types} \\
%			&\overset{\ref{prop:V'stacks}} {\Leftrightarrow} \text{2.}
%		\end{align*}
		\item[2 . $\Rightarrow$ 3.]
		$S$ is a covering / geometric $n$-stack
		\item [3. $\Rightarrow$ 2]
		$Y$ admits a map $S \to Y$  fibered in covering $n$-stacks with $S$ (covering) affine, so the composition $S \to X$ will have the same property by \ref{lemma:coversstableundercomp}.
	\end{enumerate}
\end{proof}
\begin{prop}
	We have inclusions 
	\[W_n \subset \CS_{n} \subset W_{n+1}\]
\end{prop}
\begin{proof}
	
\end{proof}

\subsection{Truncatedness}
\begin{lemma}{\label{lemma:truncTrg}}
	Let $X$ be an $n+1$-type and $Y$ a stack. If $X \to Y$ is a $n$-truncated $\bT$-surjective map, then $Y$ is an $n+1$-type.
\end{lemma}
\begin{proof}
	Use that $\mathsf{is-n-truncated} (y=y')$ is a stack for $y , y' : Y.$
\end{proof}

\begin{corollary}
	Every geometric stack is $n$-truncated for some $n : \bN$.
\end{corollary}
\begin{proof}
	Apply the prop \ref{prop:OneToRuleThemAll}. Use \ref{lemma:truncTrg}. For a stack $X$, is-$n$-truncated $X$ is indeed a stack.
\end{proof}


%\subsection{Descent}
%$\St$ a class of stacks in $\cV$, such that $\bT$ is contained in it and for any $\bT$-cover $X \to Y$ of stacks in $\cV$, $X \in \St$ iff $Y \in \St$. We call types in this class stacky.
%\begin{lemma}{\label{lemma:stackificationHasTCover}}
%	Let $\bT$ satisfy descent, i.e. beeing affine in the topology is a stack. If $Y$ admits a $\bT$-cover $f : X \to Y$ where $Y$ is seperated, then there is a $\bT$-cover $X \to L_\bT Y$.
%\end{lemma}
%\begin{proof}
%	
%	Consider $X \overset{f}{\to} Y \overset{\eta}{\to} L_\bT Y$. As beeing affine in $\bT$ is  a stack, we may just show that for all $y : Y$ , the fibers over $\eta y : L_\bT Y$ are in $\bT$. As $\eta$ is a monomorphism by \ref{lemma:sep} , $\eta$ restricts to an equivalence
%	\[
%	\fib_f y \to \fib_{ \eta f}(\eta y)
%	\]
%	
%	But the left hand side is in $\bT$ by assumption. 
%\end{proof}
%\begin{lemma}
%	Assume $\bT$ have descent.
%	Let $X \in \St$ and $Y$ a type.	Let $f : X \twoheadrightarrow Y$ be fibered in $\bT$ and surjective. Then $L_\bT Y$ is stacky.
%\end{lemma}
%\begin{proof}
%	As $X$ is stacky, it suffices to show, that $L_\bT Y$ admits a $\bT$-cover.
%	We want to apply \ref{lemma:stackificationHasTCover}. So it remains to show, that $Y$ is seperated. By surjectivity of $f$ we may only show that for any $x : X, y : Y$, the type $f x =_Y y$ is a stack. If we define $U$ to be the fiber over $y$, it is in $\bT$ by assumption. But then $f x =_Y y$ is the outer pullback
%	% https://q.uiver.app/#q=WzAsNixbMCwwLCJmIHggPSB5Il0sWzEsMCwiVSBcXGluXFxiVCJdLFswLDEsIjEiXSxbMSwxLCJYIl0sWzIsMCwiMSJdLFsyLDEsIlkiXSxbMyw1LCJmIl0sWzQsNSwieSIsMl0sWzIsMywieCJdLFsxLDNdLFsxLDRdLFsxLDUsIiIsMSx7InN0eWxlIjp7Im5hbWUiOiJjb3JuZXItaW52ZXJzZSJ9fV0sWzAsMl0sWzAsMV1d
%	\[\begin{tikzcd}
%		{f x = y} & {U} & 1 \\
%		\arrow["\ulcorner"{anchor=center, pos=0.125}, draw=none, from=1-1, to=2-2]
%		1 & X & Y
%		\arrow[from=1-1, to=1-2]
%		\arrow[from=1-1, to=2-1]
%		\arrow[from=1-2, to=1-3]
%		\arrow[from=1-2, to=2-2]
%		\arrow["\ulcorner"{anchor=center, pos=0.125}, draw=none, from=1-2, to=2-3]
%		\arrow["y"', from=1-3, to=2-3]
%		\arrow["x", from=2-1, to=2-2]
%		\arrow["f", from=2-2, to=2-3]
%	\end{tikzcd}\]
%	of stacky types, in particular stacks. \qed(Claim) \\\\
%	
%\end{proof}
%\begin{theorem}
%	Assume $\bT$ have descent. Then $\St$ is a stack.
%\end{theorem}
%\begin{proof}
%	$\St$ is seperated: This follows from the embedding $\St$ into the seperated type of stacks \ref{lemma:stacksHaveDescent}. \\
%	Let $U \in \bT$ and $P : \|U\| \to \St$. We want to construct a filler 
%	% https://q.uiver.app/#q=WzAsMyxbMCwwLCJcXHwgVVxcfCJdLFswLDEsIjEiXSxbMSwwLCJcXFNtU3QiXSxbMCwyLCJQIl0sWzAsMV0sWzEsMiwiIiwyLHsic3R5bGUiOnsiYm9keSI6eyJuYW1lIjoiZGFzaGVkIn19fV1d
%	\[\begin{tikzcd}
%		{\| U\|} & \St \\
%		1
%		\arrow["P", from=1-1, to=1-2]
%		\arrow[from=1-1, to=2-1]
%		\arrow[dashed, from=2-1, to=1-2]
%	\end{tikzcd}\]
%	Claim: $L_\bT (\sum_{x: \|U\|} P x)$ is stacky.
%	\begin{proof} of the claim. We want to apply the previous lemma to the surjection 
%		\[\sum_{x : U} P | x | \to \sum_{x : \| U\|} P x \]
%		The domain is in $\St$ by stability under $\sum$. The fibers are equivalent to $U \in \bT \subset \St$.				
%	\end{proof}
%	The claim provides the map $1 \to \St$. The diagram commutes: Assuming $x : \|\Spec A\|$ we wish to show $P x = \sum_{x: \|U\|} P x$. Using univalence, we may show that the maps 
%	\[P x \to \sum_{x: \|U\|} P x \overset{\eta}{\to} L_\bT \sum_{x: \|U\|} P x\]
%	are both equivalences.
%	The first one is an equivalence as $\|U\|$ is contractible. Hence the middle term is a stack, thus the unit map is an equivalence as well. \\
%	
%	
%	
%\end{proof}

\subsection{Descent}


\begin{theorem}
	Let $\bT$ be subcanonical.
	%	geometric stacks have descent.
	Consider a class of stacks $\St$ stable under $\sum$ such that $\bT \subset \St$ and whenever you have a $\bT$-cover $X \to Y$ between stacks then $X \in \St$ implies $Y \in \St$.
	Then $\St$ has descent.
\end{theorem}

\begin{proof}
	$\St$ is separated: This follows from the embedding $\GS$ into the separated type of sheaves \ref{lemma:SheavesHaveDescent}. \\
	Let $U \in \bT$ and $P : \|U\| \to \St$. We want to construct a filler 
	% https://q.uiver.app/#q=WzAsMyxbMCwwLCJcXHwgVVxcfCJdLFswLDEsIjEiXSxbMSwwLCJcXFNtU3QiXSxbMCwyLCJQIl0sWzAsMV0sWzEsMiwiIiwyLHsic3R5bGUiOnsiYm9keSI6eyJuYW1lIjoiZGFzaGVkIn19fV1d
	\[\begin{tikzcd}
		{\| U\|} & \GS \\
		1
		\arrow["P", from=1-1, to=1-2]
		\arrow[from=1-1, to=2-1]
		\arrow[dashed, from=2-1, to=1-2]
	\end{tikzcd}\]
	
	Given $U \in \bT$ and a map $P : \|U\| \to \St$. Claim: $L_T \sum_{x: \|U\|} P x \in \St$ . 
	If the claim is proven, the diagram commutes: Assuming $x : \|U\|$ we wish to show $P x = L_T \sum_{x: \|U\|} P x$. Using univalence, we may show that the maps 
	\[P x \to \sum_{x: \|U\|} P x \overset{\eta}{\to} L_\bT \sum_{x: \|U\|} P x\]
	are both equivalences.
	The first one is an equivalence as $\|U\|$ is contractible. Hence the middle term is a stack, thus the unit map is an equivalence as well. \\
	Proof of the claim:
	
	%I have shown above, that in this case the triangle in the sheaf-condition commutes. \\

	We introduce notation 
	\[ \sum_{x : U} P x \overset{f}{\to} \sum_{x: \|U\|} P x \equiv: Y \overset{\eta}{\hookrightarrow} L_T Y  .\] \\
	Claim: For any $y : L_T Y$, the map $\fib_{\eta f}{y} \hookrightarrow \sum_{x: U} P x \to U$ is an equivalence.  \\
	Proof: To ask that a map between stack is an equivalence is a stack, hence we may replace $y$ by $\eta y'$ with $y' : Y$.
	Consider the following commutative diagram
	% https://q.uiver.app/#q=WzAsNCxbMCwwLCJcXGZpYl97XFxldGEgZn0gXFxldGEgeSciXSxbMSwwLCJcXHN1bV97eCA6IFVfaX0gUCB4Il0sWzEsMSwiVV9pIl0sWzAsMSwiXFxmaWJfZiB5JyJdLFswLDFdLFszLDBdLFszLDIsIlxcc2ltZXEiLDJdLFsxLDJdXQ==
	\[\begin{tikzcd}
		{\fib_{\eta f} \eta y'} & {\sum_{x : U} P x} \\
		{\fib_{f} y'} & {U}
		\arrow[from=1-1, to=1-2]
		\arrow[from=1-2, to=2-2]
		\arrow[from=2-1, to=1-1]
		\arrow["\simeq"', from=2-1, to=2-2]
	\end{tikzcd}\]
	The left vertical map is an equivalence, as $\sum_{x:\|U\|} P x$ is seperated (the geometric stacks $P x$ are stacks, so in particular seperated). \\
	%We have that the map $U_i \overset{\simeq}{\leftarrow} \fib_{f_i} y' \to \fib_{\eta f_i}{\eta y'}$ is an equivalence. \qed(Claim) \\
	As $U \in \St$ and $\St$ is $\sum$-stable,  $\sum_{x: U} P x \in \St$ By the assumption of the theorem $L_T Y \in \St$ %\ref{lemma:GSPlusStability} 
	
\end{proof}
In the proof we have learned the following:
\begin{lemma}{\label{lemma:stackificationHasTCover}}
	If $Y$ is seperated and admits some $U \in \bT$ and a map $f : X \to Y$ such that every fiber is equivalent to $U$, then there is a $\bT$-cover $X \to L_\bT Y$.
\end{lemma}
\begin{corollary}
	If $\bT$ has descent, (covering) geometric stacks satisfy descent.
\end{corollary}

\begin{corollary}
	If $\bT$ has descent. For all $n : \bN $, the class of (covering) ($n$-)stacks has descent.
\end{corollary}
\begin{proof}
	The class of (covering) geometric $n$-stacks is the intersection of (covering) geometric stacks and $n$-truncated stacks. Both have descent.
	%If $n-stack \ni X \to Y$ is fibered in $\bT$ where $Y$ is an $n$-type. Then its stackification is an $n$-type as well by the lemma.
\end{proof}




\section{Saturated Topologies}
%Consider a topology $\bT$ finer than the Zariski topology.
\begin{definition}
	Consider the partial order
	\[
	\Top = \{\bT : \Prop^\Aff \ | \ 1 \in \bT \land \bT \sum-stable \}
	\]
	ordered by inclusion.
	An inflation $P$ on $\Top$ is a monotone endofunction such that $X \subset P X$. 
	$P$ is stack-preserving if for any topology $\bT$, $P \bT \subset \bT$-merely inhabited types.\\
	it is covering-stack-preserving if for any $X : P \bT$, $X$ is a $\bT$-covering stack.
%   that preserves $\sum$-stability and satisfies that 
\end{definition}
Note that covering-stack-preserving implies stack-preserving, as $\bT$-covering stacks are $\bT$-merely inhabited.
\begin{prop}{\label{prop:TopologyMonad}}
	Given a stack-preserving inflation $P$. Then for any topology $\bT$, A Type $Y$ is a stack wrt to $P \bT$ iff it is a stack wrt to $\bT$. \\
	If $P$ is idempotent, then the class $P \bT$ is the smallest $P$-fixpoint topology containing $\bT$. \\
	If $P$ is covering-stack preserving, $\bT$ and $P \bT$ will induce the same covering stacks.
\end{prop}

\begin{proof}
	$\bT \subset P \bT$ by inflationarity. 	Regarding Stacks: As $\bT \subset \bT'$ the $\rightarrow$ direction is clear. Now, let $X \in \bT'$. We have
	% https://q.uiver.app/#q=WzAsMyxbMCwwLCJcXHxYXFx8Il0sWzEsMCwiVCJdLFswLDEsIlxcfFhcXHxfXFxiVCBcXHNpbWVxIDEiXSxbMCwxLCJcXGZvcmFsbCJdLFswLDJdLFsyLDEsIlxcZXhpc3RzISIsMl1d
	\[\begin{tikzcd}
		{\|X\|} & Y \\
		{\|X\|_\bT}
		\arrow["\forall", from=1-1, to=1-2]
		\arrow[from=1-1, to=2-1]
		\arrow["{\exists!}"', dashed, from=2-1, to=1-2]
	\end{tikzcd}\]
	by the stack-preserving-property $\|X\|_\bT  \simeq 1$. Hence $T$ is $\|X\|$-local	
	If $P$ is idempotent, every other fixpoint $X$ containg $\bT$ satisfies $P T \subset P X = X$ by monotonicity. \\
	If $P$ is covering-stack-preserving, notice that every $\bT$-covering stack is also a $P \bT$-covering stack as $\bT \subset P \bT$. For the converse we use the recursion principle: For $X$ a $P \bT$-covering stack, consider the predicate 'is $P \bT$-covering'. 1 has it. If $P \bT \ni \Spec A \to X$ is a $\bT$-geometric atlas, i.e. whose fibers are $\bT$-covering stacks, as $\Spec A$ is a $\bT$-covering stack by the covering-stack-preservation, by quotient stability of $\bT$-covering stacks $X$ is a $\bT$-covering stack as well
\end{proof}

\begin{definition}
	A catlas of $X$ is  some $\hat X \in \bT , \hat X \to X \text{ $\bT$-cover }$
\end{definition}
\begin{prop}
	The assignment
		\begin{align*}
		\Top &\to \Top \\
		\bT &\mapsto \bT' \equiv \{X \in \Aff \ | \  \exists \text{ catlas of } X \}
		\end{align*}
		%i.e. the affine covering $0$-stacks.	
	covering-stack-preserving idempotent Monad, called the saturation monad. \\
	 $\bT'$ is the class of covering $\Aff$-stacks.
\end{prop}
\begin{proof}
	\begin{itemize}
		\item 	$\bT'$ is $\sum$-stable by \ref{thm:atlasStableSum}. \\
		\item $\bT \subset \bT'$ is clear.
		\item Monotonicity clear
		\item Idempotentency:  consider some $\bT'$-cover $\bT' \ni X' \to X$. By replacing $X'$ with some smooth atlas, we may assume that $X' \in \bT$. As every fiber $X'_x \in \bT'$, we merely find a smooth atlas $\tilde X'_x \to X'_x$. Then by Zariski local choice there exists a Zariski atlas $\hat X \to X$ and a commutative diagram 
	% https://q.uiver.app/#q=WzAsNCxbMCwwLCJcXHN1bV97eCA6IFxcaGF0IFh9XFx0aWxkZSBYJ194Il0sWzAsMSwiXFxoYXQgWCJdLFsxLDEsIlgiXSxbMSwwLCJcXHN1bV97eCA6IFh9WCdfeCJdLFszLDJdLFsxLDIsIlphciIsMl0sWzAsMV0sWzAsM11d
	\[\begin{tikzcd}
		Y \equiv {\sum_{x : \hat X}\tilde X'_x} & {\sum_{x : X}X'_x} = X' \\
		{\hat X} & X
		\arrow[from=1-1, to=1-2]
		\arrow[from=1-1, to=2-1]
		\arrow[from=1-2, to=2-2]
		\arrow["Zar"', from=2-1, to=2-2]
	\end{tikzcd}\]
	As $X' \in \bT$ and $Y \to X'$ is fibered in $\bT$ (\ref{lemma:AtlasSum}) we have $Y \in \bT$. But $Y \to \hat X$ is a $\bT$-cover and $\hat X \to X$ is a $\bT$-cover, $Y \to X$ is a $\bT$-cover. Hence $X \in \bT'$.

		\item covering-stack-preserving: For any $\Spec A : \bT'$ we merely have some $\bT$-catlas $\bT \ni X \to \Spec A$, witnessing that $\Spec A$ is a covering stack.
		\end{itemize}
		For the last claim, just observe that $\bT'$ is definitely contained in covering $\Aff$-stacks.
	%But $Y \to X$ is a $\bT$-cover and $\hat X \to X$ is a $\bT$-cover, $Y \to X$ is a $\bT$-cover. Hence $X \in \bT'$.
	%covering $n$-stacks are stable under dependent sums \ref{thM:stabSums}
	%Obviously $1 \in \bT'$. We say a type $X$ has covering local choice if for all $\bT$-surjections $S \to S'$ and a map $X \to S'$ there exists a catlas of $X$ lifting to $S$. 
	% %First observe that every $X \in \bT$ has covering local choice:
	% Now any $X \in \bT'$ satisfies local choice wrt catlasses because it has a catlas and catlasses are stable under composition and Zariski covers are in $\bT$. 
	% Hence ...
%		As $\bT'$ is definitely contained in the saturation, it suffices to show, that the class $\bT'$ defined above is saturated.
\end{proof}
\begin{lemma}
 if $\Spec B \to \Spec A$ is faithfully flat and $\Spec B$ is flat, then $\Spec A$ is flat.
\end{lemma}
\begin{proof}
	Consider an injection of $R$-modules $M \hookrightarrow N$. We wish to show, that $A \otimes_R M \to A \otimes_R N$ is injective. As $B$ is faithfully flat over $A$ it suffices to show, that $B \otimes_R M \cong B \otimes_A A \otimes_R N \to B \otimes_A A \otimes_R N = B \otimes_R N$ is injective. This follows as $B$ is flat over $R$.
\end{proof}
\begin{example}
	The fppf-Topology is saturated.
\end{example}
\begin{proof}
	Given a faithfully flat algebra homomorphism $A \to B$ with $B$ faithfully flat, we want to show, that $A$ is faithfully flat. First observe, that $A$ is flat by the previous lemma. Then if $M \otimes_R A = 0$ for some $R$-module $M$, then $M \otimes_R B = M \otimes_R A \otimes_A B = 0$. As $B$ is faithfully flat over $R$, we conclude $M = 0$.
\end{proof}

\begin{example}
	The unramified-topology (unramified + fppf) is saturated.
\end{example}
\begin{proof}
	Let $\Spec B \to \Spec A$ be unramified + fppf and $\Spec B$ unramified + fppf. We have to show that $\Spec A$ is unramified (fppf is the above example). For this, we may show that identity types $x = y$ are $\lnot \lnot$-stable. So assume $\lnot \lnot (x = y).$\\
	As $\Spec A$ admits a faithfully flat map with flat affine domain, the identity type $x = y$ admits such a map $\Spec B' \to x = y$ as well. As its fibers are $\lnot \lnot$-inhabited, we can conclude that the flat $\Spec B'$ is $\lnot \lnot $-inhabited, hence fppf. But now $x = y$ is a fppf-covering -1-stack, hence contractible \ref{lemma:covM1Stacks}.
\end{proof}
%
%\begin{lemma}
%
%\end{lemma}
%\begin{proof}
%
%%We have to show that $T \to T^{\|X\|}$ is an equivalence. Choose $\bT \ni Y \to X$. Then we have a commutative diagram
%%	% https://q.uiver.app/#q=WzAsMyxbMCwwLCJUIl0sWzEsMCwiVF57XFx8WFxcfH0iXSxbMSwxLCJUXntcXHxZXFx8fSJdLFswLDFdLFsxLDJdLFswLDIsIlxcc2ltZXEiLDJdXQ==
%%	\[\begin{tikzcd}
%%		T & {T^{\|X\|}} \\
%%		& {T^{\|Y\|}}
%%		\arrow[from=1-1, to=1-2]
%%		\arrow["\simeq"', from=1-1, to=2-2]
%%		\arrow[from=1-2, to=2-2]
%%	\end{tikzcd}\]
%%	So $T \to T^{\|X\|}$ has a left-inverse. Thus it suffices to show that any $f : T^{\|X\|}$ has a preimage. Choose $t : T$, s.th. $\mathrm{cnst}^Y_t$ is the composite $\|Y\| \to \|X\| \overset{f}{\to} T$. We have $\|Y\| \to (\mathrm{cnst}^X_t = f)$. But as $Y \in \bT$ and $\Delta_t = f$ is a stack (as an identitytype in the stack $T^{\|X\|}$) we are done.
%\end{proof}
%\begin{rmk}
%	We never used that we only talk about $\bT$-covers.
%\end{rmk}
%
%\begin{question}
%	Does the converse hold, i.e. is every $\bT$-merely inhabited affine saturated?    
%\end{question}

% \begin{lemma}
%     A type $T$ is a saturated $\bT$-stack if for all $X \in \bT$ the diagonal
%     \[
%     T \to T^{L_\bT \|X \|}
%     \]
%     is an equivalence.
% \end{lemma}
% \begin{proof}
%     If $X \in \bT'$, we can choose a catlas $\hat X \to X$. As it is in particular $\bT$-surjective we have $\|X\| \overset{\sim}{\to} L_T \|\hat X\| $ which gives us that the composite
%     \[
%     T \overset{\simeq}{\to} T^{L_\bT \|X'\|}  \overset{\simeq}{\to}  T^{\|X\|}
%     \]
%     is an equivalence. 
% \end{proof}

\section{Local properties}

\begin{definition}
	Let $\cV \subset \cU$ a subclass of types be stable under finite limits. 
	We call a property $P$ of morphisms of types in $\cV$ $\bT$-local, if 
	\begin{enumerate}
		\item its satisfied by identities in $\cV$,
		\item stable under composition 
		\item
		Given a commutative triangle in $\cV$% https://q.uiver.app/#q=WzAsMyxbMCwwLCJYIl0sWzEsMCwiWSJdLFsxLDEsIloiXSxbMCwxLCJmIl0sWzEsMiwiZyJdLFswLDIsImgiLDJdXQ==
		\[\begin{tikzcd}
			X & Y \\
			& Z
			\arrow["f", from=1-1, to=1-2]
			\arrow["h"', from=1-1, to=2-2]
			\arrow["g", from=1-2, to=2-2]
		\end{tikzcd}\]
		with $X \to Y$ a geometric cover (wrt to $\bT$). Then $h$ has $P$ iff $g$ has $P$
		
	
	\end{enumerate}

\end{definition}
\begin{definition}
	$P$ has descent along geometric covers: Given $X,Y,Z,W \in \cV$. if $Y \to W$ is a geometric cover , then 
	% https://q.uiver.app/#q=WzAsNCxbMCwwLCJYIl0sWzAsMSwiWSJdLFsxLDEsIlciXSxbMSwwLCJaIl0sWzAsM10sWzAsMSwiZiciLDJdLFsxLDJdLFszLDIsImYiXSxbMCwyLCIiLDEseyJzdHlsZSI6eyJuYW1lIjoiY29ybmVyLWludmVyc2UifX1dXQ==
	\[\begin{tikzcd}
		X & Z \\
		Y & W
		\arrow[from=1-1, to=1-2]
		\arrow["{f'}"', from=1-1, to=2-1]
		\arrow["\ulcorner"{anchor=center, pos=0.125}, draw=none, from=1-1, to=2-2]
		\arrow["f", from=1-2, to=2-2]
		\arrow[from=2-1, to=2-2]
	\end{tikzcd}\]
	If $f$ has $P$ then $f'$ has $P$.
\end{definition}
	\begin{lemma}{\label{lemma:local}}
	If $P$ is local, then 
	\begin{itemize}
		\item geometric covers have P
		\item in descent, The statement 'If $f'$ has $P$ then $f$ has $P$' is automatic by Point 3.
	\end{itemize}
\end{lemma}
\begin{lemma}
	Beeing a geometric cover is local.
\end{lemma}
\begin{proof}
	Reduce to the case of $Z = 1$. If $X \to Y$ is a geometric cover, then $X$ is a covering stack iff $Y$ is a covering stack by stability under quotients and under sums. If both are coverings stacks, then the fibers 
\end{proof}

\begin{lemma}{\label{lemma:atlasOfMap}}
	Let $P$ be a local property of morphisms of geometric stacks. For
	A morphism between geometric stacks $f : X \to Y$  TFAE 
	\begin{enumerate}
		\item $f$ has $P$
		
		\item For any Atlas $\Spec A \to Y$ and any atlas $S \to X \times_Y \Spec A$ the composite $S \to \Spec A$ has $P$
		\item $f$ has an atlas that has $P$.
	\end{enumerate}
\end{lemma}
\begin{proof}
	\begin{itemize}
		\item [1 $\Rightarrow$ 2]
		Given a geometric atlas $\Spec A \to Y$ and taking the pullback % https://q.uiver.app/#q=WzAsNCxbMCwwLCJYIFxcdGltZXNfWSBcXFNwZWMgQSJdLFswLDEsIlgiXSxbMSwwLCJcXFNwZWMgQSJdLFsxLDEsIlkiXSxbMSwzLCJmIl0sWzAsMiwiZiciXSxbMiwzXSxbMCwxXV0=
		\[\begin{tikzcd}
			{X \times_Y \Spec A} & {\Spec A} \\
			X & Y
			\arrow["{f'}", from=1-1, to=1-2]
			\arrow[from=1-1, to=2-1]
			\arrow[from=1-2, to=2-2]
			\arrow["f", from=2-1, to=2-2]
		\end{tikzcd}\]
		$f'$ has $P$ as a basechange of $f$ along a geometric cover. Given a geometric atlas $S \to X \times_Y \Spec A$, it will have $P$, the composition $S \to \Spec A$ will be in $P$.
		\item [2 $\Rightarrow$ 3]
		$Y$ is a geometric stack, hence admits some geom atlas $\Spec A \to Y$. Again, $X \times_Y \Spec A$ is a geometric stack hence admits a geometric atlass.
		\item [3 $\Rightarrow$ 1]
		If we have an atlas $\tilde f : \tilde X \to \tilde Y$, then $\tilde X \to \tilde Y \to Y$ has $P$ by stability under composition. Then by (4) $X \to Y$ has $P$, as $\tilde X \to X$ is a geometric cover \\
		
	\end{itemize}
\end{proof}

So we may extend algebraic notions of maps to all geometric stacks:
\begin{definition}
	Let $P$ be a property of morphisms $\bT$-local in affine schemes.
	
	We define a morphism of geometric stacks $f : X \to Y$ to have $P$ iff
	there exist atlasses and a $P$-map on affines 
	% https://q.uiver.app/#q=WzAsNCxbMCwwLCJcXFNwZWMgQSJdLFswLDEsIlgiXSxbMSwxLCJZIl0sWzEsMCwiXFxTcGVjIEIiXSxbMywyXSxbMCwxXSxbMSwyLCJmIiwxXSxbMCwzLCJcXGhhdCBmIiwwLHsic3R5bGUiOnsiYm9keSI6eyJuYW1lIjoiZGFzaGVkIn19fV1d
	\[\begin{tikzcd}
		{\Spec A} & {\Spec B} \\
		X & Y
		\arrow["{\hat f}", dashed, from=1-1, to=1-2]
		\arrow[from=1-1, to=2-1]
		\arrow[from=1-2, to=2-2]
		\arrow["f"{description}, from=2-1, to=2-2]
	\end{tikzcd}\]	
\end{definition}
\begin{lemma}
	Let $P$  be a local property of affine schemes. The induced property of morphisms of geometric stacks is local. Additionally descent is inherited.
\end{lemma}
\begin{proof}
	\begin{enumerate}
		\item Ok
		\item Ok
	
		\item geometric covers have $P$ and we have proven point 2., so one direction is clear.  Now assume 
		%https://q.uiver.app/#q=WzAsMyxbMCwwLCJYIl0sWzEsMCwiWSJdLFsxLDEsIloiXSxbMCwxLCJmIl0sWzEsMiwiZyJdLFswLDIsImgiLDJdXQ==
		\[\begin{tikzcd}
			X & Y \\
			& Z
			\arrow["f", from=1-1, to=1-2]
			\arrow["h"', from=1-1, to=2-2]
			\arrow["g", from=1-2, to=2-2]
		\end{tikzcd}\]
		Such that $f$ is a geometric cover and $h$ has $P$. \\
		We first reduce to the case where $Z$ is affine. Choose a geometric atlas $\tilde Z \to Z$. Then take the pullbacks
		% https://q.uiver.app/#q=WzAsNixbMiwwLCJcXHRpbGRlIFoiXSxbMiwxLCJaIl0sWzEsMCwiWSciXSxbMSwxLCJZIl0sWzAsMCwiWCciXSxbMCwxLCJYIl0sWzIsM10sWzMsMV0sWzAsMV0sWzIsMF0sWzIsMSwiIiwxLHsic3R5bGUiOnsibmFtZSI6ImNvcm5lci1pbnZlcnNlIn19XSxbNSwzXSxbNCw1XSxbNSwxLCJQIiwxLHsiY3VydmUiOjJ9XSxbNCwyXSxbNCwzLCIiLDEseyJzdHlsZSI6eyJuYW1lIjoiY29ybmVyLWludmVyc2UifX1dXQ==
		\[\begin{tikzcd}
			{X'} & {Y'} & {\tilde Z} \\
			X & Y & Z
			\arrow[from=1-1, to=1-2]
			\arrow[from=1-1, to=2-1]
			\arrow["\ulcorner"{anchor=center, pos=0.125}, draw=none, from=1-1, to=2-2]
			\arrow[from=1-2, to=1-3]
			\arrow[from=1-2, to=2-2]
			\arrow["\ulcorner"{anchor=center, pos=0.125}, draw=none, from=1-2, to=2-3]
			\arrow[from=1-3, to=2-3]
			\arrow[from=2-1, to=2-2]
			\arrow["P"{description}, curve={height=12pt}, from=2-1, to=2-3]
			\arrow[from=2-2, to=2-3]
		\end{tikzcd}\]
		$X' \to Y'$ is a geometric cover and By 3. $X' \to \tilde Z$ has $P$. \\
		So we may assume that $Z$ is affine. Then take a geometric atlas $\tilde X \to X$. The map $Y \to Z$ has the atlas $\tilde X \to X \to Z$ which has $P$ by stability under composition. Hence $Y \to Z$ has $P$.
			\item We show also descent: By \ref{lemma:local} we only need to show stability under basechange. Let $Z \to W$ have $P$, Given $Y \to W$ a geometric cover. We want to show that a basechagne $Y \times_W Z \to Y$ has $P$. The idea is to construct a common atlas of $Z \to W$ and its basechange. Choose an atlas $\tilde Y \to Y$. Then $\tilde Y \times_W Z \to Y \times_W Z$ is a geometric cover: It is a basechange of $\tilde Y \to Y$, because the outer diagram is a pullback
		% https://q.uiver.app/#q=WzAsNixbMCwwLCJcXHRpbGRlIFkgXFx0aW1lc19XIFoiXSxbMSwwLCJZIFxcdGltZXNfVyBaIl0sWzAsMSwiXFx0aWxkZSBZIl0sWzEsMSwiWSJdLFsyLDAsIloiXSxbMiwxLCJXIl0sWzAsMl0sWzAsMV0sWzIsM10sWzEsM10sWzMsNV0sWzEsNF0sWzQsNV0sWzEsNSwiIiwxLHsic3R5bGUiOnsibmFtZSI6ImNvcm5lci1pbnZlcnNlIn19XSxbMCwzLCIiLDEseyJzdHlsZSI6eyJuYW1lIjoiY29ybmVyLWludmVyc2UifX1dXQ==
		\[\begin{tikzcd}
			{\tilde Y \times_W Z} & {Y \times_W Z} & Z \\
			{\tilde Y} & Y & W
			\arrow[from=1-1, to=1-2]
			\arrow[from=1-1, to=2-1]
			\arrow["\ulcorner"{anchor=center, pos=0.125}, draw=none, from=1-1, to=2-2]
			\arrow[from=1-2, to=1-3]
			\arrow[from=1-2, to=2-2]
			\arrow["\ulcorner"{anchor=center, pos=0.125}, draw=none, from=1-2, to=2-3]
			\arrow[from=1-3, to=2-3]
			\arrow[from=2-1, to=2-2]
			\arrow[from=2-2, to=2-3]
		\end{tikzcd}\]
		Now choose any geometric atlas $S \to \tilde Y \times_W Z$. By composition this induce a map $S \to \tilde Y$: 
		It is both an atlas of the $P$-map $Z \to W$ and of $ Y\times_W Z \to Y$. So by \ref{lemma:atlasOfMap} $S \to \tilde Y$ has $P$ and thus $Y \times_W Z \to Y$ has $P$. 
		
	\end{enumerate}
\end{proof}
\subsection{Local properties of stacks}
\begin{definition}
	Let $\cV \subset \cU$ be a subclass of types stable under finite limits. A property $P$ of types in $\cV$ is local if 
	\begin{enumerate}
		\item $1 \in P$
		\item $P$ is $\sum$-stable
		\item If $X \to Y$ is a geometric cover between types in $\cV$, then $X$ has $P$ iff $Y$ has $P$.
		
	\end{enumerate}
	We say $P$ has descent if for all $X : \cV$, then $X$ having $P$ is a $\bT$-sheaf.
\end{definition}
\begin{lemma}
	Every local property of types in $\cV$ induces a local property of morphisms of types in $\cV$, by asking the property fiberwise.
\end{lemma}
\begin{proof}
	Use descent for the descent along a geometric cover ($\bT$-surjective!).
\end{proof}
\begin{lemma}
	Let $P$ be a $\sum$-stable-property of affines containg $\bT$. The induced property of geometric stacks is $\bT$-local.
\end{lemma}
\begin{proof}
	The $\sum$-stability is \ref{thm:atlasStableSum}. Covering stacks have $P$, as $\bT \subset P$. The quotient stability is straightforward.
\end{proof}
\subsection{Seperatedness}
\begin{definition}
Let $P$ be a $\bT$-local property of stacks.
We call a stack $P$-seperated, iff its identity types are $P$ stacks.
\end{definition}


\begin{lemma}{\label{lemma:SeperationIsLocal}}
	Let $P$ be a $\bT$-local property of stacks. Then beeing $P$-seperated is a $\bT$-local property %Then $\mathrm{Sep}(P)$ is local, defined as $X \in \mathrm{Sep}(P)$ iff $x = y$ has $P$ for all $x ,y : X$.
\end{lemma}
\begin{proof}
	Let $f : X \to Y$ be a geometric cover with $X$ beeing $P$-seperated. Let $x , y : Y$. Then by the construction in \ref{lemma:havingAbstractAtlasClosedUnderId} the map
	\[
	\fib_f x \times_X \fib_f y \to x = y
	\]
	is a geometric cover, whose domain has $P$ as $X$ is $P$-seperated and $P$ is stable under $\sum$. As $P$ is local, $x = y$ has $P$. \\
%	TODO Beeing $\bP$-seperated has descent.
\end{proof}
%
%\begin{example}
%	If $\cP$ is formally \etale, then beeing $\cP$-seperated means formally unramified.
%\end{example}
%\begin{lemma}{\label{lemma:SetUnramfied}}
%	Given morphisms of types $X \to Y \to Z$ with $X \to Z$ $\cP$-seperated and $Y$ a set, then $X \to Y$ is $\cP$-seperated.
%\end{lemma}
%\begin{proof}
%	We can argue fiberwise so we may assume $Z$ beeing the point. A fiber of $f : X \to Y$ over $y$ is $\sum_{x: X} f x = y$ where $f x = y$ is a proposition,hence $\cP$-seperated. As $X$ is $\cP$-seperated we conclude as $\cP$-seperated types are $\sum$-stable.
%\end{proof}

\begin{lemma}{\label{lemma:PImpliesPsep}}
	If $\sum$-stable property of affine schemes containing $\bT$ is stable under identity types, then the induced $\bT$-local property of geometric stacks is as well. %In this case the identity types of a $P$ stack have $P$ as well. \\	
\end{lemma}
\begin{proof}
	Let $X$ be a $P$ geometric stack. Let $x,y : X$ we want to show that $x =_X y$ has $P$. Choose a geometric atlas $P \ni S \overset{f}{\to} X$ . By assumption $S$ is $P$-seperated. We have a geometric atlas $\fib_f x \times_S \fib_f y \to x = y$. The domain is a $\sum$ of types in $\bT$ and identity types of $S$, which have $P$ by stability under identity types for the affine $S$. Hence $x = y$ has $P$. 
\end{proof}
\begin{lemma}{\label{lemma:FormEtaleSelfLocal}}
	If $Y \to X$ is a formally \etale $\bT$-surjective map between stacks and $Y$ is formally \etale, then $X$ is formally \etale.
\end{lemma}
\begin{proof}
	Take $L$ to be the modality which nullafies the propositions $\|\Spec A\|$ for $\Spec A$ \etale + fppf and all close dense propositions.
	The square 
	% https://q.uiver.app/#q=WzAsNCxbMCwwLCJcXFNwZWMgQiJdLFsxLDAsIlxcU3BlYyBBIl0sWzAsMSwiRXQoXFxTcGVjIEIpIl0sWzEsMSwiRXQoXFxTcGVjIEEpIl0sWzEsM10sWzAsMiwiXFxzaW0iXSxbMiwzLCIiLDEseyJzdHlsZSI6eyJoZWFkIjp7Im5hbWUiOiJlcGkifX19XSxbMCwxXSxbMSwzXSxbMCwzLCIiLDEseyJzdHlsZSI6eyJuYW1lIjoiY29ybmVyLWludmVyc2UifX1dXQ==
	\[\begin{tikzcd}
		{Y} & {X} \\
		{L(Y)} & {L(X)}
		\arrow[from=1-1, to=1-2]
		\arrow["\sim", from=1-1, to=2-1]
		\arrow["\ulcorner"{anchor=center, pos=0.125}, draw=none, from=1-1, to=2-2]
		\arrow[from=1-2, to=2-2]
		\arrow[from=1-2, to=2-2]
		\arrow[from=2-1, to=2-2]
	\end{tikzcd}\]
	is a pullback as $L$ is lex. 
	We want to show, the right map is an equivalence. 
	Every type occuring is an \etale-stack.
	As the lower map is \etale-surjective, and the left vertical map is an equivalence, we can conclude.
\end{proof}
\begin{lemma}{\label{lemma:CSFEt}}
	Covering stacks are formally \etale.	
	%	If $\bT=$ \etale, then 
\end{lemma}
\begin{proof}
	We apply the recursion principle. Contractible types are formally \etale.
	If $\Spec B \to X$ is a formally \etale geometric cover  and $\Spec B \in \bT$  then $X$ is formally \etale by the previous lemma. % 
	
\end{proof}
\begin{lemma}{\label{lemma:FEtLocal}}
	formally \etale is an \etale-local property of geometric stacks.	
\end{lemma}
\begin{proof}
	geometric covers are formally \etale  by \ref{lemma:CSFEt}, so conclude by \ref{lemma:FormEtaleSelfLocal}
\end{proof}
\section{Flat}


\begin{definition}
	Denote $\Top$ the topologies containing $\mathsf{Bool}$, e.g. finer than the Zariski-topology.
	Let $\mathsf{FLAT}$ consists of all the classes of affines $\bP$ containing $1, \bot$ stable under $\sum$. \\
	Given $\bP: \mathsf{FLAT} , \bT: \Top$ we say $\bP$ flattens $\bT$ iff ($\bT \subset \bP$ and)
	\[
		\bT= \{X : \bP\ | \ \|X\|_\bT \}
	\]
\end{definition}
The goal of this section is to prove the following theorem
\begin{theorem}{\label{thm:Flat}}
 \
	\begin{enumerate}
		\item There is at most one $\bP$ that flattens a topology. Then we say, the topology is flatten.
		\item A topology can be idempotently flattened without changing the stacks
		\item For any $\bP: \mathsf{FLAT}$ and any Lavwere Tierney Operator $j$, $\{ X: \bP \ | \ \|X\|_j \}$ is flattened by $\bP$.
	\end{enumerate}
\end{theorem}
We first want to show the power of this theorem.
\begin{example}
	finite sums of principal opens flatten the Zariski topology. 
\end{example}
\begin{example}
	flat affines flatten the fppf topology. %Beeing flat if fppf-local.
\end{example}
\begin{proof}
	Indeed we can either put $j = \lnot \lnot$ or $j$ the fppf sheafification, because TFAE
	\begin{enumerate}
		\item  $X$ is flat and fppf-merely inhabited
		\item $X$ is flat  and $\lnot\lnot$-inhabited
		\item $X$ is fppf
	\end{enumerate}
	
	\begin{proof}
		$1 \Rightarrow 2 \Rightarrow 3 \Rightarrow 1$ \todocite
	\end{proof}	 
	%The last point: if $\Spec B \to \Spec A$ is faithfully flat and $\Spec B$ is flat, then $\Spec A$ is flat.
\end{proof}
\begin{example}
	 formally \etale + flat affines flatten the \etale topology.
	For the etale topology $=$ formally \etale $+$ fppf, we can put $\bP$ = formally \etale + flats. %$\bP$ is local.
\end{example}
%\begin{proof}
%	By the same argument as above. \\
	%The locality: If $\Spec B \to \Spec A$ is a ${\bP_{et}}$-cover (i.e. an \etale faithfully flat map) and $\Spec B$ \etale flat then we want to show $\Spec A$ is \etale. 
	%	The square 
	%	% https://q.uiver.app/#q=WzAsNCxbMCwwLCJcXFNwZWMgQiJdLFsxLDAsIlxcU3BlYyBBIl0sWzAsMSwiRXQoXFxTcGVjIEIpIl0sWzEsMSwiRXQoXFxTcGVjIEEpIl0sWzEsM10sWzAsMiwiXFxzaW0iXSxbMiwzLCIiLDEseyJzdHlsZSI6eyJoZWFkIjp7Im5hbWUiOiJlcGkifX19XSxbMCwxXSxbMSwzXSxbMCwzLCIiLDEseyJzdHlsZSI6eyJuYW1lIjoiY29ybmVyLWludmVyc2UifX1dXQ==
	%	\[\begin{tikzcd}
	%		{\Spec B} & {\Spec A} \\
	%		{Et(\Spec B)} & {Et(\Spec A)}
	%		\arrow[from=1-1, to=1-2]
	%		\arrow["\sim", from=1-1, to=2-1]
	%		\arrow["\ulcorner"{anchor=center, pos=0.125}, draw=none, from=1-1, to=2-2]
	%		\arrow[from=1-2, to=2-2]
	%		\arrow[from=1-2, to=2-2]
	%		\arrow[two heads, from=2-1, to=2-2]
	%	\end{tikzcd}\]
	%	is a pullback as the formally \etale modality is lex. As the lower map is surjective, the right vertical map is an equivalence.	
%\end{proof}
%\begin{lemma}{\label{lemma:coveringDMstacks}}
%	Any formally \etale, \etale-merely inhabited Deligne-Mumford-stack is covering.
%\end{lemma}
%\begin{proof}
%	Todo
%\end{proof}

\begin{lemma}{\label{lemma:TruncOfP}}
	Assume $\bT$ is flatten.
	If $X$ is $\bT$-flat geometric stack, then $\|X\|_{\bT}$ is a geometric prop.
\end{lemma}
\begin{proof}
	
	If $\Spec A$ is $\bT$-flat, then $\Spec A$ is weakly-flat, i.e $\| \Spec A\|_{\bT}$ is a geometric prop.
\end{proof}
\begin{lemma}{\label{lemma:detectCovering}}
	Assume $\bT$ is flatten.
	A stack is covering iff it it a $\bT$-flat geometric stack and $\bT$-merely inhabited.
\end{lemma}

\begin{lemma}
	Assume $\bT$ is flatten.
	If $X$ is a covering stack and $Y$ a $\bT$-flat geometric stack, then $X + Y$ is covering
\end{lemma}
\begin{proof}
	Let $\bP$ flatten $\bT$.
	Let ${\bP} \ni \tilde X \to X, \tilde Y \to Y$ be geometric atlasses. Then $\tilde X+ \tilde Y$ is $\bP$ and ${\bT}$-merely inhabited, hence in the topology.
\end{proof}

\subsection{Lex flatten Topologies}


\begin{definition}
	A saturated topology $\bT$ is lex-flatten, if its flattened by some lex $\bP$. % there exists a lex subclass $\bP$ of affines such that $\bT \subset \bP$ and $\cT_{\bP}^{\|\cdot\|_\bT} \equiv \{X \in \bP \ | \ \|X\|_\bT\} \subset \CS_\bT$.
\end{definition}
Note that $\bot = (left =_{1+1} right) \in \bP$ is automatic as $\bP$ is lex.
%If $\bT$ is saturated, equivalentely $\cT_{\bP}^{\|\cdot\|_\bT}  = \bT$.

\begin{example}
	The \etale topology is lex-flatten:
	formally \etale $+$ flat affines are stable under identity types , as formally \etale seperated schemes have decidable equality.  %\ref{lemma:PImpliesPsep}
\end{example}

\begin{prop}{\label{prop:LexflattenOmegaStable}}
	Let $\bT$ be lex-flatten. Then covering stacks are $\Omega$-stable. %In particular:
	%	Let $\bP$ be stable under $\id$-types and $j$ any Lavwere Tierney Operator. Then $\cT^j_{\bP}$-covering stacks are $\Omega$-stable
\end{prop}
\begin{proof}
	If $X$ is a covering stack then $\Omega X$ is a $\bT$-flat geometric stack \ref{lemma:PImpliesPsep} and $\bT$-merely inhabited. Conclude by  \ref{lemma:detectCovering}.
\end{proof}
\begin{lemma}
	Assume that $\bT$ is lex-flattened. Then any $\bT$-flat geometric stack is a 0-gerbe.
\end{lemma}
\begin{proof}
	By descent, we may only show that the fiber $\sum_{y: X} \|x = y\|_{-1}^\bT$ of $\eta_0^\bT$ over $|x|$ is a covering stack. Note that $x = y$ has $\bP$ by id-stability of $\bP$ \ref{lemma:PImpliesPsep}. The $\bT$-truncation of a $\bP$-geometric stack is a $\bP$ geometric stack \ref{lemma:TruncOfP}. by $\sum$-stability of $\bP$ the fiber is $\bP$, but its $\bT$-merely inhabited. by \ref{lemma:detectCovering} its covering.
\end{proof}
\begin{corollary}
	If $\bT$ is lex-flattened, then identity types of covering stacks are 0-gerbes.
\end{corollary}
\subsection{Proof of the theorem}
 Observe that if $X + Y$ is affine, then $X$ is affine, as $X \to X + Y$ is an affine map.

Let $\bT$ be a topology containing $2$.% finer than the Zariski-topology.
\begin{definition}
	$\cP_\bT$ is the smallest topology containing $\bT \cup \{\bot\}$
\end{definition}
\begin{lemma}{\label{lemma:SummandStable}}
	Let $\bP$ be $\sum$ stable containing $1 , \bot$. Then its stable under decidable subtypes, i.e. If $X + Y \in \bP$ then $X \in \bP$.
\end{lemma}
\begin{proof}
	Given $X + Y \in \bP$, we can define $(1, \bot) : X + Y \to \bP$ Its $\sum$ will be $X$. \\
\end{proof}

\begin{prop}
	Assume that $\bT$ is saturated. 	
	\[
	\cP_\bT= \{X  \ | \ \exists Y , X + Y \in \bT\}
	\]
\end{prop}
\begin{proof}
	By \ref{lemma:SummandStable} and as $\bT \subset \cP_\bT$, we have $'\supset'$. So it remains to show that the RHS, lets call it $\bP$, is a topology containing $\bT, \bot$.
%\begin{lemma}{\label{lemma:flatBasics}}
%
\begin{enumerate}
	
	\item $\cP_\bT \subset \Aff$.
	\item $\bot \in \bP$
	\item $\bT \subset \bP$
	\item Assume $\bT$ is saturated. Whenever $\bP \ni S \to X \in \Aff$ is a $\bT$-cover, then $X \in \bP$. Indeed : choose $S + Y \in \bT$, Then $\bT \ni S + Y \to X + Y$ is a $\bT$-cover, hence by saturatedness $X + Y \in \bT$. Thus $X \in \bP$.
	\item If $\bT$ is saturated, then $\bP$ is stable under $\sum$. Proof:
%\begin{proof}
	Let $\bP \ni X \overset{B}{\to} \bP$. Lets first handle the special case, where $B x \in \bT$ for any $x : X$. Choose $Y$ such that $X + Y \in \bT$ . Then $\sum_{x: X} B x + \sum_{y:Y} 1$ can be expressed as $\sum_{x : X + Y} (B + \mathrm{cnst}_1) x$, which belongs to $\bT$. \\	 
	Now the general case. By Zariski local choice we find a Zariski cover $p : X' \to X$ with 
	\[
	\prod_{x' : X'} \sum_{Y_{x'}} B (p x) + Y_{x'} \in \bT
	\]
	Then $\sum_{x' : X'} Y_{x'} + \sum_{x' : X'} B (p x) \in \bP$, hence by \ref{lemma:SummandStable} $\sum_{x': X'} B (p x) \in \bP$. As $\sum_{x': X'} B (p x) \to \sum_{x: X} B x \in \Aff$ is a $\bT$-cover, we conclude by (4.)
\end{enumerate}
\end{proof}

\begin{definition}
	$\bT$ is decompostable if for any type $X$
	\[ \left(\|X\|_\bT \land \exists Y , X + Y \in \bT  \right) \to X \in \bT. \]\\ 

\end{definition}

\begin{prop}
	Let $\bT$ be saturated. There exists a smallest decompostable topology $\tilde \bT$ containing $\bT$. Moreover the stacks coincide.	
\end{prop}
\begin{proof}
	Define
\begin{align*}
	\Top &\to \Top \\
	\bT &\mapsto \tilde \bT \equiv \{X \ | \ \|X\|_{\bT} \land \exists Y , X + Y \in \bT \} %\equiv \cT_{\cP_\bT}^{\|\cdot\|_\bT}%
\end{align*}

	We apply \ref{prop:TopologyMonad}.
	\begin{itemize}
		\item 	The class is stable under $\sum$  as $\cP_\bT$ and $\bT$-merely inhabited types are both  $\sum$-stable. \\
		\item Monotonicity clear.
		\item Inflationarity clear
		\item stack-preservation is clear by construction.
	%Let us first show, that the sheaves coincide.Obviously we have $\bT \subset \tilde \bT$, so it remains to show that every $\bT$-sheaf is a $\tilde \bT$-sheaf. Let $X \in \tilde \bT$. Then as in particular $\|X\|_\bT$ , every $\bT$-sheaf will be local wrt $\|X\|$. \\
	\item idempotency: %	Now we will prove that $\tilde \bT$ is decompostable. 
Let $X$ be a type such that $\|X\|_{\tilde \bT}$ and there exists a $Y$  with $X + Y \in \tilde \bT$. By the first assumption, we have $\|X\|_\bT$ as the stacks coincide by \ref{prop:TopologyMonad}. \\
The latter means in particular that we find $Z$ with $X + Y + Z \in \bT$. But this witnesses that $X  \in \tilde \bT$.		
\end{itemize}

\end{proof}



\begin{lemma}
	Let $\bT$ be a topology, such that any $X : \cP_\bT$ is $(\bT-1)$-seperated, i.e. that the identity types of $X$ belong to $\bT-1  \equiv \{X \ | \ X + 1 \in \bT\}.$. Then we have for all $X$
	\[
	(\exists Y : \bT - 1 , X + Y \in \bT) = (X \in (\bT - 1)) \to (\|X\|_\bT \to X \in CS)
	\]
\end{lemma}
\begin{proof}
	For the first equality notice that $X + Y \to X + 1$ is a $\bT$-cover. For the last implication, by descent for covering stacks we may choose a map $1 \to X$. Then $\bT \ni X + 1 \to X$ is a $\bT$-cover by assumption.
\end{proof}
\begin{warning}
	In general, the $\tilde \cdot$ -construction is presumably not covering-stack preserving: In the above lemma we would need 
	\[
	X \in \bP \to (\|X\|_\bT \to X \in CS)
	\]
\end{warning}

\begin{example}
	If any type in $\bP$ has decidable equality, then any type in $\cP$ is $(\bT-1)$seperated.
\end{example}
\begin{prop}{\label{prop:detectDecompostable}}
Let $\bT$ be saturated. TFAE
	\begin{enumerate}
		\item  $\bT$ is decompostable, i.e. for any $X \in \cP_\bT$ we have $\|X\|_\bT \to X \in \bT$.
		\item $\cP_\bT$ flattens $\bT$, i.e. $\bT = \{X : \cP_\bT \ | \ \|X\|_\bT \}$ % \}\cT_{\cP_\bT}^{\|\cdot\|_\bT}$
	\end{enumerate}
	In this case we have  $3 . \cP_\bT = \bT - 1$. If $\cP_\bT \subset (\bT-1)-$seperated and $\bT$ is saturated.  , then the converse holds.
\end{prop}
\begin{proof}
	\ \begin{itemize}
			\item [1 $\Leftrightarrow$ 2]
		We have
		\[
		\{X \in \cP_\bT \ | \ \|X \|_\bT\} = \{ X \ | \ \|X\|_\bT \land \exists Y , X + Y \in \bT \}
		\] % \cT_{\cP_\bT}^{\|\cdot\|_\bT} \equiv 
		which coincides with $\bT$ iff $\bT$ is decompostable.
	\item [1 $\Rightarrow$ 3]
	For the second observe $\bT - 1 \subset \cP_\bT$. Then If $X + Y \in \bT$, then $1 + X + Y \in \bT$ as $\bT$ is stable under $+$. By decompostability $1 + X \in \bT$. Hence $X \in \bT - 1$. %$\Zar \in \bT$.
	\item [3 $\Rightarrow$ 1]
	By the above lemma and saturatedness of the topology.

\end{itemize}	
\end{proof}

%\begin{theorem}{\label{thm:ModSec}}
%	 Denote $\tilde \Top$ for decompostable topologies. Consider the map 
%	\begin{align*}
%		\cP_\bullet : \tilde \Top &\to \mathsf{FLAT} \\ \bT &\mapsto \cP_\bT. 
%		\end{align*}
%	Then any Lavwere Tierney Operator $j$ induces a section of $\cP_\bullet$ mapping onto the class of topologies $\bT$ such that forall $X$
%	\[
%\ \|X\|_j \land (\exists Y, X + Y \in \bT) \to X \in \bT  %	\{\bT  : \Top \ | \forall X,  
%	\]
%	For any decompostable topology $\bT$ we find such a section whose image contains $\bT$.
%%	Moreover, any such topology will be decompostable.
%%	Then we have a bijection
%%	% https://q.uiver.app/#q=WzAsNixbMCwwLCJcXG1hdGhzZntGTEFUfSJdLFsxLDAsIlxcVG9wIl0sWzEsMSwiXFxiVCJdLFswLDEsIlxcY1BfXFxiVCJdLFswLDIsIlxcYlAiXSxbMSwyLCJcXGNUX3tcXGJQfV57XFxsbm90XFxsbm90fSJdLFswLDEsIlxcc2ltZXEiLDAseyJzdHlsZSI6eyJ0YWlsIjp7Im5hbWUiOiJhcnJvd2hlYWQifX19XSxbMiwzLCIiLDAseyJzdHlsZSI6eyJ0YWlsIjp7Im5hbWUiOiJtYXBzIHRvIn19fV0sWzQsNSwiIiwwLHsic3R5bGUiOnsidGFpbCI6eyJuYW1lIjoibWFwcyB0byJ9fX1dXQ==
%%	\[\begin{tikzcd}
%%		{\mathsf{FLAT}} & \{\bT  : \Top \ | \forall X,  \ \|X\|_j \land (\exists Y, X + Y \in \bT) \to X \in \bT \} \\ % _{decompostable} \\
%%		{\cP_\bT} & \bT \\
%%		\bP & {\cT_{\bP}^{j}}
%%		\arrow["\simeq", tail reversed, from=1-1, to=1-2]
%%		\arrow[maps to, from=2-2, to=2-1]
%%		\arrow[maps to, from=3-1, to=3-2]
%%	\end{tikzcd}\]
%\end{theorem}
%\begin{proof}
%	The map $\cP_\bullet$ is well-defined by \ref{lemma:flatBasics}. \\
%	
%	Given any $\sum$-stable class of affines $\bP$ containing 1 and a  Lawvere-Tierney operator $j$ , i.e. a reflective subuniverse of $\Prop$, we will construct a decompostable topology $\cT^j_{\bP}$ such that \[
%	\cT^j_{\bP} = \{ X \in \bP \ | \ \| X \|_{\cT_\bP^{j}} \}
%	\]
%	\begin{construction}
%		
%		Just set 
%	
%		Decompostability: From $X + Y \in \cT_{\bP}^{j}$ we get $X + Y\in \bP$, thus we deduce $X \in \bP$ by Summand-stability and $\|X\|_{\cT_{\bP}^j} = \|X\|_j$ by construction of $\cT^j$. This ends the construction \\		
%	\end{construction}	
%%	Given $j$, the section $\cT_\bullet^j$ is well-defined, i.e. has image in decompostable topologies: . % and $\|X\|_{\bP}^{\lnot \lnot}$ from the former we . From the latter we see $\lnot \lnot X$. Hence $X \in \cT_{\bP}^{\lnot \lnot}$. \\
%	We have 
%	\[
%	\cT_{\cP^{j}_\bT} = \bT
%	\]
%	by assumption on $\bT$.
%
%%	It remains to show that $\cT_\bP^j$ is decompostable. If $X + Y \in \cT_\bP^j$, then by Summand stability 
%\end{proof}
\begin{lemma}{\label{lemma:AtMostOneFlat}}
		For any $\bP : \mathsf{FLAT}$ and any Lavwere Tierney operator $j$,
	\[\cT^j_{\bP} := \{ X \in \bP \ | \ j \|X \| \}	 \]
	
	is flattened by $\bP$. Furthermore
	\[\bP  = \cP_{\cT^{j}_{\bP}}.\]
\end{lemma}
\begin{proof}

				This is indeed a topology as $\bP$ and $j$ are $\sum$-stable %and $j$ is $\sum$-closed.
			We need to show, that for any $X \in \bP$, we have $\|X\|_{\cT^j_{\bP}} = j\|X\|$.
			Note
			\[
			\|X\|_{\cT^j_{\bP}} = \exists Y \in \cT^j_{\bP} : \|Y\| \to \|X\|
			\]
			If $j\|X\|$, then put $Y := X$. Conversely, given $Y \in \cT^j_{\bP}$ such that $\|Y\| \to \|X\|$, we deduce from $j \|Y\|$ that $j\|X\|$. \\
			Furthermore,
		\[
	\{X \ | \ \exists Y , X + Y \in \bP \land j \|X + Y\| \} = \{X \ | \ X \in \bP\}
	\]
	by Summand-stability on $\bP$ we have $'\subset'$.
	if $X \in \bP$, then use $Y:=1$: $X + 1 \in \bP$ and $j \| X + 1\|$. \\
\end{proof}
Proof of theorem \ref{thm:Flat}:
\begin{enumerate}
	\item [1. and 2.] 	Assume that $\bP : \mathsf{FLAT}$ flattens $\bT$, i.e. $\cT_{\bP}^{\|\cdot\|_\bT}  = \bT$. We want to show that then $\bT$ is decompostable and $\bP= \cP_\cT$. 
	First observe that $\cP_\bT \subset \bP$ as $\{\bot\} \cup \bT \subset \bP$ %and $\bP$ is stable by decidable subtypes by \ref{lemma:SummandStable}. \\
	For decompostability we apply \ref{prop:detectDecompostable}.
	Observe
	\[\cT_{\cP_\bT}^{\|\cdot\|_\bT} \subset \cT_{\bP}^{\|\cdot\|_\bT} = \bT \]
	The other inclusion is automatic. This shows decompostability.
	Note
	\[
	\cP_\bT = \cP_{\cT_{\bP}^{\|\cdot\|_\bT}} \overset{\ref{lemma:AtMostOneFlat}}{=} \bP
	\]
	\item [3.] By the first point and \ref{lemma:AtMostOneFlat}.
\end{enumerate}
%\begin{lemma}
%	Let $L$ be a lex subcanonical modality. Assume $\bT$ is flattened. Assume for $X$ affine, $X \in \bT$ is an $L$-sheaf.
%	Then we have 
%	\[\{X : \cP_\bT \ | \ L \| X \|\} \subset \bT\]
%\end{lemma}
%\begin{proof}
%	$'\supset'$ is clear. 
%	We prove the other inclusion by induction over $\cP_\bT$ for the predicate $Q X := L \| X \| \to X \in \bT$. \\
%	It holds for $X \in \bT$ and $X = \bot$. Now suppose $ X: \cP_\bT$ with $Q \ni X \overset{B}{\to} Q$ is a type family, we want to show $\sum_{x: X} B x \in Q$. So assume $L \left \| \sum_{x: X} Bx \right \|$, we get $L \| X \|$, so as $X \in Q$ , we have $X \in \bT$. We may show $\|X\|_\bT$. Then as the goal is an $L$-sheaf by assumption, we may assume $(x , t) : \sum_{x: X} Bx $ . Hence we have a $\bT$-merely 
%\end{proof}
\begin{question}
	If $\bT$ is flattened, what is the difference between $\Omega$-stability for covering stacks and lex $\bP$? \\
	Are 0-gerbes $\bT$-flat ?
\end{question}


%\begin{definition}
%	Let $\bP$ contain 1 and be stable under $ \sum$ and under summands. $\bP$ flattens $\bT$ iff there exists a modality $j$ such that $\bT = \cT_\bP^j$.
%	%	Let $L$ be a lex modality.
%	%	$\bP$ flattens $L$ if for a type to be a $L$-sheaf its enough to check that its $\|X\|$-local for all $X \in \bP_L$. Here we view $L$ as a Lavwere Tierney Operator.
%	
%	%if there exists a Lavwere Tierney operator $j$, such that $L$ sheaves are exactly the types that are $\|X\|$-local for all $X \in \bP_L$. 
%\end{definition}

%Consider a lex modality $L$.
%\begin{definition}
%	A subset of affines $\bP$ \emph{flattens} $L$ if
%	\begin{itemize}
%		\item $1 \in \bP$
%		\item $\bP$ is $\sum$-stable
%		\item The topology ${\bP_L} := \{ X \in \bP \ | \ L \| X \| \}$ induces the lex modality $L$.
%
%	%	\item If $\bP \ni X \to Y \in \Aff$ is a ${\bP_L}$-cover, then $Y \in \bP$.
%%		\item $\bP$ is ${\bP_L}$-local property of affines. local satisfies descent
%	\end{itemize}
%	
%	%	\begin{itemize}
%	%		\item Either $X$ is contractible
%	%		\item Or Whenever we find $\Spec B \in {\bP_L}$ such that $\Spec B \to L \|X\|$
%	%	\end{itemize}
%	%	generates $L$.
%\end{definition}

%Things in $\bP$ we call 'flat'.


\section{Geometric propositions}
\begin{definition}
	$U : \Aff$ is called weakly-flat, if 
	\[\|U\|_\bT \to (U \in \bT)\]	
\end{definition}
\begin{lemma}{\label{lemma:geometricEquiv}}
	The converse holds always
\end{lemma}
\begin{proof}
	because things in $\bT$ are automatically $\bT$-merely inhabited
\end{proof}
\begin{example}
	Examples of weakly-flat affines for the Zariski topology
		\begin{itemize}
		\item finite sums of principal opens
		\item Closed propositions
	\end{itemize}
	for the fppf topology: flat affines . \\
	For the \etale topology: formally \etale affines
\end{example}

Recall the definition of $\bT$-atlas \ref{def:TAtlas}
\begin{definition}{\label{def:algprop}}
	Let $\bT$ be saturated. We call a modal proposition geometric, if one of the equivalent conditions is satisfied:
	\begin{enumerate}
		\item  its merely of the form $\|U\|_\bT$ for some geometric affine $U$.
		\item It is a geometric stack
%		\item There is a $\bT$-surjective map out of a geometric affine $U$.
		\item It has a $\bT$-atlas.

	\end{enumerate}
	
\end{definition}
\begin{proof} \
	\begin{enumerate}
		\item [1 $\Rightarrow$ 2]
		we show that $U \to \|U\|_\bT$ is a geometric atlas. Every fiber is in $\bT$, because $U$ is geometric. A $\bT$-atlas is a geometric atlas.
		\item [2 $\Rightarrow$ 3]
		If $P$ is a geometric -1-stack, then we may choose $U \to P$ a geometric atlas. This is a $\bT$-atlas by \ref{prop:affineCoveringStack}.

		\item [3 $\Rightarrow$ 1]
		
		Let $V \to P$ be a $\bT$-atlas.
		have to show TFAE $\|V\|_\bT \to P \to (V \in \bT) \overset{\ref{lemma:geometricEquiv}}{\to} \|V\|_\bT$. 
		Proof: $\|V\|_\bT \to P$ as $P$ is modal prop. Secondly, because $V \to P$ is a $\bT$-cover. \\
		Hence $P$ is a geometric proposition.
	\end{enumerate}
	
\end{proof}
\begin{lemma}{\label{lemma:covM1Stacks}}
	Even Without any saturatedness condition, Covering -1-stacks $X$ are contractible.
\end{lemma}
\begin{proof}
	Choose a geometric catlas $\bT \ni \Spec A \to X$. By the same trick as in the previous lemma, this induces an equivalence $1 \simeq \|\Spec A\|_\bT \overset{\sim}{\to} X$.
\end{proof}
\begin{example}
	Open / Closed Propositions are geometric.
\end{example}
\begin{question}
Is every geometric proposition a scheme?
\end{question}
It is an algebraic space that embeds into an affine, so it suffices to reproduce the statement from the presheaf model.

%\begin{lemma}[NECESSARY?]
%	geometric propositions are algebraic spaces.
%\end{lemma}
%\begin{proof}
%	We have $U \to \|U\|_\bT$ where $U$ is affine, hence an algebraic space and the fibers are in $\bT$ by geometricness of $U$, hence they are covering algebraic spaces. By stability under quotients, our geometric proposition is an algebraic space.
%\end{proof}

\section{Algebraic Space}
Recall the notion of (covering) geometric 0-stacks, which we call (covering) Algebraic Spaces. it is the smallest pair of classes that satisfies the following
\begin{itemize}
	%	\item Stability under $\sum$ \ref{thm:stabSums} 
	\item (covering) affines are (covering) algebraic spaces. %If $\bT \ni \Spec A \to X$ is a $\bT$-cover and $X$ is a $\bT \Set$, then $X$ is a covering algebraic space
	\item stable under covering quotients: If $X$ is an algebraic space, $Y$ modal 0-type and $X \to Y$ is fibered in covering algebraic spaces, then $Y$ is an algebraic space. Additionally, if $X$ is covering, then $Y$ is covering.
\end{itemize}
\subsection{Equivalence relations vs Surjections}
%\subsection{For groupoids}
%\begin{lemma}
%	If $R \twoheadrightarrow X \to X$ is a $\bT$-htpy-coequalizer diagram of two $\bT$-covers between affines, then $X$ is a  1-stack.
%\end{lemma}

%\subsection{For sets}
\begin{lemma}{\label{quotient-by-equivalence-relation}}
	Denote $\bT Set$ for the sets that are $\bT$-sheaves. Assume given a $\bT$set  $X$ then the following maps are mutually inverse
	\begin{align*}
		 \mathrm{EqRel}(X  , \bT \Prop) \equiv \sum_{R:X\to X\to \bT\Prop} R\ \mathrm{equivalence\ relation} &\simeq \sum_{Y:\bT \mathrm{\Set}} \sum_{p:X\to Y} p\ \bT\mathrm{surjective} \\
		R &\mapsto (L_\bT \| X//R \|_0,[\_]) \\
		\lambda x,y.  (p(x)=p(y)) &\mapsfrom (Y,p) 
	\end{align*}
	%where $X / R$ is defined by applying $L_T \| \_ \|_0 $ at the higher inductive type $X // R$.
\end{lemma}
\begin{question}
	Do we actually need to set-truncate? Do we want to also mod out relations which are not given as an equivalence relation?
\end{question}
\begin{proof}
	\begin{itemize}
		\item Well-definedness: The map $[\_] : X \to \|X // R\|_0 \to L_T \|X // R\|_0$ is the composition of a surjective with a $\bT$-surjective map \todocite, hence its $\bT$-surjective. \\
		Conversely given $(Y,p)$ as $Y$ is a sheaf, we have for all $x,y : X$ that $p(x) =_Y p(y)$ is a sheaf.
		\item If $x,y : X$ then we have a chain of equivalences 
		\[
		R(x,y) \simeq (\bar x =_{\|X//R\|_0} \bar y) \overset{\mathsf{ap}_\eta}{\to} ([x] =_{L_T\|X//R\|_0} [y])
		\]
		where the first map is plain HoTT, meaning that $\|X//R\|_0$ is seperated. The second map is an equivalence by \ref{lemma:sep}. %, i.e. the unit of the modality \ref{lemma:idTypesOfSheafification}, but as the $\bar x =_{\|X//R\|_0} \bar y$ is already a sheaf, it is an isomorphism as well. \\
		\item Let $(Y,p)$ be in the RHS. Let $R(x,y) = (p(x)=p(y)) : \bT \Prop$. By plain HoTT, There is a map $\eta :  X // R  \to Y$ ( defined by the universal property of the set truncation and by induction on the higher inductive type $ X // R$ on canonical terms through the map $p : X \to Y$). I claim $\eta$ exhibits $Y$ as the localization for $\bT \Set$-modality of $X // R$. Let $T$ be another $\bT \Set$ equipped with a map $X // R  \to T$. By precomposition we obtain a map $X \to T$. 
		Claim: it factors uniquely through $p : X \to Y$. 
		% https://q.uiver.app/#q=WzAsNCxbMCwwLCJYIl0sWzEsMCwiXFx8WCAvIFJcXHwiXSxbMiwwLCJUIl0sWzEsMSwiWSJdLFswLDFdLFsxLDJdLFswLDNdLFszLDIsIlxcZXhpc3RzISIsMix7InN0eWxlIjp7ImJvZHkiOnsibmFtZSI6ImRhc2hlZCJ9fX1dXQ==
		\[\begin{tikzcd}
			X & {X // R} & T \\
			& Y
			\arrow[from=1-1, to=1-2]
			\arrow[from=1-1, to=2-2]
			\arrow[from=1-2, to=1-3]
			\arrow["{\exists!}"', dashed, from=2-2, to=1-3]
		\end{tikzcd}\]
		Proof: \\
		Existence: We want to define a map $Y \to T$. Let $y : Y$. As $p$ is $\bT$-surjective and $T$ is a sheaf, we may assume we merely have some element in the fiber of $p$ over $y$. Now push this element through     
		\[\|\fib_p y\| \to \|X // R\|_0 \to T\]
		where the first map is by Plain HoTT and the second one is induced from $X // R \to T$ by assumption and the fact that $T$ is a set.. One can easily check this makes the diagram commute.
		Uniqueness follows from $X \to Y$ beeing $\bT$-surjective and the following
		Fact: Two parellel maps $Y \rightrightarrows T$ into a $\bT \Set$ $T$ are already equal if the become equal after precomposition with a $\bT$-surjection $X \to Y$.  \\
		Proof of the fact : Let $y : Y$. The goal is an identity type of a $\bT \Set$, hence a $\bT \Prop$. Hence As the fiber over $y$ in $X$ is $\bT$-merely inhabited, we may assume an actual term in the fiber. 	As $X \to Y$ equalizes the arrows, this term allows us to conclude. \qed (fact)	\qed(Claim) \\
		We apply the fact to the ($\bT$-)surjectivity of $X \to X // R $ to get a unique factorization 
		% https://q.uiver.app/#q=WzAsNCxbMCwwLCJYIl0sWzEsMCwiXFx8WCAvIFJcXHwiXSxbMiwwLCJUIl0sWzEsMSwiWSJdLFswLDEsIiIsMCx7InN0eWxlIjp7ImhlYWQiOnsibmFtZSI6ImVwaSJ9fX1dLFsxLDJdLFswLDNdLFszLDIsIlxcZXhpc3RzISIsMix7InN0eWxlIjp7ImJvZHkiOnsibmFtZSI6ImRhc2hlZCJ9fX1dLFsxLDNdXQ==
		\[\begin{tikzcd}
			X & {X // R} & T \\
			& Y
			\arrow[two heads, from=1-1, to=1-2]
			\arrow[from=1-1, to=2-2]
			\arrow[from=1-2, to=1-3]
			\arrow[from=1-2, to=2-2]
			\arrow["{\exists!}"', dashed, from=2-2, to=1-3]
		\end{tikzcd}\]
		making the right triangle commute. This is what we wanted to show.
	\end{itemize}
\end{proof}



	\begin{definition}{\label{def:coveringEqRel}}
		%A modal equivalence relation $R : U^2 \to \bT \Prop$ on a set $U$ is covering if the fibers $R_s \equiv \sum_{t: S} R(s,t)$  are covering 0-stacks.
		Let $S$ be a geometric stack. An equivalence relation $R$ on $S$ is called covering, if all the propositions $R(s,t)$ are sheaves and every fiber $R_s \equiv \sum_{t: S} R(s,t)$ is a covering stack. 
	\end{definition}
\begin{lemma}{\label{lemma:CovRelGeomProp}}
	If $R$ is covering on $S$, then the propsotions $R(x,y)$ are geometric propositions.
\end{lemma}
\begin{proof}
	 For all $s , t : S$, $R(s,t)$ is a geometric proposition: $R(s,t)$ is the fiber of the projection $\sum_{t : S} R(s,t) \to S$ between geometric stacks, which are stable under finite limits. \\
\end{proof}
\begin{lemma}{\label{lemma:CovRel}}
	If $S$ is affine, then a modal equivalence relation on $S$ is covering iff every fiber $R_s \equiv \sum_{t: S} R(s,t)$ merely admits a $\bT$-catlas.
\end{lemma}
\begin{proof}
	Every sheaf admitting a $\bT$-catlas is a covering 0-stack. 
	Conversely: Let $s : S$ such that the fiber $R_s$ is a covering 0-stacks. We want to construct at $\bT$-catlas of $R_s$.
	The $R(s,t)$ are geometric propositions by \ref{lemma:CovRelGeomProp}.
	For all $t : S$ we there merely is a geometric atlas $\Spec A_t \to R(s,t)$ by \ref{def:algprop}. By Zariski Local choice we find a Zariski cover $f : S' \to S$ equipped with a Geometric atlas $\Spec A_{t'} \to R(s,f (t'))$ for all $t' : S$. Then
	\[
	\sum_{t:S'} \Spec A_{t'} \to \sum_{t : S} R(s,t)
	\]
	is a $\bT$-atlas by \ref{lemma:AtlasSum}. As $\sum_{t : S} R(s,t)$ is a covering 0-stack by assumption, the map has to be a $\bT$-catlas by \ref{lemma:atlasIsCatlas}. 
\end{proof}

\begin{lemma}{\label{lemma:fundamental-property-algebraic-spaces}}
	%Assume that $\bT$ satisfies descent for propositions and for sets 
%	Assume that the topology has descent.
	Given an affine $X$, the following types are equivalent:
	\begin{itemize}
		\item The type of covering equivalence relations on $X$.
		\item The type of $\bT$sets $Y$ equipped with a map $X \to Y$ fibered in types admitting a $\bT$-catlas.
	\end{itemize}
\end{lemma}

\begin{proof}
	By the equivalence in \ref{quotient-by-equivalence-relation} it is enough to check that
%	\begin{itemize}
%		\item The identity types in $X/R$ are 
%		(-1)-stacks if and only if the relation $R$ is redundant . For any $x,y:X$ we know that:
%		\[R(x,y) \simeq [x] =_{X/R}[y]\]
%		so the direct direction is immediate. For the converse we use the assumption that a modal proposition being a  (-1)-stack is a sheaf and that the map $[\_]:X\to X/R$ is $\bT$-surjective.
		%\item 
		The fibers of: 
		\[[\_]:X\to L_\bT \| X//R \|_0\] 
		merely admit a $\bT$-catlas if and only if the relation $R$ is covering. For any $y:X$ we have that:
		\[\sum_{x:X} R(x,y) \simeq \mathrm{fib}_{[\_]}([y])\]
		so the direct direction is immediate. The converse follows from $\bT$-surjectivity of $[\_]$ and from \ref{cor:DescentCatlas}.
%	\end{itemize}
\end{proof}

\section{Algebraic Space}
Recall the notion of (covering) 0-stacks. it is the smallest pair of classes that satisfies the following
\begin{itemize}
	\item Stability under $\sum$ \ref{thm:stabSums} 
	\item (covering) affines are (covering) algebraic spaces. %If $\bT \ni \Spec A \to X$ is a $\bT$-cover and $X$ is a $\bT \Set$, then $X$ is a covering algebraic space
	\item stable under covering quotients: If $X$ is an algebraic space, $Y$ modal 0-type and $X \to Y$ is fibered in covering algebraic spaces, then $Y$ is an algebraic space. Additionally, if $X$ is covering, then $Y$ is covering.
\end{itemize}
\subsection{Geometric propositions}
\begin{definition}
	An affine Scheme $U$ is called geometric, if 
	\[\|U\|_\bT \to (U \in \bT)\]	
\end{definition}
\begin{lemma}{\label{lemma:geometricEquiv}}
	The converse holds always
\end{lemma}
\begin{proof}
	 because things in $\bT$ are automatically $\bT$-merely inhabited
\end{proof}
Recall the definition of $\bT$-atlas \ref{def:TAtlas}
\begin{definition}{\label{def:algprop}}
	We call a modal proposition geometric, if one of the equivalent conditions is satisfied:
	\begin{enumerate}
		\item  its merely of the form $\|U\|_\bT$ for some geometric affine $U$.
		\item There is a $\bT$-surjective map out of a geometric affine $U$.
		\item It has a $\bT$-atlas.
	\end{enumerate}

\end{definition}
\begin{proof} \
	\begin{enumerate}
		\item [1 $\Leftrightarrow$ 2]
		Clear.
		\item [1 $\Rightarrow$ 3]
			we show that $U \to \|U\|_\bT$ is a $\bT$-atlas. Every fiber is in $\bT$, because $U$ is geometric.
		\item [3 $\Rightarrow$ 1]
		
				Let $V \to P$ be a $\bT$-atlas.
		have to show TFAE $\|V\|_\bT \to P \to (V \in \bT) \overset{\ref{lemma:geometricEquiv}}{\to} \|V\|_\bT$. 
		Proof: $\|V\|_\bT \to P$ as $P$ is modal prop. Secondly, because $V \to P$ is a $\bT$-cover. \\
		Hence $P$ is a geometric proposition.	
	\end{enumerate}

\end{proof}

\begin{lemma}
	geometric propositions are algebraic spaces.
\end{lemma}
\begin{proof}
	We have $U \to \|U\|_\bT$ where $U$ is affine, hence an algebraic space and the fibers are in $\bT$ by geometricness of $U$, hence they are covering algebraic spaces. By stability under quotients, our geometric proposition is an algebraic space.
\end{proof}
\subsection{Algebraic spaces}
\begin{definition}{\label{def:coveringEqRel}}
		%A modal equivalence relation $R : U^2 \to \bT \Prop$ on a set $U$ is covering if the fibers $R_s \equiv \sum_{t: S} R s t$  are covering 0-stacks.
		Consider a modal equivalence relation $R : U^2 \to \mathsf{GeomProp}$ on an affine $U$. We call it covering if every fiber $R_s \equiv \sum_{t: S} R s t$ satisfy one of the following equivalent conditions
		\begin{itemize}
			\item admit a $\bT$-catlas.
			\item is a covering 0-stack.
\end{itemize}
\end{definition}
\begin{proof}
Every type admitting a $\bT$-catlas is a covering 0-stack. 
Conversely: if the fibers are covering 0-stacks. For all $t : S$ we can choose a geometric atlas $\Spec A_t \to R s t$ by \ref{def:algprop}. Then 
\[
\sum_{t:S} \Spec A_t \to \sum_{t : S} R s t
\]
is a $\bT$-atlas. As $\sum_{t : S} R s t$ is a covering 0-stack by assumption, the map has to be a $\bT$-catlas by \ref{lemma:atlasIsCatlas}. 
\end{proof}


\begin{definition}
	A modal set $X$ is an algebraic space iff it is merely of the form $L_\bT (U / R)$ for some affine $U$ and  $R : U^2 \to \Prop$ a covering equivalence relation. Equivalently there exists some map $U \to X$ whose fibers merely have $\bT$-catlasses. We call $X$ covering if $U$ can be choosen to be in $\bT$.
\end{definition}
\begin{lemma}
	Every (covering) algebraic space is a (covering) geometric 0-stack.

\end{lemma}
\begin{proof}
	Choose a presentation $ R: U^2 \to \Prop$.
	It suffices to show, that the map $f : U \to L_\bT ( U / R)$ is a geometric (c)atlas. The map $f$ is $\bT$-surjective by the well-definedness of the bijection $\ref{quotient-by-equivalence-relation}$. By descent we may just show, that the fibers $\fib_f (f(s))$ for $s : U$ are covering 0-stacks. But by the bijection in \ref{quotient-by-equivalence-relation} those are equivalent to the fibers $R_s$, which are covering 0-stacks as the equivalence relation is covering.
\end{proof}
\begin{corollary}
	The identity types of algebraic spaces are geometric propositions.
\end{corollary}
\begin{proof}
	By the previous lemma and \ref{lemma:geometricStacksClosedUnderId}
\end{proof}

\begin{lemma}{\label{lemma:detectGeomProp}}
	Let $P$ be a sheaf and a proposition that admits a map $\Spec A \to P$ fibered in covering algebraic spaces. Then $P$ is a geometric proposition.
\end{lemma}
\begin{proof}
	The fibers are covering algebraic spaces and affine, hence covering affine. By \ref{def:algprop} we conclude.
\end{proof}
\begin{theorem}
	Let $X$ be a sheaf of sets. Let $S$ be (covering-) affine and $f : S \to X$ be fibered in covering algebraic spaces. Then $X$ is a (covering) algebraic space.
\end{theorem}
\begin{proof}
	The identity types of $X$ admit a map fibered in covering algebraic spaces (todo check stability under $\sum$) out of an affine by \ref{lemma:havingAbstractAtlasClosedUnderId}. by \ref{lemma:detectGeomProp} they are geometric propositions. The equivalence relation determined by $f$ is covering \ref{def:coveringEqRel} , because the fibers of $f$ are covering 0-stacks.
\end{proof}


\subsection{Stability under covers TODO}
In this subsection we want to prove the following:
\begin{theorem}[TODO]
	The class of covering $\cV$-stacks is the smallest intermediate class $\bT \subset \tilde T \subset \cV$ such that whenever $X \in \tilde \bT, Y \in \cV$ and $X \to Y$ is fibered in $\tilde \bT$, then $Y \in \tilde \bT$.
\end{theorem}

\begin{lemma}
	Covering stacks are stable by dependent sums: If $X \in \CS_\cV$, $Y : X \to \CS_\cV$, then $\sum_{x: X} Y x \in CS$.
\end{lemma}
\begin{proof}
	Lets first prove the special case where $X \in \bT$. By choice of $X$ we can choose a $C$-atlas $Q x \to Y x$ for every $x$. Now $\sum_{x : X} Q x \to \sum_{x: X} Y x$ is fibered in $C$ by \ref{lemma:AtlasSum} and the domain is in $\bT$ by $\sum$-stability of $\bT$. \\
	For the general case, choose a $C$-atlas $p : T \to X$ with $T \in \bT$. Then we have a map
	\[
	\sum_{t : T} Y (p t) \to \sum_{x : X} Y x
	\]
	where every fiber is equivalent to a fiber of $p$, i.e. its a covering $C$-stack. As its domain is a covering $C$-stack by the previous case, we can choose an atlas .
\end{proof}
\begin{proof}
	The first class is definitely contained in the second class. To show that they coincide we need to show, that the first class is stable under $\sum$ and under quotients. For the first point we use choice of affines. The second point 
\end{proof}


The first point is the minimal definition which is good mapping out of the class of coverings stacks and the second one is useful to keep in mind the stability results.
The closedness under covers assumption is the conjunction of closed under $\sum$ (as $C$ $\sum$-stable) and closed under quotients. \\
\begin{lemma}
	covering $C$-stacks contain $1$ and are closed under $\sum$.
\end{lemma}


\begin{example}
	covering Aff-stacks $=$ saturation of $\bT$. Indeed: By definition, the saturation of $\bT$ is is obtained by quotients of $\bT$ by $\bT$-covers. We have shown, that its closed under covers between affines.
\end{example}

\begin{definition}
	We call $X$ a $C'$-stack, iff there merely exists some affine $\Spec A \to X$ fibered in covering $C$-stacks. \\
	We call $X$ a $C$-stack, iff its a $C'$-stack and $X \in C$.
\end{definition}
\begin{definition}
	The (covering )$\infty$-stacks are the (covering) $\cU$-stacks.
\end{definition}
\begin{lemma}
	$X$ is a $n'$ stack iff its an $n+1$-stack
\end{lemma}
\begin{proof}
	If its 
\end{proof}
\begin{lemma}{\label{lemma:geometricStacksClosedUnderId}}
	$C$-stacks are closed under $\id$-types.
\end{lemma}
\begin{proof}
	
	This is similar to \ref{lemma:havingAbstractAtlasClosedUnderId}.
\end{proof}
\begin{warning}
	The previous lemma does not hold for covering stacks: Identity types of things in $\bT$ could be empty.
\end{warning}

THIS IS UNUSUAL, but surprisingly useful.
Let $n \ge 0$.
%\begin{lemma}[TODO]
%	If $D \subset C$, then the cover-closure of (covering $C$-stacks $\cap D$) inside $D$ are  covering $D$-stacks. 
%\end{lemma}
\begin{example}
	Affine covering $0$-stacks are the saturation of $\bT$.
\end{example}
%\begin{corollary}[Indepedence of the truncation level]
%	(covering) $n+1$-type-stacks that are $n$-types $=$ (covering) $n$-type stacks.
%\end{corollary}
%\begin{proof}
%	Only need to show $\subset$. First the covering part. For this just show the LHS is closed under covers between $n$-types by the previos lemma. \\
%	For the non covering part, let $\Spec A \to X$ be fibered in covering $n+1$-type-stacks where $X$ is a $n$-type. Then the fibers are $n$-types, hence by the covering case, they are covering $n$-type stacks.
%\end{proof}
\begin{definition}
	$X$ is a (covering) 0-stack, if its a (covering) 0-type-stack.
\end{definition}
\begin{theorem}[TODO]
	Let $X$ be a type. TFAE for all $n$ :
	\begin{enumerate}
		\item $X$ is a covering $n$-type-stack.
		
		\item Inductively, There merely exists some $U \in \bT$ with a map $U \to X$ fibered in covering $n-1$-stacks.
		\item[2'] Inductively, as the previous one but additionally the $\id$-types of $X$ are $n-1$-stacks.
		\item Inductively, There merely exists some covering $n-1$-stack $U$ with a map $U \to X$ fibered in covering $n-1$-stacks.		
		\item[3'] Inductively, as the previous one but additionally the $\id$-types of $X$ are $n-1$-stacks.
	\end{enumerate}
	If one of the conditions is satisfied we call $X$ a covering $n$-stack.
\end{theorem}
\begin{proof}
	Induction $n-1 \mapsto n$ , $n \ge 1$.
	\begin{enumerate}
		\item[1. $\Rightarrow$ 2] We have to show, that the class in 2. is closed under $\sum$ and closed under quotients between $n$-types. This was already done.
		\item [2. $\Rightarrow$ 3] Clear
		\item [3. $\Rightarrow$ 3'] By  \ref{lemma:geometricStacksClosedUnderId} and independence of the truncation level (TODO).
		\item [3'. $\Rightarrow$ 3 , 2' $\Rightarrow$ 2] Clear
		
		\item [3'. $\Rightarrow$ 1.]  by induction, covering $n-1$-stacks $=$ covering $n-1$-type-stacks $\subset$ covering $n$-type-stacks. Now use stability under covers between $n$-types.
		\item [3' $\Rightarrow$ 2'] Use 3 $\Rightarrow 1 \Rightarrow 2$.
		%	\item [2. $\Rightarrow$ 3] By some argument of Hugo, an $X$ as in $2.$ is an $n$-type.
	\end{enumerate}
\end{proof}
%We want: = covering $n$-stacks. The $\subset$-direction is clear, as covering $n$-stacks should be stable under covers between $n$-types. for $\supset$, make sure that covering $n$-stacks for $n$ small can be constructed as a quotient by a $\bT$-cover




\section{Deloopings and Truncations}

We denote $\| \cdot \|_n^\bT := L_\bT \| \cdot \|_n$, which is a modality.
\begin{definition}
	A pointed stack $(X,x)$ is called $\bT$-connected if for any $y : X$ we have $\|x = y\|_\bT$.
\end{definition}
\begin{definition}
	Let $G$ be a stack. A delooping stack of $G$ is a  pointed $\bT$-connected stack $B G$ equipped with an equivalence $\Omega B G \simeq G$.
\end{definition}

\begin{lemma}{\label{lemma:deloopingCS}}
	Let $G$ be a covering stack, that admits a delooping stack $B G$. Then $B G$ is a covering stack.
\end{lemma}
\begin{proof}
	Now assume $G$ is a covering stack. 	To show, that  $B G$ is a covering stack, we may show that the map $\bT \ni 1 \to B G$ is a geometric atlas. As $B G$ is $\bT$-connected, every point is $\bT$-merely equal to the basepoint. By descent for covering stacks, we may just show that the fiber over the basepoint is a covering stack
	But this is equivalent to $\Omega B G \simeq G$. 
\end{proof}
\begin{corollary}
	If $G$ is a covering group 0-stack,  that admits an $n$-fold delooping stack $B^n G$, then this will be a covering $n$-stack.
\end{corollary}
\begin{lemma}
	The fiber of $X \to \|X\|_n^\bT$ over $|x|$ is $\sum_{y : X} \| x = y\|_{(n-1)\bT}$
\end{lemma}
\begin{proof}
	For any $x : X$, we may show that the type family
	\begin{align*}
		B : \|X\|_n^\bT &\to \cU_{n-1}^\bT \\
		\|y\|_n &\mapsto \| x = y \|_{n-1}^\bT
	\end{align*}
	defined using the $n$ truncatedness of the stack $\cU_{n-1,\bT}$, is a unary identity system of $\|X\|_{n}^\bT$ at $|x|$. 
	By the fundamental system of identity types its enough to construct for all $y : \| X\|_n^\bT$, a section of the map $|x| = y \to B y$ induced by path induction. \\
	As the space of sections of a map between $n$-stacks is in particular an $n$-stack, we may just for all $y : X$ construct a section of the map 
	\[\mathrm{ind} : |x| =_{\|X\|_n^\bT} |y| \to \| x = y \|_{n-1}^\bT\]
	But $|x| = |y|$ is an $n-1$-stack, so there is a unique dashed map $\sigma$ such that the above triangle
% https://q.uiver.app/#q=WzAsNCxbMCwxLCJ8eHw9fHl8Il0sWzEsMSwiXFx8eCA9X1ggeSBcXHxfe259XlxcYlQiXSxbMSwwLCJ4ID1fWCB5Il0sWzEsMiwiXFx8eCA9X1ggeSBcXHxfe259XlxcYlQiXSxbMiwxLCJcXGV0YSJdLFsyLDAsIlxcYXAiLDJdLFsxLDAsIlxcc2lnbWEiLDAseyJzdHlsZSI6eyJib2R5Ijp7Im5hbWUiOiJkYXNoZWQifX19XSxbMSwzLCJcXGlkIl0sWzAsMywiXFxtYXRocm17aW5kfSIsMix7ImN1cnZlIjoyfV1d
\[\begin{tikzcd}
	& {x =_X y} \\
	{|x|=|y|} & {\|x =_X y \|_{n}^\bT} \\
	& {\|x =_X y \|_{n}^\bT}
	\arrow["\ap"', from=1-2, to=2-1]
	\arrow["\eta", from=1-2, to=2-2]
	\arrow["{\mathrm{ind}}"', curve={height=12pt}, from=2-1, to=3-2]
	\arrow["\exists! \sigma", dashed, from=2-2, to=2-1]
	\arrow["\id", from=2-2, to=3-2]
\end{tikzcd}\]
	commutes. This is indeed a section of the above map, because the maps $\mathrm{ind} \circ \sigma$ and $\id$ targeting an $n$-stack become equal after postcomposition with the unit $\eta$ of the modality $L_\bT \| \cdot \|_n$.
	
%	
%	
%	We may show, that the map
%	
%		For this we need that for all $x ,y : X$ the map
%	\[
%	\|x =_X y \|_{n-1,\bT} \to |x| =_{\|X\|_{n\bT}} |y|
%	\]
%	is an equivalence. 
 	
%	\[
%	\isContr \sum_{y : \|X\|_n^\bT} B y \simeq \prod_{y : \|X\|_n^\bT} \prod_{t : By}(p : |x| = y) \times (\tp_p r = t)
%	\]
%	where $r : B |x|$ is defined as $| \refl_x|$. \\
%	So Let $x : X$. For any such $y$ the type $\prod_{t : By}(p : |x| =_{\|X\|_n^\bT} y) \times (\tp_p r = t)$ is an $n$-stack so we may just provide a term in
%	\[
%	\prod_{y : X} \prod_{t : \| x = y\|_{n-1,\bT}}(p : |x| =_{\|X\|_n^\bT} |y|) \times (\tp_p r = t)	
%	\]
%	Let $y : X$. For any such $t$, as $(p : |x| =_{\|X\|_n^\bT} |y|) \times (|p| = t)$ is a $n-1$-stack, we may just construct a term in
%	\[
%	\prod_{t : x = y} (p : |x| = |y|) \times \tp_p r = |t|
%	\]
%	this is obvious by setting $p := \mathsf{ap}_{|\_|} t$ and we compute
%	\[
%	\tp_{\ap_{|\_|} t} |\refl_x| = | t \cdot \refl_x | = |t|
%	\]
\end{proof}
\begin{lemma}
	For any $X$ and any $n \ge -1$, the map $X \to \|X\|_n^\bT$ is $\bT$-surjective.
\end{lemma}
\begin{proof}
	It factors as $X \to \|X\|_n \to L_\bT \|X\|_n$ where the latter map is $\bT$-surjective. So it sufficess to show, that the former map is surjective. As $X \to \|X\|_0$ is surjective it suffices to show, that $\ap$ of the map $\|X\|_n \to \|X\|_0$ is surjective. TODO %But for any type $T$ the map $T \to \|T\|_0$ is surjective.
\end{proof}
\begin{prop}
	Let $X$ be a geometric stack, such that for all $x : X$, $\Omega^{n+1} (X , x)$ is a covering stack for all $x : X$. Then $\|X\|_n^\bT$ is a geometric $n$-stack.
\end{prop}
\begin{proof}
	As $X$ is a geometric stack it suffices to show, that the $\bT$-surjection map $X \to \|X\|_n^\bT$ is a geometric cover.
	Let us show by induction over $k = -1,\hdots,n$ that 
	\[\Omega^{n - k} X \to \|\Omega^{n - k} X\|_k^\bT\]
	is a geometric cover.  \\
	$k=-1$ is okay as $\Omega^{n+1} X$ is a covering stack and $\bT$-truncations of covering stacks are contractible. \\
	For the induction step $k - 1 \mapsto k$:
	Set $X' := \Omega^{n-k} X$, so we want to show that $X' \to \|X'\|_k^\bT$ is a geometric cover.
	Every fiber is modal so the fiber beeing a covering stack has descent, so we may just show that the fiber over the image of some $x : X$ is a covering stack. The fiber $\sum_{y : X} \| x = y\|_{(k-1)\bT}$ is $\bT$-connected, so its a delooping stack of 
	\[(p : \Omega X) \times \left(\tp_p r =_{\|\Omega X\|_{k-1,\bT}} r  \right),\]
	where $ r= |\refl|$.
	By \ref{lemma:deloopingCS} it suffices to show that this is a covering stack. But we calculate $\tp_p r = |p|$, so it is the fiber of 
	\[
	\Omega X  \to \|\Omega X \|_{k-1, \bT}
	\]
	over the basepoint $|\refl|$.
\end{proof}


\begin{definition}
	A higher group stack is a pointed $\bT$-connected stack.
\end{definition}
Let $BG$ be a higher group stack and $X$ be a geometric stack equipped with an action $\rho : BG \to \GS$. We use the standart notation

\[
X // G :\equiv \sum_{BG} \rho
\]
\begin{lemma}
 If $G$ is covering, then $X // G$ is a geometric stack
\end{lemma}
\begin{proof}
	$BG$ is a covering stack, as $G$ is a covering stack \ref{lemma:deloopingCS}. Hence $X // G :\equiv \sum_{BG} \rho$ is a geometric stack.
\end{proof}
\begin{prop}
	%Let $BG$ be a higher group stack and $X : BG \to \GS$ be a geometric $G$-stack. 
	If $X // G$ is a geometric stack (e.g. if $G$ is covering) and the isotropy stacks $\sum_{g : G} g x = x$ are covering stacks, then $\| X // G \|_0^\bT$ is an algebraic space.
\end{prop}
\begin{proof}
	To apply the prop, we have to show, that for all $x : X // G$, $\Omega (X // G,x)$ is a covering stack. As $X \to X // G$ is $\bT$-surjective (todo details), we may just show this for $x : X$.
	\[
	\Omega (X // G , [x]) \simeq \sum_{g: G} g x = x
	\]
	which was covering by assumption
\end{proof}
\begin{corollary}
	Let $X$ be a geometric stack, whose identity types are covering stacks. Let $G$ be a finite group acting on $X$. Then $L_\bT (X / G)$ is a geometric stack.
\end{corollary}
\begin{proof}
	 The isotropy stacks are covering, as they are $\sum$ of covering stacks.
\end{proof}
We can also reprove \ref{lemma:algSpacesStabFreeQuots}: $G$ is a finite type by assumption, hence covering. The isotropy stacks are assumed to be propositions, but they are inhabited, so they are covering \qed(lemma) \\

TODO: Find a good example of a non covering $G$.


%Observe by stability under $\sum$ this is an geometric stack, if $BG$ is geometric.


%The definition is reasonable because the following
\end{document}

 