\documentclass{article}
\newcommand{\Cov}{\emph{Cov} }
\newcommand{\affineA}{(affine ?)}
\newcommand{\Cover}{\emph{Cov}er }
\newcommand{\truncation}{$\bT$runcation}
%\newcommand{\Atl}{\emph{Atlas}}
\usepackage{graphicx} % Required for inserting images
\newcommand{\red}[1]{{\color{red} #1}}
\newcommand{\St}{\mathsf{St}}
\newcommand{\CS}{\mathsf{CS}}


%%%%%%%%%%%%%%%%%%%%%%%%%%%%%%%%%%%%%%%%%%%%%%%%
% load packages 

\usepackage[a4paper,nohead,left=3.5cm,right=3.5cm,top=4cm,bottom=3cm]{geometry}
%\usepackage{german}    % only for German articles
%\usepackage{a4wide}   % longer lines
\usepackage[intlimits,tbtags]{amsmath}   % for more basic mathematical symbols
\usepackage{amssymb}   % for more mathematical symbols
\usepackage{amsfonts}
\usepackage[utf8]{inputenc}
\usepackage{textcomp}
\usepackage{mathtools}
% \usepackage{stmaryrd}  % for more mathematical symbols
\usepackage{latexsym}  % for more mathematical symbols,
% already contained in amsmath-package
% \usepackage{accents}  % for more dots etc on symbols
\usepackage{amsxtra}   % for accents as superscripts
\usepackage{amstext}   % for text in formulas, accents, etc.
\usepackage{bm}        % boldface for non-latin letters
\usepackage{amsthm}    % for theorem-environments
\usepackage{amscd}     % for commutative diagrams
%\usepackage{MnSymbol}
%\usepackage[ansinew]{inputenc}
\usepackage{enumitem}  % Better enumerations.

\usepackage{graphicx}
\usepackage[dvipsnames]{xcolor}
\usepackage[arrow, matrix, curve]{xy}
\usepackage[colorlinks=true, citecolor=Blue, linkcolor=Blue, urlcolor=Blue]{hyperref}
\usepackage{xfrac}
\usepackage[utf8]{inputenc}
\usepackage{scalerel}
\newcommand{\tA}{{\hspace{1pt}\sim}_{A}\hspace{1pt}}
\newcommand{\tR}{\hspace{1pt}{\sim}_{R}\hspace{1pt}}
\newcommand{\bA}{\mathbb{A}}
\newcommand{\bB}{\mathbb{B}}
\newcommand{\bC}{\mathbb{C}}
\newcommand{\bD}{\mathbb{D}}
\newcommand{\bE}{\mathbb{E}}
\newcommand{\bF}{\mathbb{F}}
\newcommand{\bG}{\mathbb{G}}
\newcommand{\bH}{\mathbb{H}}
\newcommand{\bI}{\mathbb{I}}
\newcommand{\bJ}{\mathbb{J}}
\newcommand{\bK}{\mathbb{K}}
\newcommand{\bL}{\mathbb{L}}
\newcommand{\bM}{\mathbb{M}}
\newcommand{\bN}{\mathbb{N}}
\newcommand{\bO}{\mathbb{O}}
\newcommand{\bP}{\mathbb{P}}
\newcommand{\bQ}{\mathbb{Q}}
\newcommand{\bR}{\mathbb{R}}
\newcommand{\bS}{\mathbb{S}}
\newcommand{\bT}{\mathbb{T}}
\newcommand{\bU}{\mathbb{U}}
\newcommand{\bV}{\mathbb{V}}
\newcommand{\bW}{\mathbb{W}}
\newcommand{\bX}{\mathbb{X}}
\newcommand{\bY}{\mathbb{Y}}
\newcommand{\bZ}{\mathbb{Z}}
\DeclareMathOperator{\Sh}{Sh}

%%%%%%%%% calligraphic %%%%%%%%%%%%%%%%%%%%%%%
\newcommand{\mc}[1]{\mathcal{#1}}
\newcommand{\R}{\Rightarrow}
\newcommand{\cA}{\mathcal{A}}
\newcommand{\cB}{\mathcal{B}}
\newcommand{\cC}{\mathcal{C}}
\newcommand{\cD}{\mathcal{D}}
\newcommand{\cE}{\mathcal{E}}
\newcommand{\cF}{\mathcal{F}}
\newcommand{\cG}{\mathcal{G}}
\newcommand{\cH}{\mathcal{H}}
\newcommand{\cI}{\mathcal{I}}
\newcommand{\cJ}{\mathcal{J}}
\newcommand{\cK}{\mathcal{K}}
\newcommand{\cL}{\mathcal{L}}
\newcommand{\cM}{\mathcal{M}}
\newcommand{\cN}{\mathcal{N}}
\newcommand{\cO}{\mathcal{O}}
\newcommand{\cP}{\mathcal{P}}
\newcommand{\cQ}{\mathcal{Q}}
\newcommand{\cR}{\mathcal{R}}
\newcommand{\cS}{\mathcal{S}}
\newcommand{\cT}{\mathcal{T}}
\newcommand{\cU}{\mathcal{U}}
\newcommand{\cV}{\mathcal{V}}
\newcommand{\cW}{\mathcal{W}}
\newcommand{\cX}{\mathcal{X}}
\newcommand{\cY}{\mathcal{Y}}
\newcommand{\cZ}{\mathcal{Z}}
\DeclareMathOperator{\chains}{Chains}

%%%%%%%%%%%%% mathematical fraktur  %%%%%%%%%%%%%%%%%%%%%
\newcommand{\mf}[1]{\mathfrak{#1}}
\newcommand{\senk}{\ \big \vert \ }
\newcommand{\fA}{\mathfrak{A}}
\newcommand{\fB}{\mathfrak{B}}
\newcommand{\fC}{\mathfrak{C}}
\newcommand{\fD}{\mathfrak{D}}
\newcommand{\fE}{\mathfrak{E}}
\newcommand{\fF}{\mathfrak{F}}
\newcommand{\fG}{\mathfrak{G}}
\newcommand{\fH}{\mathfrak{H}}
\newcommand{\fI}{\mathfrak{I}}
\newcommand{\fJ}{\mathfrak{J}}
\newcommand{\fK}{\mathfrak{K}}
\newcommand{\fL}{\mathfrak{L}}
\newcommand{\fM}{\mathfrak{M}}
\newcommand{\fN}{\mathfrak{N}}
\newcommand{\fO}{\mathfrak{O}}
\newcommand{\fP}{\mathfrak{P}}
\newcommand{\fQ}{\mathfrak{Q}}
\newcommand{\fR}{\mathfrak{R}}
\newcommand{\fS}{\mathfrak{S}}
\newcommand{\fT}{\mathfrak{T}}
\newcommand{\fU}{\mathfrak{U}}
\newcommand{\fV}{\mathfrak{V}}
\newcommand{\fW}{\mathfrak{W}}
\newcommand{\fX}{\mathfrak{X}}
\newcommand{\fY}{\mathfrak{Y}}
\newcommand{\fZ}{\mathfrak{Z}}
\newcommand{\lp}{_\flat} %{\boldsymbol{\cdot}}
\newcommand{\hp}{^\sharp}
\newcommand{\cp}{\boldsymbol{\cdot}}

\newtheorem{theorem}{Theorem}[section]
\newtheorem{satz}[theorem]{Satz}
\newtheorem{lemma}[theorem]{Lemma}
\newtheorem{korollar}[theorem]{Korollar}
\newtheorem{example}[theorem]{Example}
\newtheorem{prop}[theorem]{Proposition}
\DeclareMathOperator{\Spec}{Spec}
\newtheorem{corollary}[theorem]{Corollary}
\theoremstyle{definition}

\newtheorem{definition}[theorem]{Definition}
\newtheorem{ziel}[theorem]{Ziel}
\newtheorem{frage}[theorem]{Frage}
\newtheorem*{notation}{Notation}
\newtheorem*{slogan}{Slogan}
\newtheorem*{construction}{Construction}
\newtheorem*{bemerkung}{Bemerkung}
\newtheorem*{exercise}{Exercise}

\newtheorem*{note*}{Note}

\newtheorem{rmk}{Remark}
\newtheorem{bsp}[theorem]{Beispiel}
\newtheorem{aufgabe}[theorem]{Aufgabe}
\newtheorem*{beweis}{\it Beweis}
\newcommand{\gray}[1]{{\color{gray} #1}}
%%%%%%%%%%    Math operators    %%%%%%%%%%%%%%%%%%%%%%%%%%%

\DeclareMathOperator{\id}{id}             % identity morphism
% \DeclareMathOperator{\ker}{ker}           % kernel
\DeclareMathOperator{\im}{im}             % image
\DeclareMathOperator{\Hom}{Hom}           % homomorphisms
\DeclareMathOperator{\End}{End}           % endomorphisms
\DeclareMathOperator{\Span}{Span}         % linear span

\usepackage{tikz-cd}
\DeclareMathOperator{\pr}{pr}
\usepackage{quiver}
\renewcommand{\:}{\colon}
\DeclareMathOperator{\isContr}{isContr}

\newcommand{\type}{\ \mathrm{Type}}
\usepackage{stmaryrd}



\newcommand{\op}{^{op}}
%\renewcommand{\subset}{\subseteq}
%\newcommand{\colim}[1]{\mathrm{colim} \limits_{#1}}
\newcommand{\colim}[1]{\underset{#1}{\mathrm{colim} \ }}
\DeclareMathOperator{\sSet}{\mathsf {sSet}}
\DeclareMathOperator{\Pos}{\mathsf {Pos}}
\DeclareMathOperator{\Set}{\mathsf {Set}}
\DeclareMathOperator{\Fun}{Fun}
\DeclareMathOperator{\Cat}{\mathsf {Cat}}
\DeclareMathOperator{\const}{const}
\DeclareMathOperator{\Vect}{\mathsf{Vect}}
\DeclareMathOperator{\Top}{\mathsf{Top}}
\DeclareMathOperator{\Ring}{\mathsf{Ring}}
\DeclareMathOperator{\Field}{\mathsf{Field}}

\DeclareMathOperator{\Ab}{\mathsf{Ab}}
\DeclareMathOperator{\GL}{GL}
\DeclareMathOperator{\Ch}{\mathsf{Ch}}
\DeclareMathOperator{\Grp}{\mathsf{Grp}}

\DeclareMathOperator{\HomC}{\Hom_{\cC}}
\DeclareMathOperator{\HomD}{\Hom_{\cD}}
\usepackage{ascii}
\DeclareMathOperator{\Ob}{Ob}
\DeclareMathOperator{\FinVect}{FinVect}
\setlength\parindent{0pt} % Keine Einrueckung von Absaetzen
\newcommand{\etale}{\' etale }
\newcommand{\Etale}{\' Etale }
\DeclareMathOperator{\fib}{fib}
\newcommand{\todo}{{\color{Red} Todo}}
\newcommand{\todocite}{[ref?]}
\newcommand{\el}{\in}
\usepackage{wasysym}
\newcommand{\ci}{\fullmoon}
\DeclareMathOperator{\isProp}{isProp}
\DeclareMathOperator{\Prop}{Prop}
%\renewcommand{\in}{\colon}

\newcommand{\details}{[...]}
\DeclareMathOperator{\tp}{tp}
\DeclareMathOperator{\Nat}{Nat}
\renewcommand{\contentsname}{Inhalt}
\font\maljapanese=dmjhira at 2.5ex
\newcommand{\yo}{\textrm{\!\maljapanese\char"48}}
\DeclareMathOperator{\Aut}{Aut}
\DeclareMathOperator{\Mod}{\mathsf{Mod}}
\DeclareMathOperator{\Mat}{Mat}

\DeclareMathOperator{\isInv}{isInv}
\DeclareMathOperator{\Alg}{Alg}
\newtheorem{axiom}{Axiom}
\newtheorem{question}{Question}
\renewcommand{\mid}{\ | \ }

\newcommand{\fun}[4]{
	\begin{align*} 
		#1 &\to #2 \\ 
		#3 &\mapsto #4 
\end{align*}}
\newcommand{\funn}[5]{
	\begin{align*} 
		#1 \colon #2 &\to #3 \\ 
		#4 &\mapsto #5
\end{align*}}

\newcommand{\RHom}{R \cH om}
\newcommand{\Ltimes}{\overset{\mathrm{L}}{\otimes}}
\renewcommand{\GS}{\mathsf{GS}}
\title{Thesis}
\author{Tim Lichtnau }
\date{May 2024}

\begin{document}
\newtheorem*{warning}{warning}
\maketitle

\section{Preparation}
\begin{lemma}{\label{lemma:havingAbstractAtlasClosedUnderId}}
	Let $C$ be a class of types stable under $\sum$. %HOPE WE DONT NEED (because we want to apply it to \cV = covering stacks) finite limits, i.e. containing 1, stable under dependent sums and finite limits
	The class $\mathsf{HasAtlas}_C$ of types $Y$ which admit a map $\Spec A \to Y$ fibered in $C$ is stable under identity types. %.  If it contains $C$ and is closed under dependent sums, then it is closed under taking identity types.
\end{lemma}
\begin{proof}
%	Obviously 1 has an atlas, and the class of types admitting an atlas is stable by $\sum$ by \ref{thm:atlasStableSum}.
%	It remains to show, that identity types in $Y$ have an atlas provided that $Y$ has an atlas.
	
	%This is a special case of stability under dependent sums. But lets prove it anyway.
	By assumption we can choose a map $p : V \to Y$ out of an affine fibered in $C$. Let $y,y' : Y$.  Then we have the map
	\begin{align*}
		(\fib_p y) \times_V (\fib_p y') &\to y = y' \\
		(v , q : y = p v) , (v', q' : y' = p v') , (h : v = v') &\mapsto q \cdot h \cdot q'^{-1}
	\end{align*}
	
	The fiber over $j : y = y'$ looks like  %because $y$ and $y'$ are free we may only show the statement for the fiber over the path  $\mathsf{refl}_y : y = y$. 
	\[
	\sum_{v}  ( \underbrace{\sum_{v'} (h : v = v')}_{\mathsf{isContr}}) \times (q : y = p v) \times (q'  : y' = p v') \times (q \cdot h \cdot q'^{-1} = j) \simeq \sum_v (v = py) \simeq \fib_p y
	\]
	Hence the map is fibered in $C$. It suffices to show, that	$(\fib_p y) \times_V (\fib_p y')$ has an atlas, because then we can compose such an atlas with the above map to obtain an atlas of $y = y'$.
	By assumption the fibers of $p$ have an atlas, so we can choose $q : W \to \fib_p y, q' : W' \to \fib_p y'$ atlasses. Then $W \times_V W' \to (\fib_p y) \times_V (\fib_p y')$ is an atlas: The domain is a fiber product of affines, hence affine. The fiber over $(x,x')$ is equivalent to the product of fibers $(\fib_q x) \times (\fib_{q'} x')$ which is in $C$ by stability under dependent sums (so in particular under finite products).
	
\end{proof}



\section{Covering stacks}
Fix $\bT$ a topology, which we call the covering-affines.
\begin{definition}
	Covering geometric stacks are the smallest intermediate class containing $\bT$ such that: If $Y$ is a sheaf and $\bT \ni S \to Y$ is fibered in covering geometric stacks, then $Y$ is a covering geometric stack.	
\end{definition}


We call such map $X \to Y$ whose fibers are covering $\cV$-stacks a $\cV$-cover. If $X$ is affine we call it an $\cV$-atlas. If $X$ is in $\bT$ we call it a $\cV$-catlas. In Case of $\cV = \cU_\bT$ the sheaves we call it a geometric cover / geometric atlas / geometric catlas.
\begin{prop}[Recursion principle for covering stacks]
 Let $P : \cV \to \Prop$ be a property of types in $\cV$. Assume
 \begin{itemize}
 	\item Every covering affine has $P$
 	\item If $\bT \ni S \to Y$ is fibered in $P$ then $Y$ has $P$
 \end{itemize}
Then every covering $\cV$-stack has $P$.
\end{prop}
\begin{proof}
		Replace $P$ by $P \land \mathsf{is-covering-stack}$. Then usual induction

\end{proof}
\begin{definition}
	We call $X$ a geometric stack if it merely has a geometric atlas, i.e some $\Spec A \to X$ fibered in covering geometric stacks.
\end{definition}

\begin{lemma}
	The class of (covering) geometric stacks is $\sum$-stable.
\end{lemma}
\begin{proof}
	Define the predicate $P X$ as ' the sum of every family $B$ of (covering) geometric $\cV$-stacks is a (covering) geometric stack
	If $X$ is a covering affine, by choice of $X$ we can choose $\cV$-catlasses $S_x \to B x$ for all $x : X$. Then $\sum_{x :X } S_x \to \sum_x B x$ is a $\cV$-catlas. \\
	If $f : S \to X$ is a map fibered in $P$ with $S \in T$, then let $B : X \to \CS_\cV$. By choice of $S$ we can choose $\cV$-catlasses $\tilde B s \to B (f s)$ for all $s : S$. Then consider $\sum_{s : S} \tilde B s \to \sum_{x : X} B x$. Its domain is in $\bT$. It remains to show, that the fiber over $(x,t)$ is a covering stack. It is a dependent sum over $\fib_f x$, which by induction satisfies $P$ that lets us conclude by definition of $P$.
\end{proof}
\begin{lemma}{\label{lemma:coversstableundercomp}}
	$\cV$-covers are stable under composition.
\end{lemma}
\begin{proof}
	covering $\cV$-stacks are stable under $\sum$.
\end{proof}
TODO same prop for geometric stack as well?
\begin{prop}{\label{prop:csHasAtlas}}
	Every covering $\cV$-stack $X$ merely admits a $\cV$-catlas, i.e. a $\cV$-cover $Y \to X$ with $Y \in \bT$. 
\end{prop}

\begin{proof}
We apply the recursion principle of covering stacks 
	\begin{itemize}
		\item If 	$X$ is covering affine, then $X \to X$ is a $\cV$-catlas with covering domain. \\
\item If $X$ is obtained as a quotient then it already is equipped with a $\cV$-atlas. %, i.e. if its equipped with a cover $Y \to X$ with $Y$ a covering stack, then by induction $Y$ admits a $\cV$-catlas $S \to Y$. Then $S \to Y \to X$ is a $\cV$-catlas by  \ref{lemma:coversstableundercomp}. \\
%\item If $X$ is obtained as a sum, i.e. we have a $\cV$-cover $f : X \to Y$, then by induction $Y$ admits an $\cV$-catlas $g : S \to Y$ and the fibers merely have $\cV$-catlasses $S_y \to \fib_f y$ s. By choice of $S$, we can choose such catlasses $S_{g s} \to \fib_f (g s)$ for all $s : S$. By \ref{lemma:AtlasSum} the map 
%\[
%\sum_{s : S} S_{gs} \to (\sum_{y: Y} \fib_f y ) \simeq X
%\]
%is a $\cV$-cover. Its domain is a covering affine as $\bT$ is $\sum$-stable. Hence $X$ admits a $\cV$-catlas .

	\end{itemize}
\end{proof}
\begin{prop}
	The class of covering $\cV$-stacks is stable under quotients: If $X \to Y$ is fibered in covering $\cV$-stacks and $X$ is a covering $\cV$-stack and $Y \in \cV$, then $Y$ is a covering $\cV$-stack.
\end{prop}
\begin{proof}
	Choose an $\cV$-catlas of $X$. Then the composition with the map $X \to Y$ is a $\cV$-cover by \ref{lemma:coversstableundercomp}. Surely its a $\cV$-catlas.
\end{proof}
Now we want to show that the clash of terminology regarding 'covering' is reasonable:


\begin{prop}{\label{prop:affineCoveringStack}}
	Let $\bT$ be saturated.
	A covering stack $X$ is affine iff its a covering affine.
\end{prop}
\begin{proof}
	The converse is clear. The direct direction follows by the recursion principle. choosing a $\cV$-catlas  $S \to X$. As both $S$ and $X$ are affine the fibers are affine. By induction the fibers are covering affines. By saturatedness of the topology $X$ is covering affine.
\end{proof}
\begin{lemma}{\label{lemma:atlasIsCatlas}}
		Let $\bT$ be saturated. Let $X$ be a covering $\cV$-stack. Let $f : \Spec A \to X$ be a $\cV$-atlas. Then $\Spec A \in \bT$
\end{lemma}
\begin{proof}
	As $\Spec A \simeq \sum_{x : X} \fib_f x$ is a dependent sum of covering $\cV$-stacks, it is a covering $\cV$-stack again. We conclude by \ref{prop:affineCoveringStack}.
\end{proof}	

\subsection{Geometric stacks}

\begin{lemma}{\label{lemma:geometricStacksClosedUnderId}}
	geometric $\cV$-stacks are closed under $\id$-types.
\end{lemma}
\begin{proof}
	
	This is \ref{lemma:havingAbstractAtlasClosedUnderId}, using that covering stacks are closed under $\sum$.
\end{proof}

\begin{warning}
	The previous lemma does not hold for covering stacks: Identity types of things in $\bT$ could be empty.
\end{warning}
\begin{prop}[Recursion principle for geometric stacks]
	Let $P : \GS \to \Prop$ be a property of types. Assume
	\begin{itemize}
		\item Every affine has $P$
		\item If $S \to Y$ is fibered in covering stacks that have $P$ then $Y$ has $P$
	\end{itemize}
	Then every $\cV$-stack has $P$.
\end{prop}
\begin{proof}
	One could explain geometric stacks as the smallest class containing all affines and if $\Spec A \to X$ is fibered in geometric stacks that happens to be covering, then $X$ is a geometric stack.
\end{proof}
\subsection{About the smallest class in a subuniverse}
\begin{definition}
	Let $\cV \supset \bT$ be a superclass stable under $\sum$. covering stacks are the smallest intermediate class $\bT \subset \CS_\cV \subset \cV$ such that: If $X : \bT$  $Y : \cV$ and $X \to Y$ is fibered in $\CS_\cV$, then $Y \in \CS_\cV$
\end{definition}
\begin{example}
	covering Aff-stacks $=$ saturation of $\bT$. Indeed: By definition, the saturation of $\bT$ is is obtained by quotients of $\bT$ by $\bT$-covers. We have shown, that its closed under covers between affines.
\end{example}



\begin{prop}{\label{prop:coveringVstackDescr}}
	Let $\cV$ be stable under finite limits and containing (covering) affines. $X$ is a (covering) $\cV$-stack iff it is in $\cV$ and a (covering) geometric stack.
\end{prop}
\begin{proof} 	
	The direct direction is clear. For the converse we apply the recursion principle to the property '$X \in \cV$ implies $X$ is a (covering) $\cV$-stack'. If $X \in \bT$, its clear. Otherwise its equipped with a geometric (c)atlas. The fibers are in $\cV$, as they can be written as a fiberproduct of $S, X, 1 \in \cV$.  By induction all fibers are covering $\cV$-stacks. %We are left to show that $F$ is a covering $\cV$-stack. \\
%	We can choose $S \to F$ a $\cV$-atlas, so in particular a geometric atlas of $F$, which was assumed to be a covering geometric stack. Then $S \in \bT$ by \ref{lemma:atlasIsCatlas}. So we actually have a $\cV$-catlas.
\end{proof}

\begin{prop}{\label{prop:V'stacks}}
	A sheaf $X$ merely admits some affine $\Spec A \to X$ fibered in covering $\cV$-stacks, iff its a geometric stack whose identity types are $\cV$-stacks. \\	
\end{prop}
\begin{proof}
	The direct direction: By \ref{lemma:havingAbstractAtlasClosedUnderId} the identity types are $\cV$-stacks. \\
	The converse direction: Choose a $\cV$-atlas $S \to X$. As each fiber is in $V$ and is a covering $\cV$-stack, its a covering $\cV$-stack by \ref{prop:coveringVstackDescr}.
\end{proof}
\begin{definition}
	Let $n \ge -2$. A (covering) geometric $n$-stack is a (covering) geometric stack that is an $n$-type.
\end{definition}
\begin{prop}
	Let $X$ be a sheaf. For all $n \ge 0$, the following are equivalent:
	\begin{enumerate}
		\item $X$ is a (covering) geometric $n+1$-stack
		\item $X$ merely admits some map $S \to X$ out of a (covering) affine fibered in covering $n$-stacks
		\item $X$ merely admits some (covering) geometric $n$-stack $Y \to X$ fibered in covering $n$-stacks.
	\end{enumerate}
\end{prop}
\begin{proof}
	\
	\begin{enumerate}
	\item[1 . $\Leftrightarrow$ 2.]
	$X$ is a (covering) geometric $n+1$ stack iff ( \ref{lemma:geometricStacksClosedUnderId}) its a (covering) geometric stack whose identity types are geometric $n$-stack iff (\ref{prop:V'stacks}) 2. 
	\item[2 . $\Rightarrow$ 3.]
	$S$ is a (covering) geometric $n$-stack
	\item [3. $\Rightarrow$ 2]
	$Y$ admits a map $S \to Y$  fibered in covering $n$-stacks with $S$ (covering) affine, so the composition $S \to X$ will have the same property by \ref{lemma:coversstableundercomp}.
		\end{enumerate}
\end{proof}
\subsection{Truncatedness}
In this subsection we want to prove
\begin{theorem}
	Every geometric stack is $n$-truncated for some $n : \bN$.
\end{theorem}


\begin{lemma}{\label{lemma:cstinh}}
	Every covering $\cV$-stack $X$ is $\bT$-merely inhabited.

\end{lemma}
\begin{proof} 
	\begin{itemize}
		\item If $X$ is in $\bT$ then its clear.
		\item  If $X$ is obtained by a quotient, we have a map $\Spec A \to X$ with domain in $\bT$. Now use that we get a map on $\bT$-propositional-truncations and that Spec A is T-merely inhabited.
%		\item if $X$ is obtained by  $X = \sum_{y: Y} B y$ for $Y$ a covering $\cV$-stack and $B y$ covering $\cV$-stacks, by induction all the $B y$ are $\bT$-merely inhabited. Hence, for all $y : Y$, we can conclude $\| X\|_\bT$. As $Y$ is $\bT$-merely inhabited by induction and the goal is a sheaf, we can conclude. 
		\end{itemize}
\end{proof}
\begin{lemma}{\label{lemma:truncTrg}}
	Let $X$ be an $n+1$-type and $Y$ a sheaf. If $X \to Y$ is a $n$-truncated $\bT$-surjective map, then $Y$ is an $n+1$-type.
\end{lemma}
\begin{proof}
	Use that $\mathsf{is-n-truncated} (y=y')$ is a sheaf for $y , y' : Y.$
\end{proof}

\begin{proof} of the theorem.
	We apply the recursion principle for geometric stacks.
\begin{itemize}
	\item If $Y$ is affine its clear with $n = 0$.
	\item Assume $Y$ is equipped with a $\cV$-atlas $f : S \to Y$, such that every fiber in $n$-truncated for some $n$. f is $\bT$-surjetive by \ref{lemma:cstinh}.
	We apply \ref{lemma:truncTrg}.
	So it remains to find an $n$ such that all fibers are $n$-truncated.
	For any $x : S$, By induction $\fib_f (f x)$ is $n$-truncated for some $n$. By projectivity of $S$, we find some $n$ such that $\fib_f (f x)$ is $n$-truncated for all $x : S$. For general $y : Y$, using that is-$n$-truncated $\fib_f y$ is a sheaf , we can conclude by $\bT$-surjectivity of $f$.
%	\item Let $X$ be an $n$-truncated covering geometric stack. By \ref{prop:csHasAtlas} we find a geometric catlas $S \to X$. All the fibers are (at most) $n$-truncated. 
\end{itemize}

%

\end{proof}

\subsection{Descent}
For this subsection lets assume $\cV$ a subuniverse (stable under $\sum$), that satisfies: If $Y \in \cV$, then $L_\bT Y \in \cV$.  $\St$ a class of sheaves, such that $\bT$ is contained in it and for any map $X \to Y$ of sheaves in $\cV$ fibered in $\bT$, $X \in \St$ iff $Y \in \St$. We call types in this class stacky.
\begin{lemma}{\label{lemma:sheafificationHasTCover}}
	Let $\bT$ satisfy descent, i.e. beeing affine in the topology is a sheaf. If $Y$ admits a $\bT$-cover $f : X \to Y$ where $Y \in \cV$ is seperated, then there is a $\bT$-cover $X \to \bullet Y$.
\end{lemma}
\begin{proof}
	
	Consider $X \overset{f}{\to} Y \overset{\eta}{\to} \bullet Y$. As beeing affine in $\bT$ is  a sheaf, we may just show that for all $y : Y$ , the fibers over $\eta y : \bullet Y$ are in $\bT$. As $\eta$ is a monomorphism by \ref{lemma:sep} , $\eta$ restricts to an equivalence
	\[
	\fib_f y \to \fib_{ \eta f}(\eta y)
	\]
	
	But the left hand side is in $\bT$ by assumption. 
\end{proof}
\begin{lemma}
 Assume $\bT$ have descent.
Let $X \in \St$ and $Y \in \cV$.	Let $f : X \twoheadrightarrow Y$ be fibered in $\bT$ and surjective. Then $\bullet Y$ is stacky.
\end{lemma}
\begin{proof}
As $X$ is stacky, it suffices to show, that $\bullet Y$ admits a $\bT$-cover.
We want to apply \ref{lemma:sheafificationHasTCover}. So it remains to show, that $Y$ is seperated. By surjectivity of $f$ we may only show that for any $x : X, y : Y$, the type $f x =_Y y$ is a sheaf. If we define $U$ to be the fiber over $y$, it is in $\bT$ by assumption. But then $f x =_Y y$ is the outer pullback
% https://q.uiver.app/#q=WzAsNixbMCwwLCJmIHggPSB5Il0sWzEsMCwiVSBcXGluXFxiVCJdLFswLDEsIjEiXSxbMSwxLCJYIl0sWzIsMCwiMSJdLFsyLDEsIlkiXSxbMyw1LCJmIl0sWzQsNSwieSIsMl0sWzIsMywieCJdLFsxLDNdLFsxLDRdLFsxLDUsIiIsMSx7InN0eWxlIjp7Im5hbWUiOiJjb3JuZXItaW52ZXJzZSJ9fV0sWzAsMl0sWzAsMV1d
\[\begin{tikzcd}
	{f x = y} & {U \in\bT} & 1 \\
	\arrow["\ulcorner"{anchor=center, pos=0.125}, draw=none, from=1-1, to=2-2]
	1 & X & Y
	\arrow[from=1-1, to=1-2]
	\arrow[from=1-1, to=2-1]
	\arrow[from=1-2, to=1-3]
	\arrow[from=1-2, to=2-2]
	\arrow["\ulcorner"{anchor=center, pos=0.125}, draw=none, from=1-2, to=2-3]
	\arrow["y"', from=1-3, to=2-3]
	\arrow["x", from=2-1, to=2-2]
	\arrow["f", from=2-2, to=2-3]
\end{tikzcd}\]
of stacky types, in particular sheaves. \qed(Claim) \\\\

\end{proof}
\begin{theorem}
	Assume $\bT$ have descent. Then $\St$ is a sheaf.
\end{theorem}
\begin{proof}
	$\St$ is seperated: This follows from the embedding $\St$ into the seperated (TODO) type of sheaves. \\
	Let $U \in \bT$ and $P : \|U\| \to \St$. We want to construct a filler 
	% https://q.uiver.app/#q=WzAsMyxbMCwwLCJcXHwgVVxcfCJdLFswLDEsIjEiXSxbMSwwLCJcXFNtU3QiXSxbMCwyLCJQIl0sWzAsMV0sWzEsMiwiIiwyLHsic3R5bGUiOnsiYm9keSI6eyJuYW1lIjoiZGFzaGVkIn19fV1d
	\[\begin{tikzcd}
		{\| U\|} & \St \\
		1
		\arrow["P", from=1-1, to=1-2]
		\arrow[from=1-1, to=2-1]
		\arrow[dashed, from=2-1, to=1-2]
	\end{tikzcd}\]
	Claim: $\bullet (\sum_{x: \|U\|} P x)$ is stacky.
	\begin{proof} of the claim. We want to apply the previous lemma to the map 
		\[\sum_{x : U} P | x | \to \sum_{x : \| U\|} P x \]
		The domain is in $\St$ by stability under $\sum$. The fibers are equivalent to $U \in \bT \subset \St$.				
	\end{proof}
    The claim provides the map $1 \to \St$. The diagram commutes: Assuming $x : \|\Spec A\|$ we wish to show $P x = \sum_{x: \|U\|} P x$. Using univalence, we may show that the maps 
	\[P x \to \sum_{x: \|U\|} P x \overset{\eta}{\to} \bullet \sum_{x: \|U\|} P x\]
	are both equivalences.
	The first one is an equivalence as $\|U\|$ is contractible. Hence the middle term is a sheaf, thus the unit map is an equivalence as well. \\
	
	
	
\end{proof}
\begin{corollary}
	(covering) geometric stacks satisfy descent.
\end{corollary}
\begin{lemma}[TODO]
	If $Y$ is an $n$ type , then $L_\bT Y$ is an $n$-type.
\end{lemma}
\begin{corollary}
	For all $n : \bN $, the class of (covering) ($n$-)stacks has descent.
\end{corollary}
\begin{proof}
 We set $\cV$ as the $n$-truncated-types which is fine by the lemma.: %If $n-stack \ni X \to Y$ is fibered in $\bT$ where $Y$ is an $n$-type. Then its sheafification is an $n$-type as well by the lemma.
\end{proof}



\section{Saturated Topologies}
%Consider a topology $\bT$ finer than the Zariski topology.
\begin{definition}
	Consider the partial order
	\[
	\Top = \{\bT : \Prop^\Aff \ | \ 1 \in \bT \land \bT \sum-stable \}
	\]
	ordered by inclusion.
	An inflation $P$ on $\Top$ is a monotone endofunction such that $X \subset P X$. 
	$P$ is stack-preserving if for any topology $\bT$, $P \bT \subset \bT$-merely inhabited types.\\
	it is covering-stack-preserving if for any $X : P \bT$, $X$ is a $\bT$-covering stack.
	%   that preserves $\sum$-stability and satisfies that 
\end{definition}
Note that covering-stack-preserving implies stack-preserving, as $\bT$-covering stacks are $\bT$-merely inhabited.
\begin{prop}{\label{prop:TopologyMonad}}
	Given a stack-preserving inflation $P$. Then for any topology $\bT$, A Type $Y$ is a stack wrt to $P \bT$ iff it is a stack wrt to $\bT$. \\
	If $P$ is idempotent, then the class $P \bT$ is the smallest $P$-fixpoint topology containing $\bT$. \\
	If $P$ is covering-stack preserving, $\bT$ and $P \bT$ will induce the same covering stacks.
\end{prop}

\begin{proof}
	$\bT \subset P \bT$ by inflationarity. 	Regarding Stacks: As $\bT \subset \bT'$ the $\rightarrow$ direction is clear. Now, let $X \in \bT'$. We have
	% https://q.uiver.app/#q=WzAsMyxbMCwwLCJcXHxYXFx8Il0sWzEsMCwiVCJdLFswLDEsIlxcfFhcXHxfXFxiVCBcXHNpbWVxIDEiXSxbMCwxLCJcXGZvcmFsbCJdLFswLDJdLFsyLDEsIlxcZXhpc3RzISIsMl1d
	\[\begin{tikzcd}
		{\|X\|} & Y \\
		{\|X\|_\bT}
		\arrow["\forall", from=1-1, to=1-2]
		\arrow[from=1-1, to=2-1]
		\arrow["{\exists!}"', dashed, from=2-1, to=1-2]
	\end{tikzcd}\]
	by the stack-preserving-property $\|X\|_\bT  \simeq 1$. Hence $T$ is $\|X\|$-local	
	If $P$ is idempotent, every other fixpoint $X$ containg $\bT$ satisfies $P T \subset P X = X$ by monotonicity. \\
	If $P$ is covering-stack-preserving, notice that every $\bT$-covering stack is also a $P \bT$-covering stack as $\bT \subset P \bT$. For the converse we use the recursion principle: For $X$ a $P \bT$-covering stack, consider the predicate 'is $P \bT$-covering'. 1 has it. If $P \bT \ni \Spec A \to X$ is a $\bT$-geometric atlas, i.e. whose fibers are $\bT$-covering stacks, as $\Spec A$ is a $\bT$-covering stack by the covering-stack-preservation, by quotient stability of $\bT$-covering stacks $X$ is a $\bT$-covering stack as well
\end{proof}

\begin{definition}
	A catlas of $X$ is  some $\hat X \in \bT , \hat X \to X \text{ $\bT$-cover }$
\end{definition}
\begin{prop}
	The assignment
	\begin{align*}
		\Top &\to \Top \\
		\bT &\mapsto \bT' \equiv \{X \in \Aff \ | \  \exists \text{ catlas of } X \}
	\end{align*}
	%i.e. the affine covering $0$-stacks.	
	covering-stack-preserving idempotent Monad, called the saturation monad. \\
	$\bT'$ is the class of covering $\Aff$-stacks.
\end{prop}
\begin{proof}
	\begin{itemize}
		\item 	$\bT'$ is $\sum$-stable by \ref{thm:atlasStableSum}. \\
		\item $\bT \subset \bT'$ is clear.
		\item Monotonicity clear
		\item Idempotentency:  consider some $\bT'$-cover $\bT' \ni X' \to X$. By replacing $X'$ with some smooth atlas, we may assume that $X' \in \bT$. As every fiber $X'_x \in \bT'$, we merely find a smooth atlas $\tilde X'_x \to X'_x$. Then by Zariski local choice there exists a Zariski atlas $\hat X \to X$ and a commutative diagram 
		% https://q.uiver.app/#q=WzAsNCxbMCwwLCJcXHN1bV97eCA6IFxcaGF0IFh9XFx0aWxkZSBYJ194Il0sWzAsMSwiXFxoYXQgWCJdLFsxLDEsIlgiXSxbMSwwLCJcXHN1bV97eCA6IFh9WCdfeCJdLFszLDJdLFsxLDIsIlphciIsMl0sWzAsMV0sWzAsM11d
		\[\begin{tikzcd}
			Y \equiv {\sum_{x : \hat X}\tilde X'_x} & {\sum_{x : X}X'_x} = X' \\
			{\hat X} & X
			\arrow[from=1-1, to=1-2]
			\arrow[from=1-1, to=2-1]
			\arrow[from=1-2, to=2-2]
			\arrow["Zar"', from=2-1, to=2-2]
		\end{tikzcd}\]
		As $X' \in \bT$ and $Y \to X'$ is fibered in $\bT$ (\ref{lemma:AtlasSum}) we have $Y \in \bT$. But $Y \to \hat X$ is a $\bT$-cover and $\hat X \to X$ is a $\bT$-cover, $Y \to X$ is a $\bT$-cover. Hence $X \in \bT'$.
		
		\item covering-stack-preserving: For any $\Spec A : \bT'$ we merely have some $\bT$-catlas $\bT \ni X \to \Spec A$, witnessing that $\Spec A$ is a covering stack.
	\end{itemize}
	For the last claim, just observe that $\bT'$ is definitely contained in covering $\Aff$-stacks.
	%But $Y \to X$ is a $\bT$-cover and $\hat X \to X$ is a $\bT$-cover, $Y \to X$ is a $\bT$-cover. Hence $X \in \bT'$.
	%covering $n$-stacks are stable under dependent sums \ref{thM:stabSums}
	%Obviously $1 \in \bT'$. We say a type $X$ has covering local choice if for all $\bT$-surjections $S \to S'$ and a map $X \to S'$ there exists a catlas of $X$ lifting to $S$. 
	% %First observe that every $X \in \bT$ has covering local choice:
	% Now any $X \in \bT'$ satisfies local choice wrt catlasses because it has a catlas and catlasses are stable under composition and Zariski covers are in $\bT$. 
	% Hence ...
	%		As $\bT'$ is definitely contained in the saturation, it suffices to show, that the class $\bT'$ defined above is saturated.
\end{proof}
\begin{lemma}{\label{lemma:flatDescendsAlongFppf}}
	if $\Spec B \to \Spec A$ is faithfully flat and $\Spec B$ is flat, then $\Spec A$ is flat.
\end{lemma}
\begin{proof}
	Consider an injection of $R$-modules $M \hookrightarrow N$. We wish to show, that $A \otimes_R M \to A \otimes_R N$ is injective. As $B$ is faithfully flat over $A$ it suffices to show, that $B \otimes_R M \cong B \otimes_A A \otimes_R N \to B \otimes_A A \otimes_R N = B \otimes_R N$ is injective. This follows as $B$ is flat over $R$.
\end{proof}
\begin{example}
	The fppf-Topology is saturated.
\end{example}
\begin{proof}
	Given a faithfully flat algebra homomorphism $A \to B$ with $B$ faithfully flat, we want to show, that $A$ is faithfully flat. First observe, that $A$ is flat by the previous lemma. Then if $M \otimes_R A = 0$ for some $R$-module $M$, then $M \otimes_R B = M \otimes_R A \otimes_A B = 0$. As $B$ is faithfully flat over $R$, we conclude $M = 0$.
\end{proof}

\begin{example}
	The unramified-topology (unramified + fppf) is saturated.
\end{example}
\begin{proof}
	Let $\Spec B \to \Spec A$ be unramified + fppf and $\Spec B$ unramified + fppf. We have to show that $\Spec A$ is unramified (fppf is the above example). For this, we may show that identity types $x = y$ are $\lnot \lnot$-stable. So assume $\lnot \lnot (x = y).$\\
	As $\Spec A$ admits a faithfully flat map with flat affine domain, the identity type $x = y$ admits such a map $\Spec B' \to x = y$ as well. As its fibers are $\lnot \lnot$-inhabited, we can conclude that the flat $\Spec B'$ is $\lnot \lnot $-inhabited, hence fppf. But now $x = y$ is a fppf-covering -1-stack, hence contractible \ref{lemma:covM1Stacks}.
\end{proof}

\begin{lemma}
	The \etale topology is saturated
\end{lemma}
\begin{proof}
	fppf is clear by saturatedness of the fppf topology. %${\bP_{et}}$-cover (i.e. an \etale faithfully flat map)
	Conclude By \ref{lemma:FEtLocal}
\end{proof}
%
%\begin{lemma}
%
%\end{lemma}
%\begin{proof}
%
%%We have to show that $T \to T^{\|X\|}$ is an equivalence. Choose $\bT \ni Y \to X$. Then we have a commutative diagram
%%	% https://q.uiver.app/#q=WzAsMyxbMCwwLCJUIl0sWzEsMCwiVF57XFx8WFxcfH0iXSxbMSwxLCJUXntcXHxZXFx8fSJdLFswLDFdLFsxLDJdLFswLDIsIlxcc2ltZXEiLDJdXQ==
%%	\[\begin{tikzcd}
%%		T & {T^{\|X\|}} \\
%%		& {T^{\|Y\|}}
%%		\arrow[from=1-1, to=1-2]
%%		\arrow["\simeq"', from=1-1, to=2-2]
%%		\arrow[from=1-2, to=2-2]
%%	\end{tikzcd}\]
%%	So $T \to T^{\|X\|}$ has a left-inverse. Thus it suffices to show that any $f : T^{\|X\|}$ has a preimage. Choose $t : T$, s.th. $\mathrm{cnst}^Y_t$ is the composite $\|Y\| \to \|X\| \overset{f}{\to} T$. We have $\|Y\| \to (\mathrm{cnst}^X_t = f)$. But as $Y \in \bT$ and $\Delta_t = f$ is a stack (as an identitytype in the stack $T^{\|X\|}$) we are done.
%\end{proof}
%\begin{rmk}
%	We never used that we only talk about $\bT$-covers.
%\end{rmk}
%
%\begin{question}
%	Does the converse hold, i.e. is every $\bT$-merely inhabited affine saturated?    
%\end{question}

% \begin{lemma}
%     A type $T$ is a saturated $\bT$-stack if for all $X \in \bT$ the diagonal
%     \[
%     T \to T^{L_\bT \|X \|}
%     \]
%     is an equivalence.
% \end{lemma}
% \begin{proof}
%     If $X \in \bT'$, we can choose a catlas $\hat X \to X$. As it is in particular $\bT$-surjective we have $\|X\| \overset{\sim}{\to} L_T \|\hat X\| $ which gives us that the composite
%     \[
%     T \overset{\simeq}{\to} T^{L_\bT \|X'\|}  \overset{\simeq}{\to}  T^{\|X\|}
%     \]
%     is an equivalence. 
% \end{proof}


\section{(Lex) Modalities}
\begin{lemma}[Stability resuls]{\label{lemma:ModalityStability}}
	Modalities are stable under 
	\begin{enumerate}
		\item Conjunction
		\item Composition
	\end{enumerate}
	
\end{lemma}
\begin{lemma}{\label{lemma:ModalitySumStable}}
	Let $\ci$ be a modality. Let $X$ be $\ci$-modal and $B : X \to \cU_{\ci}$ be a family of modal types. Then $\sum_{x : X} B_x$ is $\ci$-modal
\end{lemma}
\begin{lemma}{\label{lemma:mod_comm_sum}}
	Let $B  : \bullet X \to \cU$. Then $\bullet (\sum_{x : X} B (\eta x)) = \sum_{x : \bullet X} \bullet B x$
\end{lemma}
\begin{proof}
	Observe that 
	\[
	\sum_{x : X} B \eta x \to \sum_{x : \bullet X} B x
	\]
	is a $\bullet$-equivalence, because for all modal types $T$, the type $B x \to T$ is modal for any $x : \bullet X$. \\
	Then it follows by \todocite.
\end{proof}
\begin{lemma}{\label{lemma:idTypesOfSheafification}}
	Let $\bullet$ be a lex modality. Let $x , y : X$. The map
	\[
	\bullet(x = y) \to \eta x =_{\bullet X} \eta y
	\]
	induced by $ap_\eta : x = y \to \eta x =_{\bullet X} \eta y$ is an equivalence
\end{lemma}
\begin{proof}
 	By Modalities Theorem 3.1 [ix].
\end{proof}
\begin{definition}{\label{lemma:sep}}
	Let $\bullet$ be a lex modality. we call a type $X$ $\bullet$-seperated if one of the following equivalent conditions hold
	\begin{itemize}
		\item the identity types of $X$ are modal
		\item the unit $X \to \bullet X$ is an embedding
	\end{itemize}
In this case
\end{definition}
\begin{proof}

	by \ref{lemma:idTypesOfSheafification} the vertical map in the commutative diagram
	% https://q.uiver.app/#q=WzAsMyxbMCwwLCJ4ID1fWCB5Il0sWzEsMCwiTCh4PXkpIl0sWzEsMSwiXFxldGEgeCA9X3tMWH0gXFxldGEgeSJdLFswLDIsImFwX3tcXGV0YV9YfSIsMl0sWzEsMiwiXFxzaW1lcSJdLFswLDEsIlxcZXRhX3t4PXl9Il1d
	\[\begin{tikzcd}
		{x =_X y} & {L(x=y)} \\
		& {\eta x =_{LX} \eta y}
		\arrow["{\eta_{x=y}}", from=1-1, to=1-2]
		\arrow["{ap_{\eta_X}}"', from=1-1, to=2-2]
		\arrow["\simeq", from=1-2, to=2-2]
	\end{tikzcd}\]
is an equivalence.
	So $x = y$ is a sheaf if $\eta_{x=y}$ is an equivalence iff $\eta_X$ is an embedding.
\end{proof}
\begin{lemma}{\label{lemma:SheavesHaveDescent}}
	If $\bullet$ is a lex modality, then $\bullet U$ is modal.
\end{lemma}
\section{Atlas}
\begin{definition}{\label{def:TAtlas}}
	% Let $T \subset \cU$ be any subtype of the universe. 
	% A $\bT$-cover 
	Given $\cV \subset \cU$ a subclass stable under $\sum$, a $\cV$-cover is a map fibered in $\cV$.
	A $\cV$-atlas of $X$ is a $\bT$-cover $\Spec A \to X$ out of an affine scheme. \\
	In the context of a topology $\bT$, We call a $\cV$-atlas $\Spec A \to X$ a $\cV$-catlas, if the domain $\Spec A$ belongs to $\bT$.
	% of a type $\bT$ is an affine Scheme $\Spec A$ with a formally \etale \etale-surjective map
	% \[
	% \Spec A \to T.
	% \]    
\end{definition}

% \begin{lemma}
%     $\bT$-atlasses are stable under composition.
% \end{lemma}
%\begin{rmk}{\label{rmk:defatlas}}
%	Any good enough TODO scheme has a Zariski atlas. If $\bT$ is finer than the Zariski-topology then in the definition we may replace affine scheme by good enough scheme, if its just about the question whether a type admits an atlas.
%\end{rmk}

\begin{example}
	Let $X$ be a (1-)type. $X$ has a $\Zar$-atlas, iff there exists some $f : \Spec A \to X$ fibered in types of the form $\Spec (R_{f_1} \times \hdots \times R_{f_n})$ for $(f_1,\hdots,f_n) \in Um(R)$. 
\end{example}
\begin{rmk}{\label{ZLCGivesZariski}}
	If one applies ZLC to an affine scheme $\Spec A$ the resulting principal open cover $D(f_i), f_i \in A$ will induce indeed a zariski atlas $\bigsqcup D(f_i) \to \Spec A$, because the fiber over $x : \Spec A$ is $\bigsqcup D(f_i(x))$.
\end{rmk}
Question: Does every zariski atlas of $\Spec A$ have this form? \nameref{ex:weirdZarAtlasses}
%Let $\cZ$ be the class of types which have a Zariski cover.

\begin{example}{\label{ex:PnIsStack}}
	$\bP^n$ has a zariski atlas given by the standart homogeneous principal opens $\sum_{i=0}^n D_+(x_i)$. The fiber over a point $[y_0 : \hdots . y_n]$ is $D(y_0) + \hdots D(y_n)$ where $(y_1,\hdots,y_n) \in Um(R)$. % Indeed the standart principal opens $D_+(x_0) , \hdots , D_+(x_n)$ form a presentable open cover.    
\end{example}

\begin{definition}
	A Zariski sheaf $X$ is a scheme if there merely exists some affine $S$  map $S \to X$ whose fibers are Zariski-merely inhabited finite sums of open propositions 
\end{definition}
\begin{lemma}{\label{lemma:IsScheme}}
	Every $\Zar$-sheaf that admits a $\Zar$-atlas is a scheme. 
\end{lemma}
\begin{proof}
	Obvious.
\end{proof}
% Warning: Let $X$ be a . Then $X$ is already affine if it has a Zariski cover, i.e.  there exists some $f : \Spec A \to X$ fibered in types of the form $\Spec (R_{f_1} \times \hdots \times R_{f_n})$ for $(f_1,\hdots,f_n) \in Um(R)$. More generally: We can facto

\section{Local Choice}
One of the goals of this chapter is to show descent for types admitting a $\bT$-(c)atlas.
In this section let $\bT$ denote a topology finer than the zariski topology.
\begin{definition}
	Let \Cov be a class of morphisms (which we think of $n$-atlasses of some $n$), containing $\bT$-atlas, (stable under pullback NECESSARY TODO?)
	A type $S$ has \emph{local choice} wrt \Cov if for any $\bT$ -surjective map $X \to Y$ and any map $f : S \to Y$ there exists a map  $p' : S' \to S$ in \Cov and a commutative diagram
	% https://q.uiver.app/#q=WzAsNCxbMCwwLCJUJyJdLFswLDEsIlQiXSxbMSwwLCJYIl0sWzEsMSwiWSJdLFsxLDNdLFsyLDNdLFswLDIsIiIsMix7InN0eWxlIjp7ImJvZHkiOnsibmFtZSI6ImRhc2hlZCJ9fX1dLFswLDFdXQ==
	% https://q.uiver.app/#q=WzAsNCxbMCwwLCJUJyJdLFswLDEsIlQiXSxbMSwwLCJYIl0sWzEsMSwiWSJdLFsxLDNdLFsyLDMsInAiLDJdLFswLDIsIiIsMix7InN0eWxlIjp7ImJvZHkiOnsibmFtZSI6ImRhc2hlZCJ9fX1dLFswLDFdXQ==
	\[\begin{tikzcd}
		{S'} & X \\
		S & Y
		\arrow[dashed, from=1-1, to=1-2]
		\arrow[from=1-1, to=2-1]
		\arrow["p"', from=1-2, to=2-2]
		\arrow["f",from=2-1, to=2-2]
	\end{tikzcd}\]
	%We say $S$ has affine local choice, if one can arrange $S'$ to be affine.
\end{definition}
\begin{prop}{\label{prop:LocalChoice}}
	%Let $\bT$ be a finer topology than the zariski topology.
	Assume that \Cov is stable under composition. %and that Zariski-covers are in \Cov.
	\begin{itemize}
		\item If $\hat S \to S$ is a \Cover and $\hat S$ has $\bT$-local choice, then $S$ has $\bT$-local choice. 
		\item Affine schemes have $\bT$-local choice.
		\item Any type admitting a \Cov - Atlas $\Spec A \to S$ has $\bT$-local choice.
	\end{itemize}
%	$S$ has  $\bT$-local choice wrt \Cov if it has a projective \Cover, i.e. there exists a projective (or, assuming ZLC, affine scheme resp.)  $\hat{S}$ with a map $g : \hat{S} \to S$ in \Cov. %, satisfying local choice wrt \Cov
\end{prop}
\begin{proof}
	%We may assume that $f = \id_S$.
	The first point follows from stability under composition of \Cov. %  We may assume that $g : \hat{S} \to S$ is the identity.
	the third point follows from the second. 
	By the first point, we may assume that $S$ is affine.
	As $p$ is $\bT$-surjective, for any $x : S$ there merely is a $\Spec B_x \in T$  and a map $\Spec B_x \to \| \fib_p (x) \| $. 
	%Claim: No matter on the assumptions (on $S = \hat{S}$), there exists a Zariski cover $S' \overset{p'}{\to} S$ with $S'$ projective (affine resp.) 
	As $S$ is projective, we have a term in
	\[\prod_{x : S} \sum_{\Spec B_x \in T} \Spec B_x \to \| \fib_p (f x)\| \]
	%	Proof: In the case of projectivity, just use $p' = \id_S$ and in the case of having ZLC and $S$ beeing affine, use ZLC (\ref{ZLCGivesZariski}). \qed(Claim)\\    
	By setting 
	\[(S' := \sum_{x : S} \Spec B_x) \overset{\pi}{\longrightarrow} S \]
	
	the projection, we are now in the situation that for any $t : S'$ we merely have a point in $\fib_p((p'(t)))$ and $S' \to S$ is a $\bT$-cover, thus it is in \Cov. Moreover, $S'$ is affine, as it is a dependent sum of affines. Hence again we now can find a lift $S' \to X$ %By replacing $S''$ again with a Zariski cover we find a lift $S'' \to X$     
	making
% https://q.uiver.app/#q=WzAsNCxbMCwwLCJTJyJdLFswLDEsIlMiXSxbMSwxLCJYIl0sWzEsMCwiWSJdLFsxLDIsImYiXSxbMywyLCJwIiwyXSxbMCwzXSxbMCwxLCJwJyIsMl1d
\[\begin{tikzcd}
	{S'} & Y \\
	S & X
	\arrow[from=1-1, to=1-2]
	\arrow["{p'}"', from=1-1, to=2-1]
	\arrow["p"', from=1-2, to=2-2]
	\arrow["f", from=2-1, to=2-2]
\end{tikzcd}\]	commute. %Now $S'' \to S' \to S$ as the composition of Zariski-covers and \Cover is a \Cover \details as desired.
	%For the general case, the previous proof is enough. \todo
\end{proof}
The next lemma shows, that the class of types equipped with a $\bT$-atlas is stable under dependent sums.

\begin{theorem}{\label{thm:atlasStableSum}}
	Let $\cU'$ be a class stable under dependent sums.
	The class of types admitting a $\cU'$-atlas is closed under dependent sums. If $\bT$ is a topology, the same holds for $\cU'$-catlasses.
\end{theorem}
\begin{proof}
	The stability under quotients is easy: 
	Let us construct some atlas $\Spec A \to \sum_{x : X} B_x$ %fibered in smooth $n$-stacks.
	For any $x : X$ we merely have an atlas $V_x \to B_x$, i.e. with $V_x$ affine. %fibered in smooth $n$-stacks . \\
	$X$ has local choice wrt atlasses by (\ref{prop:LocalChoice}) using $\cU'$ is $\sum$-stable (we use the trivial topology).\\
	If additionally, all the $B_x$ and $X$ are smooth $n$-stacks, just observe that we can choose the affine $V_{p u}$ to lie in $\bT$, Accordingly $\sum_{u : U} V_{p u} \in T$ as $\bT$ is stable under $\Sigma$.

%	$n$-atlasses contain zariski-atlasses, because $\bT$ is finer than the Zariski topology.
%	$\cU' are stable under dependent sum by induction, thus $n$-atlasses are stable under composition.         
%	\qed(Claim)\\
	By Local choice for $X$, we merely find $U$ affine, an atlas $p : U \to X$ % fibered in smooth $n$-stacks 
	with
	\[
	\prod_{u : U} \sum_{V_{p(u)} \in T} (q : V_{p(u)} \to B_{p(u)}) \times (q \ \text{fibered in smooth } n \text{ stacks } )
	\]
	Now the desired map is $\sum_{u : U} V_{p u} \to \sum_{x : X} B_x$, because it is  an atlas %fibered in smooth $(n)$-stacks 
	by \ref{lemma:AtlasSum} \\
\end{proof}
\begin{prop}{\label{prop:atlasStableCover}}
	Let $\cU'$ be a class stable under dependent sums.
	The class of types admitting a $\cU'$-(c)atlas is closed under $\cU'$-covers: If $X \to Y$ is a $\cU'$-cover, then $X$ admits a $\cU'$-(c)atlas iff $Y$ admits a $\cU'$-(c)atlas.
\end{prop}
\begin{proof}
One direction is the stability under dependent sums. For the other, if $S \to X$ is a $\cU'$-atlas, then $S \to X \to Y$ is a $\cU'$-atlas by $\sum$-stability of $\cU'$.
\end{proof}

\begin{corollary}{\label{cor:DescentCatlas}}
	If $\bT$ has descent, The class of sheaves merely admitting a $\bT$-catlas has descent.
\end{corollary}
\begin{proof}
	We can set $\cV = \cU$, and we have to show, that if $X \to Y$ is a $\bT$-cover than $X$ admits a $\bT$-catlas iff $Y$ admits a $\bT$-catlas. This follows from \ref{prop:atlasStableCover}.
\end{proof}
\section{Fundamental Theorem of algebraic spaces}
\subsection{For groupoids}
\begin{lemma}
	If $R \twoheadrightarrow X \to X$ is a $\bT$-htpy-coequalizer diagram of two $\bT$-covers between affines, then $X$ is a  1-stack.
\end{lemma}

\subsection{For sets}
\begin{lemma}\label{quotient-by-equivalence-relation}
	Denote $\bT Set$ for the sets that are $\bT$-sheaves. Assume given a $\bT$set  $X$ then the following maps are mutually inverse
	\begin{align*}
		\sum_{R:X\to X\to \bT\Prop} R\ \mathrm{equivalence\ relation} &\simeq \sum_{Y:\bT \mathrm{\Set}} \sum_{p:X\to Y} p\ \bT\mathrm{surjective} \\
		R &\mapsto (X/R,[\_]) \\
		\lambda x,y.  (p(x)=p(y)) &\mapsfrom (Y,p) 
	\end{align*}
	where $X / R$ is defined by applying $L_T \| \_ \|_0 $ at the higher inductive type $X // R$.
\end{lemma}

\begin{proof}
	\begin{itemize}
		\item Well-definedness: The map $[\_] : X \to \|X // R\|_0 \to L_T \|X // R\|_0$ is the composition of a surjective with a $\bT$-surjective map \todocite, hence its $\bT$-surjective. \\
		Conversely given $(Y,p)$ as $Y$ is a sheaf, we have for all $x,y : X$ that $p(x) =_Y p(y)$ is a sheaf.
		\item If $x,y : X$ then we have a chain of equivalences 
		\[
		R(x,y) \simeq (\bar x =_{\|X//R\|_0} \bar y) \to ([x] =_{L_T\|X//R\|_0} [y])
		\]
		where the first map is plain HoTT and the second map is $\mathsf{ap}$, i.e. the unit of the modality \todocite, but as the $\bar x =_{\|X//R\|_0} \bar y$ is already a sheaf, it is an isomorphism as well. \\
		\item Let $(Y,p)$ be in the RHS. Let $R(x,y) = (p(x)=p(y)) : \bT \Prop$. By plain HoTT, There is a map $\eta :  X // R  \to Y$ ( defined by the universal property of the set truncation and by induction on the higher inductive type $ X // R$ on canonical terms through the map $p : X \to Y$). I claim $\eta$ exhibits $Y$ as the localization for $\bT \Set$-modality of $X // R$. Let $T$ be another $\bT \Set$ equipped with a map $X // R  \to T$. By precomposition we obtain a map $X \to T$. 
		Claim: it factors uniquely through $p : X \to Y$. 
		% https://q.uiver.app/#q=WzAsNCxbMCwwLCJYIl0sWzEsMCwiXFx8WCAvIFJcXHwiXSxbMiwwLCJUIl0sWzEsMSwiWSJdLFswLDFdLFsxLDJdLFswLDNdLFszLDIsIlxcZXhpc3RzISIsMix7InN0eWxlIjp7ImJvZHkiOnsibmFtZSI6ImRhc2hlZCJ9fX1dXQ==
		\[\begin{tikzcd}
			X & {X // R} & T \\
			& Y
			\arrow[from=1-1, to=1-2]
			\arrow[from=1-1, to=2-2]
			\arrow[from=1-2, to=1-3]
			\arrow["{\exists!}"', dashed, from=2-2, to=1-3]
		\end{tikzcd}\]
		Proof: \\
		Existence: We want to define a map $Y \to T$. Let $y : Y$. As $p$ is $\bT$-surjective and $T$ is a sheaf, we may assume we merely have some element in the fiber of $p$ over $y$. Now push this element through     
		\[\|\fib_p y\| \to \|X // R\|_0 \to T\]
		where the first map is by Plain HoTT and the second one is induced from $X // R \to T$ by assumption and the fact that $T$ is a set.. One can easily check this makes the diagram commute.
		Uniqueness follows from $X \to Y$ beeing $\bT$-surjective and the following
		Fact: Two parellel maps $Y \rightrightarrows T$ into a $\bT \Set$ $T$ are already equal if the become equal after precomposition with a $\bT$-surjection $X \to Y$.  \\
		Proof of the fact : Let $y : Y$. The goal is an identity type of a $\bT \Set$, hence a $\bT \Prop$. Hence As the fiber over $y$ in $X$ is $\bT$-merely inhabited, we may assume an actual term in the fiber. 	As $X \to Y$ equalizes the arrows, this term allows us to conclude. \qed (fact)	\qed(Claim) \\
		We apply the fact to the ($\bT$-)surjectivity of $X \to X // R $ to get a unique factorization 
		% https://q.uiver.app/#q=WzAsNCxbMCwwLCJYIl0sWzEsMCwiXFx8WCAvIFJcXHwiXSxbMiwwLCJUIl0sWzEsMSwiWSJdLFswLDEsIiIsMCx7InN0eWxlIjp7ImhlYWQiOnsibmFtZSI6ImVwaSJ9fX1dLFsxLDJdLFswLDNdLFszLDIsIlxcZXhpc3RzISIsMix7InN0eWxlIjp7ImJvZHkiOnsibmFtZSI6ImRhc2hlZCJ9fX1dLFsxLDNdXQ==
		\[\begin{tikzcd}
			X & {X // R} & T \\
			& Y
			\arrow[two heads, from=1-1, to=1-2]
			\arrow[from=1-1, to=2-2]
			\arrow[from=1-2, to=1-3]
			\arrow[from=1-2, to=2-2]
			\arrow["{\exists!}"', dashed, from=2-2, to=1-3]
		\end{tikzcd}\]
		making the right triangle commute. This is what we wanted to show.
	\end{itemize}
\end{proof}

\begin{definition}
	An equivalence relation $R$ on a type $X$ is called:
	\begin{itemize}
		\item redundant if for all $x,y:X$ the proposition $R(x,y)$ is a  $-1$-stack.
		\item covering if its  and for any $y:X$ its fibers:
		\[R_y :\equiv \sum_{x:X} R(x,y)\]
		are affine in $\bT$.
	\end{itemize}
\end{definition}

\begin{lemma}\label{fundamental-propriety-algebraic-spaces}
	Assume that $\bT$ satisfies descent for propositions and for sets \ref{thm:descent}, i.e. that a modal proposition being a  (-1)-stack is a sheaf. Assume that a modal set beeing affine in $\bT$ is a sheaf.
	Assume given a $\bT$set $X$, then the following types are equivalent:
	\begin{itemize}
		\item The type of redundant covering equivalence relations over $X$.
		\item The type of $\bT$sets $Y$ with identity types beeing  stacks and an $-1$-atlas $X$ to $Y$ (in V2 a $\bT$-cover).
	\end{itemize}
\end{lemma}

\begin{proof}
	By the equivalence in \ref{quotient-by-equivalence-relation}, it is enough to check that:
	\begin{itemize}
		\item The identity types in $X/R$ are 
		(-1)-stacks if and only if the relation $R$ is redundant . For any $x,y:X$ we know that:
		\[R(x,y) \simeq [x] =_{X/R}[y]\]
		so the direct direction is immediate. For the converse we use the assumption that a modal proposition being a  (-1)-stack is a sheaf and that the map $[\_]:X\to X/R$ is $\bT$-surjective.
		\item The fibers of: 
		\[[\_]:X\to X/R\] 
		are affine in $\bT$ if and only if the relation $R$ is covering. For any $y:X$ we have that:
		\[\sum_{x:X} R(x,y) \simeq \mathrm{fib}_{[\_]}([y])\]
		so the direct direction is immediate. Here as well the converse follows from $\bT$-surjectivity of $[\_]$ and that the topology has descent.
	\end{itemize}
\end{proof}
\begin{corollary}
	Assume $\bT$ satisfies descent for propositions and for sets.
	A type is a  0-stack iff its merely the $\bT$-quotient of an affine scheme by a covering equivalence relation.
\end{corollary}
\begin{theorem}{\label{thm:QuotientOfAlgebraicSpace}}
	Assume $\bT$ satisfies descent for propositions. 
	The quotient of a  $0$-stack $X \in \bT \Set$ by an $0$-covering equivalence relation $R$ is a  $0$-stack. TODO
\end{theorem}

\begin{proof}
	The identity types in $X / R$ are propositional  0-stacks, hence $(-1)$-\truncation s of  -1-stacks by \ref{lemma:prop0stacks} as desired. \\
	How to find an atlas: todo. How to proceed, if we could choose all atlasses we want at the same time?
	% Motivation why the choice of atlasse should work: Let $T = X / R$. 
	%  If we could choose -1-atlasses $\tilde X_t$ for the covering 0-stacks $\fib_{[]}(t)$ for all $t : T$ at the same time, then $\sum_{t : T} \tilde X_t \to \sum_{t : T} \fib_{[]}(t) \to T$ has as domain a is fibered in covering $-1$-stacks, as the fiber over $t$ would be $\tilde X_t$ which is an affine scheme in the topology. Moreover, This is enough as \\ %, hence by definition a covering -1 stack. \\
	
	%     Given $p_1 , p_2 : R \rightrightarrows X$ fibered in covering $0$-stacks, hence the fibers merely have $-1$-atlasses.
	%     %Claim: There exists a $-1$-atlas $R' \to R$
	%     As $X$ has Local choice with respect to $-1$-atlasses, we find a $-1$ atlas $f : X' \to X$ with 
	%     \[
	%     \prod_{x' : X'} \text{-1-atlas}(\sum_{x : X} R(x,f(x'))
	%     \]
	
\end{proof}
\begin{rmk}
	This is equivalent to saying that  $1$-stacks that are $0$-types are geomeric $0$-stacks: One direction we prove later. If $R$ is a 0-covering equivalence relation on a  0-stack $X$, then $ X/ R$ is a  1-stack by observing that any -1-atlas $X' \to X$ gives a 0-atlas $X' \to X \to X/ R$. Moreover, $ X/ R$ is a 0-type, hence by assumption a  0-stack.
\end{rmk}

\begin{example}
	There are open affine subschemes $U$ of affine schemes $\Spec A$, which are not (disjoints unions of) principal open
\end{example}
\begin{proof}
	Consider $A = R[x,y,u,v]/(xy + ux^2 + vy^2), X= \Spec A$ and consider the open $U = D(x,y)$. \\
	We cant expect $U$ to be a disjoint union of principal opens (todo). However, $D(x,y)$ is affine: We have maps $U \to R$ given by
	$f = -v/x = (y+ux)/y^2 , g= -u/y = (x+vy)/x^2$. 
	Then $D(f) \cup D(g) = \Spec R^X$ , as $yf + xg = 1$ in $R^U$.
	Taking preimages under the affinization map, $U_f \cup U_g = X$ and one checks this defines an open affine cover (for example : $U_f \simeq \Spec R[x,u,f^{\pm 1}, g] / (xy + ux^2 + uy^2)$ with $y := (1-gx)/f$.)
	But on both of this open subsets the affinization map is an isomorphism
	hence the affinization of $X$ is an isomorphism.
	%\[R^X \simeq R^{D(f) \cup D(g)} \simeq R^{D(f)} \times_{R^{D(fg}} D(g) \]
	compare (Hartshorne II.2.17)
\end{proof}
% \begin{example}
%     The Zariski topology does not descent along $\bT$-covers between affines
% \end{example}
% \begin{proof}
% Assume it would hold.
% By the previous example pick such an open affine subset $U \subset \Spec A$ and pick a Zariski atlas $V \to U$ such that $V$ is mereley of the form $D(a_1) + \hdots + D(a_n)$ for some $a_i \in A$. Let $x : \Spec A$. Then pulling pack the Zariski atlas along $U(x) \to U$ gives us a Zariski atlas of the open proposition $V' \to U(x)$. Now $V' + 1 \to U(x) + 1$ is a Zariski atlas with total space in the Zariski topology. By assumption, $U(x) + 1$ is in the topology, hence $U(x)$ would be a sum of principal opens. As it is a propososition, it would be a principal open subset of $1$. 
% This is not a contradiction, because an open subset can be non principal although all the fibers are principal open props...
% This is a contradiction by the assumption on $U \subset \Spec A$ beeing not principal open.

% \end{proof}
\begin{lemma}
	Let $f : X \to Y$ be surjective. There exists a Zariski Cover $X' \to X$ such that $X' \to Y$ is a Zariski cover iff there exists a Zariski Cover $X' \to X$, some $n : \bN$ and an open affine embedding $X' \hookrightarrow Y^n$ over $Y$.
\end{lemma}

% \begin{lemma}
%     A morphism $f : X \to Y$ of  $n$-stacks is fibered in covering $n$-stacks if there exists a covering $n$-atlas of $f$.
% \end{lemma}





\subsection{Algebraic spaces}


%Let us also mention what we learned in the proof:
%\begin{lemma}[NECESSARY?@²]
%	A covering equivalence relation $R : S^2 \to \bT \Prop$ has values in geometric propositions.
%\end{lemma}

%\begin{corollary}
%	The identity types of algebraic spaces are geometric propositions.
%\end{corollary}
%\begin{proof}
%	By the previous lemma and \ref{lemma:geometricStacksClosedUnderId}
%\end{proof}
%
%\begin{lemma}{\label{lemma:detectGeomProp}}
%	Let $P$ be a sheaf and a proposition that admits a map $\Spec A \to P$ fibered in covering algebraic spaces. Then $P$ is a geometric proposition.
%\end{lemma}
%\begin{proof}
%	The fibers are covering algebraic spaces and affine, hence covering affine. By \ref{def:algprop} we conclude.
%\end{proof}
\begin{theorem}
	Let $X$ be a modal set. The following are equivalent:
	\begin{enumerate}
		\item $X$ is a (covering) geometric 0-stack
		\item $X$ is merely of the form $L_\bT (U / R)$ for some (covering) affine $U$ and  $R : U^2 \to \Prop_{\ci}$ a covering equivalence relation. 
		\item there exists some map $S \to X$ with $S$ (covering) affine whose fibers merely have $\bT$-catlasses.
	\end{enumerate}
	We call this class (covering) algebraic spaces.
\end{theorem}
\begin{proof}
\ 	\begin{enumerate}
	\item [2 $\leftrightarrow$ 3]
		This is \ref {lemma:fundamental-property-algebraic-spaces}
	\item [2 $\to$ 1]
	Choose a presentation $ R: U^2 \to \Prop$.
	It suffices to show, that the map $f : U \to L_\bT ( U / R)$ is a geometric (c)atlas. The map $f$ is $\bT$-surjective by the well-definedness of the bijection $\ref{quotient-by-equivalence-relation}$. By descent we may just show, that the fibers $\fib_f (f(s))$ for $s : U$ are covering 0-stacks. But by the bijection in \ref{quotient-by-equivalence-relation} those are equivalent to the fibers $R_s$, which are covering 0-stacks as the equivalence relation is covering. \\
	\item [1 $\to$ 2]
	This can be reformulated in the following way, using the recursion principle for (covering) geometric 0-stacks:
	Let $X$ be a sheaf of sets. Let $S$ be (covering-) affine and $f : S \to X$ be fibered in covering algebraic spaces. Then $X$ is a (covering) algebraic space.
%	The identity types of $X$ admit a map fibered in covering algebraic spaces (todo check stability under $\sum$) out of an affine by \ref{lemma:havingAbstractAtlasClosedUnderId}. by \ref{lemma:detectGeomProp} they are geometric propositions. 
This follows from the observation, that the equivalence relation determined by $f$ is covering \ref{def:coveringEqRel} , because the fibers of $f$ are covering 0-stacks.
	\end{enumerate}
\end{proof}
\begin{prop}
	For any $n \ge 1$, we have inclusions 
	\[W_{n} \subset \CS_{n-1} \subset W_{n+1}\]
\end{prop}
\begin{proof}
	Induction. $n = 1$ gives
	\[
	\mathsf{HasCatlas}_\bT \subset \CS_0 \subset \text{ types admitting a catlas fibered in } W_1
	\]
	the latter inclusion is the previous theorem. \\
	The induction step is obtained by \ref{prop:nstack}
\end{proof}


\subsection{Schemes are algebraic Spaces for the Zariski Topology}
\begin{definition}
 A proposition $U$ is open iff its merely of the form $f_1 \ inv \lor \hdots f_n inv$ for some $f_i : R$.
\end{definition}

\begin{lemma}
	Given $f_1, \hdots,f_n : R$ such that $\| D(f_1) + \hdots + D(f_n) \|$ then $\sum_{i=1}^n D(f_i) \in \Zar$.
\end{lemma}
\begin{prop}
	Every Zariski-merely-inhabited type that is merely of the form $U_1 + \hdots + U_n$ for open propositions $U_i$ admits a $\Zar$-catlas.
\end{prop}
\begin{proof}
	By definition of openness, We can choose a surjection $\coprod_{j=1}^{n_i} D(f_{ij}) \twoheadrightarrow U_i$ for any $i$. We want to show, that the map
	\[
	\coprod_{i , j} D(f_{ij}) \twoheadrightarrow U_1 + \hdots U_n
	\]
	is a $\Zar$-catlas. 
	\begin{itemize}
		\item Let us first show that the fibers are in $\Zar$. Assume $U_i$ holds. So we find a term in $\coprod_j D(f_{ij})$. In particular we have $\| \coprod_j D(f_{ij})\|_{\Zar}$. By the lemma we conclude, that the fiber $\sum_j D(f_{ij})$ belongs to $\Zar$.\\
		\item The total space is in $\Zar$: This follows as the surjection after propositional truncation becomes an equivalence. As we have $\| U_1 + \hdots + U_n\|$, we can conclude by the lemma.
	\end{itemize}
	
\end{proof}
\begin{warning}
	The converse does not hold! We want to apply \ref{lemma:stackificationHasTCover}, to the map
	\[\Zar \ni 1 + 1 \to \sum D(f) \]
	\begin{itemize}
		\item 	$\sum D(f)$ is seperated as $D(f)$ is a sheaf.
		\item 	All the fibers are equivalent to $1 + X$, hence they are in the Zariski topology.
	\end{itemize}	
\end{warning}
\begin{lemma}
	let $X$ be a scheme. There merely exists some affine $S$  map $S \to X$ whose fibers are merely inhabited finite sums of open propositions 
\end{lemma}

\begin{corollary}		
		Every scheme is an algebraic space.
\end{corollary}


\begin{lemma}
	If $X$ is an algebraic space, then the global sections embed via a $R$-algebra homomorphisms into a finitely presented $R$-algebra.
\end{lemma}
\begin{proof}
	Choose an atlas $S \to X$, in particular $\bT$-surjective. As $\bT$ is subcanonical the map $R^X \to R^S$ is an injection.
\end{proof}
\begin{question}
	Is it an open embedding of types?
\end{question}
%\begin{lemma}
%	Consider a morphism $X \to Y$ between algebraic spaces such that $R^F$ is finitely presented and the affinizations of the fibers $F \to \Spec R^F$ are open immersions. If $Y$ is a scheme, then $X$ is a scheme.
%\end{lemma}
%\begin{proof}
%	Consider the stein factorization
%	\[
%	X = \sum_{y : Y} \fib_f y \to \sum_{y : Y} \Spec R^{\fib_f y} \to Y
%	\]
%	The first map 
%\end{proof}
\subsection{Examples}
The goal of this subsection is to construct algebraic spaces. The first example actually gives us a scheme:
\begin{example}
	Let $p \neq 0$ be a prime. You can let $\mu_p := \Spec(R[X] / (X^p - 1))$ act on $\bA^\times$ via multiplication. Set $\bT= fppf$. Then the $p$.th power map
	
	\[
	pow : \| \bA^\times // \mu_p \|_0^\bT \to \bA^\times
	\]
	is an equivalence.
	\begin{itemize}
		\item 	 It is an embedding: 
		First note, that $\|\bA^\times // \mu_p \|_0$ is $\bT$-seperated:
		
		as $\mu_p$ act freely on $\bA^\times$, $\bA^\times // \mu_p$ is already a set. Meaning that the identity types of the set-quotient are $\sum_{g: \mu_p} g x =_{\bA^\times} y$ , hence sheaves. \\
		On the other hand the map $\|\bA^\times // \mu_p \|_0 \to \bA^\times$ is an embedding, as for any $x , y : \bA^\times$ the map $(\sum_{g : \mu_p} g x = y) \to (x^p = y^p)$ is an equivalence. 
		\item 	It is $\bT$-surjective, as for any $\lambda : \bA^\times$, we find $S = \Spec R [X] / (X^p - \lambda) \in \bT$ with 
		\[
		S \to \fib_{\bA^\times / \mu_p \to \bA^\times}(\lambda)
		\]
		hence 
		\[
		1 = \|S\|_\bT \to \|\fib_{pow}\|_0^\bT
		\]
	\end{itemize}
	
	
\end{example}
\begin{example}[TODO]
	The sheaf quotient of $\bA^1$ by the $\mu_\ell$ action is probably not an algebraic space.
\end{example}

\begin{lemma}{\label{lemma:AlmostEverywhere}}
	Let $p : A$ be reguar. If $f : \Spec A \to R$ such that $f(x) = 0$ for all $x \in D(p)$, then $f(x) = 0$ for all $x : \Spec A$.
\end{lemma}
\begin{proof}
	$f$ is in the kernel of the diagonal map
% https://q.uiver.app/#q=WzAsNCxbMSwwLCJSXlIiXSxbMSwxLCJSXntSIFxcc2V0bWludXMgXFx7MFxcfX0iXSxbMCwwLCJSW1hdIl0sWzAsMSwiUltYXntcXHBtIDF9XSJdLFswLDFdLFsyLDMsIiIsMCx7InN0eWxlIjp7InRhaWwiOnsibmFtZSI6Imhvb2siLCJzaWRlIjoidG9wIn19fV0sWzAsMiwiIiwxLHsibGV2ZWwiOjIsInN0eWxlIjp7ImhlYWQiOnsibmFtZSI6Im5vbmUifX19XSxbMSwzLCIiLDEseyJsZXZlbCI6Miwic3R5bGUiOnsiaGVhZCI6eyJuYW1lIjoibm9uZSJ9fX1dXQ==
\[\begin{tikzcd}
	{A} & {R^{\Spec A}} \\
	{A_p} & {R^{D(p)}}
	\arrow[hook, from=1-1, to=2-1]
	\arrow[Rightarrow, no head, from=1-2, to=1-1]
	\arrow[from=1-2, to=2-2]
	\arrow[Rightarrow, no head, from=2-2, to=2-1]
\end{tikzcd}\]
	which is injective, as $p$ is regular in $A$. \\
	Thus $f = 0$ in $A$.
%	 Thus $f = 0 $ in $(R \setminus \{0\} \to R) = R[X^{\pm 1}]$ hence $f \cdot X^n = 0$ in $R[X]$ for some $n$, thus $f = 0$.
\end{proof}
Let $\ell \neq 0$ denote a prime. Consider $\mu_\ell = R[X] / (X^\ell - 1)$.
%\[\mu_\ell'  = \Spec R[X] / (X^{p-1} + \hdots + 1) = \mu_\ell \setminus\{1\}\]

%\begin{lemma}
%	Let $0 \in D(p)$. For a function $\phi : D(p) \to R$ TFAE % We say some $\phi: R[X]_p$ is $\ell$-even if one of the following equivalent conditions is satisfied:
%	\begin{enumerate}
%
%	\item The function $\phi : D(p) \to R$ is a $\mu_{\ell}$-invariant function, i.e. $\phi(x) = \phi(g x)$ for all $ x : D(p)$ and each $g : \mu_\ell$.
%	\item The function $\phi|_{D(p) \setminus 0} : D(p) \setminus \{0\} \to R$ is an $\mu_{\ell}$-invariant function
%%	\item[1']  There exists $f : R[X], n : \bN$ such that $\phi= f/ p^n$ and $f (g.p)^n (x) = (g.f) p^n (x)$ for each $x : R , g : \mu_\ell$.
%%	\item[2'] The same as 3. but only for $x : D(p) \setminus\{0\}$
%\end{enumerate}
%\end{lemma}
%\begin{proof}
%	 we can apply  \ref{lemma:AlmostEverywhere}, observing $\phi - g.\phi = 0$ on $D(X / 1) \subset \Spec R[X]_p$, where $X/1 : R[X]_p$ is regular, because $X$ is regular in $R[X]$. %$: 
%\end{proof}
\begin{prop}
	Let $G$ be a formally \etale flat affine group, such that $\lnot \lnot$ its finite with cardinality invertible in $R$. Let it act on an affine scheme $\Spec A$ with a fixpoint 0, such that the group action is free away from 0.
	Let $p : A^G$. Define $R: D(p)^2 \to \Prop$ as
	\[
	R(x,y) = (x = y) + (x \neq 0) \times \sum_{g : G \setminus \{1\}} g x = y
	\]
	Then the sheaf quotient $D(p) / G$ is an algebraic space.
\end{prop}
\begin{proof}
%	Define
%	\[
%	E(x,y) = (x = y) + (x \neq 0 \land \sum_{g : \mu_\ell \setminus \{1\}} gx = y)
%	\]
	This is a proposition: First note, that both summands are propositions because $G$ acts freely on $\bA^1 \setminus \{0\}$. If both summands are inhabited we get a contradiction, as $x = y$ and $gx = y$ implies $(g-1) x = y - x = 0$, but as $g-1$ is invertible $x = 0$. \\
	The relation is covering: 
	The propositions are affines, thus sheaves. Furthermore, for any $y : D(p)$ we have
	\[
	\sum_{x : D(p)}  (x = y) + (x \neq 0 \times \sum_{g : G \setminus \{1\}} gx = y) = 1 + (y \neq 0 \times G \setminus \{1\}) \in \bT
	\]
	as $G \setminus \{1\} = \sum_{g : G} g \neq 1$ is a $\sum$ of formally \etale + flat affines (recall that formally \etale affines have decidable equality).
	
\end{proof}
%\begin{prop}
%	Let $p : R[X]^{\mu_\ell}$ with $0 \in D(p)$ % be such that $0 \in D(p)$ and $x \in D(p)$ implies $gx \in D(p)$ for all $g : \mu_\ell$.
%	The sheaf-quotient of $D(p)$ by the relation which identifies $x$ and $gx$ when $x \neq 0, g : \mu_\ell \setminus \{1\}$ is not an affine scheme.
%\end{prop}
%\begin{proof}
%
%\end{proof}
\begin{lemma}
	Let $G$ be a finite group whose cardinality is invertible in $R$. Let $G$ act on an affine scheme equipped with a fixpoint $0$. Let $V$ be an invariant open neighborhood of 0. Then we find an invariant principal open neigbhorhood contained in $V$. Invariant means here that $g(U) = U$ for all $g : G$.%Let $A$ be a finitely presented algebra and $0 \in \Spec A$ a basepoint. 
\end{lemma}
\begin{proof}
	Let $U$ be an invariant open neighborhood. Choose a principal open neighborhood $0 \in D(p) \subset U$. $G$ acts on $R[X]$, via $(g.p)(x) = p(g x)$.Then 
	\[p' = \sum_{g : G} g . p : R[X]\]
	is a $G$-invariant polynomial, in particular $D(p)$ is $G$-invariant. Moreover $0 \in D(p')$ as
	\[
	p'(0) = \sum_{g : G} p(g(0)) = \sum_{g : G} p (0) = |G| \cdot p(0) 
	\]
	is invertible, as $|G|$ and $p(0)$ are both invertible. Furthermore, as $U$ was $G$ invariant and contained $D(p)$ it also has to contain $D(p')$.
\end{proof}

\begin{prop}
	Let $G$ be a group, such that $\lnot \lnot$ its finite with cardinality invertible in $R$. Let it act on an affine scheme $\Spec A$ with a fixpoint 0, such that the group action is free away from 0.
	For any $p : A^G$ define $R: D(p)^2 \to \Prop$ as
	\[
	R(x,y) = (x = y) + (x \neq 0) \times \sum_{g : G \setminus \{1\}} g x = y
	\]
	Assume the sheaf quotient $D(p) / G$ is not affine for any $p : A^G$. Then $(\Spec A) / G$ is an algebraic space that is not a scheme.
\end{prop}
\begin{proof}
	It is an algebraic space by putting $p \equiv 1 : R[X]$ in the previous prop. \\
	Assume the quotient is a scheme. 
	The preimage along the quotient map obtained from the relation induces a open neigbhorhood $V$ of $0$ in $\bA^1$. As we want to prove a contradiction we may assume that $\mu_\ell$ consists of $\ell$ many elements, where $\ell \neq 0 $ in $R$. We apply the previous lemma to $V$ to obtain an invariant principal open neigborhood $0 \in D(p) \subset V \subset \bA^1$. As its invariant, $p : \bA^1 \to R$ descends to $X \to R$. Restricting to $V$ yields a map $p' : V \to R$, such that setting $U \equiv D(p')$ yields such that $q^{-1}(V) =q^{-1}(D(p')) = D(p)$ . We are now in the following situation
	% https://q.uiver.app/#q=WzAsNCxbMCwwLCJEKHApIl0sWzEsMCwiXFxiQcK5Il0sWzEsMSwiWCJdLFswLDEsIlUiXSxbMCwxLCIiLDAseyJzdHlsZSI6eyJ0YWlsIjp7Im5hbWUiOiJob29rIiwic2lkZSI6InRvcCJ9fX1dLFswLDNdLFszLDIsIiIsMix7InN0eWxlIjp7InRhaWwiOnsibmFtZSI6Imhvb2siLCJzaWRlIjoidG9wIn19fV0sWzEsMl0sWzAsMiwiIiwxLHsic3R5bGUiOnsibmFtZSI6ImNvcm5lci1pbnZlcnNlIn19XV0=
	\[\begin{tikzcd}
		{D(p)} & {\bA^1} \\
		U & X
		\arrow[hook, from=1-1, to=1-2]
		\arrow[from=1-1, to=2-1]
		\arrow["\ulcorner"{anchor=center, pos=0.125}, draw=none, from=1-1, to=2-2]
		\arrow[from=1-2, to=2-2]
		\arrow[hook, from=2-1, to=2-2]
	\end{tikzcd}\]
	where $U$ is an open affine neighborhood of 0. \\
	Then we have $D(p) / \sim' \simeq U$ with restricted equivalence relation. By assumption we see $U$ cannot be affine. Contradiction.\\		
\end{proof}
\begin{corollary}
	Let $\ell \neq 0$ be prime. The sheaf quotient of $\bA^1$ by the relation that identifies each  $x \neq 0$ with $g x$ if $g^p = 1, g \neq 1$ is an algebraic space that is not a scheme.
\end{corollary}
\begin{proof}
	Let $p$ be as above.
	The conditions on $p$ give $p(0) \neq 0$ and $p(x) \neq 0 \to p(gx) \neq 0$ for all $g : \mu_\ell$.
	
	Lets call this quotient $X$.
	
	Define 
	\[
	A = \{\phi : R^{D(p)} \ | \ \phi \text{  is $\mu_{\ell}$-invariant }\}
	\]
	This is an $R$-subalgebra: for any $r : R$, $r : R[X]_p$ is $\mu_{\ell}$-invariant. $\mu_{\ell}$-invariant functions are stable under addition and multiplication . \\
	
	Claim: The affinization map of $X$ is the induced dashed map $f : X \to \Spec A$ in
	
	% https://q.uiver.app/#q=WzAsNCxbMCwwLCJEKHApIl0sWzEsMCwiXFxTcGVjIFJbWF1fcCJdLFswLDEsIlgiXSxbMSwxLCJcXFNwZWMgQSJdLFswLDIsInEiXSxbMiwzLCJcXGV4aXN0ISBmIiwwLHsic3R5bGUiOnsiYm9keSI6eyJuYW1lIjoiZGFzaGVkIn19fV0sWzEsM10sWzAsMSwiIiwyLHsibGV2ZWwiOjIsInN0eWxlIjp7ImhlYWQiOnsibmFtZSI6Im5vbmUifX19XV0=
	\[\begin{tikzcd}
		{D(p)} & {\Spec R^{D(p)}} \\
		X & {\Spec A}
		\arrow[Rightarrow, no head, from=1-1, to=1-2]
		\arrow["q", from=1-1, to=2-1]
		\arrow["q'",from=1-2, to=2-2]
		\arrow["{\exists! f}", dashed, from=2-1, to=2-2]
	\end{tikzcd}\]
	Proof: A function $\phi : D(p) \to R$ factors through $q$ iff $\phi|_{D(p) \setminus\{0\}}$ is $\mu_{\ell}$-invariant away from 0. We have to show, that then $\phi$ is $\mu_{\ell}$ invariant, because then the embedding (using that $R$ is a sheaf) $R^X \hookrightarrow R^{D(p)}$ has image $A$. We can apply  \ref{lemma:AlmostEverywhere}, observing $\phi - g.\phi = 0$ on $D(X / 1) \subset \Spec R[X]_p$, where $X/1 : R[X]_p$ is regular, because $X$ is regular in $R[X]$. $\qed$(Claim). 	\\ \\ %respective $\phi : R[X]_p$ satisfies $	\phi (x) = \phi(-x)  $, i.e. (1) if $\phi$ is $\mu_{\ell}$-invariant. 
	Proof that $X$ is not an affine:	Assume that $X$ were affine. Then the map $f$ would be in particular an embedding. 
	We may assume a term $g : \mu_\ell \setminus \{1\}$: Indeed, as we want to prove a contradiction we may assume a term in $g : \Spec R[X] / (\sum_{i=0}^{\ell-1} X^i)$. But this type is equivalent to $\mu_\ell \setminus \{1\}$, using that $\sum_{i=0}^{\ell-1} X^i | X^\ell -1 $ and $\ell \neq 0$. \\
	Let $V \subset \bA^1$ be a non-contractible neighborhood of 0 that is infinitesimal, i.e.  $\lnot \lnot x =0$ for every $x : V$. (e.g $\Spec R[X] / X^n$ for some $n >1$ is non contractible, because $R[X] / X^n \to R[X] / X$ is not an algebra isomorphism). %$\cN_{\lnot \lnot}(0) = \{x : \bA^1 \ | \ \lnot \lnot x = 0\}$ be be a non-contractible subtype that is $\lnot \lnot$-connected. 
	Then for any $\varepsilon : V \subset D(p)$ (using that invertibility is $\lnot \lnot$ stable) we have
	\begin{align*}
		(q\varepsilon =_X q (g \varepsilon)) \overset{\ref{quotient-by-equivalence-relation}}{=} (\varepsilon = g\varepsilon) + (\varepsilon \neq 0 \land \sum_{h \neq 1} \varepsilon = h g \varepsilon) = ((g-1)\varepsilon = 0) = (\varepsilon = 0)
	\end{align*}
	But we have 
	\[(q' \varepsilon =_{\Spec A} q' (g \varepsilon)) = \prod_{\phi : A} \phi(q' \varepsilon) = \phi(q' (g \varepsilon)) = \prod_{\substack{\phi : R^{D(p)} \\ \phi \in A}} \phi (\varepsilon) = \phi(g \varepsilon) ,\] 
	%as for any $\phi : A$ we have $\phi(\varepsilon) = \phi(g \varepsilon)$ 
	and the right hand side is inhabited as each $\phi$ satisfies the condition (1). \\
	So we conclude the the embedding $\{0\} \hookrightarrow V$ is an equivalence.  But we asked $V$ to be non contractible.
\end{proof}
\begin{example}
	The scheme 
	\[
	\sum_{x , y : R} x y = 0
	\]
	sheaf-quotiented by the relation that identifies $(x,0)$ and $(0,x)$ if $x \neq 0$ is an algebraic space.
\end{example}
\begin{proof}
	The equivalence relation is given by
	\[
	E((x,y) , (x',y')) = (x = x' \land y = y') + (x \neq 0 \land x = y' \land x' = 0)
	\]
	as $x \neq 0$ implies $y = 0$ as $x y =0$. This is a covering relation, as for any $x' y' = 0$ we have
	\[
	\sum_{x,y : R} x y = 0 \land E((x,y) , (x' , y')) = 1 + (y' \neq 0) \in \Zar \subset \bT
	\]		
\end{proof}



\section{Group quotients}
For this section let $G$ denote a group that is a covering 0-stack. Let $X$ be a sheaf equipped with a $G$ action.
\begin{lemma}
	 $\mu_p = \Spec R[X] / (X^p - 1)$ is covering for $p \neq 0$ prime.
\end{lemma}
\begin{proof}
	It is fppf + \etale as $X^p - 1$ is monic seperable. TODO
\end{proof}
\begin{definition}
	A $G$ action on $X$ is free, if for all $x , y : X$ the type 
	\[
	\sum_{g: G} g x = y
	\]
	is a proposition. 
\end{definition}
\begin{lemma}
	Let $G$ act freely on a sheaf $X$. Then the relation
	\[
	x , y\mapsto \sum_{g : G} g x = y
	\]
	is a covering equivalence relation on $X$
\end{lemma}
\begin{proof}
	All those propositions are modal as $X$ and $G$ are sheaves. For all $x : X$ , the fiber
	\[
	\sum_{y : X} \sum_{g : G} g x = y \simeq \sum_{g : G} \sum_{y: X} g x = y \simeq G
	\]
	is a covering 0-stack by assumption.
\end{proof}
\begin{lemma}{\label{lemma:algSpacesStabFreeQuots}}
	%For $n \ge 0$, geometric $(n)$-stacks 
	Algebraic spaces are stable by free quotients of covering group 0-stacks.
\end{lemma}
\begin{proof}
	The map $ X \to L_T (X / G)$ is fibered in covering 0-stacks, so in particular covering $0$-stacks. As $X$ is a geometric $0$-stack, the quotient is a geometric $0$-stack as well, This follows by the description in \label{prop:nstack}, choosing a geometric atlas of $X$ and postcomposing this to get a geometric atlas of the quotient.
\end{proof}

%\begin{lemma}
%	Let $X$ be a geometric stack, whose identity types are covering stacks. Let $G$ be a finite group acting on $X$. Then $L_\bT (X / G)$ is a geometric stack.
%\end{lemma}
%\begin{proof}
%	Consider for $x , y : X$ , $R(x,y) \equiv \| \sum_{g : G} g x = y\|_\bT$ which is indeed modal. We have to check that the relation is covering, i.e. that for all $x : X$, 
%	\[
%	\sum_{y: X} \|\sum_{g: G} g x = y\|_\bT
%	\]
%	is a covering stack. \\
%	To prove this, as covering stacks are stable under quotients, it suffices to show, that the map
%	\[
%	G \simeq \left (\sum_{y : X} \sum_{g : G} g x = y \right) \to \sum_{y: X} \|\sum_{g: G} g x = y\|_\bT
%	\]
%	is a geometric cover. But the fibers look like $\sum_{g : G} gx  = y$ which is a finite sum of identity types in $X$, which were assumed to be covering stacks. By \ref{lemma:geomStackPlusStable} the fibers are covering stacks.
%	
%\end{proof}



\section{Stability under Quotients}
\begin{definition}
	A morphism between $n$-stacks is covering if it is fibered in 
	\begin{itemize}
		\item $\bT$ if $n \le 0$
		\item covering $n$-stacks if $n > 0$.
	\end{itemize}
\end{definition}

\begin{theorem}{\label{thm:quotients}}
	Let $f : X \to Y$ be a $\bT$-surjective covering morphism between modal $n$-types. If $X$ is a (covering) stack , then $Y$ a  (covering) stack.
\end{theorem}
(*) This can only hold if we define -1-stacks to be  modal propositions with a $-2$-atlas $\Spec A \to P$, i.e. algebraic propositions \ref{def:algprop} %\begin{lemma}[have to force this]{\label{lemma:missing}}
%	To check wether a modal proposition $P$ is a $-1$-stack its enough to find a $-2$-atlas $\Spec A \to P$.
%\end{lemma}
\begin{proof}
	Induction.
	For $n = -2$ its clear.
	Let $X$ be a  $n$-stack. Lets first construct the $n-1$-atlas of $Y$.
	We merely find a $V \twoheadrightarrow X$ which is an $n-1$-atlas.  Then $V \to X \to Y$ is an $n$-atlas because it is $\bT$-surjective and is fibered in the correct $\sum$-stable class of types, i.e. $\bT$ if $n \le 1$ and  covering $n-1$-stacks for $n > 1$. Hence $Y$ is an $n+1$-stack. As $Y$ is an $n$-type, $Y$ is an $n$-stack \ref{thm:stack}. \\
	If additionally $X$ is assumed to be covering, then $V$ can be assumed to lie in $\bT$ which directly gives us that $Y$ has a covering atlas. \\
	It remains to show that the identity types of $Y$ are  $n-1$-stacks. As $Y$ has an $n-1$-atlas, by \ref{lemma:havingAbstractAtlasClosedUnderId} we  find some $n-1$-atlas $p : W \to y=y'$. The map is covering. %, because the fiRbers of $p$ are covering. % because for $n \le 0$ $\bT$ and for $n > 0$ covering $n-1$-stacks are stable under finite products. 
	If $n=0$, $y = y'$ is a $-1$-stack by (*). If $n > 0$, $W$ is an $n-1$-stack and $p$ is covering, so by induction $y = y'$ is an $n-1$-stack. \\

	
	
	%This is enough because choosing an $n$-atlas of the stack $(\fib_p y) \times_Y (\fib_p y')$ gives us by composition an $n$-atlas of $y = y'$. By induction ? $y = y'$ is an $n+1$-stack. By \ref{thm:stack} its an $n-1$-stack as desired.
\end{proof}
\begin{rmk}[Using descent but not induction]
		Hugo suggested an alternative argument proving that the identity types of $Y$ are $n-1$-stacks, which presumable avoids \ref{thm:stack} but uses descent for $n$-stacks: 
	For $x : X, y: Y$ we have that 
	\[
	(f(x) = y) \simeq (1 \times_X \fib_f y)
	\]
	is an $n$-stack by stability under $\sum$. Because it is an $n-1$-type, it is a $n-1$-stack by \ref{thm:stack}. Now conclude that every identity type of $Y$ is an $n-1$-stack by using descent for $n-1$-stacks and $\bT$-surjectivity of $f$.
\end{rmk}
\section{Local properties}

\begin{definition}
Let \Cov be the property of morphisms of  $n$-stacks defined by asking that the morphism is $\bT$-surjective and fibered in covering $n$-stacks. Its stable under basechange. A property of  $n$-stacks is local if $P(1)$ holds, $P$ is stable by dependent sums and given a \Cover  $X \to Y$ we have $P X$ iff $P Y$.
\end{definition}
\begin{example}    
    beeing covering $n$-stack is a local property of stacks.
\end{example}
\begin{proof}
    We have to show: If $f : X \to Y$ is a $\bT$-surjective map fibered in covering $n$-stacks between  $n$-stacks, then $X$ is a covering $n$-stack iff $Y$ is a covering $n$-stack.
    The only if is clear by stability under dependent sums. The other direction is \ref{thm:quotients}.
    
\end{proof}

\begin{definition}
    A property of morphisms between $n$-stacks is local, if it is satisfied by identities, stable under composition and basechange/descent along \Cov-maps, precomposition/right cancellability with \Cov-maps.
\end{definition}
\begin{lemma}
    Given a local property of types $P$. Then beeing fibered in $P$ is a local property of morphisms.
\end{lemma}
\begin{lemma}[\todocite]
    Given a local property $P$ of morphisms of $n$-stacks, a morphism $f : X \to Y$ has $P$ if there exists an $n$-atlas of $f$ having $P$.
\end{lemma}
\begin{example}
    A morphism of $n$-stacks is covering iff there exists an $n$-atlas of $f$ 
    % https://q.uiver.app/#q=WzAsNCxbMCwwLCJcXFNwZWMgQSJdLFsxLDAsIlxcU3BlYyBCIl0sWzAsMSwiWCJdLFsxLDEsIlkiXSxbMiwzLCJmIl0sWzAsMl0sWzEsM10sWzAsMSwiXFx0aWxkZSBmIiwyXV0=
\[\begin{tikzcd}
	{\Spec A} & {\Spec B} \\
	X & Y
	\arrow["{\tilde f}"', from=1-1, to=1-2]
	\arrow[from=1-1, to=2-1]
	\arrow[from=1-2, to=2-2]
	\arrow["f", from=2-1, to=2-2]
\end{tikzcd}\]
such that $\tilde f$ is a $\bT$-cover.
\end{example}
The previous lemma tells us that we have the correct notion of covering morphisms between  $n$-stacks for $n = 0,1$.


\end{document}

