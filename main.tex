\documentclass{article}
\newcommand{\Cov}{\emph{Cov} }
\newcommand{\affineA}{(affine ?)}
\newcommand{\Cover}{\emph{Cov}er }
\newcommand{\truncation}{$\bT$runcation}
%\newcommand{\Atl}{\emph{Atlas}}
\usepackage{graphicx} % Required for inserting images
\newcommand{\red}[1]{{\color{red} #1}}
\newcommand{\St}{\mathsf{St}}
\newcommand{\CS}{\mathsf{CS}}
\newcommand{\Zar}{\mathsf{Zar}}
\newcommand{\refl}{\mathsf{refl}}
\newcommand{\ap}{\mathsf{ap}}
\newcommand{\Aff}{\mathsf{Aff}}
\newcommand{\Sep}{\mathsf{Sep}}
\newcommand{\Pres}{\mathsf{Pres}}
\newcommand{\EF}{\mathsf{EF}}
\newcommand{\Susp}{\mathsf{Susp}}
\newcommand{\Ann}{\mathsf{Ann}}
\newcommand{\Fin}{\mathsf{Fin} \  }
\usepackage{biblatex}
\addbibresource{form.bib}

%%%%%%%%%%%%%%%%%%%%%%%%%%%%%%%%%%%%%%%%%%%%%%%%
% load packages 

\usepackage[a4paper,nohead,left=3.5cm,right=3.5cm,top=4cm,bottom=3cm]{geometry}
%\usepackage{german}    % only for German articles
%\usepackage{a4wide}   % longer lines
\usepackage[intlimits,tbtags]{amsmath}   % for more basic mathematical symbols
\usepackage{amssymb}   % for more mathematical symbols
\usepackage{amsfonts}
\usepackage[utf8]{inputenc}
\usepackage{textcomp}
\usepackage{mathtools}
% \usepackage{stmaryrd}  % for more mathematical symbols
\usepackage{latexsym}  % for more mathematical symbols,
% already contained in amsmath-package
% \usepackage{accents}  % for more dots etc on symbols
\usepackage{amsxtra}   % for accents as superscripts
\usepackage{amstext}   % for text in formulas, accents, etc.
\usepackage{bm}        % boldface for non-latin letters
\usepackage{amsthm}    % for theorem-environments
\usepackage{amscd}     % for commutative diagrams
%\usepackage{MnSymbol}
%\usepackage[ansinew]{inputenc}
\usepackage{enumitem}  % Better enumerations.

\usepackage{graphicx}
\usepackage[dvipsnames]{xcolor}
\usepackage[arrow, matrix, curve]{xy}
\usepackage[colorlinks=true, citecolor=Blue, linkcolor=Blue, urlcolor=Blue]{hyperref}
\usepackage{xfrac}
\usepackage[utf8]{inputenc}
\usepackage{scalerel}
\newcommand{\tA}{{\hspace{1pt}\sim}_{A}\hspace{1pt}}
\newcommand{\tR}{\hspace{1pt}{\sim}_{R}\hspace{1pt}}
\newcommand{\bA}{\mathbb{A}}
\newcommand{\bB}{\mathbb{B}}
\newcommand{\bC}{\mathbb{C}}
\newcommand{\bD}{\mathbb{D}}
\newcommand{\bE}{\mathbb{E}}
\newcommand{\bF}{\mathbb{F}}
\newcommand{\bG}{\mathbb{G}}
\newcommand{\bH}{\mathbb{H}}
\newcommand{\bI}{\mathbb{I}}
\newcommand{\bJ}{\mathbb{J}}
\newcommand{\bK}{\mathbb{K}}
\newcommand{\bL}{\mathbb{L}}
\newcommand{\bM}{\mathbb{M}}
\newcommand{\bN}{\mathbb{N}}
\newcommand{\bO}{\mathbb{O}}
\newcommand{\bP}{\mathbb{P}}
\newcommand{\bQ}{\mathbb{Q}}
\newcommand{\bR}{\mathbb{R}}
\newcommand{\bS}{\mathbb{S}}
\newcommand{\bT}{\mathbb{T}}
\newcommand{\bU}{\mathbb{U}}
\newcommand{\bV}{\mathbb{V}}
\newcommand{\bW}{\mathbb{W}}
\newcommand{\bX}{\mathbb{X}}
\newcommand{\bY}{\mathbb{Y}}
\newcommand{\bZ}{\mathbb{Z}}
\DeclareMathOperator{\Sh}{Sh}

%%%%%%%%% calligraphic %%%%%%%%%%%%%%%%%%%%%%%
\newcommand{\mc}[1]{\mathcal{#1}}
\newcommand{\R}{\Rightarrow}
\newcommand{\cA}{\mathcal{A}}
\newcommand{\cB}{\mathcal{B}}
\newcommand{\cC}{\mathcal{C}}
\newcommand{\cD}{\mathcal{D}}
\newcommand{\cE}{\mathcal{E}}
\newcommand{\cF}{\mathcal{F}}
\newcommand{\cG}{\mathcal{G}}
\newcommand{\cH}{\mathcal{H}}
\newcommand{\cI}{\mathcal{I}}
\newcommand{\cJ}{\mathcal{J}}
\newcommand{\cK}{\mathcal{K}}
\newcommand{\cL}{\mathcal{L}}
\newcommand{\cM}{\mathcal{M}}
\newcommand{\cN}{\mathcal{N}}
\newcommand{\cO}{\mathcal{O}}
\newcommand{\cP}{\mathcal{P}}
\newcommand{\cQ}{\mathcal{Q}}
\newcommand{\cR}{\mathcal{R}}
\newcommand{\cS}{\mathcal{S}}
\newcommand{\cT}{\mathcal{T}}
\newcommand{\cU}{\mathcal{U}}
\newcommand{\cV}{\mathcal{V}}
\newcommand{\cW}{\mathcal{W}}
\newcommand{\cX}{\mathcal{X}}
\newcommand{\cY}{\mathcal{Y}}
\newcommand{\cZ}{\mathcal{Z}}
\DeclareMathOperator{\chains}{Chains}

%%%%%%%%%%%%% mathematical fraktur  %%%%%%%%%%%%%%%%%%%%%
\newcommand{\mf}[1]{\mathfrak{#1}}
\newcommand{\senk}{\ \big \vert \ }
\newcommand{\fA}{\mathfrak{A}}
\newcommand{\fB}{\mathfrak{B}}
\newcommand{\fC}{\mathfrak{C}}
\newcommand{\fD}{\mathfrak{D}}
\newcommand{\fE}{\mathfrak{E}}
\newcommand{\fF}{\mathfrak{F}}
\newcommand{\fG}{\mathfrak{G}}
\newcommand{\fH}{\mathfrak{H}}
\newcommand{\fI}{\mathfrak{I}}
\newcommand{\fJ}{\mathfrak{J}}
\newcommand{\fK}{\mathfrak{K}}
\newcommand{\fL}{\mathfrak{L}}
\newcommand{\fM}{\mathfrak{M}}
\newcommand{\fN}{\mathfrak{N}}
\newcommand{\fO}{\mathfrak{O}}
\newcommand{\fP}{\mathfrak{P}}
\newcommand{\fQ}{\mathfrak{Q}}
\newcommand{\fR}{\mathfrak{R}}
\newcommand{\fS}{\mathfrak{S}}
\newcommand{\fT}{\mathfrak{T}}
\newcommand{\fU}{\mathfrak{U}}
\newcommand{\fV}{\mathfrak{V}}
\newcommand{\fW}{\mathfrak{W}}
\newcommand{\fX}{\mathfrak{X}}
\newcommand{\fY}{\mathfrak{Y}}
\newcommand{\fZ}{\mathfrak{Z}}
\newcommand{\lp}{_\flat} %{\boldsymbol{\cdot}}
\newcommand{\hp}{^\sharp}
\newcommand{\cp}{\boldsymbol{\cdot}}

\newtheorem{theorem}{Theorem}[section]
\newtheorem{satz}[theorem]{Satz}
\newtheorem{lemma}[theorem]{Lemma}
\newtheorem{korollar}[theorem]{Korollar}
\newtheorem{example}[theorem]{Example}
\newtheorem{prop}[theorem]{Proposition}
\DeclareMathOperator{\Spec}{Spec}
\newtheorem{corollary}[theorem]{Corollary}
\theoremstyle{definition}

\newtheorem{definition}[theorem]{Definition}
\newtheorem{ziel}[theorem]{Ziel}
\newtheorem{frage}[theorem]{Frage}
\newtheorem*{notation}{Notation}
\newtheorem*{slogan}{Slogan}
\newtheorem*{construction}{Construction}
\newtheorem*{bemerkung}{Bemerkung}
\newtheorem*{exercise}{Exercise}

\newtheorem*{note*}{Note}

\newtheorem{rmk}{Remark}
\newtheorem{bsp}[theorem]{Beispiel}
\newtheorem{aufgabe}[theorem]{Aufgabe}
\newtheorem*{beweis}{\it Beweis}
\newcommand{\gray}[1]{{\color{gray} #1}}
%%%%%%%%%%    Math operators    %%%%%%%%%%%%%%%%%%%%%%%%%%%

\DeclareMathOperator{\id}{id}             % identity morphism
% \DeclareMathOperator{\ker}{ker}           % kernel
\DeclareMathOperator{\im}{im}             % image
\DeclareMathOperator{\Hom}{Hom}           % homomorphisms
\DeclareMathOperator{\End}{End}           % endomorphisms
\DeclareMathOperator{\Span}{Span}         % linear span

\usepackage{tikz-cd}
\DeclareMathOperator{\pr}{pr}
\usepackage{quiver}
\renewcommand{\:}{\colon}
\DeclareMathOperator{\isContr}{isContr}

\newcommand{\type}{\ \mathrm{Type}}
\usepackage{stmaryrd}



\newcommand{\op}{^{op}}
%\renewcommand{\subset}{\subseteq}
%\newcommand{\colim}[1]{\mathrm{colim} \limits_{#1}}
\newcommand{\colim}[1]{\underset{#1}{\mathrm{colim} \ }}
\DeclareMathOperator{\sSet}{\mathsf {sSet}}
\DeclareMathOperator{\Pos}{\mathsf {Pos}}
\DeclareMathOperator{\Set}{\mathsf {Set}}
\DeclareMathOperator{\Fun}{Fun}
\DeclareMathOperator{\Cat}{\mathsf {Cat}}
\DeclareMathOperator{\const}{const}
\DeclareMathOperator{\Vect}{\mathsf{Vect}}
\DeclareMathOperator{\Top}{\mathsf{Top}}
\DeclareMathOperator{\Ring}{\mathsf{Ring}}
\DeclareMathOperator{\Field}{\mathsf{Field}}

\DeclareMathOperator{\Ab}{\mathsf{Ab}}
\DeclareMathOperator{\GL}{GL}
\DeclareMathOperator{\Ch}{\mathsf{Ch}}
\DeclareMathOperator{\Grp}{\mathsf{Grp}}

\DeclareMathOperator{\HomC}{\Hom_{\cC}}
\DeclareMathOperator{\HomD}{\Hom_{\cD}}
\usepackage{ascii}
\DeclareMathOperator{\Ob}{Ob}
\DeclareMathOperator{\FinVect}{FinVect}
\setlength\parindent{0pt} % Keine Einrueckung von Absaetzen
\newcommand{\etale}{\' etale }
\newcommand{\Etale}{\' Etale }
\DeclareMathOperator{\fib}{fib}
\newcommand{\todo}{{\color{Red} Todo}}
\newcommand{\todocite}{[ref?]}
\newcommand{\el}{\in}
\usepackage{wasysym}
\newcommand{\ci}{\fullmoon}
\DeclareMathOperator{\isProp}{isProp}
\DeclareMathOperator{\Prop}{Prop}
%\renewcommand{\in}{\colon}

\newcommand{\details}{[...]}
\DeclareMathOperator{\tp}{tp}
\DeclareMathOperator{\Nat}{Nat}
\renewcommand{\contentsname}{Inhalt}
\font\maljapanese=dmjhira at 2.5ex
\newcommand{\yo}{\textrm{\!\maljapanese\char"48}}
\DeclareMathOperator{\Aut}{Aut}
\DeclareMathOperator{\Mod}{\mathsf{Mod}}
\DeclareMathOperator{\Mat}{Mat}

\DeclareMathOperator{\isInv}{isInv}
\DeclareMathOperator{\Alg}{Alg}
\newtheorem{axiom}{Axiom}
\newtheorem{question}{Question}
\renewcommand{\mid}{\ | \ }

\newcommand{\fun}[4]{
	\begin{align*} 
		#1 &\to #2 \\ 
		#3 &\mapsto #4 
\end{align*}}
\newcommand{\funn}[5]{
	\begin{align*} 
		#1 \colon #2 &\to #3 \\ 
		#4 &\mapsto #5
\end{align*}}

\newcommand{\RHom}{R \cH om}
\newcommand{\Ltimes}{\overset{\mathrm{L}}{\otimes}}


\renewcommand{\GS}{\mathsf{GS}}
\title{Thesis}
\author{Tim Lichtnau }
\date{May 2024}
\newcommand{\emdash}{\text{---}}
\begin{document}
\newtheorem*{warning}{Warning}
\newtheorem*{why}{Why I did it this way}
\newtheorem*{think}{Think about}
\maketitle
\tableofcontents
\section{Atlas}
\begin{definition}{\label{def:TAtlas}}
	% Let $T \subset \cU$ be any subtype of the universe. 
	% A $\bT$-cover 
	Given $\cV \subset \cU$ a subclass stable under $\sum$, a $\cV$-cover is a map fibered in $\cV$.
	A $\cV$-atlas of $X$ is a $\bT$-cover $\Spec A \to X$ out of an affine scheme. \\
	In the context of a topology $\bT$, We call a $\cV$-atlas $\Spec A \to X$ a $\cV$-catlas, if the domain $\Spec A$ belongs to $\bT$.
	% of a type $\bT$ is an affine Scheme $\Spec A$ with a formally \etale \etale-surjective map
	% \[
	% \Spec A \to T.
	% \]    
\end{definition}

% \begin{lemma}
%     $\bT$-atlasses are stable under composition.
% \end{lemma}
%\begin{rmk}{\label{rmk:defatlas}}
%	Any good enough TODO scheme has a Zariski atlas. If $\bT$ is finer than the Zariski-topology then in the definition we may replace affine scheme by good enough scheme, if its just about the question whether a type admits an atlas.
%\end{rmk}

\begin{example}
	Let $X$ be a (1-)type. $X$ has a $\Zar$-atlas, iff there exists some $f : \Spec A \to X$ fibered in types of the form $\Spec (R_{f_1} \times \hdots \times R_{f_n})$ for $(f_1,\hdots,f_n) \in Um(R)$. 
\end{example}
\begin{rmk}{\label{ZLCGivesZariski}}
	If one applies ZLC to an affine scheme $\Spec A$ the resulting principal open cover $D(f_i), f_i \in A$ will induce indeed a zariski atlas $\bigsqcup D(f_i) \to \Spec A$, because the fiber over $x : \Spec A$ is $\bigsqcup D(f_i(x))$.
\end{rmk}
Question: Does every zariski atlas of $\Spec A$ have this form? \nameref{ex:weirdZarAtlasses}
%Let $\cZ$ be the class of types which have a Zariski cover.

\begin{example}{\label{ex:PnIsStack}}
	$\bP^n$ has a zariski atlas given by the standart homogeneous principal opens $\sum_{i=0}^n D_+(x_i)$. The fiber over a point $[y_0 : \hdots . y_n]$ is $D(y_0) + \hdots D(y_n)$ where $(y_1,\hdots,y_n) \in Um(R)$. % Indeed the standart principal opens $D_+(x_0) , \hdots , D_+(x_n)$ form a presentable open cover.    
\end{example}

\begin{definition}
	A Zariski sheaf $X$ is a scheme if there merely exists some affine $S$  map $S \to X$ whose fibers are Zariski-merely inhabited finite sums of open propositions 
\end{definition}
\begin{lemma}{\label{lemma:IsScheme}}
	Every $\Zar$-sheaf that admits a $\Zar$-atlas is a scheme. 
\end{lemma}
\begin{proof}
	Obvious.
\end{proof}
% Warning: Let $X$ be a . Then $X$ is already affine if it has a Zariski cover, i.e.  there exists some $f : \Spec A \to X$ fibered in types of the form $\Spec (R_{f_1} \times \hdots \times R_{f_n})$ for $(f_1,\hdots,f_n) \in Um(R)$. More generally: We can facto
\section{HoTT}
\subsection{Subtypes}
\section{Introduction to SAG}
\begin{lemma}{\label{lemma:Locality}}
	$R$ is local, i.e. if $x, y : R$ and $x \neq y$, then $x$ is invertible or $y$ is invertible.
\end{lemma}
\begin{lemma}
	If $char \neq 2$, Let $\rho \neq 0$, then $x^2 = \rho^2$ implies $x = \rho$ or $x = -\rho$
\end{lemma}
\begin{proof}
	Indeed, as $\rho \neq - \rho$ , one of them is invertible by \ref{lemma:Locality} %By locality, $x / 2 - \rho /2$ or $x / 2 + \rho /2 $is invertible.
\end{proof}


Example for zariski local choice

\begin{example}{\label{ex:Divisibility}}
	For some $A$ and $g,g' : A$ define
	\[
	g \ |_A \ g' \equiv \| \{ h : A \mid h g =_A g' \} \|
	\]
	Claim: For any $g,g' : A$, we have 
	\[
	g |_{A} g' \leftrightarrow \forall x : \Spec A , g x |_R g' x
	\]
\end{example}
\begin{proof}
	$\rightarrow$ is obvious using that the duality map is an algebra isomorphism.\\
	$\leftarrow$. For any $x : \Spec A$ we merely find some $h : R$ with $h \cdot g(x) = g'(x)$, i.e. we define our family of inhabited types as 
	\[
	B x =  \{h  : R \mid  h \cdot g(x) = g'(x) \}
	\]
	
	By zariski local choice we merely find some principal open cover $\Spec A = \bigcup_{i=1}^n D(f_i)$ and local sections
	\begin{align*}
		&\prod_{x : D(f_i)} \{h_i  : R \mid h_i \cdot g(x) = g'(x) \} \\
		&\overset{\ref{prop:TTAC}}{\simeq} \{h_i : D(f_i) \to R \mid (h_i x) \cdot g(x) = g'(x) \}\\
		&\overset{ \nameref{Duality}}{{\color{green} \simeq}} \left\{h_i : A_{f_i} \mid h_i \cdot \frac{g}{1} =_{A_{f_i}} \frac{g'}{1} \right \}   
	\end{align*}
	We can multiply $h_i$ by high enough powers of $f_i$ to obtain some $h_i : A$ with $h_i \cdot g = g' \cdot f_i^{n}$ for some $n : \bN$. we may assume that $n$ does not depend on $i = 1 , \hdots, n$ by taking the maximum and multiplying the $h_i$ again with enough powers of $f_i$. Now use \ref{lemma:powersofUnimod} to write $ 1 = \sum_{i=1}^n \ell_i f_i^n$ for some $\ell_i : A$ and then \[
	(\sum_i \ell_i h_i) \cdot g = \sum_i \ell_i f_i^n g' = 1 g' = g'
	\]    
\end{proof}
\section{Topology}
\begin{lemma}{\label{lemma:Equalizer}}
	Let $f, g : X \to Y$ be two functions into a seperated scheme where $X = \Spec A$ for $A$ a reduced ring. If $f$ and $g$ coincide on a dense subset, then $f = g$.
\end{lemma}
\begin{proof}
	The equalizer 
	\[
	Z \equiv \sum_{x : X} f x =_Y g x
	\]
	is a closed subset of $X$, as $Y$ is seperated. As its open complement does not intersect the given dense subset, its empty. In other words $\lnot \lnot Z = X$. Writing $I \subset A$ as the ideal of functions that vanish on $Z$, By \cite{SAGtopology}, we have
	\[
	\Spec A = \lnot \lnot Z = \bigcup_{n} \Spec A / I^n
	\]
	But by the strong boundedness principle, we find some $n$ such that $\Spec A = \Spec A / I^n$, in other words, $I^n = 0$. As $A$ is a reduced ring, we conclude $I = 0$, so $Z = \Spec A / I = \Spec A$.
\end{proof}


\begin{definition}
	A point $0 : \Spec B$ is regular, if $\Spec B \setminus \{0\} = D(p_1,\hdots,p_n)$ for some $p_1,\hdots,p_n : B$ jointly-reguar, i.e. if $p_i^m \cdot b = 0$ for all $i=1,\hdots,n$ then $b = 0$. \\
	If $0 : X$ is a point of a scheme, we call it regular, if one of the following equivalent conditions is satisfied 
	\begin{enumerate}
		\item it admits some open affine neigbhorhood $U$ such that $0 : U$ is regular.\\
		\item It is a regular point of any open affine neigbhorhood.
	\end{enumerate}
\end{definition}
\begin{proof}
	Consider an open affine neigbhorhood $0 : D(f) \subset U = \Spec B$.
	We will show
	\begin{enumerate}
		\item If $0$ is regular in $D(f)$, then it is regular in $\Spec B$:
		Consider $g_1, \hdots, g_n : B$ such that 
		\[
		B_f \to \prod_i B_{f g_i}
		\]
		is injective. Define $g_0 := f - f(0)$, where $0 \not \in D(g_0)$.
		Let us show, that $g_0,\hdots,g_n$ are jointly surjective in $B$.
		Let $b : B$ such that $g_i^n b = 0$ for all $0 \le i \le n$. Then in particular $b / 1 =_{B_f} 0$. Thus $b$ is in the kernel of $B \to B_{g_0} \times B_f$. But $D(g_0) \cup D(f)$ forms an open cover of $\Spec B$ as $(f,g_0)$ generate the unit ideals. Thus $b : \Spec B \to R$ equals 0 on an open cover, thus its 0.
		\item If $0$ is regular in $\Spec B$, then it is regular in $D(f)$:
		Assume $B \to \prod B_{g_i}$ is injective. Let $f : B$. Let us show that $B_f \to \prod B_{g_i f}$ is injective. If $(g_i f)^n b = 0$, then $(g_i)^n (f^n b) = 0$, thus $f^n b = 0$ by assumption. Thus $b / 1 =_{B /f} 0$ as desired.
	\end{enumerate}
\end{proof}
\begin{lemma}{\label{lemma:AlmostEverywhere}}
	If $0 : X$ is a regular point in a scheme, then both holds:
	\begin{enumerate}
		\item $X \setminus \{0\}$ is dense
		\item $R^X \to R^{X \setminus \{0\}}$ is injective. %Probably the converse also holds.
	\end{enumerate}
\end{lemma}

\begin{proof}
	\begin{enumerate}
		\item We write $A \perp B$ for $A \cap B = \varnothing$. We reduce to affine case: Let $0 \in \Spec B \subset X$. Let $U \subset X$ be open such that $U \perp X \setminus \{0\}$. Then $U \perp X \setminus \{0\} \Rightarrow U \perp \Spec B \setminus \{0\} \Rightarrow U \perp \Spec B$ so $U \perp (\Spec B \cup X \setminus \{0\}) = X$, thus $U = \varnothing$. \\
		So we may assume that $X = \Spec B$ is affine: Then by \cite{SAGtopology}, an open subset of $\Spec B$ is dense iff it is of the form $D(g_1,\hdots,g_n)$ for nilregular functions $g_i : B$. Conclude, as regular implies nilregular \cite{SAGtopology}.
		\item Lets first reduce to the affine case. Choose an open affine neibhorhood $U$ of $0$ such that $0 : U$ is regular. Then the surjection $U + X \setminus \{0\} \twoheadrightarrow X$ induces a vertical left injection 
	% https://q.uiver.app/#q=WzAsNCxbMCwwLCJSXntVICsgWCBcXHNldG1pbnVzIFxcezBcXH19Il0sWzAsMSwiUl5YIl0sWzEsMCwiUl57VSBcXHNldG1pbnVzIFxcezBcXH0gKyBYIFxcc2V0bWludXMgXFx7MFxcfX0iXSxbMSwxLCJSXntYIFxcc2V0bWludXMgXFx7MFxcfX0iXSxbMSwwLCIiLDAseyJzdHlsZSI6eyJ0YWlsIjp7Im5hbWUiOiJob29rIiwic2lkZSI6InRvcCJ9fX1dLFszLDJdLFswLDIsIiIsMSx7InN0eWxlIjp7InRhaWwiOnsibmFtZSI6Imhvb2siLCJzaWRlIjoidG9wIn19fV0sWzEsM11d
	\[\begin{tikzcd}
		{R^{U + X \setminus \{0\}}} & {R^{U \setminus \{0\} + X \setminus \{0\}}} \\
		{R^X} & {R^{X \setminus \{0\}}}
		\arrow[hook, from=1-1, to=1-2]
		\arrow[hook, from=2-1, to=1-1]
		\arrow[from=2-1, to=2-2]
		\arrow[from=2-2, to=1-2]
	\end{tikzcd}\]
	So we may assume that $X = \Spec A$ is affine. \\
	Let $p_1,\hdots,p_n : A$ be jointly-reguar, i.e. if $p_i^m \cdot a = 0$ for all $i=1,\hdots,n$ then $a = 0$. If $f : \Spec A \to R$ such that $f(x) = 0$ for all $x \in D(p_1,\hdots,p_n)$, then $f(x) = 0$ for all $x : \Spec A$.
	$f$ is in the kernel of the diagonal map
	% https://q.uiver.app/#q=WzAsNSxbMSwwLCJSXntcXFNwZWMgQX0iXSxbMSwxLCJ7Ul57XFxjb3Byb2RfaSBEKHBfaSl9fSJdLFswLDAsIlxcQSJdLFswLDEsIlx0e1xccHJvZF97aT0xfV5uIEFfe3BfaX19ICJdLFsyLDEsIlJee1xcY3VwIEQocF9pKX0iXSxbMCwxXSxbMiwzLCIiLDAseyJzdHlsZSI6eyJ0YWlsIjp7Im5hbWUiOiJob29rIiwic2lkZSI6InRvcCJ9fX1dLFswLDIsIiIsMSx7ImxldmVsIjoyLCJzdHlsZSI6eyJoZWFkIjp7Im5hbWUiOiJub25lIn19fV0sWzEsMywiIiwxLHsibGV2ZWwiOjIsInN0eWxlIjp7ImhlYWQiOnsibmFtZSI6Im5vbmUifX19XSxbMSw0LCIiLDAseyJzdHlsZSI6eyJ0YWlsIjp7Im5hbWUiOiJob29rIiwic2lkZSI6InRvcCJ9fX1dXQ==
	\[\begin{tikzcd}
		A & {R^{\Spec A}} \\
		{	{\prod_{i=1}^n A_{p_i}} } & {{R^{\coprod_i D(p_i)}}} & {R^{\cup D(p_i)}}
		\arrow[hook, from=1-1, to=2-1]
		\arrow[Rightarrow, no head, from=1-2, to=1-1]
		\arrow[from=1-2, to=2-2]
		\arrow[Rightarrow, no head, from=2-2, to=2-1]
		\arrow[hook, from=2-2, to=2-3]
	\end{tikzcd}\]
	which is injective, as $p_1,\hdots,p_n$ are jointly-regular in $A$. \\
	Thus $f = 0$ in $A$.
		\end{enumerate}
	%	 Thus $f = 0 $ in $(R \setminus \{0\} \to R) = R[X^{\pm 1}]$ hence $f \cdot X^n = 0$ in $R[X]$ for some $n$, thus $f = 0$.
\end{proof}
\begin{rmk}
	If $A$ is an algebra that is reduced as a ring, then for $X = \Spec A$, 1. implies 2. by \ref{lemma:Equalizer}
\end{rmk}
\begin{prop}{\label{prop:NotLocClosed}}
the subtype $\{0\} + 0^c \subset \Spec B$ is not locally closed whenever one of the following conditions is satisfied:
	\begin{enumerate}
		\item 	$\Spec B \setminus \{0\}$ is dense 
		\item $R^{\Spec B} \to R^{\Spec B \setminus \{0\}}$ is injective
	\end{enumerate}
%or  (e.g. both satisfied if $0 : \Spec B$ is a regular point) , Then . %whenever it is a bridge point, i.e. there merely is a regular $g : \Spec B \to R$ vanishing at 0, such that $\fib_g(0)$ is infinitesimal  (i.e. each two points are $\lnot \lnot$-equal). 
	%	
\end{prop}
\begin{proof}
	%We have	$x \neq 0 \leftrightarrow g (x) \neq 0$ %(and by the previous assumption also the converse)
	%	Let us show this more generally for $(\Spec B , 0)$ with the infinitesimal neigbhorhood of 0 beeing not open. \\
	Let us first show, that the infinitesimal neigbhorhood of $0$ is not open. 
	\begin{enumerate}
		\item If $0^c \subset \Spec A$ is dense) :
		The non-empty open $\cN_\infty$  does not intersect the dense subset $0^c$.
		
		\item If $R^{\Spec B} \to R^{\Spec B \setminus \{0\}}$ is injective:
		If it would, we find a principal open smaller neighborhood$0 \in D(g) \subset \cN_\infty(0)$, which however already cotains the whole infinitesimal one, thus $\cN_\infty(0) = D(g)$ \\
		Then for any $x \neq 0$, we have $\lnot \lnot g(x) = 0$. As $\Spec B \setminus \{0\}$ is a scheme, it admits a boundedness principle, thus we find some $n$, such that $g^n (x) = 0$ for all $x \neq 0$. \\ %, in particular for all $x : D(p)$. By \ref{lemma:AlmostEverywhere} applied to the givenregular $p$,  
		by \ref{lemma:AlmostEverywhere} we have that $R^{\Spec B} \to R^{\Spec B \setminus \{0\}}$ is injective, so we deduce $g^n =0$, hence $D(g) = D(g^n) = \varnothing$ contradiction.
	\end{enumerate}
	Just assumg that the infinitesimal neighborhood is not open,	The subtype $\{0\} + 0^c \subset \Spec B$ is not locally closed.	Let $U , C \subset \Spec B$ be an open subset and a closed subset respectively, such that $(x \neq 0) + (x \neq 0) \leftrightarrow x \in U \land x \in C$. Then, for any $x : U$ , 
	\[
	(x = 0) + (x \neq 0) = x \in C
	\]
	is a closed proposition. Thus the decidable subtype $x \neq 0$ is a closed proposition. To contradict the assumption, we may convince ourself that the right vertical map
	% https://q.uiver.app/#q=WzAsNCxbMSwwLCJcdFxcc3VtX3t4IDogXFxTcGVjIEF9IFxcbG5vdCBcXGxub3QgeCA9IDAiXSxbMSwxLCJcXFNwZWMgQSJdLFswLDAsIlxcc3VtX3t4OiBVfSBcXGxub3QgXFxsbm90IHggPSAwIl0sWzAsMSwiVSJdLFsyLDMsIiIsMix7InN0eWxlIjp7InRhaWwiOnsibmFtZSI6Imhvb2siLCJzaWRlIjoidG9wIn19fV0sWzIsMCwiXFxzaW0iLDJdLFswLDFdLFszLDEsIiIsMSx7InN0eWxlIjp7InRhaWwiOnsibmFtZSI6Imhvb2siLCJzaWRlIjoidG9wIn19fV1d
	\[\begin{tikzcd}
		{\sum_{x: U} \lnot \lnot x = 0} & {	\sum_{x : \Spec B} \lnot \lnot x = 0} \\
		U & {\Spec B}
		\arrow["\sim"', from=1-1, to=1-2]
		\arrow[hook, from=1-1, to=2-1]
		\arrow[from=1-2, to=2-2]
		\arrow[hook, from=2-1, to=2-2]
	\end{tikzcd}\]
	is an open embedding
	
	where the upper horizontal map is indeed an equivalence as for any $x : \Spec B$ , $x \in U$ is $\lnot \lnot$-stable, but $\lnot \lnot x = 0$ and $0 \in U$, thus $x \in U$.
	
\end{proof}
\section{Preparation}

\begin{lemma}[Strong boundedness, NEEDED?]{\label{lemma:StrongBoundedness}}
	Consider a sequence of embeddings of types 
	\[
	X_0 \overset{\iota_0}{\hookrightarrow} X_1 \overset{\iota_1}{\hookrightarrow} X_2 \hdots
	\]
	Then any map $f : \Spec A \to \colim{n} X_n \equiv: \bigcup_n X_n$ factors through some $\kappa_m : X_m \hookrightarrow \colim{n} X_n$.
\end{lemma}
\begin{proof}	
	For every term $x : \Spec A$ consider the subset $S_x$ of natural numbers $n$, such that $f(x) \in \im \kappa_m$. Its a merely inhabited upwards closed subset. By the strong boundedness principle \todocite, the subset $\bigcap_{x : \Spec A} S_x$ is merely inhabited.
\end{proof}
\begin{lemma}{\label{lemma:SeqUnionSmooth}}

	Let $Y$ be a type, which admits a jointly surjective family of maps with smooth domain $X_i \to Y$ %together with a sequence of embeddings 
%	\[
%	X_0 \overset{\iota_0}{\hookrightarrow} X_1 \overset{\iota_1}{\hookrightarrow} X_2 \hdots
%	\]
%	making the triangles with $Y$ commute. 
	%With maps, i.e. $ \bigcup_n X_n \twoheadrightarrow Y$. 
	Then $Y$ is formally smooth. %as well as $\bigcup_n \| X_n\|$ is formally smooth. 
\end{lemma}
\begin{proof}
	$ \sum_{n : \bN} X_n \to Y$ is surjective with formally smooth domain, as $\bN$ is formally smooth.
%	 , we may just prove the special case $Y \equiv \sum_n X_n$.
%	Let $P$ be closed dense with a map $P \to \sum_n X_n$. By the strong boundedness principle \ref{lemma:StrongBoundedness} we merely find some $n : \bN$ such that the map factors through $\kappa_n$.
%	By smoothness of $X_n$ there exists some a filler (2.)
%	% https://q.uiver.app/#q=WzAsNCxbMCwwLCJQIl0sWzEsMCwiUl5uIl0sWzEsMSwiXFxtYXRoc2Z7T1BFTn0iXSxbMCwxLCIxIl0sWzAsMl0sWzAsMSwiMS4iLDAseyJzdHlsZSI6eyJib2R5Ijp7Im5hbWUiOiJkYXNoZWQifX19XSxbMCwzXSxbMSwyXSxbMywxLCIyLiIsMSx7ImxhYmVsX3Bvc2l0aW9uIjozMCwic3R5bGUiOnsiYm9keSI6eyJuYW1lIjoiZGFzaGVkIn19fV0sWzMsMiwiMy4iLDEseyJzdHlsZSI6eyJib2R5Ijp7Im5hbWUiOiJkYXNoZWQifX19XV0=
%	\[\begin{tikzcd}
%		P & {X_n} \\
%		1 & {\sum_n X_n}
%		\arrow["{1.}", dashed, from=1-1, to=1-2]
%		\arrow[from=1-1, to=2-1]
%		\arrow[from=1-1, to=2-2]
%		\arrow[from=1-2, to=2-2]
%		\arrow["{2.}"{description, pos=0.3}, dashed, from=2-1, to=1-2]
%		\arrow["{3.}"{description}, dashed, from=2-1, to=2-2]
%	\end{tikzcd}\]
%	By composition we obtain a filler $1 \to \sum_n X_n$.
\end{proof}
\begin{corollary}[Monoid is smooth]{\label{lemma:SmoothMonoid}}
	Let $(Y , +)$ be a magma, which is generated by a map with smooth domain $f : X \to Y$, i.e. every $a : Y$ can merely be written as a finite sum \[
a = f(x_1) + \hdots + f(x_n)
\]
Then $Y$ is formally smooth.
\end{corollary}
\begin{lemma}{\label{lemma:havingAbstractAtlasClosedUnderId}}
	Let $C$ be a class of types stable under $\sum$. Let $\bP \subset \Aff$ (in most cases $\bP := \Aff$) be any subclass of affines stable under finite limits.  %HOPE WE DONT NEED (because we want to apply it to \cV = covering stacks) finite limits, i.e. containing 1, stable under dependent sums and finite limits
	The class $\mathsf{HasAtlas}_C^\bP$ of types $Y$ which admit a map $\bP \ni S \to Y$ fibered in $C$ is stable under identity types. \\
	
	%.  If it contains $C$ and is closed under dependent sums, then it is closed under taking identity types.
\end{lemma}
\begin{proof}
%	Obviously 1 has an atlas, and the class of types admitting an atlas is stable by $\sum$ by \ref{thm:atlasStableSum}.
%	It remains to show, that identity types in $Y$ have an atlas provided that $Y$ has an atlas.
	
	%This is a special case of stability under dependent sums. But lets prove it anyway.
	By assumption we can choose a map $\bP \ni V \overset{p}{\to} Y$ fibered in $C$. Let $y,y' : Y$.  Then we have the map
	\begin{align*}
		(\fib_p y) \times_V (\fib_p y') &\to y = y' \\
		(v , q : y = p v) , (v', q' : y' = p v') , (h : v = v') &\mapsto q \cdot h \cdot q'^{-1}
	\end{align*}
	
	The fiber over $j : y = y'$ looks like  %because $y$ and $y'$ are free we may only show the statement for the fiber over the path  $\mathsf{refl}_y : y = y$. 
	\[
	\sum_{v}  ( \underbrace{\sum_{v'} (h : v = v')}_{\mathsf{isContr}}) \times (q : y = p v) \times (q'  : y' = p v') \times (q \cdot h \cdot q'^{-1} = j) \simeq \sum_v (v = py) \simeq \fib_p y
	\]
	Hence the map is fibered in $C$. It suffices to show, that	$(\fib_p y) \times_V (\fib_p y')$ has an atlas, because then we can compose such an atlas with the above map to obtain an atlas of $y = y'$.
	By assumption the fibers of $p$ have an atlas, so we can choose $q : W \to \fib_p y, q' : W' \to \fib_p y'$ atlasses. Then $W \times_V W' \to (\fib_p y) \times_V (\fib_p y')$ is an atlas: The domain is a fiber product of types in $\bP$, hence it belongs to $\bP$. The fiber over $(x,x')$ is equivalent to the product of fibers $(\fib_q x) \times (\fib_{q'} x')$ which is in $C$ by stability under dependent sums (so in particular under finite products).
	
\end{proof}
\begin{lemma}{\label{lemma:AtlasSum}}
	Let $\cU' \subset \cU$ be stable under dependent sums.
	Let $X$ be a type with a  map $p : U \to X$ fibered in $\cU'$.  For any $x : X$, let $Y_x$ be a type and moreover for any $u : U$, we are given a map $q_u : V_u \to Y_{p(u)}$ fibered in $\cU'$. Then the induced map
	\[
	p : \sum_{u : U} V_u \to \sum_{x : X} Y_{x}
	\]
	is fibered in $\cU'$
\end{lemma}
\begin{proof}
	The fiber of $p$ over some $(x,y) \in \sum_{x :X} Y_x$ is given by
	\[
	\sum_{u : \fib_p x} \fib_{q_u} (y') 
	\]
	where $y' : Y_{p(u)}$ (depending on $u$) is the transport of $y : Y_x$ along $x = p(u)$. As $\cU'$ is stable under dependent sum %(\ref{lemma:LexSumStable}, \ref{lemma:LexStability}), 
	those fibers are again in $\cU'$. This shows the result.
\end{proof}

\section{(Lex) Modalities}
\begin{lemma}[Stability resuls]{\label{lemma:ModalityStability}}
	Modalities are stable under 
	\begin{enumerate}
		\item Conjunction
		\item Composition
	\end{enumerate}
	
\end{lemma}
\begin{lemma}{\label{lemma:ModalitySumStable}}
	Let $\ci$ be a modality. Let $X$ be $\ci$-modal and $B : X \to \cU_{\ci}$ be a family of modal types. Then $\sum_{x : X} B_x$ is $\ci$-modal
\end{lemma}
\begin{lemma}{\label{lemma:mod_comm_sum}}
	Let $B  : \bullet X \to \cU$. Then $\bullet (\sum_{x : X} B (\eta x)) = \sum_{x : \bullet X} \bullet B x$
\end{lemma}
\begin{proof}
	Observe that 
	\[
	\sum_{x : X} B \eta x \to \sum_{x : \bullet X} B x
	\]
	is a $\bullet$-equivalence, because for all modal types $T$, the type $B x \to T$ is modal for any $x : \bullet X$. \\
	Then it follows by \todocite.
\end{proof}
\begin{lemma}{\label{lemma:idTypesOfSheafification}}
	Let $\bullet$ be a lex modality. Let $x , y : X$. The map
	\[
	\bullet(x = y) \to \eta x =_{\bullet X} \eta y
	\]
	induced by $ap_\eta : x = y \to \eta x =_{\bullet X} \eta y$ is an equivalence
\end{lemma}
\begin{proof}
 	By Modalities Theorem 3.1 [ix].
\end{proof}
\begin{definition}{\label{lemma:sep}}
	Let $\bullet$ be a lex modality. we call a type $X$ $\bullet$-seperated if one of the following equivalent conditions hold
	\begin{itemize}
		\item the identity types of $X$ are modal
		\item the unit $X \to \bullet X$ is an embedding
	\end{itemize}
In this case
\end{definition}
\begin{proof}

	by \ref{lemma:idTypesOfSheafification} the vertical map in the commutative diagram
	% https://q.uiver.app/#q=WzAsMyxbMCwwLCJ4ID1fWCB5Il0sWzEsMCwiTCh4PXkpIl0sWzEsMSwiXFxldGEgeCA9X3tMWH0gXFxldGEgeSJdLFswLDIsImFwX3tcXGV0YV9YfSIsMl0sWzEsMiwiXFxzaW1lcSJdLFswLDEsIlxcZXRhX3t4PXl9Il1d
	\[\begin{tikzcd}
		{x =_X y} & {L(x=y)} \\
		& {\eta x =_{LX} \eta y}
		\arrow["{\eta_{x=y}}", from=1-1, to=1-2]
		\arrow["{ap_{\eta_X}}"', from=1-1, to=2-2]
		\arrow["\simeq", from=1-2, to=2-2]
	\end{tikzcd}\]
is an equivalence.
	So $x = y$ is a sheaf if $\eta_{x=y}$ is an equivalence iff $\eta_X$ is an embedding.
\end{proof}
\begin{lemma}{\label{lemma:SheavesHaveDescent}}
	If $\bullet$ is a lex modality, then $\bullet U$ is modal.
\end{lemma}

\section{Local Choice}
One of the goals of this chapter is to show descent for types admitting a $\bT$-(c)atlas.
In this section let $\bT$ denote a topology finer than the zariski topology.
\begin{definition}
	Let \Cov be a class of morphisms (which we think of $n$-atlasses of some $n$), containing $\bT$-atlas, (stable under pullback NECESSARY TODO?)
	A type $S$ has \emph{local choice} wrt \Cov if for any $\bT$ -surjective map $X \to Y$ and any map $f : S \to Y$ there exists a map  $p' : S' \to S$ in \Cov and a commutative diagram
	% https://q.uiver.app/#q=WzAsNCxbMCwwLCJUJyJdLFswLDEsIlQiXSxbMSwwLCJYIl0sWzEsMSwiWSJdLFsxLDNdLFsyLDNdLFswLDIsIiIsMix7InN0eWxlIjp7ImJvZHkiOnsibmFtZSI6ImRhc2hlZCJ9fX1dLFswLDFdXQ==
	% https://q.uiver.app/#q=WzAsNCxbMCwwLCJUJyJdLFswLDEsIlQiXSxbMSwwLCJYIl0sWzEsMSwiWSJdLFsxLDNdLFsyLDMsInAiLDJdLFswLDIsIiIsMix7InN0eWxlIjp7ImJvZHkiOnsibmFtZSI6ImRhc2hlZCJ9fX1dLFswLDFdXQ==
	\[\begin{tikzcd}
		{S'} & X \\
		S & Y
		\arrow[dashed, from=1-1, to=1-2]
		\arrow[from=1-1, to=2-1]
		\arrow["p"', from=1-2, to=2-2]
		\arrow["f",from=2-1, to=2-2]
	\end{tikzcd}\]
	%We say $S$ has affine local choice, if one can arrange $S'$ to be affine.
\end{definition}
\begin{prop}{\label{prop:LocalChoice}}
	%Let $\bT$ be a finer topology than the zariski topology.
	Assume that \Cov is stable under composition. %and that Zariski-covers are in \Cov.
	\begin{itemize}
		\item If $\hat S \to S$ is a \Cover and $\hat S$ has $\bT$-local choice, then $S$ has $\bT$-local choice. 
		\item Affine schemes have $\bT$-local choice.
		\item Any type admitting a \Cov - Atlas $\Spec A \to S$ has $\bT$-local choice.
	\end{itemize}
%	$S$ has  $\bT$-local choice wrt \Cov if it has a projective \Cover, i.e. there exists a projective (or, assuming ZLC, affine scheme resp.)  $\hat{S}$ with a map $g : \hat{S} \to S$ in \Cov. %, satisfying local choice wrt \Cov
\end{prop}
\begin{proof}
	%We may assume that $f = \id_S$.
	The first point follows from stability under composition of \Cov. %  We may assume that $g : \hat{S} \to S$ is the identity.
	the third point follows from the second. 
	By the first point, we may assume that $S$ is affine.
	As $p$ is $\bT$-surjective, for any $x : S$ there merely is a $\Spec B_x \in T$  and a map $\Spec B_x \to \| \fib_p (x) \| $. 
	%Claim: No matter on the assumptions (on $S = \hat{S}$), there exists a Zariski cover $S' \overset{p'}{\to} S$ with $S'$ projective (affine resp.) 
	As $S$ is projective, we have a term in
	\[\prod_{x : S} \sum_{\Spec B_x \in T} \Spec B_x \to \| \fib_p (f x)\| \]
	%	Proof: In the case of projectivity, just use $p' = \id_S$ and in the case of having ZLC and $S$ beeing affine, use ZLC (\ref{ZLCGivesZariski}). \qed(Claim)\\    
	By setting 
	\[(S' := \sum_{x : S} \Spec B_x) \overset{\pi}{\longrightarrow} S \]
	
	the projection, we are now in the situation that for any $t : S'$ we merely have a point in $\fib_p((p'(t)))$ and $S' \to S$ is a $\bT$-cover, thus it is in \Cov. Moreover, $S'$ is affine, as it is a dependent sum of affines. Hence again we now can find a lift $S' \to X$ %By replacing $S''$ again with a Zariski cover we find a lift $S'' \to X$     
	making
% https://q.uiver.app/#q=WzAsNCxbMCwwLCJTJyJdLFswLDEsIlMiXSxbMSwxLCJYIl0sWzEsMCwiWSJdLFsxLDIsImYiXSxbMywyLCJwIiwyXSxbMCwzXSxbMCwxLCJwJyIsMl1d
\[\begin{tikzcd}
	{S'} & Y \\
	S & X
	\arrow[from=1-1, to=1-2]
	\arrow["{p'}"', from=1-1, to=2-1]
	\arrow["p"', from=1-2, to=2-2]
	\arrow["f", from=2-1, to=2-2]
\end{tikzcd}\]	commute. %Now $S'' \to S' \to S$ as the composition of Zariski-covers and \Cover is a \Cover \details as desired.
	%For the general case, the previous proof is enough. \todo
\end{proof}
The next lemma shows, that the class of types equipped with a $\bT$-atlas is stable under dependent sums.

\begin{theorem}{\label{thm:atlasStableSum}}
	Let $\cU'$ be a class stable under dependent sums.
	The class of types admitting a $\cU'$-atlas is closed under dependent sums. If $\bT$ is a topology, the same holds for $\cU'$-catlasses.
\end{theorem}
\begin{proof}
	The stability under quotients is easy: 
	Let us construct some atlas $\Spec A \to \sum_{x : X} B_x$ %fibered in smooth $n$-stacks.
	For any $x : X$ we merely have an atlas $V_x \to B_x$, i.e. with $V_x$ affine. %fibered in smooth $n$-stacks . \\
	$X$ has local choice wrt atlasses by (\ref{prop:LocalChoice}) using $\cU'$ is $\sum$-stable (we use the trivial topology).\\
	If additionally, all the $B_x$ and $X$ are smooth $n$-stacks, just observe that we can choose the affine $V_{p u}$ to lie in $\bT$, Accordingly $\sum_{u : U} V_{p u} \in T$ as $\bT$ is stable under $\Sigma$.

%	$n$-atlasses contain zariski-atlasses, because $\bT$ is finer than the Zariski topology.
%	$\cU' are stable under dependent sum by induction, thus $n$-atlasses are stable under composition.         
%	\qed(Claim)\\
	By Local choice for $X$, we merely find $U$ affine, an atlas $p : U \to X$ % fibered in smooth $n$-stacks 
	with
	\[
	\prod_{u : U} \sum_{V_{p(u)} \in T} (q : V_{p(u)} \to B_{p(u)}) \times (q \ \text{fibered in smooth } n \text{ stacks } )
	\]
	Now the desired map is $\sum_{u : U} V_{p u} \to \sum_{x : X} B_x$, because it is  an atlas %fibered in smooth $(n)$-stacks 
	by \ref{lemma:AtlasSum} \\
\end{proof}
\begin{prop}{\label{prop:atlasStableCover}}
	Let $\cU'$ be a class stable under dependent sums.
	The class of types admitting a $\cU'$-(c)atlas is closed under $\cU'$-covers: If $X \to Y$ is a $\cU'$-cover, then $X$ admits a $\cU'$-(c)atlas iff $Y$ admits a $\cU'$-(c)atlas.
\end{prop}
\begin{proof}
One direction is the stability under dependent sums. For the other, if $S \to X$ is a $\cU'$-atlas, then $S \to X \to Y$ is a $\cU'$-atlas by $\sum$-stability of $\cU'$.
\end{proof}

\begin{corollary}{\label{cor:DescentCatlas}}
	If $\bT$ has descent, The class of sheaves merely admitting a $\bT$-catlas has descent.
\end{corollary}
\begin{proof}
	We can set $\cV = \cU$, and we have to show, that if $X \to Y$ is a $\bT$-cover than $X$ admits a $\bT$-catlas iff $Y$ admits a $\bT$-catlas. This follows from \ref{prop:atlasStableCover}.
\end{proof}

\section{Covering stacks}
Fix $\bT$ a topology, which we call the covering-affines.
\begin{definition}
	Covering geometric stacks are the smallest class containing contractible Types such that: If $Y$ is a stack and $\bT \ni S \to Y$ is fibered in covering geometric stacks, then $Y$ is a covering geometric stack.	
\end{definition}
We call such map $X \to Y$ whose fibers are covering stacks a geometric cover. If $X$ is affine we call it a geometric atlas. If $X$ is in $\bT$ we call it a geometric catlas. 
\begin{definition}
	We call $X$ a geometric stack if it merely has a geometric atlas, i.e some $\Spec A \to X$ fibered in covering geometric stacks.
\end{definition}
\begin{prop}[Recursion principle for (covering) geometric stacks]
	Let $P$ be a property of (covering) geometric stacks. Assume
	\begin{itemize}
		\item contractibles have $P$
		\item If $S$ is (covering) affine and $S \to Y$ is fibered in covering stacks having $P$ then $Y$ has $P$
	\end{itemize}
	Then every (covering) geometric stack has $P$.
\end{prop}
%\begin{proof}
%	Replace $P$ by $P \land \mathsf{is-covering-stack}$. Then usual induction
%	
%\end{proof}
\begin{why}
	Should $P$ be defined more generally for all sheaves?
	No, because we want for the recursion principle for geometric stacks, that the fibers are covering stacks (proof of truncatedness).
%	If $P$ is defined only for covering stacks, do we need to talk about $P$-covers between covering stacks without knowing that the fibers are covering stacks as well?
\end{why}
\begin{prop}{\label{prop:csHasAtlas}}
Every covering geometric stack $X$ merely admits a geometric catlas. %, i.e. a geometric cover $Y \to X$ with $Y \in \bT$. 
\end{prop}

\begin{proof}
%We apply the recursion principle of (covering) geometric stacks 
\begin{itemize}
	\item If 	$X$ is covering affine, then $X \to X$ is a geometric catlas. 
	\item If $X$ is obtained as a quotient then it already is equipped with a catlas. %, i.e. if its equipped with a cover $Y \to X$ with $Y$ a covering stack, then by induction $Y$ admits a $\cV$-catlas $S \to Y$. Then $S \to Y \to X$ is a $\cV$-catlas by  \ref{lemma:coversstableundercomp}. \\
	%\item If $X$ is obtained as a sum, i.e. we have a $\cV$-cover $f : X \to Y$, then by induction $Y$ admits an $\cV$-catlas $g : S \to Y$ and the fibers merely have $\cV$-catlasses $S_y \to \fib_f y$ s. By choice of $S$, we can choose such catlasses $S_{g s} \to \fib_f (g s)$ for all $s : S$. By \ref{lemma:AtlasSum} the map 
	%\[
	%\sum_{s : S} S_{gs} \to (\sum_{y: Y} \fib_f y ) \simeq X
	%\]
	%is a $\cV$-cover. Its domain is a covering affine as $\bT$ is $\sum$-stable. Hence $X$ admits a $\cV$-catlas .
	
\end{itemize}
\end{proof}

\subsection{Needing finitely many steps}
%\subsection{Truncatedness}
%In this subsection we want to prove that every geometric stack is a geometric $n$-stack for some $n$.

\begin{lemma}{\label{lemma:cstinh}}
	Every covering stack $X$ is $\bT$-merely inhabited.
\end{lemma}
\begin{proof} 
	\begin{itemize}
		\item If $X$ is in $\bT$ then its clear.
		\item  If $X$ is obtained by a quotient, we have a map $\Spec A \to X$ with domain in $\bT$. Now use that we get a map on $\bT$-propositional-truncations and that Spec A is T-merely inhabited.
		%		\item if $X$ is obtained by  $X = \sum_{y: Y} B y$ for $Y$ a covering $\cV$-stack and $B y$ covering $\cV$-stacks, by induction all the $B y$ are $\bT$-merely inhabited. Hence, for all $y : Y$, we can conclude $\| X\|_\bT$. As $Y$ is $\bT$-merely inhabited by induction and the goal is a stack, we can conclude. 
	\end{itemize}
\end{proof}
\begin{prop}{\label{prop:FindCommonN}}
	Given a geometric stack $Y$ and a family $M : Y \to (\bN \to \Prop_{\ci})$  be a family of upwards closed merely inhabited subsets of $\bN$. Then there exists some $n$, such that $M y n$ for all $y : Y$.
\end{prop}
\begin{proof}
	Write $M_n = \{y : Y \ | \ M y n\}$.
	Choose a geometric atlas $f : S \to Y$.
	For any $x : S$, $M(f x) n$ for some $n$. By foundations Prop 3.3.5, we merely find some $n$ such that $f(x) \in M_n$ for all $x : S$. Let us show that for general $y : Y$ we have $y \in M_n$. Using that $y \in M_n$ is modal , we can conclude by $\bT$-surjectivity of $f$, which follows from \ref{lemma:cstinh}
	
\end{proof}
\begin{prop}{\label{prop:OneToRuleThemAll}}
	Let $W : \GS \to (\bN \to \Prop_{\ci})$ be a family of upwards closed subsets of $\bN$. Assume
	\begin{itemize}
		\item $W 1$ is merely inhabited
		\item whenever there is some $n : \bN$ and a geometric atlas $S \to X$ fibered in covering stacks $F$ satisfying $W F n \equiv: W_n F$, then $W_{n+1} X$.
	\end{itemize}  %(or more generally $W X$ is merely inhabited). 
	Then for any $X \in \GS$, $W X$ is merely inhabited.
\end{prop}

\begin{proof}
		We apply the recursion principle for geometric stacks.
	\begin{itemize}
		\item If $Y$ is contractible its clear by assumption
		\item Assume $Y$ is equipped with a geometric atlas $f : S \to Y$, such that every fiber has $W_n$ for some $n$. Apply \ref{prop:FindCommonN} to $M y n = W_n (\fib_f y)$ to find some $n$ such that $W_n (\fib_f y)$ for all $y : Y$.
		Then we can conclude by applying the assumption.
	
		%	\item Let $X$ be an $n$-truncated covering geometric stack. By \ref{prop:csHasAtlas} we find a geometric catlas $S \to X$. All the fibers are (at most) $n$-truncated. 	
	\end{itemize}
\end{proof}


\begin{corollary}
	Define \begin{align*}
		W_0 &\equiv \bT \\
		W_{n+1} &\equiv \| \sum \bT \ni S \to X \text{ fibered in covering stacks having } W_n \|_\bT
	\end{align*}
	Then every covering geometric stack has $W_n$ for some $n$.
\end{corollary}
\begin{proof}
	We need that $W_n$ is a sheaf 
\end{proof}
\begin{why}
	$W 0 $ is not defined as $\isContr$, because for $\sum$ stability later, we want to apply \ref{thm:atlasStableSum}, so we need that Zariski covers are allowed covers.
\end{why}
\subsection{Stability}
\begin{theorem}{\label{thm:CSSum}}
	The class of (covering) geometric stacks is $\sum$-stable.
\end{theorem}
\begin{proof}
	The geometric case follows from the covering geometric case by \ref{thm:atlasStableSum}.
	Let $X$ be a covering stack and $B : X \to \CS$ a family of covering stacks.
	By \ref{prop:FindCommonN} we merely find an $n : \bN$ such that $W_n (B x)$ for all $x : X$. By making $n$ larger, we may assume $X$ has $W n$ for some $n$. \\
	So we are left to show that for all $n : \bN$, $W_n$ covering stacks are $\sum$-stable. \\
	
	
	Induction over $n$. If $n = 1$, then this is the stability under $\sigma$ of $\bT$ \\
	If we wish to prove the statement for $n+1$, we may assume that $W_n$ covering stacks are $\sum$-stable. We have $\Zar \subset \bT \subset W_n$. So we can apply \ref{thm:atlasStableSum}. \\

\end{proof}
\begin{lemma}{\label{lemma:coversstableundercomp}}
	geometric covers are stable under composition.
\end{lemma}
\begin{proof}
	covering stacks are stable under $\sum$.
\end{proof}


\begin{prop}{\label{prop:stackQuot}}
	The class of (covering) geometric stacks is stable under quotients: If $X \to Y$ is fibered in covering stacks and $X$ is a (covering) stack and $Y$ is a stack then $Y$ is a (covering) geometric stack.
\end{prop}
\begin{proof}
	Choose a geometric (c)atlas of $X$. Then the composition with the map $X \to Y$ is a cover by \ref{lemma:coversstableundercomp}. As the domain is (covering) affine, its a geometric (c)atlas.
\end{proof}
Now we want to show that the clash of terminology regarding 'covering' is reasonable:


\begin{prop}{\label{prop:affineCoveringStack}}
	Let $\bT$ be saturated.
	A covering stack $X$ is affine iff its a covering affine.
\end{prop}
\begin{proof}
	The converse is clear. The direct direction follows by the recursion principle. choosing a geometric catlas  $S \to X$. As both $S$ and $X$ are affine the fibers are affine. By induction the fibers are covering affines. By saturatedness of the topology $X$ is covering affine.
\end{proof}
\begin{lemma}{\label{lemma:atlasIsCatlas}}
	Let $\bT$ be saturated. Let $X$ be a covering stack. Let $f : \Spec A \to X$ be a geometric atlas. Then $\Spec A \in \bT$
\end{lemma}
\begin{proof}
	As $\Spec A \simeq \sum_{x : X} \fib_f x$ is a dependent sum of covering stacks, it is a covering stack again by \ref{thm:CSSum}. We conclude by \ref{prop:affineCoveringStack}.
\end{proof}	

%\subsection{Geometric stacks}

\begin{lemma}{\label{lemma:geometricStacksClosedUnderId}}
	geometric stacks are closed under $\id$-types.
\end{lemma}
\begin{proof}
	
	This is \ref{lemma:havingAbstractAtlasClosedUnderId}, using that covering stacks are closed under $\sum$ (\ref{thm:CSSum})
\end{proof}

\begin{warning}
	The previous lemma does not hold for covering stacks: Identity types of things in $\bT$ could be empty.
\end{warning}

\subsection{About the covering stacks in a subuniverse}
\begin{definition}
	Let $\cV \supset \mathsf{Aff}$ be a superclass stable under $\sum$. covering geometric $\cV$ stacks are the smallest intermediate class $\bT \subset \CS_\cV \subset \cV$ such that: If $X : \bT$ ,  $Y : \cV$ and $X \to Y$ is fibered in $\CS_\cV$, then $Y \in \CS_\cV$. \\
	$X$ is a geometric $\cV$-stack if its in $\cV$ and it merely admits a map $\Spec A \to X$ fibered in $\CS_\cV$.
\end{definition}
%We call such map $X \to Y$ whose fibers are covering geometric $\cV$-stacks a geometric-$\cV$-cover. If $X$ is affine we call it an geometric-$\cV$ atlas. If $X$ is in $\bT$ we call it a geometric-$\cV$-catlas. 
\begin{definition}
	We define the saturation of $\bT$ as the class of covering Aff-stacks. We call a topology $\bT$ saturated if it coincides with its saturation, or more concretely: Every affine schemes that has a catlas lies itself in $\bT$. \\ 
\end{definition}
In a further chapter we will develop this theory further.



\begin{prop}{\label{prop:coveringVstackDescr}}
	Let $\cV$ be stable under finite limits and containing (covering) affines. $X$ is a (covering) $\cV$-stack iff it is in $\cV$ and a (covering) geometric stack.
\end{prop}
\begin{proof} 	
	The direct direction is clear. For the converse we apply the recursion principle to the property '$X \in \cV$ implies $X$ is a (covering) $\cV$-stack'. If $X$ is contractible, its clear. Otherwise its equipped with a geometric (c)atlas. The fibers are in $\cV$, as they can be written as a fiberproduct of $S, X, 1 \in \cV$.  By induction all fibers are covering $\cV$-stacks (we may show the covering part of the proposition first). %We are left to show that $F$ is a covering $\cV$-stack. \\
	%	We can choose $S \to F$ a $\cV$-atlas, so in particular a geometric atlas of $F$, which was assumed to be a covering geometric stack. Then $S \in \bT$ by \ref{lemma:atlasIsCatlas}. So we actually have a $\cV$-catlas.
\end{proof}
\begin{prop}{\label{prop:CSVSum}}
	(covering) $\cV$-stacks are stable under dependent sums. In particular the saturation of a topology defines a topology.
\end{prop}
\begin{proof}
	Both the classes $\cV$ and (covering) stacks are stable under dependent sums. Hence the intersection of them is $\sum$-stable as well. \\
	The saturation is a class of affines, that in particular contains $1 \in \bT$. We have argued its stable under $\sum$.
\end{proof}
\begin{prop}{\label{prop:V'stacks}}
	A stack $X$ merely admits some map $S \to X$ out of a (covering) affine fibered in covering $\cV$-stacks, iff its a (covering) geometric stack whose identity types are in $\cV$. 
\end{prop}
\begin{proof}
	The direct direction: By \ref{lemma:havingAbstractAtlasClosedUnderId} the identity types are geometric $\cV$-stacks. \\
	The converse direction: Choose a geometric (c)atlas $f : S \to X$. As each fiber $\sum_{s : S} f s =_X x$ is in $V$ by $\sum$-stability of $\cV$ and is a covering stack, its a covering $\cV$-stack by \ref{prop:coveringVstackDescr}.
\end{proof}
\begin{definition}
	Let $n \ge -2$. A (covering) geometric $n$-stack is a (covering) geometric stack that is an $n$-type.
\end{definition}
\begin{prop}
	Let $X$ be a stack. For all $n \ge 0$, the following are equivalent:
	\begin{enumerate}
		\item $X$ is a (covering) geometric $n+1$-stack
		\item $X$ merely admits some map $S \to X$ out of a (covering) affine fibered in covering $n$-stacks
		\item $X$ merely admits some (covering) geometric $n$-stack $Y$ and a map $Y \to X$ fibered in covering $n$-stacks.
	\end{enumerate}
\end{prop}
\begin{proof}
	\
	\begin{enumerate}
		\item[1 . $\Leftrightarrow$ 2.]
		$X$ is a (covering) geometric $n+1$ stack iff its a (covering) geometric stack whose identity types are $n$-types. But this is equivalent to 2. by \ref{prop:V'stacks}.
%		\begin{align*}
%			& \text{$X$ is a (covering) geometric $n+1$ stack} \\
%			&\overset{ \ref{lemma:geometricStacksClosedUnderId}} {\Leftrightarrow} \text{$X$ is a (covering) geometric stack whose identity types are $n$-types} \\
%			&\overset{\ref{prop:V'stacks}} {\Leftrightarrow} \text{2.}
%		\end{align*}
		\item[2 . $\Rightarrow$ 3.]
		$S$ is a (covering) geometric $n$-stack
		\item [3. $\Rightarrow$ 2]
		$Y$ admits a map $S \to Y$  fibered in covering $n$-stacks with $S$ (covering) affine, so the composition $S \to X$ will have the same property by \ref{lemma:coversstableundercomp}.
	\end{enumerate}
\end{proof}

\subsection{Truncatedness}
\begin{lemma}{\label{lemma:truncTrg}}
	Let $X$ be an $n+1$-type and $Y$ a stack. If $X \to Y$ is a $n$-truncated $\bT$-surjective map, then $Y$ is an $n+1$-type.
\end{lemma}
\begin{proof}
	Use that $\mathsf{is-n-truncated} (y=y')$ is a stack for $y , y' : Y.$
\end{proof}

\begin{corollary}
	Every geometric stack is $n$-truncated for some $n : \bN$.
\end{corollary}
\begin{proof}
	Apply the prop \ref{prop:OneToRuleThemAll}. Use \ref{lemma:truncTrg}. For a stack $X$, is-$n$-truncated $X$ is indeed a stack.
\end{proof}


\subsection{Descent}
For this subsection lets assume $\cV$ a subuniverse (stable under $\sum$), that satisfies: \\
If $Y \in \cV$ is seperated, then $L_\bT Y \in \cV$. (*) \\
$\St$ a class of sheaves in $\cV$, such that $\bT$ is contained in it and for any $\bT$-cover $X \to Y$ of sheaves in $\cV$, $X \in \St$ iff $Y \in \St$. We call types in this class stacky.
\begin{lemma}{\label{lemma:sheafificationHasTCover}}
	Let $\bT$ satisfy descent, i.e. beeing affine in the topology is a sheaf. If $Y$ admits a $\bT$-cover $f : X \to Y$ where $Y \in \cV$ is seperated, then there is a $\bT$-cover $X \to L_\bT Y$.
\end{lemma}
\begin{proof}
	
	Consider $X \overset{f}{\to} Y \overset{\eta}{\to} L_\bT Y$. As beeing affine in $\bT$ is  a sheaf, we may just show that for all $y : Y$ , the fibers over $\eta y : L_\bT Y$ are in $\bT$. As $\eta$ is a monomorphism by \ref{lemma:sep} , $\eta$ restricts to an equivalence
	\[
	\fib_f y \to \fib_{ \eta f}(\eta y)
	\]
	
	But the left hand side is in $\bT$ by assumption. 
\end{proof}
\begin{lemma}
	Assume $\bT$ have descent.
	Let $X \in \St$ and $Y \in \cV$.	Let $f : X \twoheadrightarrow Y$ be fibered in $\bT$ and surjective. Then $L_\bT Y$ is stacky.
\end{lemma}
\begin{proof}
	As $X$ is stacky, it suffices to show, that $L_\bT Y$ admits a $\bT$-cover.
	We want to apply \ref{lemma:sheafificationHasTCover}. So it remains to show, that $Y$ is seperated, because then we  also know $L_\bT Y \in \cV$ by (*). By surjectivity of $f$ we may only show that for any $x : X, y : Y$, the type $f x =_Y y$ is a sheaf. If we define $U$ to be the fiber over $y$, it is in $\bT$ by assumption. But then $f x =_Y y$ is the outer pullback
	% https://q.uiver.app/#q=WzAsNixbMCwwLCJmIHggPSB5Il0sWzEsMCwiVSBcXGluXFxiVCJdLFswLDEsIjEiXSxbMSwxLCJYIl0sWzIsMCwiMSJdLFsyLDEsIlkiXSxbMyw1LCJmIl0sWzQsNSwieSIsMl0sWzIsMywieCJdLFsxLDNdLFsxLDRdLFsxLDUsIiIsMSx7InN0eWxlIjp7Im5hbWUiOiJjb3JuZXItaW52ZXJzZSJ9fV0sWzAsMl0sWzAsMV1d
	\[\begin{tikzcd}
		{f x = y} & {U} & 1 \\
		\arrow["\ulcorner"{anchor=center, pos=0.125}, draw=none, from=1-1, to=2-2]
		1 & X & Y
		\arrow[from=1-1, to=1-2]
		\arrow[from=1-1, to=2-1]
		\arrow[from=1-2, to=1-3]
		\arrow[from=1-2, to=2-2]
		\arrow["\ulcorner"{anchor=center, pos=0.125}, draw=none, from=1-2, to=2-3]
		\arrow["y"', from=1-3, to=2-3]
		\arrow["x", from=2-1, to=2-2]
		\arrow["f", from=2-2, to=2-3]
	\end{tikzcd}\]
	of stacky types, in particular sheaves. \qed(Claim) \\\\
	
\end{proof}
\begin{theorem}
	Assume $\bT$ have descent. Then $\St$ is a sheaf.
\end{theorem}
\begin{proof}
	$\St$ is seperated: This follows from the embedding $\St$ into the seperated type of sheaves \ref{lemma:SheavesHaveDescent}. \\
	Let $U \in \bT$ and $P : \|U\| \to \St$. We want to construct a filler 
	% https://q.uiver.app/#q=WzAsMyxbMCwwLCJcXHwgVVxcfCJdLFswLDEsIjEiXSxbMSwwLCJcXFNtU3QiXSxbMCwyLCJQIl0sWzAsMV0sWzEsMiwiIiwyLHsic3R5bGUiOnsiYm9keSI6eyJuYW1lIjoiZGFzaGVkIn19fV1d
	\[\begin{tikzcd}
		{\| U\|} & \St \\
		1
		\arrow["P", from=1-1, to=1-2]
		\arrow[from=1-1, to=2-1]
		\arrow[dashed, from=2-1, to=1-2]
	\end{tikzcd}\]
	Claim: $L_\bT (\sum_{x: \|U\|} P x)$ is stacky.
	\begin{proof} of the claim. We want to apply the previous lemma to the surjection 
		\[\sum_{x : U} P | x | \to \sum_{x : \| U\|} P x \]
		The domain is in $\St$ by stability under $\sum$. The fibers are equivalent to $U \in \bT \subset \St$.				
	\end{proof}
	The claim provides the map $1 \to \St$. The diagram commutes: Assuming $x : \|\Spec A\|$ we wish to show $P x = \sum_{x: \|U\|} P x$. Using univalence, we may show that the maps 
	\[P x \to \sum_{x: \|U\|} P x \overset{\eta}{\to} L_\bT \sum_{x: \|U\|} P x\]
	are both equivalences.
	The first one is an equivalence as $\|U\|$ is contractible. Hence the middle term is a sheaf, thus the unit map is an equivalence as well. \\
	
	
	
\end{proof}
\begin{corollary}
	If $\bT$ has descent, (covering) geometric stacks satisfy descent.
\end{corollary}

\begin{corollary}
	If $\bT$ has descent. For all $n : \bN $, the class of (covering) ($n$-)stacks has descent.
\end{corollary}
\begin{proof}
	We set $\cV$ as the $n$-truncated-type. We have to check the condition (*):
	If $Y$ is a seperated $n$ type , then $L_\bT Y$ is an $n$-type.
	As a sheaf beeing $n$-truncated is a sheaf, we may just show that $\eta x = \eta y$ is $n-1$-truncated for all $x , y : Y$.
	Apply \ref{lemma:sep} to the seperated $Y$, we know $\eta x =_{L X} \eta y \simeq (x=y) $ beeing an $n-1$-type.
	
	%If $n-stack \ni X \to Y$ is fibered in $\bT$ where $Y$ is an $n$-type. Then its sheafification is an $n$-type as well by the lemma.
\end{proof}




\section{Saturated Topologies}
%Consider a topology $\bT$ finer than the Zariski topology.
\begin{definition}
	Consider the partial order
	\[
	\Top = \{\bT : \Prop^\Aff \ | \ 1 \in \bT \land \bT \sum-stable \}
	\]
	ordered by inclusion.
	An inflation $P$ on $\Top$ is a monotone endofunction such that $X \subset P X$. 
	$P$ is stack-preserving if for any topology $\bT$, $P \bT \subset \bT$-merely inhabited types.\\
	it is covering-stack-preserving if for any $X : P \bT$, $X$ is a $\bT$-covering stack.
	%   that preserves $\sum$-stability and satisfies that 
\end{definition}
Note that covering-stack-preserving implies stack-preserving, as $\bT$-covering stacks are $\bT$-merely inhabited.
\begin{prop}{\label{prop:TopologyMonad}}
	Given a stack-preserving inflation $P$. Then for any topology $\bT$, A Type $Y$ is a stack wrt to $P \bT$ iff it is a stack wrt to $\bT$. \\
	If $P$ is idempotent, then the class $P \bT$ is the smallest $P$-fixpoint topology containing $\bT$. \\
	If $P$ is covering-stack preserving, $\bT$ and $P \bT$ will induce the same covering stacks.
\end{prop}

\begin{proof}
	$\bT \subset P \bT$ by inflationarity. 	Regarding Stacks: As $\bT \subset \bT'$ the $\rightarrow$ direction is clear. Now, let $X \in \bT'$. We have
	% https://q.uiver.app/#q=WzAsMyxbMCwwLCJcXHxYXFx8Il0sWzEsMCwiVCJdLFswLDEsIlxcfFhcXHxfXFxiVCBcXHNpbWVxIDEiXSxbMCwxLCJcXGZvcmFsbCJdLFswLDJdLFsyLDEsIlxcZXhpc3RzISIsMl1d
	\[\begin{tikzcd}
		{\|X\|} & Y \\
		{\|X\|_\bT}
		\arrow["\forall", from=1-1, to=1-2]
		\arrow[from=1-1, to=2-1]
		\arrow["{\exists!}"', dashed, from=2-1, to=1-2]
	\end{tikzcd}\]
	by the stack-preserving-property $\|X\|_\bT  \simeq 1$. Hence $T$ is $\|X\|$-local	
	If $P$ is idempotent, every other fixpoint $X$ containg $\bT$ satisfies $P T \subset P X = X$ by monotonicity. \\
	If $P$ is covering-stack-preserving, notice that every $\bT$-covering stack is also a $P \bT$-covering stack as $\bT \subset P \bT$. For the converse we use the recursion principle: For $X$ a $P \bT$-covering stack, consider the predicate 'is $P \bT$-covering'. 1 has it. If $P \bT \ni \Spec A \to X$ is a $\bT$-geometric atlas, i.e. whose fibers are $\bT$-covering stacks, as $\Spec A$ is a $\bT$-covering stack by the covering-stack-preservation, by quotient stability of $\bT$-covering stacks $X$ is a $\bT$-covering stack as well
\end{proof}

\begin{definition}
	A catlas of $X$ is  some $\hat X \in \bT , \hat X \to X \text{ $\bT$-cover }$
\end{definition}
\begin{prop}
	The assignment
	\begin{align*}
		\Top &\to \Top \\
		\bT &\mapsto \bT' \equiv \{X \in \Aff \ | \  \exists \text{ catlas of } X \}
	\end{align*}
	%i.e. the affine covering $0$-stacks.	
	covering-stack-preserving idempotent Monad, called the saturation monad. \\
	$\bT'$ is the class of covering $\Aff$-stacks.
\end{prop}
\begin{proof}
	\begin{itemize}
		\item 	$\bT'$ is $\sum$-stable by \ref{thm:atlasStableSum}. \\
		\item $\bT \subset \bT'$ is clear.
		\item Monotonicity clear
		\item Idempotentency:  consider some $\bT'$-cover $\bT' \ni X' \to X$. By replacing $X'$ with some smooth atlas, we may assume that $X' \in \bT$. As every fiber $X'_x \in \bT'$, we merely find a smooth atlas $\tilde X'_x \to X'_x$. Then by Zariski local choice there exists a Zariski atlas $\hat X \to X$ and a commutative diagram 
		% https://q.uiver.app/#q=WzAsNCxbMCwwLCJcXHN1bV97eCA6IFxcaGF0IFh9XFx0aWxkZSBYJ194Il0sWzAsMSwiXFxoYXQgWCJdLFsxLDEsIlgiXSxbMSwwLCJcXHN1bV97eCA6IFh9WCdfeCJdLFszLDJdLFsxLDIsIlphciIsMl0sWzAsMV0sWzAsM11d
		\[\begin{tikzcd}
			Y \equiv {\sum_{x : \hat X}\tilde X'_x} & {\sum_{x : X}X'_x} = X' \\
			{\hat X} & X
			\arrow[from=1-1, to=1-2]
			\arrow[from=1-1, to=2-1]
			\arrow[from=1-2, to=2-2]
			\arrow["Zar"', from=2-1, to=2-2]
		\end{tikzcd}\]
		As $X' \in \bT$ and $Y \to X'$ is fibered in $\bT$ (\ref{lemma:AtlasSum}) we have $Y \in \bT$. But $Y \to \hat X$ is a $\bT$-cover and $\hat X \to X$ is a $\bT$-cover, $Y \to X$ is a $\bT$-cover. Hence $X \in \bT'$.
		
		\item covering-stack-preserving: For any $\Spec A : \bT'$ we merely have some $\bT$-catlas $\bT \ni X \to \Spec A$, witnessing that $\Spec A$ is a covering stack.
	\end{itemize}
	For the last claim, just observe that $\bT'$ is definitely contained in covering $\Aff$-stacks.
	%But $Y \to X$ is a $\bT$-cover and $\hat X \to X$ is a $\bT$-cover, $Y \to X$ is a $\bT$-cover. Hence $X \in \bT'$.
	%covering $n$-stacks are stable under dependent sums \ref{thM:stabSums}
	%Obviously $1 \in \bT'$. We say a type $X$ has covering local choice if for all $\bT$-surjections $S \to S'$ and a map $X \to S'$ there exists a catlas of $X$ lifting to $S$. 
	% %First observe that every $X \in \bT$ has covering local choice:
	% Now any $X \in \bT'$ satisfies local choice wrt catlasses because it has a catlas and catlasses are stable under composition and Zariski covers are in $\bT$. 
	% Hence ...
	%		As $\bT'$ is definitely contained in the saturation, it suffices to show, that the class $\bT'$ defined above is saturated.
\end{proof}
\begin{lemma}{\label{lemma:flatDescendsAlongFppf}}
	if $\Spec B \to \Spec A$ is faithfully flat and $\Spec B$ is flat, then $\Spec A$ is flat.
\end{lemma}
\begin{proof}
	Consider an injection of $R$-modules $M \hookrightarrow N$. We wish to show, that $A \otimes_R M \to A \otimes_R N$ is injective. As $B$ is faithfully flat over $A$ it suffices to show, that $B \otimes_R M \cong B \otimes_A A \otimes_R N \to B \otimes_A A \otimes_R N = B \otimes_R N$ is injective. This follows as $B$ is flat over $R$.
\end{proof}
\begin{example}
	The fppf-Topology is saturated.
\end{example}
\begin{proof}
	Given a faithfully flat algebra homomorphism $A \to B$ with $B$ faithfully flat, we want to show, that $A$ is faithfully flat. First observe, that $A$ is flat by the previous lemma. Then if $M \otimes_R A = 0$ for some $R$-module $M$, then $M \otimes_R B = M \otimes_R A \otimes_A B = 0$. As $B$ is faithfully flat over $R$, we conclude $M = 0$.
\end{proof}

\begin{example}
	The unramified-topology (unramified + fppf) is saturated.
\end{example}
\begin{proof}
	Let $\Spec B \to \Spec A$ be unramified + fppf and $\Spec B$ unramified + fppf. We have to show that $\Spec A$ is unramified (fppf is the above example). For this, we may show that identity types $x = y$ are $\lnot \lnot$-stable. So assume $\lnot \lnot (x = y).$\\
	As $\Spec A$ admits a faithfully flat map with flat affine domain, the identity type $x = y$ admits such a map $\Spec B' \to x = y$ as well. As its fibers are $\lnot \lnot$-inhabited, we can conclude that the flat $\Spec B'$ is $\lnot \lnot $-inhabited, hence fppf. But now $x = y$ is a fppf-covering -1-stack, hence contractible \ref{lemma:covM1Stacks}.
\end{proof}

\begin{lemma}
	The \etale topology is saturated
\end{lemma}
\begin{proof}
	fppf is clear by saturatedness of the fppf topology. %${\bP_{et}}$-cover (i.e. an \etale faithfully flat map)
	Conclude By \ref{lemma:FEtLocal}
\end{proof}
%
%\begin{lemma}
%
%\end{lemma}
%\begin{proof}
%
%%We have to show that $T \to T^{\|X\|}$ is an equivalence. Choose $\bT \ni Y \to X$. Then we have a commutative diagram
%%	% https://q.uiver.app/#q=WzAsMyxbMCwwLCJUIl0sWzEsMCwiVF57XFx8WFxcfH0iXSxbMSwxLCJUXntcXHxZXFx8fSJdLFswLDFdLFsxLDJdLFswLDIsIlxcc2ltZXEiLDJdXQ==
%%	\[\begin{tikzcd}
%%		T & {T^{\|X\|}} \\
%%		& {T^{\|Y\|}}
%%		\arrow[from=1-1, to=1-2]
%%		\arrow["\simeq"', from=1-1, to=2-2]
%%		\arrow[from=1-2, to=2-2]
%%	\end{tikzcd}\]
%%	So $T \to T^{\|X\|}$ has a left-inverse. Thus it suffices to show that any $f : T^{\|X\|}$ has a preimage. Choose $t : T$, s.th. $\mathrm{cnst}^Y_t$ is the composite $\|Y\| \to \|X\| \overset{f}{\to} T$. We have $\|Y\| \to (\mathrm{cnst}^X_t = f)$. But as $Y \in \bT$ and $\Delta_t = f$ is a stack (as an identitytype in the stack $T^{\|X\|}$) we are done.
%\end{proof}
%\begin{rmk}
%	We never used that we only talk about $\bT$-covers.
%\end{rmk}
%
%\begin{question}
%	Does the converse hold, i.e. is every $\bT$-merely inhabited affine saturated?    
%\end{question}

% \begin{lemma}
%     A type $T$ is a saturated $\bT$-stack if for all $X \in \bT$ the diagonal
%     \[
%     T \to T^{L_\bT \|X \|}
%     \]
%     is an equivalence.
% \end{lemma}
% \begin{proof}
%     If $X \in \bT'$, we can choose a catlas $\hat X \to X$. As it is in particular $\bT$-surjective we have $\|X\| \overset{\sim}{\to} L_T \|\hat X\| $ which gives us that the composite
%     \[
%     T \overset{\simeq}{\to} T^{L_\bT \|X'\|}  \overset{\simeq}{\to}  T^{\|X\|}
%     \]
%     is an equivalence. 
% \end{proof}




\section{Geometric propositions}
\begin{definition}
	$U : \Aff$ is called weakly-flat, if 
	\[\|U\|_\bT \to (U \in \bT)\]	
\end{definition}
\begin{lemma}{\label{lemma:geometricEquiv}}
	The converse holds always
\end{lemma}
\begin{proof}
	because things in $\bT$ are automatically $\bT$-merely inhabited
\end{proof}
\begin{example}
	Examples of weakly-flat affines for the Zariski topology
		\begin{itemize}
		\item finite sums of principal opens
		\item Closed propositions
	\end{itemize}
	for the fppf topology: flat affines . \\
	For the \etale topology: formally \etale affines
\end{example}

Recall the definition of $\bT$-atlas \ref{def:TAtlas}
\begin{definition}{\label{def:algprop}}
	Let $\bT$ be saturated. We call a modal proposition geometric, if one of the equivalent conditions is satisfied:
	\begin{enumerate}
		\item  its merely of the form $\|U\|_\bT$ for some geometric affine $U$.
		\item It is a geometric stack
%		\item There is a $\bT$-surjective map out of a geometric affine $U$.
		\item It has a $\bT$-atlas.

	\end{enumerate}
	
\end{definition}
\begin{proof} \
	\begin{enumerate}
		\item [1 $\Rightarrow$ 2]
		we show that $U \to \|U\|_\bT$ is a geometric atlas. Every fiber is in $\bT$, because $U$ is geometric. A $\bT$-atlas is a geometric atlas.
		\item [2 $\Rightarrow$ 3]
		If $P$ is a geometric -1-stack, then we may choose $U \to P$ a geometric atlas. This is a $\bT$-atlas by \ref{prop:affineCoveringStack}.

		\item [3 $\Rightarrow$ 1]
		
		Let $V \to P$ be a $\bT$-atlas.
		have to show TFAE $\|V\|_\bT \to P \to (V \in \bT) \overset{\ref{lemma:geometricEquiv}}{\to} \|V\|_\bT$. 
		Proof: $\|V\|_\bT \to P$ as $P$ is modal prop. Secondly, because $V \to P$ is a $\bT$-cover. \\
		Hence $P$ is a geometric proposition.
	\end{enumerate}
	
\end{proof}
\begin{example}
	Open / Closed Propositions are geometric.
\end{example}
\begin{question}
Is every geometric proposition a scheme?
\end{question}
It is an algebraic space that embeds into an affine, so it suffices to reproduce the statement from the presheaf model.

%\begin{lemma}[NECESSARY?]
%	geometric propositions are algebraic spaces.
%\end{lemma}
%\begin{proof}
%	We have $U \to \|U\|_\bT$ where $U$ is affine, hence an algebraic space and the fibers are in $\bT$ by geometricness of $U$, hence they are covering algebraic spaces. By stability under quotients, our geometric proposition is an algebraic space.
%\end{proof}

\section{Fundamental Theorem of algebraic spaces}
\subsection{For groupoids}
\begin{lemma}
	If $R \twoheadrightarrow X \to X$ is a $\bT$-htpy-coequalizer diagram of two $\bT$-covers between affines, then $X$ is a  1-stack.
\end{lemma}

\subsection{For sets}
\begin{lemma}\label{quotient-by-equivalence-relation}
	Denote $\bT Set$ for the sets that are $\bT$-sheaves. Assume given a $\bT$set  $X$ then the following maps are mutually inverse
	\begin{align*}
		\sum_{R:X\to X\to \bT\Prop} R\ \mathrm{equivalence\ relation} &\simeq \sum_{Y:\bT \mathrm{\Set}} \sum_{p:X\to Y} p\ \bT\mathrm{surjective} \\
		R &\mapsto (X/R,[\_]) \\
		\lambda x,y.  (p(x)=p(y)) &\mapsfrom (Y,p) 
	\end{align*}
	where $X / R$ is defined by applying $L_T \| \_ \|_0 $ at the higher inductive type $X // R$.
\end{lemma}

\begin{proof}
	\begin{itemize}
		\item Well-definedness: The map $[\_] : X \to \|X // R\|_0 \to L_T \|X // R\|_0$ is the composition of a surjective with a $\bT$-surjective map \todocite, hence its $\bT$-surjective. \\
		Conversely given $(Y,p)$ as $Y$ is a sheaf, we have for all $x,y : X$ that $p(x) =_Y p(y)$ is a sheaf.
		\item If $x,y : X$ then we have a chain of equivalences 
		\[
		R(x,y) \simeq (\bar x =_{\|X//R\|_0} \bar y) \to ([x] =_{L_T\|X//R\|_0} [y])
		\]
		where the first map is plain HoTT and the second map is $\mathsf{ap}$, i.e. the unit of the modality \todocite, but as the $\bar x =_{\|X//R\|_0} \bar y$ is already a sheaf, it is an isomorphism as well. \\
		\item Let $(Y,p)$ be in the RHS. Let $R(x,y) = (p(x)=p(y)) : \bT \Prop$. By plain HoTT, There is a map $\eta :  X // R  \to Y$ ( defined by the universal property of the set truncation and by induction on the higher inductive type $ X // R$ on canonical terms through the map $p : X \to Y$). I claim $\eta$ exhibits $Y$ as the localization for $\bT \Set$-modality of $X // R$. Let $T$ be another $\bT \Set$ equipped with a map $X // R  \to T$. By precomposition we obtain a map $X \to T$. 
		Claim: it factors uniquely through $p : X \to Y$. 
		% https://q.uiver.app/#q=WzAsNCxbMCwwLCJYIl0sWzEsMCwiXFx8WCAvIFJcXHwiXSxbMiwwLCJUIl0sWzEsMSwiWSJdLFswLDFdLFsxLDJdLFswLDNdLFszLDIsIlxcZXhpc3RzISIsMix7InN0eWxlIjp7ImJvZHkiOnsibmFtZSI6ImRhc2hlZCJ9fX1dXQ==
		\[\begin{tikzcd}
			X & {X // R} & T \\
			& Y
			\arrow[from=1-1, to=1-2]
			\arrow[from=1-1, to=2-2]
			\arrow[from=1-2, to=1-3]
			\arrow["{\exists!}"', dashed, from=2-2, to=1-3]
		\end{tikzcd}\]
		Proof: \\
		Existence: We want to define a map $Y \to T$. Let $y : Y$. As $p$ is $\bT$-surjective and $T$ is a sheaf, we may assume we merely have some element in the fiber of $p$ over $y$. Now push this element through     
		\[\|\fib_p y\| \to \|X // R\|_0 \to T\]
		where the first map is by Plain HoTT and the second one is induced from $X // R \to T$ by assumption and the fact that $T$ is a set.. One can easily check this makes the diagram commute.
		Uniqueness follows from $X \to Y$ beeing $\bT$-surjective and the following
		Fact: Two parellel maps $Y \rightrightarrows T$ into a $\bT \Set$ $T$ are already equal if the become equal after precomposition with a $\bT$-surjection $X \to Y$.  \\
		Proof of the fact : Let $y : Y$. The goal is an identity type of a $\bT \Set$, hence a $\bT \Prop$. Hence As the fiber over $y$ in $X$ is $\bT$-merely inhabited, we may assume an actual term in the fiber. 	As $X \to Y$ equalizes the arrows, this term allows us to conclude. \qed (fact)	\qed(Claim) \\
		We apply the fact to the ($\bT$-)surjectivity of $X \to X // R $ to get a unique factorization 
		% https://q.uiver.app/#q=WzAsNCxbMCwwLCJYIl0sWzEsMCwiXFx8WCAvIFJcXHwiXSxbMiwwLCJUIl0sWzEsMSwiWSJdLFswLDEsIiIsMCx7InN0eWxlIjp7ImhlYWQiOnsibmFtZSI6ImVwaSJ9fX1dLFsxLDJdLFswLDNdLFszLDIsIlxcZXhpc3RzISIsMix7InN0eWxlIjp7ImJvZHkiOnsibmFtZSI6ImRhc2hlZCJ9fX1dLFsxLDNdXQ==
		\[\begin{tikzcd}
			X & {X // R} & T \\
			& Y
			\arrow[two heads, from=1-1, to=1-2]
			\arrow[from=1-1, to=2-2]
			\arrow[from=1-2, to=1-3]
			\arrow[from=1-2, to=2-2]
			\arrow["{\exists!}"', dashed, from=2-2, to=1-3]
		\end{tikzcd}\]
		making the right triangle commute. This is what we wanted to show.
	\end{itemize}
\end{proof}

\begin{definition}
	An equivalence relation $R$ on a type $X$ is called:
	\begin{itemize}
		\item redundant if for all $x,y:X$ the proposition $R(x,y)$ is a  $-1$-stack.
		\item covering if its  and for any $y:X$ its fibers:
		\[R_y :\equiv \sum_{x:X} R(x,y)\]
		are affine in $\bT$.
	\end{itemize}
\end{definition}

\begin{lemma}\label{fundamental-propriety-algebraic-spaces}
	Assume that $\bT$ satisfies descent for propositions and for sets \ref{thm:descent}, i.e. that a modal proposition being a  (-1)-stack is a sheaf. Assume that a modal set beeing affine in $\bT$ is a sheaf.
	Assume given a $\bT$set $X$, then the following types are equivalent:
	\begin{itemize}
		\item The type of redundant covering equivalence relations over $X$.
		\item The type of $\bT$sets $Y$ with identity types beeing  stacks and an $-1$-atlas $X$ to $Y$ (in V2 a $\bT$-cover).
	\end{itemize}
\end{lemma}

\begin{proof}
	By the equivalence in \ref{quotient-by-equivalence-relation}, it is enough to check that:
	\begin{itemize}
		\item The identity types in $X/R$ are 
		(-1)-stacks if and only if the relation $R$ is redundant . For any $x,y:X$ we know that:
		\[R(x,y) \simeq [x] =_{X/R}[y]\]
		so the direct direction is immediate. For the converse we use the assumption that a modal proposition being a  (-1)-stack is a sheaf and that the map $[\_]:X\to X/R$ is $\bT$-surjective.
		\item The fibers of: 
		\[[\_]:X\to X/R\] 
		are affine in $\bT$ if and only if the relation $R$ is covering. For any $y:X$ we have that:
		\[\sum_{x:X} R(x,y) \simeq \mathrm{fib}_{[\_]}([y])\]
		so the direct direction is immediate. Here as well the converse follows from $\bT$-surjectivity of $[\_]$ and that the topology has descent.
	\end{itemize}
\end{proof}
\begin{corollary}
	Assume $\bT$ satisfies descent for propositions and for sets.
	A type is a  0-stack iff its merely the $\bT$-quotient of an affine scheme by a covering equivalence relation.
\end{corollary}
\begin{theorem}{\label{thm:QuotientOfAlgebraicSpace}}
	Assume $\bT$ satisfies descent for propositions. 
	The quotient of a  $0$-stack $X \in \bT \Set$ by an $0$-covering equivalence relation $R$ is a  $0$-stack. TODO
\end{theorem}

\begin{proof}
	The identity types in $X / R$ are propositional  0-stacks, hence $(-1)$-\truncation s of  -1-stacks by \ref{lemma:prop0stacks} as desired. \\
	How to find an atlas: todo. How to proceed, if we could choose all atlasses we want at the same time?
	% Motivation why the choice of atlasse should work: Let $T = X / R$. 
	%  If we could choose -1-atlasses $\tilde X_t$ for the covering 0-stacks $\fib_{[]}(t)$ for all $t : T$ at the same time, then $\sum_{t : T} \tilde X_t \to \sum_{t : T} \fib_{[]}(t) \to T$ has as domain a is fibered in covering $-1$-stacks, as the fiber over $t$ would be $\tilde X_t$ which is an affine scheme in the topology. Moreover, This is enough as \\ %, hence by definition a covering -1 stack. \\
	
	%     Given $p_1 , p_2 : R \rightrightarrows X$ fibered in covering $0$-stacks, hence the fibers merely have $-1$-atlasses.
	%     %Claim: There exists a $-1$-atlas $R' \to R$
	%     As $X$ has Local choice with respect to $-1$-atlasses, we find a $-1$ atlas $f : X' \to X$ with 
	%     \[
	%     \prod_{x' : X'} \text{-1-atlas}(\sum_{x : X} R(x,f(x'))
	%     \]
	
\end{proof}
\begin{rmk}
	This is equivalent to saying that  $1$-stacks that are $0$-types are geomeric $0$-stacks: One direction we prove later. If $R$ is a 0-covering equivalence relation on a  0-stack $X$, then $ X/ R$ is a  1-stack by observing that any -1-atlas $X' \to X$ gives a 0-atlas $X' \to X \to X/ R$. Moreover, $ X/ R$ is a 0-type, hence by assumption a  0-stack.
\end{rmk}

\begin{example}
	There are open affine subschemes $U$ of affine schemes $\Spec A$, which are not (disjoints unions of) principal open
\end{example}
\begin{proof}
	Consider $A = R[x,y,u,v]/(xy + ux^2 + vy^2), X= \Spec A$ and consider the open $U = D(x,y)$. \\
	We cant expect $U$ to be a disjoint union of principal opens (todo). However, $D(x,y)$ is affine: We have maps $U \to R$ given by
	$f = -v/x = (y+ux)/y^2 , g= -u/y = (x+vy)/x^2$. 
	Then $D(f) \cup D(g) = \Spec R^X$ , as $yf + xg = 1$ in $R^U$.
	Taking preimages under the affinization map, $U_f \cup U_g = X$ and one checks this defines an open affine cover (for example : $U_f \simeq \Spec R[x,u,f^{\pm 1}, g] / (xy + ux^2 + uy^2)$ with $y := (1-gx)/f$.)
	But on both of this open subsets the affinization map is an isomorphism
	hence the affinization of $X$ is an isomorphism.
	%\[R^X \simeq R^{D(f) \cup D(g)} \simeq R^{D(f)} \times_{R^{D(fg}} D(g) \]
	compare (Hartshorne II.2.17)
\end{proof}
% \begin{example}
%     The Zariski topology does not descent along $\bT$-covers between affines
% \end{example}
% \begin{proof}
% Assume it would hold.
% By the previous example pick such an open affine subset $U \subset \Spec A$ and pick a Zariski atlas $V \to U$ such that $V$ is mereley of the form $D(a_1) + \hdots + D(a_n)$ for some $a_i \in A$. Let $x : \Spec A$. Then pulling pack the Zariski atlas along $U(x) \to U$ gives us a Zariski atlas of the open proposition $V' \to U(x)$. Now $V' + 1 \to U(x) + 1$ is a Zariski atlas with total space in the Zariski topology. By assumption, $U(x) + 1$ is in the topology, hence $U(x)$ would be a sum of principal opens. As it is a propososition, it would be a principal open subset of $1$. 
% This is not a contradiction, because an open subset can be non principal although all the fibers are principal open props...
% This is a contradiction by the assumption on $U \subset \Spec A$ beeing not principal open.

% \end{proof}
\begin{lemma}
	Let $f : X \to Y$ be surjective. There exists a Zariski Cover $X' \to X$ such that $X' \to Y$ is a Zariski cover iff there exists a Zariski Cover $X' \to X$, some $n : \bN$ and an open affine embedding $X' \hookrightarrow Y^n$ over $Y$.
\end{lemma}

% \begin{lemma}
%     A morphism $f : X \to Y$ of  $n$-stacks is fibered in covering $n$-stacks if there exists a covering $n$-atlas of $f$.
% \end{lemma}





\subsection{Algebraic spaces}


%Let us also mention what we learned in the proof:
%\begin{lemma}[NECESSARY?@²]
%	A covering equivalence relation $R : S^2 \to \bT \Prop$ has values in geometric propositions.
%\end{lemma}

%\begin{corollary}
%	The identity types of algebraic spaces are geometric propositions.
%\end{corollary}
%\begin{proof}
%	By the previous lemma and \ref{lemma:geometricStacksClosedUnderId}
%\end{proof}
%
%\begin{lemma}{\label{lemma:detectGeomProp}}
%	Let $P$ be a sheaf and a proposition that admits a map $\Spec A \to P$ fibered in covering algebraic spaces. Then $P$ is a geometric proposition.
%\end{lemma}
%\begin{proof}
%	The fibers are covering algebraic spaces and affine, hence covering affine. By \ref{def:algprop} we conclude.
%\end{proof}
\begin{theorem}
	Let $X$ be a modal set. The following are equivalent:
	\begin{enumerate}
		\item $X$ is a (covering) geometric 0-stack
		\item $X$ is merely of the form $L_\bT (U / R)$ for some (covering) affine $U$ and  $R : U^2 \to \Prop_{\ci}$ a covering equivalence relation. 
		\item there exists some map $S \to X$ with $S$ (covering) affine whose fibers merely have $\bT$-catlasses.
	\end{enumerate}
	We call this class (covering) algebraic spaces.
\end{theorem}
\begin{proof}
\ 	\begin{enumerate}
	\item [2 $\leftrightarrow$ 3]
		This is \ref {lemma:fundamental-property-algebraic-spaces}
	\item [2 $\to$ 1]
	Choose a presentation $ R: U^2 \to \Prop$.
	It suffices to show, that the map $f : U \to L_\bT ( U / R)$ is a geometric (c)atlas. The map $f$ is $\bT$-surjective by the well-definedness of the bijection $\ref{quotient-by-equivalence-relation}$. By descent we may just show, that the fibers $\fib_f (f(s))$ for $s : U$ are covering 0-stacks. But by the bijection in \ref{quotient-by-equivalence-relation} those are equivalent to the fibers $R_s$, which are covering 0-stacks as the equivalence relation is covering. \\
	\item [1 $\to$ 2]
	This can be reformulated in the following way, using the recursion principle for (covering) geometric 0-stacks:
	Let $X$ be a sheaf of sets. Let $S$ be (covering-) affine and $f : S \to X$ be fibered in covering algebraic spaces. Then $X$ is a (covering) algebraic space.
%	The identity types of $X$ admit a map fibered in covering algebraic spaces (todo check stability under $\sum$) out of an affine by \ref{lemma:havingAbstractAtlasClosedUnderId}. by \ref{lemma:detectGeomProp} they are geometric propositions. 
This follows from the observation, that the equivalence relation determined by $f$ is covering \ref{def:coveringEqRel} , because the fibers of $f$ are covering 0-stacks.
	\end{enumerate}
\end{proof}
\begin{prop}
	For any $n \ge 1$, we have inclusions 
	\[W_{n} \subset \CS_{n-1} \subset W_{n+1}\]
\end{prop}
\begin{proof}
	Induction. $n = 1$ gives
	\[
	\mathsf{HasCatlas}_\bT \subset \CS_0 \subset \text{ types admitting a catlas fibered in } W_1
	\]
	the latter inclusion is the previous theorem. \\
	The induction step is obtained by \ref{prop:nstack}
\end{proof}


\subsection{Schemes are algebraic Spaces for the Zariski Topology}
\begin{definition}
 A proposition $U$ is open iff its merely of the form $f_1 \ inv \lor \hdots f_n inv$ for some $f_i : R$.
\end{definition}

\begin{lemma}
	Given $f_1, \hdots,f_n : R$ such that $\| D(f_1) + \hdots + D(f_n) \|$ then $\sum_{i=1}^n D(f_i) \in \Zar$.
\end{lemma}
\begin{prop}
	Every Zariski-merely-inhabited type that is merely of the form $U_1 + \hdots + U_n$ for open propositions $U_i$ admits a $\Zar$-catlas.
\end{prop}
\begin{proof}
	By definition of openness, We can choose a surjection $\coprod_{j=1}^{n_i} D(f_{ij}) \twoheadrightarrow U_i$ for any $i$. We want to show, that the map
	\[
	\coprod_{i , j} D(f_{ij}) \twoheadrightarrow U_1 + \hdots U_n
	\]
	is a $\Zar$-catlas. 
	\begin{itemize}
		\item Let us first show that the fibers are in $\Zar$. Assume $U_i$ holds. So we find a term in $\coprod_j D(f_{ij})$. In particular we have $\| \coprod_j D(f_{ij})\|_{\Zar}$. By the lemma we conclude, that the fiber $\sum_j D(f_{ij})$ belongs to $\Zar$.\\
		\item The total space is in $\Zar$: This follows as the surjection after propositional truncation becomes an equivalence. As we have $\| U_1 + \hdots + U_n\|$, we can conclude by the lemma.
	\end{itemize}
	
\end{proof}
\begin{warning}
	The converse does not hold! We want to apply \ref{lemma:stackificationHasTCover}, to the map
	\[\Zar \ni 1 + 1 \to \sum D(f) \]
	\begin{itemize}
		\item 	$\sum D(f)$ is seperated as $D(f)$ is a sheaf.
		\item 	All the fibers are equivalent to $1 + X$, hence they are in the Zariski topology.
	\end{itemize}	
\end{warning}
\begin{lemma}
	let $X$ be a scheme. There merely exists some affine $S$  map $S \to X$ whose fibers are merely inhabited finite sums of open propositions 
\end{lemma}

\begin{corollary}		
		Every scheme is an algebraic space.
\end{corollary}


\begin{lemma}
	If $X$ is an algebraic space, then the global sections embed via a $R$-algebra homomorphisms into a finitely presented $R$-algebra.
\end{lemma}
\begin{proof}
	Choose an atlas $S \to X$, in particular $\bT$-surjective. As $\bT$ is subcanonical the map $R^X \to R^S$ is an injection.
\end{proof}
\begin{question}
	Is it an open embedding of types?
\end{question}
%\begin{lemma}
%	Consider a morphism $X \to Y$ between algebraic spaces such that $R^F$ is finitely presented and the affinizations of the fibers $F \to \Spec R^F$ are open immersions. If $Y$ is a scheme, then $X$ is a scheme.
%\end{lemma}
%\begin{proof}
%	Consider the stein factorization
%	\[
%	X = \sum_{y : Y} \fib_f y \to \sum_{y : Y} \Spec R^{\fib_f y} \to Y
%	\]
%	The first map 
%\end{proof}
\subsection{Examples}
The goal of this subsection is to construct algebraic spaces. The first example actually gives us a scheme:
\begin{example}
	Let $p \neq 0$ be a prime. You can let $\mu_p := \Spec(R[X] / (X^p - 1))$ act on $\bA^\times$ via multiplication. Set $\bT= fppf$. Then the $p$.th power map
	
	\[
	pow : \| \bA^\times // \mu_p \|_0^\bT \to \bA^\times
	\]
	is an equivalence.
	\begin{itemize}
		\item 	 It is an embedding: 
		First note, that $\|\bA^\times // \mu_p \|_0$ is $\bT$-seperated:
		
		as $\mu_p$ act freely on $\bA^\times$, $\bA^\times // \mu_p$ is already a set. Meaning that the identity types of the set-quotient are $\sum_{g: \mu_p} g x =_{\bA^\times} y$ , hence sheaves. \\
		On the other hand the map $\|\bA^\times // \mu_p \|_0 \to \bA^\times$ is an embedding, as for any $x , y : \bA^\times$ the map $(\sum_{g : \mu_p} g x = y) \to (x^p = y^p)$ is an equivalence. 
		\item 	It is $\bT$-surjective, as for any $\lambda : \bA^\times$, we find $S = \Spec R [X] / (X^p - \lambda) \in \bT$ with 
		\[
		S \to \fib_{\bA^\times / \mu_p \to \bA^\times}(\lambda)
		\]
		hence 
		\[
		1 = \|S\|_\bT \to \|\fib_{pow}\|_0^\bT
		\]
	\end{itemize}
	
	
\end{example}
\begin{example}[TODO]
	The sheaf quotient of $\bA^1$ by the $\mu_\ell$ action is probably not an algebraic space.
\end{example}

\begin{lemma}{\label{lemma:AlmostEverywhere}}
	Let $p : A$ be reguar. If $f : \Spec A \to R$ such that $f(x) = 0$ for all $x \in D(p)$, then $f(x) = 0$ for all $x : \Spec A$.
\end{lemma}
\begin{proof}
	$f$ is in the kernel of the diagonal map
% https://q.uiver.app/#q=WzAsNCxbMSwwLCJSXlIiXSxbMSwxLCJSXntSIFxcc2V0bWludXMgXFx7MFxcfX0iXSxbMCwwLCJSW1hdIl0sWzAsMSwiUltYXntcXHBtIDF9XSJdLFswLDFdLFsyLDMsIiIsMCx7InN0eWxlIjp7InRhaWwiOnsibmFtZSI6Imhvb2siLCJzaWRlIjoidG9wIn19fV0sWzAsMiwiIiwxLHsibGV2ZWwiOjIsInN0eWxlIjp7ImhlYWQiOnsibmFtZSI6Im5vbmUifX19XSxbMSwzLCIiLDEseyJsZXZlbCI6Miwic3R5bGUiOnsiaGVhZCI6eyJuYW1lIjoibm9uZSJ9fX1dXQ==
\[\begin{tikzcd}
	{A} & {R^{\Spec A}} \\
	{A_p} & {R^{D(p)}}
	\arrow[hook, from=1-1, to=2-1]
	\arrow[Rightarrow, no head, from=1-2, to=1-1]
	\arrow[from=1-2, to=2-2]
	\arrow[Rightarrow, no head, from=2-2, to=2-1]
\end{tikzcd}\]
	which is injective, as $p$ is regular in $A$. \\
	Thus $f = 0$ in $A$.
%	 Thus $f = 0 $ in $(R \setminus \{0\} \to R) = R[X^{\pm 1}]$ hence $f \cdot X^n = 0$ in $R[X]$ for some $n$, thus $f = 0$.
\end{proof}
Let $\ell \neq 0$ denote a prime. Consider $\mu_\ell = R[X] / (X^\ell - 1)$.
%\[\mu_\ell'  = \Spec R[X] / (X^{p-1} + \hdots + 1) = \mu_\ell \setminus\{1\}\]

%\begin{lemma}
%	Let $0 \in D(p)$. For a function $\phi : D(p) \to R$ TFAE % We say some $\phi: R[X]_p$ is $\ell$-even if one of the following equivalent conditions is satisfied:
%	\begin{enumerate}
%
%	\item The function $\phi : D(p) \to R$ is a $\mu_{\ell}$-invariant function, i.e. $\phi(x) = \phi(g x)$ for all $ x : D(p)$ and each $g : \mu_\ell$.
%	\item The function $\phi|_{D(p) \setminus 0} : D(p) \setminus \{0\} \to R$ is an $\mu_{\ell}$-invariant function
%%	\item[1']  There exists $f : R[X], n : \bN$ such that $\phi= f/ p^n$ and $f (g.p)^n (x) = (g.f) p^n (x)$ for each $x : R , g : \mu_\ell$.
%%	\item[2'] The same as 3. but only for $x : D(p) \setminus\{0\}$
%\end{enumerate}
%\end{lemma}
%\begin{proof}
%	 we can apply  \ref{lemma:AlmostEverywhere}, observing $\phi - g.\phi = 0$ on $D(X / 1) \subset \Spec R[X]_p$, where $X/1 : R[X]_p$ is regular, because $X$ is regular in $R[X]$. %$: 
%\end{proof}
\begin{prop}
	Let $G$ be a formally \etale flat affine group, such that $\lnot \lnot$ its finite with cardinality invertible in $R$. Let it act on an affine scheme $\Spec A$ with a fixpoint 0, such that the group action is free away from 0.
	Let $p : A^G$. Define $R: D(p)^2 \to \Prop$ as
	\[
	R(x,y) = (x = y) + (x \neq 0) \times \sum_{g : G \setminus \{1\}} g x = y
	\]
	Then the sheaf quotient $D(p) / G$ is an algebraic space.
\end{prop}
\begin{proof}
%	Define
%	\[
%	E(x,y) = (x = y) + (x \neq 0 \land \sum_{g : \mu_\ell \setminus \{1\}} gx = y)
%	\]
	This is a proposition: First note, that both summands are propositions because $G$ acts freely on $\bA^1 \setminus \{0\}$. If both summands are inhabited we get a contradiction, as $x = y$ and $gx = y$ implies $(g-1) x = y - x = 0$, but as $g-1$ is invertible $x = 0$. \\
	The relation is covering: 
	The propositions are affines, thus sheaves. Furthermore, for any $y : D(p)$ we have
	\[
	\sum_{x : D(p)}  (x = y) + (x \neq 0 \times \sum_{g : G \setminus \{1\}} gx = y) = 1 + (y \neq 0 \times G \setminus \{1\}) \in \bT
	\]
	as $G \setminus \{1\} = \sum_{g : G} g \neq 1$ is a $\sum$ of formally \etale + flat affines (recall that formally \etale affines have decidable equality).
	
\end{proof}
%\begin{prop}
%	Let $p : R[X]^{\mu_\ell}$ with $0 \in D(p)$ % be such that $0 \in D(p)$ and $x \in D(p)$ implies $gx \in D(p)$ for all $g : \mu_\ell$.
%	The sheaf-quotient of $D(p)$ by the relation which identifies $x$ and $gx$ when $x \neq 0, g : \mu_\ell \setminus \{1\}$ is not an affine scheme.
%\end{prop}
%\begin{proof}
%
%\end{proof}
\begin{lemma}
	Let $G$ be a finite group whose cardinality is invertible in $R$. Let $G$ act on an affine scheme equipped with a fixpoint $0$. Let $V$ be an invariant open neighborhood of 0. Then we find an invariant principal open neigbhorhood contained in $V$. Invariant means here that $g(U) = U$ for all $g : G$.%Let $A$ be a finitely presented algebra and $0 \in \Spec A$ a basepoint. 
\end{lemma}
\begin{proof}
	Let $U$ be an invariant open neighborhood. Choose a principal open neighborhood $0 \in D(p) \subset U$. $G$ acts on $R[X]$, via $(g.p)(x) = p(g x)$.Then 
	\[p' = \sum_{g : G} g . p : R[X]\]
	is a $G$-invariant polynomial, in particular $D(p)$ is $G$-invariant. Moreover $0 \in D(p')$ as
	\[
	p'(0) = \sum_{g : G} p(g(0)) = \sum_{g : G} p (0) = |G| \cdot p(0) 
	\]
	is invertible, as $|G|$ and $p(0)$ are both invertible. Furthermore, as $U$ was $G$ invariant and contained $D(p)$ it also has to contain $D(p')$.
\end{proof}

\begin{prop}
	Let $G$ be a group, such that $\lnot \lnot$ its finite with cardinality invertible in $R$. Let it act on an affine scheme $\Spec A$ with a fixpoint 0, such that the group action is free away from 0.
	For any $p : A^G$ define $R: D(p)^2 \to \Prop$ as
	\[
	R(x,y) = (x = y) + (x \neq 0) \times \sum_{g : G \setminus \{1\}} g x = y
	\]
	Assume the sheaf quotient $D(p) / G$ is not affine for any $p : A^G$. Then $(\Spec A) / G$ is an algebraic space that is not a scheme.
\end{prop}
\begin{proof}
	It is an algebraic space by putting $p \equiv 1 : R[X]$ in the previous prop. \\
	Assume the quotient is a scheme. 
	The preimage along the quotient map obtained from the relation induces a open neigbhorhood $V$ of $0$ in $\bA^1$. As we want to prove a contradiction we may assume that $\mu_\ell$ consists of $\ell$ many elements, where $\ell \neq 0 $ in $R$. We apply the previous lemma to $V$ to obtain an invariant principal open neigborhood $0 \in D(p) \subset V \subset \bA^1$. As its invariant, $p : \bA^1 \to R$ descends to $X \to R$. Restricting to $V$ yields a map $p' : V \to R$, such that setting $U \equiv D(p')$ yields such that $q^{-1}(V) =q^{-1}(D(p')) = D(p)$ . We are now in the following situation
	% https://q.uiver.app/#q=WzAsNCxbMCwwLCJEKHApIl0sWzEsMCwiXFxiQcK5Il0sWzEsMSwiWCJdLFswLDEsIlUiXSxbMCwxLCIiLDAseyJzdHlsZSI6eyJ0YWlsIjp7Im5hbWUiOiJob29rIiwic2lkZSI6InRvcCJ9fX1dLFswLDNdLFszLDIsIiIsMix7InN0eWxlIjp7InRhaWwiOnsibmFtZSI6Imhvb2siLCJzaWRlIjoidG9wIn19fV0sWzEsMl0sWzAsMiwiIiwxLHsic3R5bGUiOnsibmFtZSI6ImNvcm5lci1pbnZlcnNlIn19XV0=
	\[\begin{tikzcd}
		{D(p)} & {\bA^1} \\
		U & X
		\arrow[hook, from=1-1, to=1-2]
		\arrow[from=1-1, to=2-1]
		\arrow["\ulcorner"{anchor=center, pos=0.125}, draw=none, from=1-1, to=2-2]
		\arrow[from=1-2, to=2-2]
		\arrow[hook, from=2-1, to=2-2]
	\end{tikzcd}\]
	where $U$ is an open affine neighborhood of 0. \\
	Then we have $D(p) / \sim' \simeq U$ with restricted equivalence relation. By assumption we see $U$ cannot be affine. Contradiction.\\		
\end{proof}
\begin{corollary}
	Let $\ell \neq 0$ be prime. The sheaf quotient of $\bA^1$ by the relation that identifies each  $x \neq 0$ with $g x$ if $g^p = 1, g \neq 1$ is an algebraic space that is not a scheme.
\end{corollary}
\begin{proof}
	Let $p$ be as above.
	The conditions on $p$ give $p(0) \neq 0$ and $p(x) \neq 0 \to p(gx) \neq 0$ for all $g : \mu_\ell$.
	
	Lets call this quotient $X$.
	
	Define 
	\[
	A = \{\phi : R^{D(p)} \ | \ \phi \text{  is $\mu_{\ell}$-invariant }\}
	\]
	This is an $R$-subalgebra: for any $r : R$, $r : R[X]_p$ is $\mu_{\ell}$-invariant. $\mu_{\ell}$-invariant functions are stable under addition and multiplication . \\
	
	Claim: The affinization map of $X$ is the induced dashed map $f : X \to \Spec A$ in
	
	% https://q.uiver.app/#q=WzAsNCxbMCwwLCJEKHApIl0sWzEsMCwiXFxTcGVjIFJbWF1fcCJdLFswLDEsIlgiXSxbMSwxLCJcXFNwZWMgQSJdLFswLDIsInEiXSxbMiwzLCJcXGV4aXN0ISBmIiwwLHsic3R5bGUiOnsiYm9keSI6eyJuYW1lIjoiZGFzaGVkIn19fV0sWzEsM10sWzAsMSwiIiwyLHsibGV2ZWwiOjIsInN0eWxlIjp7ImhlYWQiOnsibmFtZSI6Im5vbmUifX19XV0=
	\[\begin{tikzcd}
		{D(p)} & {\Spec R^{D(p)}} \\
		X & {\Spec A}
		\arrow[Rightarrow, no head, from=1-1, to=1-2]
		\arrow["q", from=1-1, to=2-1]
		\arrow["q'",from=1-2, to=2-2]
		\arrow["{\exists! f}", dashed, from=2-1, to=2-2]
	\end{tikzcd}\]
	Proof: A function $\phi : D(p) \to R$ factors through $q$ iff $\phi|_{D(p) \setminus\{0\}}$ is $\mu_{\ell}$-invariant away from 0. We have to show, that then $\phi$ is $\mu_{\ell}$ invariant, because then the embedding (using that $R$ is a sheaf) $R^X \hookrightarrow R^{D(p)}$ has image $A$. We can apply  \ref{lemma:AlmostEverywhere}, observing $\phi - g.\phi = 0$ on $D(X / 1) \subset \Spec R[X]_p$, where $X/1 : R[X]_p$ is regular, because $X$ is regular in $R[X]$. $\qed$(Claim). 	\\ \\ %respective $\phi : R[X]_p$ satisfies $	\phi (x) = \phi(-x)  $, i.e. (1) if $\phi$ is $\mu_{\ell}$-invariant. 
	Proof that $X$ is not an affine:	Assume that $X$ were affine. Then the map $f$ would be in particular an embedding. 
	We may assume a term $g : \mu_\ell \setminus \{1\}$: Indeed, as we want to prove a contradiction we may assume a term in $g : \Spec R[X] / (\sum_{i=0}^{\ell-1} X^i)$. But this type is equivalent to $\mu_\ell \setminus \{1\}$, using that $\sum_{i=0}^{\ell-1} X^i | X^\ell -1 $ and $\ell \neq 0$. \\
	Let $V \subset \bA^1$ be a non-contractible neighborhood of 0 that is infinitesimal, i.e.  $\lnot \lnot x =0$ for every $x : V$. (e.g $\Spec R[X] / X^n$ for some $n >1$ is non contractible, because $R[X] / X^n \to R[X] / X$ is not an algebra isomorphism). %$\cN_{\lnot \lnot}(0) = \{x : \bA^1 \ | \ \lnot \lnot x = 0\}$ be be a non-contractible subtype that is $\lnot \lnot$-connected. 
	Then for any $\varepsilon : V \subset D(p)$ (using that invertibility is $\lnot \lnot$ stable) we have
	\begin{align*}
		(q\varepsilon =_X q (g \varepsilon)) \overset{\ref{quotient-by-equivalence-relation}}{=} (\varepsilon = g\varepsilon) + (\varepsilon \neq 0 \land \sum_{h \neq 1} \varepsilon = h g \varepsilon) = ((g-1)\varepsilon = 0) = (\varepsilon = 0)
	\end{align*}
	But we have 
	\[(q' \varepsilon =_{\Spec A} q' (g \varepsilon)) = \prod_{\phi : A} \phi(q' \varepsilon) = \phi(q' (g \varepsilon)) = \prod_{\substack{\phi : R^{D(p)} \\ \phi \in A}} \phi (\varepsilon) = \phi(g \varepsilon) ,\] 
	%as for any $\phi : A$ we have $\phi(\varepsilon) = \phi(g \varepsilon)$ 
	and the right hand side is inhabited as each $\phi$ satisfies the condition (1). \\
	So we conclude the the embedding $\{0\} \hookrightarrow V$ is an equivalence.  But we asked $V$ to be non contractible.
\end{proof}
\begin{example}
	The scheme 
	\[
	\sum_{x , y : R} x y = 0
	\]
	sheaf-quotiented by the relation that identifies $(x,0)$ and $(0,x)$ if $x \neq 0$ is an algebraic space.
\end{example}
\begin{proof}
	The equivalence relation is given by
	\[
	E((x,y) , (x',y')) = (x = x' \land y = y') + (x \neq 0 \land x = y' \land x' = 0)
	\]
	as $x \neq 0$ implies $y = 0$ as $x y =0$. This is a covering relation, as for any $x' y' = 0$ we have
	\[
	\sum_{x,y : R} x y = 0 \land E((x,y) , (x' , y')) = 1 + (y' \neq 0) \in \Zar \subset \bT
	\]		
\end{proof}



\section{Group quotients}
For this section let $G$ denote a group that is a covering 0-stack. Let $X$ be a sheaf equipped with a $G$ action.
\begin{lemma}
	 $\mu_p = \Spec R[X] / (X^p - 1)$ is covering for $p \neq 0$ prime.
\end{lemma}
\begin{proof}
	It is fppf + \etale as $X^p - 1$ is monic seperable. TODO
\end{proof}
\begin{definition}
	A $G$ action on $X$ is free, if for all $x , y : X$ the type 
	\[
	\sum_{g: G} g x = y
	\]
	is a proposition. 
\end{definition}
\begin{lemma}
	Let $G$ act freely on a sheaf $X$. Then the relation
	\[
	x , y\mapsto \sum_{g : G} g x = y
	\]
	is a covering equivalence relation on $X$
\end{lemma}
\begin{proof}
	All those propositions are modal as $X$ and $G$ are sheaves. For all $x : X$ , the fiber
	\[
	\sum_{y : X} \sum_{g : G} g x = y \simeq \sum_{g : G} \sum_{y: X} g x = y \simeq G
	\]
	is a covering 0-stack by assumption.
\end{proof}
\begin{lemma}{\label{lemma:algSpacesStabFreeQuots}}
	%For $n \ge 0$, geometric $(n)$-stacks 
	Algebraic spaces are stable by free quotients of covering group 0-stacks.
\end{lemma}
\begin{proof}
	The map $ X \to L_T (X / G)$ is fibered in covering 0-stacks, so in particular covering $0$-stacks. As $X$ is a geometric $0$-stack, the quotient is a geometric $0$-stack as well, This follows by the description in \label{prop:nstack}, choosing a geometric atlas of $X$ and postcomposing this to get a geometric atlas of the quotient.
\end{proof}

%\begin{lemma}
%	Let $X$ be a geometric stack, whose identity types are covering stacks. Let $G$ be a finite group acting on $X$. Then $L_\bT (X / G)$ is a geometric stack.
%\end{lemma}
%\begin{proof}
%	Consider for $x , y : X$ , $R(x,y) \equiv \| \sum_{g : G} g x = y\|_\bT$ which is indeed modal. We have to check that the relation is covering, i.e. that for all $x : X$, 
%	\[
%	\sum_{y: X} \|\sum_{g: G} g x = y\|_\bT
%	\]
%	is a covering stack. \\
%	To prove this, as covering stacks are stable under quotients, it suffices to show, that the map
%	\[
%	G \simeq \left (\sum_{y : X} \sum_{g : G} g x = y \right) \to \sum_{y: X} \|\sum_{g: G} g x = y\|_\bT
%	\]
%	is a geometric cover. But the fibers look like $\sum_{g : G} gx  = y$ which is a finite sum of identity types in $X$, which were assumed to be covering stacks. By \ref{lemma:geomStackPlusStable} the fibers are covering stacks.
%	
%\end{proof}



\section{Deloopings and Truncations}

We denote $\| \cdot \|_n^\bT := L_\bT \| \cdot \|_n$, which is a modality. We denote 
\[
\eta_n^\bT X : X \to \| X\|_n^\bT
\]
\begin{definition}
	A pointed stack $(X,x)$ is called $\bT$-1-connected (or $\bT$-connected) if for any $y : X$ we have $\|x = y\|_\bT$. \\
	Inductively, $(X,x)$ is called $\bT$-$n+1$-connected if its $\bT$-connected and $\Omega X$ is $\bT$-$n$-connected.
\end{definition}
\begin{definition}
	Let $G$ be a stack. A delooping stack of $G$ is a  pointed $\bT$-connected stack $B G$ equipped with an equivalence $\Omega B G \simeq G$.
\end{definition}
\begin{lemma}{\label{lemma:detectDelooping}}
	For $X,Y$ pointed stacks, to construct an equivalence $X \simeq B^k Y$ we may show that $X$ is $\bT$-$k$-connected and construct an equivalence $\Omega^k X \simeq Y$.
\end{lemma}
\begin{proof}
	If $k = 1$ its fine. Then $X \simeq B^{k+1} Y$ iff $X$ is $\bT$-connected and $\Omega X \simeq B^k Y$. By induction the latter is equivalent to $\Omega X$ beeing $\bT$-$k$-connected and $\Omega^{k+1} X \simeq Y$.
\end{proof}

\begin{lemma}{\label{lemma:deloopingCS}}
	Let $G$ be a covering stack, that admits a delooping stack $B G$. Then $B G$ is a covering stack.
\end{lemma}
\begin{proof}
	Now assume $G$ is a covering stack. 	To show, that  $B G$ is a covering stack, we may show that the map $\bT \ni 1 \to B G$ is a geometric atlas. As $B G$ is $\bT$-connected, every point is $\bT$-merely equal to the basepoint. By descent for covering stacks, we may just show that the fiber over the basepoint is a covering stack
	But this is equivalent to $\Omega B G \simeq G$. 
\end{proof}
\begin{corollary}
	If $G$ is a covering group 0-stack,  that admits an $n$-fold delooping stack $B^n G$, then this will be a covering $n$-stack.
\end{corollary}
\begin{lemma}
	The fiber of $\eta_n^\bT X : X \to \|X\|_n^\bT$ over $|x|$ is $\sum_{y : X} \| x = y\|_{(n-1)\bT}$
\end{lemma}
\begin{proof}
	For any $x : X$, we may show that the type family
	\begin{align*}
		B : \|X\|_n^\bT &\to \cU_{n-1}^\bT \\
		\|y\|_n &\mapsto \| x = y \|_{n-1}^\bT
	\end{align*}
	defined using the $n$ truncatedness of the stack $\cU_{n-1,\bT}$, is a unary identity system of $\|X\|_{n}^\bT$ at $|x|$. 
	By the fundamental system of identity types its enough to construct for all $y : \| X\|_n^\bT$, a section of the map $|x| = y \to B y$ induced by path induction. \\
	As the space of sections of a map between $n$-stacks is in particular an $n$-stack, we may just for all $y : X$ construct a section of the map 
	\[\mathrm{ind} : |x| =_{\|X\|_n^\bT} |y| \to \| x = y \|_{n-1}^\bT\]
	But $|x| = |y|$ is an $n-1$-stack, so there is a unique dashed map $\sigma$ such that the above triangle
	% https://q.uiver.app/#q=WzAsNCxbMCwxLCJ8eHw9fHl8Il0sWzEsMSwiXFx8eCA9X1ggeSBcXHxfe259XlxcYlQiXSxbMSwwLCJ4ID1fWCB5Il0sWzEsMiwiXFx8eCA9X1ggeSBcXHxfe259XlxcYlQiXSxbMiwxLCJcXGV0YSJdLFsyLDAsIlxcYXAiLDJdLFsxLDAsIlxcc2lnbWEiLDAseyJzdHlsZSI6eyJib2R5Ijp7Im5hbWUiOiJkYXNoZWQifX19XSxbMSwzLCJcXGlkIl0sWzAsMywiXFxtYXRocm17aW5kfSIsMix7ImN1cnZlIjoyfV1d
	\[\begin{tikzcd}
		& {x =_X y} \\
		{|x|=|y|} & {\|x =_X y \|_{n}^\bT} \\
		& {\|x =_X y \|_{n}^\bT}
		\arrow["\ap"', from=1-2, to=2-1]
		\arrow["\eta", from=1-2, to=2-2]
		\arrow["{\mathrm{ind}}"', curve={height=12pt}, from=2-1, to=3-2]
		\arrow["\exists! \sigma", dashed, from=2-2, to=2-1]
		\arrow["\id", from=2-2, to=3-2]
	\end{tikzcd}\]
	commutes. This is indeed a section of the above map, because the maps $\mathrm{ind} \circ \sigma$ and $\id$ targeting an $n$-stack become equal after postcomposition with the unit $\eta$ of the modality $L_\bT \| \cdot \|_n$.
	
	%	
	%	
	%	We may show, that the map
	%	
	%		For this we need that for all $x ,y : X$ the map
	%	\[
	%	\|x =_X y \|_{n-1,\bT} \to |x| =_{\|X\|_{n\bT}} |y|
	%	\]
	%	is an equivalence. 
	
	%	\[
	%	\isContr \sum_{y : \|X\|_n^\bT} B y \simeq \prod_{y : \|X\|_n^\bT} \prod_{t : By}(p : |x| = y) \times (\tp_p r = t)
	%	\]
	%	where $r : B |x|$ is defined as $| \refl_x|$. \\
	%	So Let $x : X$. For any such $y$ the type $\prod_{t : By}(p : |x| =_{\|X\|_n^\bT} y) \times (\tp_p r = t)$ is an $n$-stack so we may just provide a term in
	%	\[
	%	\prod_{y : X} \prod_{t : \| x = y\|_{n-1,\bT}}(p : |x| =_{\|X\|_n^\bT} |y|) \times (\tp_p r = t)	
	%	\]
	%	Let $y : X$. For any such $t$, as $(p : |x| =_{\|X\|_n^\bT} |y|) \times (|p| = t)$ is a $n-1$-stack, we may just construct a term in
	%	\[
	%	\prod_{t : x = y} (p : |x| = |y|) \times \tp_p r = |t|
	%	\]
	%	this is obvious by setting $p := \mathsf{ap}_{|\_|} t$ and we compute
	%	\[
	%	\tp_{\ap_{|\_|} t} |\refl_x| = | t \cdot \refl_x | = |t|
	%	\]
\end{proof}
\begin{lemma}
	For any $X$ and any $n \ge -1$, the map $\eta_n^\bT X : X \to \|X\|_n^\bT$ is $\bT$-surjective.
\end{lemma}
\begin{proof}
	It factors as $X \to \|X\|_n \to L_\bT \|X\|_n$ where the latter map is $\bT$-surjective. So it sufficess to show, that the former map is surjective. As $X \to \|X\|_0$ is surjective it suffices to show, that $\ap$ of the map $\|X\|_n \to \|X\|_0$ is surjective. TODO %But for any type $T$ the map $T \to \|T\|_0$ is surjective.
\end{proof}
\begin{notation}
	Given a map $f : X \to Y$ and some $x : X$ we denote $	\fib f x$ for the pointed type
	\[
	\fib f x \equiv (\fib_f (f x) , (x , \refl))
	\]
	and $f , x$ for the pointed map 
	\[
	(f , \refl_{f x}) : (X , x) \to (Y,f(x))
	\]
\end{notation}
\begin{lemma}{\label{lemma:loopOfFiber}}
	If $(X,x)$ is a pointed stack, the looping of the fiber of $X \to \|X\|_{n}^\bT$ over $|x|$ is the basefiber of $\Omega X \to \|\Omega X\|_{n-1}^\bT$.
	\[
	\Omega (\fib(\eta_n^\bT X)(x)) \simeq \fib (\eta_{n-1}^\bT \Omega (X,x))(pt)
	\]
\end{lemma}
\begin{proof}
	We have to understand the loop space of $\sum_{y : X} \| x = y\|_{(k-1)\bT}$. It is given by
	\[(p : \Omega X) \times \left(\tp_p r =_{\|\Omega X\|_{k-1,\bT}} r  \right),\]
	where $ r= |\refl|$.
	we calculate $\tp_p r = |p|$, so it is the fiber of 
	\[
	\Omega X  \to \|\Omega X \|_{k-1, \bT}
	\]
	over the basepoint $|\refl|$.
	
	Alternative proof
% https://q.uiver.app/#q=WzAsMyxbMCwwLCJcXE9tZWdhIChYICwgeCkiXSxbMSwxLCJcXHxcXE9tZWdhIChYICwgeClcXHxfe24tMX1eXFxiVCJdLFsxLDAsIlxcT21lZ2EgKFxcfFhcXHxfbl5cXGJUICwgfHh8KSAiXSxbMCwyLCJcdFxcT21lZ2EgKFxcZXRhX25eXFxiVCBYICwgeCkiXSxbMiwxLCJcXHNpbWVxICJdLFswLDEsIlxcZXRhX3tuLTF9XlxcYlQgKFxcT21lZ2EgKFgseCkpIiwyLHsibGFiZWxfcG9zaXRpb24iOjAsIm9mZnNldCI6MX1dXQ==
\[\begin{tikzcd}
	{\Omega (X , x)} & {\Omega (\|X\|_n^\bT , |x|) } \\
	& {\|\Omega (X , x)\|_{n-1}^\bT}
	\arrow["{	\Omega (\eta_n^\bT X , x)}", from=1-1, to=1-2]
	\arrow["{\eta_{n-1}^\bT (\Omega (X,x))}"'{pos=0}, shift right, from=1-1, to=2-2]
	\arrow["{\simeq }", from=1-2, to=2-2]
\end{tikzcd}\]
	\[
		\Omega (\fib(\eta_n^\bT X)(x)) = \fib (\Omega (\eta_n^\bT X ,x)) pt = \fib (\eta_{n-1}^\bT \Omega (X ,x)) pt
	\]
\end{proof}
\begin{prop}{\label{prop:LoopingsImplyGerbe}}
	Let $n \ge 0$ , $X$ be a geometric stack, such that for all $x : X$, $\Omega^{n+1} (X , x)$ is a covering stack for all $x : X$. Then $\eta_n^\bT X : X \to \|X\|_n^\bT$ is a geometric cover. In particular, $\|X\|_n^\bT$ is a geometric $n$-stack.
\end{prop}
\begin{proof}
	Let us show by induction over $k = -1,\hdots,n$ that 
	\[\eta_k^\bT (\Omega^{n - k} X) : \Omega^{n - k} X \to \|\Omega^{n - k} X\|_k^\bT\]
	is a geometric cover.  \\
	$k=-1$ is okay as $\Omega^{n+1} X$ is a covering stack and $\bT$-truncations of covering stacks are contractible. \\
	For the induction step $k - 1 \mapsto k$:
	Set $X' := \Omega^{n-k} X$, so we want to show that $X' \to \|X'\|_k^\bT$ is a geometric cover.
	Every fiber is modal so the fiber beeing a covering stack has descent, so we may just show that the fiber over the image of some $x : X$ is a covering stack. The fiber $\sum_{y : X} \| x = y\|_{(k-1)\bT}$ is $\bT$-connected, so its a delooping stack of the basefiber of 
	\[
	\Omega X  \to \|\Omega X \|_{k-1, \bT}
	\]
	by \ref{lemma:loopOfFiber} and \ref{lemma:deloopingCS} we conclude.
\end{proof}


\begin{definition}
	A higher group stack is a pointed $\bT$-connected stack.
\end{definition}
Let $BG$ be a higher group stack and $X$ be a geometric stack equipped with an action $\rho : BG \to \GS$. We use the standart notation

\[
X // G :\equiv \sum_{BG} \rho
\]
\begin{lemma}
	If $G$ is covering, then $X // G$ is a geometric stack
\end{lemma}
\begin{proof}
	$BG$ is a covering stack, as $G$ is a covering stack \ref{lemma:deloopingCS}. Hence $X // G :\equiv \sum_{BG} \rho$ is a geometric stack.
\end{proof}
\begin{prop}[Special case of \ref{prop:htpyGroups}]
	%Let $BG$ be a higher group stack and $X : BG \to \GS$ be a geometric $G$-stack. 
	If $X // G$ is a geometric stack (e.g. if $G$ is covering) and the isotropy stacks $\sum_{g : G} g x = x$ are covering stacks, then $\| X // G \|_0^\bT$ is an algebraic space.
\end{prop}
\begin{proof}
	To apply the prop, we have to show, that for all $x : X // G$, $\Omega (X // G,x)$ is a covering stack. As $X \to X // G$ is $\bT$-surjective (todo details), we may just show this for $x : X$.
	\[
	\Omega (X // G , [x]) \simeq \sum_{g: G} g x = x
	\]
	which was covering by assumption
\end{proof}
\begin{corollary}
	 Let $G$ be a covering group sheaf (e.g. finite group), acting on a geometric stack $X$ with  $\bT$-flat identity types. Then $L_\bT (X / G)$ is a geometric stack.
\end{corollary}
\begin{proof}
	The isotropy stacks are covering by \ref{lemma:detectCovering}, as they are $\sum$ of $\bT$-flat geometric stacks and they are $\bT$-merely inhabited 
\end{proof}
We can also reprove \ref{lemma:algSpacesStabFreeQuots}: $G$ is a finite type by assumption, hence covering. The isotropy stacks are assumed to be propositions, but they are inhabited, so they are covering \qed(lemma) \\

TODO: Find a good example of a non covering $G$.
\section{Local properties}

\begin{definition}
	Let $\cV \subset \cU$ a subclass of types be stable under finite limits. 
	We call a property $P$ of morphisms of types in $\cV$ $\bT$-local, if 
	\begin{enumerate}
		\item its satisfied by identities in $\cV$,
		\item stable under composition 
		\item
		Given a commutative triangle in $\cV$% https://q.uiver.app/#q=WzAsMyxbMCwwLCJYIl0sWzEsMCwiWSJdLFsxLDEsIloiXSxbMCwxLCJmIl0sWzEsMiwiZyJdLFswLDIsImgiLDJdXQ==
		\[\begin{tikzcd}
			X & Y \\
			& Z
			\arrow["f", from=1-1, to=1-2]
			\arrow["h"', from=1-1, to=2-2]
			\arrow["g", from=1-2, to=2-2]
		\end{tikzcd}\]
		with $X \to Y$ a geometric cover (wrt to $\bT$). Then $h$ has $P$ iff $g$ has $P$
		
	
	\end{enumerate}

\end{definition}
\begin{definition}
	$P$ has descent along geometric covers: Given $X,Y,Z,W \in \cV$. if $Y \to W$ is a geometric cover , then 
	% https://q.uiver.app/#q=WzAsNCxbMCwwLCJYIl0sWzAsMSwiWSJdLFsxLDEsIlciXSxbMSwwLCJaIl0sWzAsM10sWzAsMSwiZiciLDJdLFsxLDJdLFszLDIsImYiXSxbMCwyLCIiLDEseyJzdHlsZSI6eyJuYW1lIjoiY29ybmVyLWludmVyc2UifX1dXQ==
	\[\begin{tikzcd}
		X & Z \\
		Y & W
		\arrow[from=1-1, to=1-2]
		\arrow["{f'}"', from=1-1, to=2-1]
		\arrow["\ulcorner"{anchor=center, pos=0.125}, draw=none, from=1-1, to=2-2]
		\arrow["f", from=1-2, to=2-2]
		\arrow[from=2-1, to=2-2]
	\end{tikzcd}\]
	If $f$ has $P$ then $f'$ has $P$.
\end{definition}
	\begin{lemma}{\label{lemma:local}}
	If $P$ is local, then 
	\begin{itemize}
		\item geometric covers have P
		\item in descent, The statement 'If $f'$ has $P$ then $f$ has $P$' is automatic by Point 3.
	\end{itemize}
\end{lemma}
\begin{lemma}
	Beeing a geometric cover is local.
\end{lemma}
\begin{proof}
	Reduce to the case of $Z = 1$. If $X \to Y$ is a geometric cover, then $X$ is a covering stack iff $Y$ is a covering stack by stability under quotients and under sums. If both are coverings stacks, then the fibers 
\end{proof}

\begin{lemma}{\label{lemma:atlasOfMap}}
	Let $P$ be a local property of morphisms of geometric stacks. For
	A morphism between geometric stacks $f : X \to Y$  TFAE 
	\begin{enumerate}
		\item $f$ has $P$
		
		\item For any Atlas $\Spec A \to Y$ and any atlas $S \to X \times_Y \Spec A$ the composite $S \to \Spec A$ has $P$
		\item $f$ has an atlas that has $P$.
	\end{enumerate}
\end{lemma}
\begin{proof}
	\begin{itemize}
		\item [1 $\Rightarrow$ 2]
		Given a geometric atlas $\Spec A \to Y$ and taking the pullback % https://q.uiver.app/#q=WzAsNCxbMCwwLCJYIFxcdGltZXNfWSBcXFNwZWMgQSJdLFswLDEsIlgiXSxbMSwwLCJcXFNwZWMgQSJdLFsxLDEsIlkiXSxbMSwzLCJmIl0sWzAsMiwiZiciXSxbMiwzXSxbMCwxXV0=
		\[\begin{tikzcd}
			{X \times_Y \Spec A} & {\Spec A} \\
			X & Y
			\arrow["{f'}", from=1-1, to=1-2]
			\arrow[from=1-1, to=2-1]
			\arrow[from=1-2, to=2-2]
			\arrow["f", from=2-1, to=2-2]
		\end{tikzcd}\]
		$f'$ has $P$ as a basechange of $f$ along a geometric cover. Given a geometric atlas $S \to X \times_Y \Spec A$, it will have $P$, the composition $S \to \Spec A$ will be in $P$.
		\item [2 $\Rightarrow$ 3]
		$Y$ is a geometric stack, hence admits some geom atlas $\Spec A \to Y$. Again, $X \times_Y \Spec A$ is a geometric stack hence admits a geometric atlass.
		\item [3 $\Rightarrow$ 1]
		If we have an atlas $\tilde f : \tilde X \to \tilde Y$, then $\tilde X \to \tilde Y \to Y$ has $P$ by stability under composition. Then by (4) $X \to Y$ has $P$, as $\tilde X \to X$ is a geometric cover \\
		
	\end{itemize}
\end{proof}

So we may extend algebraic notions of maps to all geometric stacks:
\begin{definition}
	Let $P$ be a property of morphisms $\bT$-local in affine schemes.
	
	We define a morphism of geometric stacks $f : X \to Y$ to have $P$ iff
	there exist atlasses and a $P$-map on affines 
	% https://q.uiver.app/#q=WzAsNCxbMCwwLCJcXFNwZWMgQSJdLFswLDEsIlgiXSxbMSwxLCJZIl0sWzEsMCwiXFxTcGVjIEIiXSxbMywyXSxbMCwxXSxbMSwyLCJmIiwxXSxbMCwzLCJcXGhhdCBmIiwwLHsic3R5bGUiOnsiYm9keSI6eyJuYW1lIjoiZGFzaGVkIn19fV1d
	\[\begin{tikzcd}
		{\Spec A} & {\Spec B} \\
		X & Y
		\arrow["{\hat f}", dashed, from=1-1, to=1-2]
		\arrow[from=1-1, to=2-1]
		\arrow[from=1-2, to=2-2]
		\arrow["f"{description}, from=2-1, to=2-2]
	\end{tikzcd}\]	
\end{definition}
\begin{lemma}
	Let $P$  be a local property of affine schemes. The induced property of morphisms of geometric stacks is local. Additionally descent is inherited.
\end{lemma}
\begin{proof}
	\begin{enumerate}
		\item Ok
		\item Ok
	
		\item geometric covers have $P$ and we have proven point 2., so one direction is clear.  Now assume 
		%https://q.uiver.app/#q=WzAsMyxbMCwwLCJYIl0sWzEsMCwiWSJdLFsxLDEsIloiXSxbMCwxLCJmIl0sWzEsMiwiZyJdLFswLDIsImgiLDJdXQ==
		\[\begin{tikzcd}
			X & Y \\
			& Z
			\arrow["f", from=1-1, to=1-2]
			\arrow["h"', from=1-1, to=2-2]
			\arrow["g", from=1-2, to=2-2]
		\end{tikzcd}\]
		Such that $f$ is a geometric cover and $h$ has $P$. \\
		We first reduce to the case where $Z$ is affine. Choose a geometric atlas $\tilde Z \to Z$. Then take the pullbacks
		% https://q.uiver.app/#q=WzAsNixbMiwwLCJcXHRpbGRlIFoiXSxbMiwxLCJaIl0sWzEsMCwiWSciXSxbMSwxLCJZIl0sWzAsMCwiWCciXSxbMCwxLCJYIl0sWzIsM10sWzMsMV0sWzAsMV0sWzIsMF0sWzIsMSwiIiwxLHsic3R5bGUiOnsibmFtZSI6ImNvcm5lci1pbnZlcnNlIn19XSxbNSwzXSxbNCw1XSxbNSwxLCJQIiwxLHsiY3VydmUiOjJ9XSxbNCwyXSxbNCwzLCIiLDEseyJzdHlsZSI6eyJuYW1lIjoiY29ybmVyLWludmVyc2UifX1dXQ==
		\[\begin{tikzcd}
			{X'} & {Y'} & {\tilde Z} \\
			X & Y & Z
			\arrow[from=1-1, to=1-2]
			\arrow[from=1-1, to=2-1]
			\arrow["\ulcorner"{anchor=center, pos=0.125}, draw=none, from=1-1, to=2-2]
			\arrow[from=1-2, to=1-3]
			\arrow[from=1-2, to=2-2]
			\arrow["\ulcorner"{anchor=center, pos=0.125}, draw=none, from=1-2, to=2-3]
			\arrow[from=1-3, to=2-3]
			\arrow[from=2-1, to=2-2]
			\arrow["P"{description}, curve={height=12pt}, from=2-1, to=2-3]
			\arrow[from=2-2, to=2-3]
		\end{tikzcd}\]
		$X' \to Y'$ is a geometric cover and By 3. $X' \to \tilde Z$ has $P$. \\
		So we may assume that $Z$ is affine. Then take a geometric atlas $\tilde X \to X$. The map $Y \to Z$ has the atlas $\tilde X \to X \to Z$ which has $P$ by stability under composition. Hence $Y \to Z$ has $P$.
			\item We show also descent: By \ref{lemma:local} we only need to show stability under basechange. Let $Z \to W$ have $P$, Given $Y \to W$ a geometric cover. We want to show that a basechagne $Y \times_W Z \to Y$ has $P$. The idea is to construct a common atlas of $Z \to W$ and its basechange. Choose an atlas $\tilde Y \to Y$. Then $\tilde Y \times_W Z \to Y \times_W Z$ is a geometric cover: It is a basechange of $\tilde Y \to Y$, because the outer diagram is a pullback
		% https://q.uiver.app/#q=WzAsNixbMCwwLCJcXHRpbGRlIFkgXFx0aW1lc19XIFoiXSxbMSwwLCJZIFxcdGltZXNfVyBaIl0sWzAsMSwiXFx0aWxkZSBZIl0sWzEsMSwiWSJdLFsyLDAsIloiXSxbMiwxLCJXIl0sWzAsMl0sWzAsMV0sWzIsM10sWzEsM10sWzMsNV0sWzEsNF0sWzQsNV0sWzEsNSwiIiwxLHsic3R5bGUiOnsibmFtZSI6ImNvcm5lci1pbnZlcnNlIn19XSxbMCwzLCIiLDEseyJzdHlsZSI6eyJuYW1lIjoiY29ybmVyLWludmVyc2UifX1dXQ==
		\[\begin{tikzcd}
			{\tilde Y \times_W Z} & {Y \times_W Z} & Z \\
			{\tilde Y} & Y & W
			\arrow[from=1-1, to=1-2]
			\arrow[from=1-1, to=2-1]
			\arrow["\ulcorner"{anchor=center, pos=0.125}, draw=none, from=1-1, to=2-2]
			\arrow[from=1-2, to=1-3]
			\arrow[from=1-2, to=2-2]
			\arrow["\ulcorner"{anchor=center, pos=0.125}, draw=none, from=1-2, to=2-3]
			\arrow[from=1-3, to=2-3]
			\arrow[from=2-1, to=2-2]
			\arrow[from=2-2, to=2-3]
		\end{tikzcd}\]
		Now choose any geometric atlas $S \to \tilde Y \times_W Z$. By composition this induce a map $S \to \tilde Y$: 
		It is both an atlas of the $P$-map $Z \to W$ and of $ Y\times_W Z \to Y$. So by \ref{lemma:atlasOfMap} $S \to \tilde Y$ has $P$ and thus $Y \times_W Z \to Y$ has $P$. 
		
	\end{enumerate}
\end{proof}
\subsection{Local properties of stacks}
\begin{definition}
	Let $\cV \subset \cU$ be a subclass of types stable under finite limits. A property $P$ of types in $\cV$ is local if 
	\begin{enumerate}
		\item $1 \in P$
		\item $P$ is $\sum$-stable
		\item If $X \to Y$ is a geometric cover between types in $\cV$, then $X$ has $P$ iff $Y$ has $P$.
		
	\end{enumerate}
	We say $P$ has descent if for all $X : \cV$, then $X$ having $P$ is a $\bT$-sheaf.
\end{definition}
\begin{lemma}
	Every local property of types in $\cV$ induces a local property of morphisms of types in $\cV$, by asking the property fiberwise.
\end{lemma}
\begin{proof}
	Use descent for the descent along a geometric cover ($\bT$-surjective!).
\end{proof}
\begin{lemma}
	Let $P$ be a $\sum$-stable-property of affines containg $\bT$. The induced property of geometric stacks is $\bT$-local.
\end{lemma}
\begin{proof}
	The $\sum$-stability is \ref{thm:atlasStableSum}. Covering stacks have $P$, as $\bT \subset P$. The quotient stability is straightforward.
\end{proof}
\subsection{Seperatedness}
\begin{definition}
Let $P$ be a $\bT$-local property of stacks.
We call a stack $P$-seperated, iff its identity types are $P$ stacks.
\end{definition}


\begin{lemma}{\label{lemma:SeperationIsLocal}}
	Let $P$ be a $\bT$-local property of stacks. Then beeing $P$-seperated is a $\bT$-local property %Then $\mathrm{Sep}(P)$ is local, defined as $X \in \mathrm{Sep}(P)$ iff $x = y$ has $P$ for all $x ,y : X$.
\end{lemma}
\begin{proof}
	Let $f : X \to Y$ be a geometric cover with $X$ beeing $P$-seperated. Let $x , y : Y$. Then by the construction in \ref{lemma:havingAbstractAtlasClosedUnderId} the map
	\[
	\fib_f x \times_X \fib_f y \to x = y
	\]
	is a geometric cover, whose domain has $P$ as $X$ is $P$-seperated and $P$ is stable under $\sum$. As $P$ is local, $x = y$ has $P$. \\
%	TODO Beeing $\bP$-seperated has descent.
\end{proof}
%
%\begin{example}
%	If $\cP$ is formally \etale, then beeing $\cP$-seperated means formally unramified.
%\end{example}
%\begin{lemma}{\label{lemma:SetUnramfied}}
%	Given morphisms of types $X \to Y \to Z$ with $X \to Z$ $\cP$-seperated and $Y$ a set, then $X \to Y$ is $\cP$-seperated.
%\end{lemma}
%\begin{proof}
%	We can argue fiberwise so we may assume $Z$ beeing the point. A fiber of $f : X \to Y$ over $y$ is $\sum_{x: X} f x = y$ where $f x = y$ is a proposition,hence $\cP$-seperated. As $X$ is $\cP$-seperated we conclude as $\cP$-seperated types are $\sum$-stable.
%\end{proof}

\begin{lemma}{\label{lemma:PImpliesPsep}}
	If $\sum$-stable property of affine schemes containing $\bT$ is stable under identity types, then the induced $\bT$-local property of geometric stacks is as well. %In this case the identity types of a $P$ stack have $P$ as well. \\	
\end{lemma}
\begin{proof}
	Let $X$ be a $P$ geometric stack. Let $x,y : X$ we want to show that $x =_X y$ has $P$. Choose a geometric atlas $P \ni S \overset{f}{\to} X$ . By assumption $S$ is $P$-seperated. We have a geometric atlas $\fib_f x \times_S \fib_f y \to x = y$. The domain is a $\sum$ of types in $\bT$ and identity types of $S$, which have $P$ by stability under identity types for the affine $S$. Hence $x = y$ has $P$. 
\end{proof}

\begin{lemma}{\label{lemma:FEtLocal}}
	formally \etale is an \etale-local property of geometric stacks.
	%	If $\bT=$ \etale, then 
\end{lemma}
\begin{proof}
	If $\Spec B \to X$ is a geometric cover  and $\Spec B \in \bT$  then we want to show $X$ is formally \etale + fppf . fppf is clear by saturatedness of the fppf topology. %${\bP_{et}}$-cover (i.e. an \etale faithfully flat map)
	Take $L$ to be the modality which nullafies the propositions $\|\Spec A\|$ for $\Spec A$ \etale + fppf and all close dense propositions.
	The square 
	% https://q.uiver.app/#q=WzAsNCxbMCwwLCJcXFNwZWMgQiJdLFsxLDAsIlxcU3BlYyBBIl0sWzAsMSwiRXQoXFxTcGVjIEIpIl0sWzEsMSwiRXQoXFxTcGVjIEEpIl0sWzEsM10sWzAsMiwiXFxzaW0iXSxbMiwzLCIiLDEseyJzdHlsZSI6eyJoZWFkIjp7Im5hbWUiOiJlcGkifX19XSxbMCwxXSxbMSwzXSxbMCwzLCIiLDEseyJzdHlsZSI6eyJuYW1lIjoiY29ybmVyLWludmVyc2UifX1dXQ==
	\[\begin{tikzcd}
		{\Spec B} & {X} \\
		{L(\Spec B)} & {L(X)}
		\arrow[from=1-1, to=1-2]
		\arrow["\sim", from=1-1, to=2-1]
		\arrow["\ulcorner"{anchor=center, pos=0.125}, draw=none, from=1-1, to=2-2]
		\arrow[from=1-2, to=2-2]
		\arrow[from=1-2, to=2-2]
		\arrow[from=2-1, to=2-2]
	\end{tikzcd}\]
	is a pullback as $L$ is lex. 
	We want to show, the right map is an equivalence. 
	Every type occuring is an \etale-stack.
	As the lower map is \etale-surjective, and the left vertical map is an equivalence, we can conclude.
\end{proof}




\subsection{Weakly-flat stacks}
\begin{definition}
	We call a geometric stack $X$ weakly-flat iff one of the following conditions is satisfied
	\begin{enumerate}
		\item $\|X\|_{-1}^\bT \to X \in \CS$
		\item For any geometric atlas $W \to X$ , $W$ is weakly-flat, i.e $\|W\|^\bT \to W \in \bT$.
	\end{enumerate}

\end{definition}
\begin{proof}
	\ \begin{enumerate}
		\item [1 $\Rightarrow$ 2] Choose a geometric atlas $W \to X$. In particular its $\bT$-surjective, hence we have $\|W\|^\bT$, so by assumption $W \in \bT$. So $X \in CS$.
		\item [2 $\Rightarrow$ 1] 
		\[\|W\|^\bT \to \|X\|^\bT \to X \in \CS \overset{\ref{lemma:atlasIsCatlas}}{\to} W \in \bT \]
	\end{enumerate}
\end{proof}
They behave bad as they are not stable under $\sum$ (and not under $\id$-types, although this holds for affines).
%		\item 
%
\begin{lemma}
For any weakly-flat geometric stack $X$, $\|X\|_{-1}^\bT$ is a geometric stack.
\end{lemma}
\begin{proof}
	$X \to \|X\|_{-1}^\bT$ is a geometric cover.
\end{proof}
\begin{prop}{\label{prop:TruncationCover}}
	We may define $X$ to be $0$-wf-seperated, iff its weakly flat and $n+1$-wf-seperated, iff identity types of $X$ are $n$-wf-seperated. \\
	%Assume that loopspaces of covering stacks are covering. TFAE
	For $X$ a geometric stack, TFAE \begin{enumerate}
	%	\item $X \to \|X\|_n^\bT$ is a geometric cover
		\item  $X$ is $n+1$-wf-seperated, i.e. all $n+1$-fold identity types of $X$ are weakly-flat.
	
		\item For any $x$, $\Omega^{n+1} (X , x)$ is covering.
		\item For any $x: X$ , $x=x$ is $n$-wf-seperated, i.e. $n$-fold identity types of $x=x$ are weakly flat. %For any sequence $x_0 : X \equiv X_0$ , $x_{k+1} : (x_{k} =_{X_k} x_k) \equiv X_{k+1}$, $x_n =_{X_n} x_n$ is covering.
	\end{enumerate}

\end{prop}
\begin{proof}
	
%	Claim: A stack beeing $n$-wf-seperated has descent.
%	\item [3 $\Rightarrow$ 1] already done
	\item [1 $\Rightarrow$ 3 $\Rightarrow$ 2] ez
%	\item [1 $\Rightarrow$ 3]

%	If for any $x$ , the fiber of $\eta_n^\bT X$ over $|x|$ is covering, then its $n+1$-fold loop space covering by assumption, so also $\Omega^{n+1} (X , x)$.	
	\item [3 $\Rightarrow$ 1] We prove this by induction. $n = 0$. To show that $x =_{X} y$ is weakly-flat, by descent we may assume that $x = y$. Then we have $(x = y) \simeq (x =_{X} x)$. By assumption this is weakly flat. \\
	Assume now, that for any $x : X$, that $x = x$ is $n$-wf-seperated. Let $x, y : X$. We want to show that $x = y$ is $n$-wf-seperated. By induction we may just prove that for any $p : x = y$, $p = p$ is $n-1$-wf-seperated. Applying $p \cdot \_$ induces an equivalence $\refl_x = \refl_x \simeq p = p$ . But $x = x$ is $n$-wf-seperated, hence $\refl_x = \refl_x$ is $n-1$-wf-seperated.
	\item [2 $\Rightarrow$ 3] Induction. $n=0$ is fine. 
	Let $x : X$. To show that $\Omega(X,x)$ is $n$-wf-seperated, just use that $\Omega^n(\Omega(X,x))$ is covering, hence by the inductive statement $2 \Rightarrow 3 \Rightarrow 1$, we now that $\Omega(X,x)$ is $n$-wf-seperated.
	%Applying $x_k \cdot \_$ from the left induces an equivalence
%	\[\refl =_{X_k} \refl \to x_k =_{X_k} x_k \]
%	Conclude inductively
\end{proof}
\section{Omega-stability and gerbes}
\begin{definition}
	A geometric stack $X$ is an $n$-gerbe iff the map $\eta_n^\bT : X \to \|X\|_n^\bT$ is a geometric cover.
\end{definition}
\begin{example}
	If $G$ is a covering group sheaf, then $BG$ is a 0-gerbe.
\end{example}
\begin{example}
	It may happen, that $\|X\|_n^\bT$ is a geometric $n$-stack for $X$ a geometric stack, although $X$ is not an $n$-gerbe. Indeed: Put $n = 0$ and $X$ any pointed $\bT$-connected geometric stack that is not covering, like $\Susp(1 + x = 0)$ for some 
\end{example}

\begin{theorem}
	Assume that Covering stacks are $\Omega$-stable, % either that loop spaces of covering spaces are covering or that .\\
	Then every geometric stack is a 1-gerbe. %For any geometric stack $X$, the map $X \to \|X\|_1^\bT$ is a geometric cover. In particular $\|X\|_1^\bT$ is a geometric stack.
\end{theorem}
\begin{proof}
	By \ref{prop:LoopingsImplyGerbe}, we need to show that for any $x : X$, $\Omega^2(X,x)$ is covering.
	choose an geometric atlas $f : S \to X$. by descent we may only show that $\Omega^2(X , fs)$ for $s : S$ is covering.
	%	 % https://q.uiver.app/#q=WzAsMTAsWzMsMSwiMSJdLFszLDIsIlgiXSxbMiwyLCJTIl0sWzIsMSwiXFxzdW1fe3QgOiBTfWZ0ID1fWCBmcyJdLFsxLDEsImZzID0gZnMiXSxbMSwyLCIxIl0sWzEsMCwiMSJdLFsyLDAsIjEiXSxbMCwwLCJcXE9tZWdhKGZzID0gZnMpIl0sWzAsMSwiMSJdLFsyLDEsImYiXSxbMCwxLCJmcyJdLFszLDJdLFszLDBdLFs1LDIsInMiXSxbNCw1XSxbNCwzXSxbNywzLCIocyxcXHJlZmwpIl0sWzYsNF0sWzYsN10sWzksNCwiXFxyZWZsIl0sWzgsNl0sWzgsOV1d
	%	 \[\begin{tikzcd}
	%	 	{\Omega(fs = fs)} & 1 & 1 \\
	%	 	1 & {fs = fs} & {\sum_{t : S}ft =_X fs} & 1 \\
	%	 	& 1 & S & X
	%	 	\arrow[from=1-1, to=1-2]
	%	 	\arrow[from=1-1, to=2-1]
	%	 	\arrow[from=1-2, to=1-3]
	%	 	\arrow[from=1-2, to=2-2]
	%	 	\arrow["{(s,\refl)}", from=1-3, to=2-3]
	%	 	\arrow["\refl", from=2-1, to=2-2]
	%	 	\arrow[from=2-2, to=2-3]
	%	 	\arrow[from=2-2, to=3-2]
	%	 	\arrow[from=2-3, to=2-4]
	%	 	\arrow[from=2-3, to=3-3]
	%	 	\arrow["fs", from=2-4, to=3-4]
	%	 	\arrow["s", from=3-2, to=3-3]
	%	 	\arrow["f", from=3-3, to=3-4]
	%	 \end{tikzcd}\]
	%By pullback pasting every square is a pullback. Hence 
	
	% 	\[
	% 	\Omega(fs = fs) \simeq \Omega(\sum_{t : S} ft = fs) 
	% 	\]
	\[
	\Omega(\sum_{t : S} ft = fs) \simeq \left (\sum_{p : \Omega(S,s)} \tp_p (\refl_{f s}) = \refl_{f s} \right) \simeq \refl =_{f s = f s} \refl
	\]
	where the last equivalence is obtained, as $\Omega(S,s)$ is contractible with center $\refl_s$.
	
	So $\Omega^2(X,fs)$ is the loop space of a covering stack, hence by assumption covering.
\end{proof}
\begin{corollary}
	Any Deligne Mumford Stack is a 1-gerbe
\end{corollary}
\begin{proof}
	Use that \etale topology is lex-flattened and \ref{prop:LexFlattenedOmegaStable}.
\end{proof}
\begin{prop}{\label{prop:GerbeIffLooping}}
	This proposition seems only interesting for $n = 0$ by the previous theorem. Assume that covering stacks are $\Omega$-stable. Then $X$ is an $n$-gerbe iff $\Omega^{n+1}(X,x)$ is covering for all $x: X$	
\end{prop}
\begin{proof}
	One direction is \ref{prop:LoopingsImplyGerbe}. The other follows
	
	By applying iteratively \ref{lemma:loopOfFiber} 
	\begin{align*}
		\Omega^{n+1} (\fib (\eta_n^\bT X) |x|) &\simeq \Omega^n \fib (\eta_{n-1}^\bT (\Omega (X,x))) pt \simeq \hdots \\
		&\simeq \Omega^{n-k} \fib(\eta_{n-k-1}^\bT \Omega^{k+1}(X,x)) pt \simeq \hdots \\
		&\simeq \fib (\eta_{-1}^\bT \Omega^{n+1} (X,x)) pt \\
		&\simeq \Omega^{n+1} (X , x)
	\end{align*}
	The LHS is covering by $\Omega$-stability.
\end{proof}
We can reprove \ref{lemma:TflatIs0-gerbe} by just observing that $\bT$-flat geometric stacks have covering loop spaces. \\

\begin{rmk}
	Put $\bT$ the \etale topology. Observe, that we have an analogous statement if we replace covering stack by formally \etale: \\
	\begin{enumerate}
		\item $\eta_0^\bT X : X \to \|X\|_0^\bT$ is formally \etale
		\item $X \to \|X\|_0^\bT$ is formally unramified
		\item for every $x : X$, $\Omega(X,x)$ is formally \etale.
	\end{enumerate}
\end{rmk}
\begin{proof}
	\begin{enumerate}
		\item[ 1 $\Leftrightarrow$ 2] Observe that the map $\eta_0^\bT$ is $\bT$-smooth.
		\item [2 $\Rightarrow$ 3] okay as the fibers of $\eta_0^\bT$ embed into $X$.
		\item [3 $\Rightarrow$ 2] Let $x, y : X$ be $\bT$-merely equal. The goal $\mathsf{isFormallyEtale}(x = y)$ is a sheaf, so we may assume that $x = y$. 
	\end{enumerate}
\end{proof}
\begin{corollary}
	If covering stacks are $\Omega$-stable, then identity types of geometric stacks are 0-gerbes.
\end{corollary}
\begin{proof}
	We need to check, that identity types of a 1-gerbe $X$ are 0-gerbes. So assume $p : x = y$. Then 
	\[
	\Omega(x = y, p) = \Omega(x = x, \refl) = \Omega^2(X,x)
	\]
	which is covering as $X$ is a 1-gerbe.
\end{proof}

%\begin{lemma}
%	Let $\EF: X \to Y$ be a $\lnot \lnot$-surjective map into a type with $\etale$-flat identity types. Then its a geometric cover.
%\end{lemma}
%\begin{prop}
%	Assume $\bT$ is lex-flattened. Let $X$ be a geometric stack. TFAE
%	\begin{enumerate}
%		\item $X$ is an etale-flat geometric stack
%		\item All the identity types are \etale-flat.
%		\item Loop spaces are covering
%		\item $X$ is a	0-gerbe
%	\end{enumerate}
%\end{prop}
%\begin{proof}
%	
%\end{proof}
%The goal of this subsection is the following:




%
%\begin{lemma}
%	Given a geometric atlas $p : X_0 \to X$ of a geometric stack $X$, $\|X_0 \times_X X_0\|_0^\bT$ is an algebraic space, iff for any $x, y : X$ , $\| x =_X y \|_0^\bT$ are algebraic spaces.	
%\end{lemma}
%\begin{proof}
%	The map $p : X_0 \to X$ is $\bT$-surjective, so by descent we may just show the respective statement for $p x, p y : X$ for $x ,y : X_0$. 
%	Now we use 	that the fibers of the map
%	\[
%	\|X_0 \times_X X_0\|_0^\bT = \sum_{x y : X_0} \| x =_X y\|_0^\bT \to X_0 \times X_0 
%	\] %= \sum_{x , y} |x| =_{\|X\|_1^\bT} |y|
%	are geometric stacks iff the domain is a geometric stack.
%	
%\end{proof}
%\begin{lemma}
%	TFAE
%	\begin{enumerate}
%		\item  Any geometric atlas $X_0 \to X$ satisfies that $p_1 : \|X_0 \times_X X_0\|_0^\bT \to X_0$ is covering
%		\item $X \to \|X\|_1^\bT$ is covering
%		\item  $\Omega^2(X,x)$ is covering for all $x : X$. 
%	\end{enumerate}
%
%\end{lemma}
%\begin{proof}
%	2. and 3. are equivalent by \ref{prop:TruncationCover}. Choose a geometric atlas $X_0 \to X$. Then 
%	\[
%	\|X_0 \times_X X_0\|_0^\bT \simeq \| \sum_{x y : X_0} p x =_X py \|_0^\bT \simeq \sum_{x y  : X_0} \| px  =_X p y \|_0^\bT = \sum_{x y : X_0} |x| =_{\|X\|_1^\bT} |y| \simeq X_0 \times_{\|X\|_1^\bT} X_0
%	\]
%	So $\|X_0 \times_X X_0\|_0^\bT \to X_0$ is covering iff $X_0 \to \|X\|_1^\bT$ is covering iff $X \to \|X\|_1^\bT$ is covering by stability under covers.
%\end{proof}

	%	$X \to \|X\|_0^\bT$ is a geometric cover iff for all $x : X$, $\sum_{y : X} \|x = y\|^\bT$ is covering. If the identity types are weakly-flat then 
%\[
%\bT \ni 1 \simeq \sum_{y : X} x = y \to \sum_{y : X} \| y = x\|^\bT
%\]
%is a geometric catlas. \\
%
\section{Flat}
Consider a lex modality $L$.
\begin{definition}
	A subset of affines $\bP$ \emph{flattens} $L$ if
	\begin{itemize}
		\item $1 \in \bP$
		\item $\bP$ is stable under finite sums
		\item The topology $\bT := \{ X \in \bP \ | \ L \| X \| \}$ induces the lex modality $L$.
		\item $\bP$ is a $\bT$-local property of affines.

	\end{itemize}
	
	%	\begin{itemize}
	%		\item Either $X$ is contractible
	%		\item Or Whenever we find $\Spec B \in \bT$ such that $\Spec B \to L \|X\|$
	%	\end{itemize}
	%	generates $L$.
\end{definition}

Things in $\bP$ we call 'flat'.
\begin{example}
	For the Zariski topology, the flat affines are the finite sums of open affines.
\end{example}
\begin{example}
	For the fppf topology, flat is flat
\end{example}
\begin{example}
	For the etale topology, $\bP$ = formally \etale + flats. Indeed if $X \in \bP$ is \etale-merely inhabited, then its $\lnot\lnot$-inhabited. As $X$ is in particular flat, its fppf \todocite. We conclude that $X$ is \etale + fppf.
\end{example}
%\begin{lemma}{\label{lemma:coveringDMstacks}}
%	Any formally \etale, \etale-merely inhabited Deligne-Mumford-stack is covering.
%\end{lemma}
%\begin{proof}
%	Todo
%\end{proof}
\begin{lemma}
	If $\Spec A \in \bP$, then $\Spec A$ is weakly-flat, i.e $\| \Spec A\|_\bT$ is a geometric prop.
\end{lemma}
\begin{lemma}{\label{lemma:detectCovering}}
	A geometric stack is covering iff its flat and $\bT$-merely inhabited.
\end{lemma}

\begin{lemma}
	If $X$ is a covering stack and $Y$ a flat geometric stack, then $X + Y$ is covering
\end{lemma}
\begin{proof}
	Let $\bT \ni \tilde X \to X, \tilde Y \to Y$ be geometric atlasses. Then $\tilde X+ \tilde Y$ is flat and $\bT$-merely inhabited, hence in the topology.
\end{proof}





\section{Covering Stacks are formally \etale}
\subsection{The \etale topology is saturated}
Let $P$ denote a closed dense proposition.
\begin{lemma}{\label{lemma:covOfEF}}
	An \etale-flat DM-stack that is $\lnot \lnot$-inhabited is covering.
\end{lemma}
\begin{proof}
	If $X$ is an \etale-flat geometric stack, we may choose a geometric atlas $W \to X$ with $W$ formally \etale + flat. Using that the fibers $W \to X$ are $\lnot \lnot$-inhabited, we have 
	\begin{align*}
		\lnot \lnot X &\to \lnot \lnot W \\
		&\to W \in \bT  ´\\
		&\to X \in \CS \\
	\end{align*}
\end{proof}
\begin{lemma}{\label{lemma:wfcover}}
	For $X \in EF$, $X \to X^P$ is a map fibered in weakly-flat stacks iff for any $x, y : X$, $(x = y)^P$ is \etale-flat.
\end{lemma}
\begin{proof}
	\begin{itemize}
		
		\item[$\leftarrow$]
		By descent for covering stacks we may only show this for the fiber over $\Delta x $ for some $x  : X$ (Indeed let $z : X^P$. Assume $\|\sum_x \Delta x = z \|_\bT$. By descent we may replace $z$ by $\Delta x$ for some $x : X$ ) 
		But then the fiber is $\sum_y (x = y)^P$, a $\sum$ of \etale-flat geometric stacks which is merely inhabited, hence covering.
		\item[$\rightarrow$]The fiber-inclusion over $\Delta x$ is $(\sum_{y : X} (x = y)^P) \to X$ as calculated above. By stability under finite limits of $\EF$ the fiber $(x = y)^P$ over $y$ is an \etale-flat geometric stack.
	\end{itemize}
\end{proof}
\begin{lemma}{\label{lemma:ETALE}}
	Let $X$ be \etale-flat geometric stack and $P$ a closed dense proposition. Then TFAE
	\begin{enumerate}
		\item $X^P \in \EF$.
		\item $X \to X^P$ is an $\EF$-cover, i.e. a map fibered in \etale-flat stacks. % for any cld $P$
		\item $X \to X^P$ is a geometric cover % for any cld $P$
		\item $\Delta(X) : X \to X^P$ is an equivalence. 
		\item For any \etale-flat geometric stack $W$ such that $\Delta(W)$ is an equivalence and for any geometric cover $W \to X$ , $W^P \to X^P$ is fibered in $\EF$-stacks.
		\item The same as 5 but 'exists $W$' instead of 'for all'.
		%		\item For any $F : X \to \CS$ such that $\Delta(\sum_{X} F)$ is an equivalence and any $x : X^P$, $\prod_p F(x_p)$ is $\bT$-merely inhabited.
	\end{enumerate}
\end{lemma}
\begin{proof}
	\	\begin{enumerate}
		\item [1 $\Rightarrow$ 2] $\EF$ is stable under finite limits
		\item [2 $\Rightarrow$ 3] \ref{lemma:covOfEF}
		\item [4 $\Rightarrow$ 3 $\Rightarrow$ 1] obvious
		\item [1 $\Rightarrow$ 4]  we do induction over the truncation level. Contractible types are okay. Now let $X$ be an \etale-flat geometric stack. It suffices to show that $X \to X^P$ is an embedding by assumption (3) and \ref{lemma:covM1Stacks}. As $X \to X^P$ is in particular a map fibered in weakly flat stacks, for any $x,y : X$, $(x=y)^P \in \EF$ by \ref{lemma:wfcover}. $x = y$ is an \etale-flat geometric stack of one truncation level lower, we may apply the induction hypothesis thus $x = y \to (x = y)^P = (\Delta X =_{X^P} \Delta Y)$ is an equivalence. This map is $\ap_\Delta$.
		\item [ 6  $\Rightarrow$ 2 $\Rightarrow$ 5 ] Let $W \to X$ be a geometric cover with $\Delta W$ beeing an equivalence. Consider the commutative diagram
		% https://q.uiver.app/#q=WzAsNCxbMCwwLCJXIl0sWzEsMCwiV15QIl0sWzEsMSwiWF5QIl0sWzAsMSwiWCJdLFszLDJdLFswLDEsIlxcc2ltIl0sWzAsM10sWzEsMl1d
		\[\begin{tikzcd}
			W & {W^P} \\
			X & {X^P}
			\arrow["\sim", from=1-1, to=1-2]
			\arrow[from=1-1, to=2-1]
			\arrow[from=1-2, to=2-2]
			\arrow[from=2-1, to=2-2]
		\end{tikzcd}\]
		As beeing fibered in \etale-flat geometric stacks is a local property of morphisms of geometric stacks, $W^P \to X^P$ is fibered in \etale-flat stacks iff $X \to X^P$ is an $\EF$-cover, which is condition 2.
		\item [$5 \Rightarrow 6$] just use the definition of $\EF$: $X$ admits an atlas with $\EF$-domain.
		
	\end{enumerate}
\end{proof}
\begin{corollary}
	Let $X : \EF$ . Then $\Delta X : X \to X^P$ is fibered in weakly-flat stacks, iff its its an embedding.
\end{corollary}
%--------------
\begin{lemma}{\label{lemma:EFPropFet}}
	\etale-flat geometric propositions are formally \etale.
\end{lemma}
\begin{proof}
	Let $X$ be such a proposition.
	We may show, that $X$ is $\lnot \lnot$ stable, as $\lnot \lnot$-stable propositions are formally \etale. If we assume $\lnot \lnot X$, then $X$ is covering by \ref{lemma:covOfEF}, thus by \ref{lemma:covM1Stacks} its contractible.
\end{proof}
\begin{corollary}[of \ref{lemma:EFPropFet}.]
	Covering sheaves are formally unramified.
\end{corollary}
%---------
\begin{lemma}{\label{lemma:OpenIsSmooth}}
	The type of open propositons $\mathsf{OPEN}$ is smooth. 
\end{lemma}
\begin{proof}
	You can write 
	\[\mathsf{OPEN} = \bigcup_n \mathsf{OPEN}_n\]
	where $ \mathsf{OPEN}_n$ consists of open propositions which are merely of the form $\mathrm{isInv} (f_1) \lor \hdots \lor \mathrm{isInv} (f_n)$
	for some $f_1,\hdots,f_n : R$. \\
	By \ref{lemma:SeqUnionSmooth} its enough to see that $\mathsf{OPEN}_n$ is formally smoots. But its equipped with a surjection out of $R^n$, which is formally smooth.
\end{proof}



%\begin{proof}
%%	Let $x , y : X \in \CS \cap \Set$. 
%\end{proof}


\begin{lemma}{\label{lemma:schemeEFIsFEt}}
	Let $X$ be a scheme that is \etale-flat as a GS. Then $X$ is formally \etale.
\end{lemma}
\begin{proof}
	By \ref{lemma:ETALE} we may show, that $X \to X^P$ is an \etale-flat cover. 	As $X$ is unramified by \ref{lemma:EFPropFet}, $x_p = y$ is an open proposition depending on $p : P$. But as the type of open propositions is smooth we find an open proposition $Q$ such that $\forall p, (x_p = y) = Q$. Then, using that $Q$ is formally \etale, % open proposition,
	\[(\prod_{p} x_p = y) = (P \to Q) = Q\] %= \lnot P \lor Q 
	But $Q$ is an open proposition, hence a formally \etale + flat scheme, thus an \etale-flat geometric stack.		
	
\end{proof}

\begin{corollary}
	The \etale topology is saturated.
\end{corollary}
\begin{proof}
	Every affine covering stack is a scheme that is \etale-flat as a GS, by \ref{lemma:schemeEFIsFEt} its formally \etale. Its also fppf by saturatedness of fppf.
\end{proof}
\subsection{Formally \etale subuniverses}
\begin{definition}{\label{lemma:FisFet}}
	A formally \etale subuniverse is a subtype $\bF \subset FET$ such that one of the following equivalent conditions is satisfied
	\begin{enumerate}
		\item  $\bF$ is formally \etale.
		\item $\bF$ is formally smooth
		\item For any $X : FET$, $X \in \bF$ is smooth.
	\end{enumerate}
\end{definition}
\begin{proof}
	Use that 	$FET$ is formally unramified.
	\begin{enumerate}
		\item [$2 \Leftrightarrow 3$] Study the fibers of $\bF \to FET$ and use that beeing formaly \etale is stable under finite limits.
		\item [$1 \Leftrightarrow 2$]  The map $\bF \to FET$ is an inclusion, thus $\bF$ is formally unramified.
	\end{enumerate}
	
\end{proof}


\begin{lemma}{\label{cor:SubUnivDepProdStable}}
	Let $S : P \to \bF$, Then $\prod_p S_p \in \mathsf{\bF}$.
	%	In particular: Let $S : \Aff$, such that  for any closed dense $P \to S \in \mathsf{StdEt}$. Then for any such $P$, $S^P \in \mathsf{StdEt}$.
\end{lemma}

\begin{proof}
	note that there exists a unique filler $\tilde Y : 1 \to \bF$, as $\bF$ is formally \etale. On the other hand the filler $1 \to FET$ is given by $\prod_p Y_p$. But $\bF \hookrightarrow FET$ is an embedding.
	% https://q.uiver.app/#q=WzAsNCxbMCwwLCJQIl0sWzEsMCwiXFxiRiJdLFsyLDAsIkZFVCJdLFswLDEsIjEiXSxbMCwxLCJZIl0sWzEsMiwiIiwwLHsic3R5bGUiOnsidGFpbCI6eyJuYW1lIjoiaG9vayIsInNpZGUiOiJ0b3AifX19XSxbMCwzXSxbMywxLCJcXGV4aXN0cyEiLDEseyJzdHlsZSI6eyJib2R5Ijp7Im5hbWUiOiJkYXNoZWQifX19XSxbMywyLCJcXHByb2RfcCBZX3AiLDJdXQ==
	\[\begin{tikzcd}
		P & \bF & FET \\
		1
		\arrow["Y", from=1-1, to=1-2]
		\arrow[from=1-1, to=2-1]
		\arrow[hook, from=1-2, to=1-3]
		\arrow["{\exists!}"{description}, dashed, from=2-1, to=1-2]
		\arrow["{\prod_p Y_p}"', from=2-1, to=1-3]
	\end{tikzcd}\]\\
	
	
	
\end{proof}
\begin{prop}
	For any modality $\ci$, there is a formally \etale subuniverse cut out by the $\ci$-modal types $FET \cap \cU_{\ci}$
\end{prop}
\begin{proof}
	We only need to show that $FET \cap \cU_{\ci}$ is formally smooth. Here you use that $\ci$-modal types are stable by depedent products over arbitrary indexing types.
\end{proof}
\begin{example}{\label{lemma:FETSt}}
	The following form formally \etale subuniverses:
	\begin{itemize}
		\item 	The class $FET \cap \St$ of formally \etale \etale-stacks
		\item 	The class of \etale propositions, i.e. propositions that are formally \etale \etale-sheaves.
	\end{itemize}	
\end{example}

\subsection{Standart \etale}
If $A$ is an fp $R$-algebra, $\Alg_A$ denotes fp $A$- algebras.
\begin{definition}
	Let $A$ be a f.p. $R$ algebra. The type of standart \etale $A$-algebra $\mathsf{StdEtAlg}_A$ is the type of f.p. flat $A$-algebras which are merely of the form
	\[
	\left (A[X_1,\hdots,X_n] / (P_1,\hdots,P_m) \right)_G
	\]
	such that $\det (\mathrm{Jac}(P_1,\hdots,P_n))$ divides $G$ in $A[X_1,\hdots,X_n] / (P_1,\hdots,P_m) $. \\
	We define $\mathsf{StdEt}_R = \sum_{X : \Aff} X^R \in \mathsf{StdEtAlg}_R$.
\end{definition}
\begin{question}
	Is standart \etale stable under finite sums?
\end{question}

\begin{definition}
	Let $A : \Alg_R$.
	 The type of Presentations of f.p. algebras over $R$ is 
	\[
	\Pres_A = \sum_{n,m} A[X_1,\hdots,X_n]^m
	\]
	We have a presentation forgetting map 
	\begin{align*}
		\mathrm{fgt} :\Pres_A &\to \Alg_A \\
		(n,m, P_1,\hdots,P_m) &\mapsto A[X_1,\hdots,X_n] / (P_1,\hdots,P_m)
	\end{align*}

	 %, where $\fp_*$ denotes the chance of coefficients $\fp \otimes_R R[X_1,\hdots,X_n] : A[X_1,\hdots,X_n] \to R[X_1,\hdots,X_n]$. \\
	
\end{definition}
\begin{construction}
	For any map of $R$-algebras $A \to B$ there is an evident pushforward map on type of presentations, which we call by abuse of notation the same as on algebras:
	% https://q.uiver.app/#q=WzAsNCxbMSwwLCJcXFByZXNfQiJdLFswLDAsIlxcUHJlc19BIl0sWzAsMSwiXFxBbGdfQSJdLFsxLDEsIlxcQWxnX0IiXSxbMCwzXSxbMSwyXSxbMiwzLCJcXF8gXFxvdGltZXNfQSBCIl0sWzEsMCwiIiwwLHsic3R5bGUiOnsiYm9keSI6eyJuYW1lIjoiZGFzaGVkIn19fV1d
	\[\begin{tikzcd}
		{\Pres_A} & {\Pres_B} \\
		{\Alg_A} & {\Alg_B}
		\arrow["\_ \otimes_A B",dashed, from=1-1, to=1-2]
		\arrow[from=1-1, to=2-1]
		\arrow[from=1-2, to=2-2]
		\arrow["{\_ \otimes_A B}", from=2-1, to=2-2]
	\end{tikzcd}\]
	making the diagram commute. \\
	It is given 
	\begin{align*}
		{\Pres_A} \simeq \sum_{n,m} A[X_1,\hdots,X_n]^m &\to \sum_{n,m} A[X_1,\hdots,X_n]^m \otimes_A B \simeq {\Pres_B} \\
		(n,m,P) &\mapsto (n,m,P\otimes 1)
	\end{align*}	
\end{construction}
\begin{lemma}
	The type $\Pres_R$ is smooth.
\end{lemma}

\begin{proof}
	As $\bN$ is formally smooth, by $\sum$-stability of formally smooth types. it remains to show that  For any $n$, The type $R[X_1,\hdots,X_n]$ is formally smooth. However we can write it as a sequential union
	\[
	R[X_1,\hdots,X_n] = \bigcup_k R[X_1,\hdots,X_n]_{\le k}
	\]
	where $ R[X_1,\hdots,X_n]_{\le k}$ is the finite free $R$-submodule generated by monomials with degree $\le k$. In particular it is a smooth type. Conclude by \ref{lemma:SeqUnionSmooth}.
\end{proof}
\begin{lemma}
	Let $R \twoheadrightarrow A$ be a quotient algebra. duality for fp algebras restricts to a bijection
	\[
	\mathsf{StdEtAlg}_A \cong (\Spec A \to \mathsf{StdEtAlg}_R)
	\]
\end{lemma}

\begin{proof}
	
	Duality enhances to a bijection for pointed algebras
	\[
	\Alg_{A,*} \to (\Spec A \to \Alg_{R,*})
	\]
	Moreover, flatness is preserved under duality by \ref{thm:FlatAffines}.
	Def: A pointed $A$-algebra $(B,G)$ admits an apropriate presentation, if there exists a presentation $B = R[X_1,\hdots,X_n] / (P_1,\hdots,P_n)$ such that $\det(Jac(P_1,\hdots,P_n))$ divides $G$ in $B$. We need to show, that a pointed $A$-algebra admits an approriate presentation iff it admits that pointwise. \\
	\begin{itemize}
		\item 
	If $B$ admits an appropriate presentation $F$ %$
	, then for any $\fp: \Spec A$, $F \otimes_A \fp$ %R[X_1,\hdots, X_n] / (\fp_* P_1,\hdots,\fp_* P_n)$ 
	is an appropriate presentation of $B \otimes_A \fp$. \\
	%, where $\fp_*$ denotes the chance of coefficients $\fp \otimes_R R[X_1,\hdots,X_n] : A[X_1,\hdots,X_n] \to R[X_1,\hdots,X_n]$. \\
	\item Let $(B , G) : Alg_{A,*}$, such that for $\fp : \Spec A$ we merely find an appropriate presentation of $B \otimes_A \fp$ as an $R$-algebra. In particular we have a solid arrow commutative diagram
% https://q.uiver.app/#q=WzAsNixbMSwwLCJcXFByZXNfUiJdLFsyLDAsIlxcQWxnX1IiXSxbMCwwLCJQIl0sWzAsMSwiMSJdLFsyLDEsIlxcQWxnX0EiXSxbMSwxLCJcXFByZXMgQSJdLFswLDFdLFsyLDBdLFsyLDNdLFsxLDQsIlxcXyBcXG90aW1lc19SIEEiXSxbMywwLCIoMSkiLDAseyJzdHlsZSI6eyJib2R5Ijp7Im5hbWUiOiJkYXNoZWQifX19XSxbMyw0LCJCIiwyLHsiY3VydmUiOjR9XSxbNSw0XSxbMCw1LCJcXF9cXG90aW1lc19SIEEiXSxbMyw1LCIoMikiLDIseyJsYWJlbF9wb3NpdGlvbiI6NDAsInN0eWxlIjp7ImJvZHkiOnsibmFtZSI6ImRhc2hlZCJ9fX1dXQ==
\[\begin{tikzcd}
	P & {\Pres_R} & {\Alg_R} \\
	1 & {\Pres A} & {\Alg_A}
	\arrow[from=1-1, to=1-2]
	\arrow[from=1-1, to=2-1]
	\arrow[from=1-2, to=1-3]
	\arrow["{\_\otimes_R A}", from=1-2, to=2-2]
	\arrow["{\_ \otimes_R A}", from=1-3, to=2-3]
	\arrow["{(1)}", dashed, from=2-1, to=1-2]
	\arrow["{(2)}"'{pos=0.4}, dashed, from=2-1, to=2-2]
	\arrow["B"', curve={height=24pt}, from=2-1, to=2-3]
	\arrow[from=2-2, to=2-3]
\end{tikzcd}\]

	 % we are given a map $P \to \Pres_R$  if $P$ then we merely find an appropriate presentation of $B$ as an $R$-algebra. 
	 By smoothness of $\Pres_R$ , we find (1) an actual presentation $F : \Pres_R$, which presents $B \fp$ whenever $\fp : P$. 
	 The other half of the diagram still commutes 	using \todocite
	 	\[B \equiv \prod_{\fp : P} B_\fp = (\Spec A \to \mathrm{fgt} F) = \mathrm{fgt} F \otimes_R A \]
	 But then we can just use (2) the presentation $F \otimes_R A : \Pres_{A}$ of $B$ as an $A$ algebra.

	
%	= \mathrm{fgt} (F \otimes_R A)

	It remains to show, that $F \otimes_R A$ is an appropriate presentation of the $A$-algebra $(B,G)$. However this is encoded by divisibility in $B$ which can be checked pointwise \ref{ex:Divisibility}. % It holds pointwise because $(B_\bullet ; G_\bullet)$ was assumed to be pointwise appropriate.
		\end{itemize}
	
	%	First observe that every $ B : \mathsf{StdEt}_A$  admits some $
	%	A StdEt presentation is a finitely presented $R[X]$ algebra
	%	Write $X = \Spec A$.
	%	Let $B : X \to \mathsf{StdEt}_R$. By ZLC for $X$ we find a Zariski cover $X' \to X$ with
	%	\[
	%	\prod_{x: X'}  
	%	\]
\end{proof}



\begin{lemma}{\label{lemma:StdEtSm}}
	%Let $B : Alg_R$ be smooth. 
	Then $\mathsf{StdEt}_R$ is a formally \etale subuniverse. %Let $A$ be a finitely presented $R$-algebra. Then '$A$ beeing standart \etale $R$-algebra' is formally smooth. \\
	%	A Zariski cover of standart \etale is standart \etale.
\end{lemma}
\begin{proof}
	\begin{itemize}
		\item standart \etale type is in particular formally \etale \todocite
		\item formally smoothness:
		Let $I^2 = 0$.
		%Choose a presentation $A = R[X_1,\hdots,X_n] / (\tilde P_1,\hdots,\tilde P_m)$. Let $P_1,\hdots P_m$ denote the images in $R/I[X_1,\hdots,X_n]$.
		Let $\Spec  R/ I \to \mathsf{StdEt}_R \simeq \mathsf{StdEtAlg}_{R}$. This corresponds to a unique $\mathsf{StdEtAlg}_{R/I}$, where $B' = R / I \otimes_R B$. choose a presentation
		\[
		T = \left (R/I[X_1,\hdots,X_n] / (P_1,\hdots,P_m) \right)_G
		\]
		and some $H : R/I[X_1,\hdots,X_n] / (P_1,\hdots,P_m)$ such that  
		\[\det (\mathrm{Jac}(P_1,\hdots,P_n)) \cdot H = G.\] \\
		Then choose lifts $\tilde P_1, \hdots, \tilde P_m \in R[X_1,\hdots,X_n]$ of the $P_i$, then a lift 
		\[\tilde H : R[X_1,\hdots,X_n] / (\tilde P_1 , \hdots, \tilde P_n)\]
		of $H$. Then define 
		\[
		\tilde G :\equiv \det (\mathrm{Jac}(\tilde P_1,\hdots, \tilde P_n)) \cdot \tilde H
		\]
		Then 
		\[\hat T \equiv \left(R[X_1,\hdots,X_n]  / (\tilde P_1,\hdots,\tilde P_n) \right)_{\tilde G} \]
		is a standart \etale $R$-algebra such that $\hat T \otimes_R R/ I = T$.
		%	It remains to construct a $R$-algebra homomorphism
		%	% https://q.uiver.app/#q=WzAsNCxbMCwwLCJCIl0sWzEsMCwiXFx0aWxkZSBUIl0sWzAsMSwiQiciXSxbMSwxLCJUIl0sWzIsM10sWzEsMywiIiwyLHsic3R5bGUiOnsiaGVhZCI6eyJuYW1lIjoiZXBpIn19fV0sWzAsMl0sWzAsMSwiIiwyLHsic3R5bGUiOnsiYm9keSI6eyJuYW1lIjoiZGFzaGVkIn19fV1d
		%	\[\begin{tikzcd}
		%		B & {\tilde T} \\
		%		{B'} & T
		%		\arrow[dashed, from=1-1, to=1-2]
		%		\arrow[from=1-1, to=2-1]
		%		\arrow[two heads, from=1-2, to=2-2]
		%		\arrow[from=2-1, to=2-2]
		%	\end{tikzcd}\]
		%	For this you need the smoothness of $B$. Indeed from the map $\Spec \tilde T \to \Spec B^{\bD 1}$ we merely obtain map $\Spec \tilde T \to \Spec B$.
		% if $I = 0$ it coincides with the $R / I$-algebra we started with.
	\end{itemize}
\end{proof}
\begin{question}
	Let $P \to \mathsf{StdEt}_R$ correspond to some $B : \mathsf{StdEt}_{R / I}$. What is $\Spec {}_R B$ where we restricted scalars ?
\end{question}

\begin{lemma}{\label{lemma:EqRelSmooth}}
	Let $\bF$ be formally \etale subuniverse. 
	The Type of $\bF$-fibered  $\mathsf{EtProp}$-valued equivalence relation on a Std\etale 
	\[\sum_{X: \mathsf{StdEt}} (R : X^2 \to \mathsf{EtProp}) \times (\prod_{x : X} R_x \in \bF) \]
	is formally \etale.
\end{lemma}
\begin{proof}
	It is formally unramified, as  $\mathsf{StdEt}$ is formally \etale and our equivalence relation is valued in formally \etale propositions. \\
	So it remains formally smoothness. 
	Let $Y : \mathsf{StdEt}^P$, $R : \prod_p Y_p^2 \to \mathsf{EtProp}$ such that for all $p$,  $R_p$ is an $\mathsf{EtProp}$-valued equivalence relation fibered in $\bF$. \\
	As $\mathsf{StdEt}$ is formally \etale, we find a unique $\tilde Y : \mathsf{StdEt}$ such that for all $p$, $\tilde Y = Y_p$. Then our relation transport to a term $\tilde R$ in the LHS
	\[
	\prod_p Y_p^2 \to \mathsf{EtProp} = (\prod_p \tilde Y^2 \to \mathsf{EtProp}) = (\tilde Y^2 \to \mathsf{EtProp}^P) = (\tilde Y^2 \to \mathsf{EtProp})
	\]
	Using that the type of etale propositions is formally \etale \ref{lemma:FETSt}. \\
	One  can check, that $\tilde R$ is indeed an equivalence relation on $\tilde Y$. It remains to show, that the relation $\tilde R$ is fibered in $\bF$. for any fiber this holds $P$-merely, which is enough by assumption \ref{lemma:FisFet} on $\bF$, as the fibers are formally \etale .
	More precisely: For this we calculate $\tilde Y = \prod_p Y_p$ and $\tilde R(y,y') = \prod_p R(y_p,y'_p)$. Let $y : \tilde Y$. 
	The fiber is
	\[
	\sum_{y' : \tilde Y} \tilde R(y , y') = \sum_{y': \tilde Y} \prod_p R(y_p,y'_p) = \prod_p \sum_{y' : Y_p} R(y_p , y')
	\]
	which is a dependent product of fibers of $R$. So conclude by \ref{cor:SubUnivDepProdStable}.
	%, the fiber is $P$-merely in $\bF$. By assumption the fiber is in $\bF$.
\end{proof}
\subsection{\etale-flat stacks form a formally \etale subuniverse}

%\begin{lemma}{\label{lemma:FETUniv}}
%	Let $\bF$ be a formally \etale subuniverse contained in $\EF$. An $\EF$ stack is formally \etale if it merely admits a geometric $\bF$-cover with formally \etale domain.
%\end{lemma}
%\begin{proof}
%
%\end{proof}


\begin{warning}
	$\EF \hookrightarrow \GS$ is probably not formally \etale.
\end{warning}
\begin{lemma}{\label{lemma:FetFromFCover}}
	An \etale-flat geometric stack is formally \etale , if it admits a geometric $\bF$-cover with formally \etale domain for $\bF$ some formally \etale subuniverse contained in $\EF$.
\end{lemma}
\begin{proof}
	Choose $f: W \to X$ a geometric $\bF$-cover with $W$ a formally \etale geometric stack. By \ref{lemma:ETALE}, we may show that $W \to X^P$ is an $\bF$-cover (thus an $\EF$-cover). The fiber over any $x :X^P$ is $\prod_p \fib_{f_p} x_p$ , a dependent product of things in $\bF$, thus in $\bF$ by \ref{cor:SubUnivDepProdStable}. 
\end{proof}
\begin{theorem}
	The type of $\EF$-stacks is a formally \etale subuniverse.
\end{theorem}
\begin{proof}
	By truncatedness of $\EF$-stacks and \ref{lemma:SeqUnionSmooth}, we may just show, that $\EF$-$n$-stacks form a formally \etale subuniverse for each $n$. 
	\begin{itemize}
		\item $n=0$: \\ 
		%For $\bF$ be a formally \etale subuniverse, we call a sheaf a $\bF$-DM-sheaf if it is merely the quotient of some std\etale by an $\mathsf{EtProp}$-valued equivalence relation fibered in $\bF \cap \CS$.
		
		An algebraic space is called 
		\begin{itemize}
			\item level 0, if its Zariski-flat, i.e. a finite sum of open propositions: \[
			\cP_{\Zar} \equiv \bigcup_n \mathsf{Open}^n % \{ X : \cU \ | \ \| \sum_{m : \bN} (Q : \mathsf{Fin} \ m \to \mathsf{Open}) \times X \simeq \sum_i Q i \| \}
			\]
			where we use the structure maps $X : \mathsf{Open}^n \mapsto (\bot , X) : \mathsf{Open}^{n+1}$
			\item level $n+1$, if it is merely the quotient of some std\etale by an $\mathsf{EtProp}$-valued equivalence relation fibered in level $n$ covering algebraic spaces.
		\end{itemize}
		Observe that, the respective levels contain in particular
		\begin{enumerate}
			\item level algebraic spaces contain the \etale-topology.
			\item level contains sheaves that merely admit an atlas $\mathsf{StdEt} \ni S' \to F$ fibered in $\bT$ \\
			\item level algebraic spaces contain already all $\EF$-algebraic spaces by \todocite
		\end{enumerate}
		We want to prove by induction, that level $k$ algebraic spaces form a formally \etale subuniverse.
		\begin{itemize}
			\item  Zariski-Flat types form a formally \etale subuniverse. 
			\begin{itemize}
				\item Every finite sum of opens is formally \etale
				\item The type $\cP_{\Zar}$ is formally smooth: by \ref{lemma:SeqUnionSmooth} and the observation that each $\mathsf{Open}^m$ is smooth.
			\end{itemize}
			\item For the induction step $n \mapsto n+1$ we may just show the following:
			%For any formally \etale subuniverse $\bF$, the type of $\bF$-DM-sheaves is a formally \etale subuniverse.
			\begin{itemize}
				\item 	First we show that the type of level $n+1$ algebraic spaces is formally smooth: We can $P$-merely choose an appropriate atlas. By \ref{lemma:EqRelSmooth} we get an actual stdetale with an equivalence relation. The quotient will give us a filler. \\
				\item Every level $n+1$ algebraic space admits a geometric cover fibered in level $n$ algebraic spaces with formally \etale domain, thus it is formally \etale by \ref{lemma:FetFromFCover}.
			\end{itemize}
		\end{itemize}
		\item By definition we need to show:
		\begin{itemize}
			\item Any \etale-flat geometric $n+1$-stack is formally \etale by \ref{lemma:FetFromFCover} using that \etale-flat geometric $n$-stacks are a formally \etale subuniverse by the induction hypothesis.
			\item To show smoothness of $\EF_{n+1}$, we may show \todocite that the domain of the surjection
			% https://q.uiver.app/#q=WzAsMyxbMCwwLCJcXHN1bV97WCA6IFxcU3QgXFxjYXAgRkVUfShYIFxcb3ZlcnNldHtGfXtcXHRvfSBcXENTX3tufSkgXFx0aW1lcyAoXFxzdW1fe1h9IEYgXFxpbiBcXEVGX24pIl0sWzEsMCwiXFxzdW1fe1ggOiBcXFN0IFxcY2FwIEZFVH0gWCBcXGluIFxcRUZfe24rMX0iXSxbMiwwLCJcXEVGX3tuKzF9Il0sWzAsMSwiIiwwLHsic3R5bGUiOnsiaGVhZCI6eyJuYW1lIjoiZXBpIn19fV0sWzEsMiwiIiwwLHsibGV2ZWwiOjIsInN0eWxlIjp7ImhlYWQiOnsibmFtZSI6Im5vbmUifX19XV0=
			\[\begin{tikzcd}
				{\sum_{X : \St \cap FET}(X \overset{F}{\to} \CS_{n}) \times (\sum_{X} F \in \EF_n)} & {\sum_{X : \St \cap FET} X \in \EF_{n+1}} & {\EF_{n+1}}
				\arrow[two heads, from=1-1, to=1-2]
				\arrow[Rightarrow, no head, from=1-2, to=1-3]
			\end{tikzcd}\]
			is formally \etale, where the right equality uses the previous paragraph. \\
			As beeing formaly \etale is a modality, we may only show that the following types are formally \etale
			\begin{itemize}
				\item $\St \cap FET$ by \ref{lemma:FETSt}
				\item $\CS_n$. It embeds into the formally \etale $\EF_n$, so it remains to show that the fibers are smooth. But for some $X : \EF_n$ beeing $\lnot \lnot$ inhabited is formally smooth, so conclude by \ref{lemma:covOfEF}.
				\item $\sum_X F \in \EF_n$. Here just use that the map $\EF_n \hookrightarrow \St \cap FET \ni \sum_X F$ between formally \etale types has formally \etale fibers.
			\end{itemize} 
		\end{itemize}
	\end{itemize}
\end{proof}
%\begin{proof}
%	Let $X : \GS$. Choose a geometric atlas $S \to X$. We have $S \in \EF$ iff $X \in \EF$. The first statement is formally \etale.
%\end{proof}

\section{Tangent Spaces}
\begin{definition}
	A pointed type $(D,0)$ is tiny if
	\begin{itemize}	
		\item it has choice
		\item for any $W : D \to \Aff$, $\prod_d W d$ is affine
		\item $D$ is flat affine % admits a strong boundedness principle (e.g. because its affine)
	\end{itemize}
\end{definition}
\begin{rmk}
	Closed dense propositions are probably not tiny: Put $W	d = \bA^1$, then if $\prod_{d: D} W d = R / \varepsilon$ is affine, it would be an affine finitely copresented module, hence maybe free?
\end{rmk}
Fix a topology $\bT$ which is stable under tiny exponentials, i.e. such that for any $D$ tiny, and any $W : D \to \bT$, the affine $\prod_{d:D} W_d$ belongs to $\bT$.
\begin{lemma}[TODO]{\label{lemma:TinyExp}}
	The following topologies are stable under tiny exponentials.
	\begin{itemize}

		\item The \etale topology
		\item The smooth topology				
	\end{itemize}
	
\end{lemma}
\begin{proof}
	\begin{itemize}
		\item TODO
		\item TODO
	\end{itemize}
\end{proof}
\begin{warning}
	The fppf topology is not stable under tiny exponentials! \ref{warning:TangentSpacesNotFlat}
\end{warning}

\begin{theorem}
	Covering / Geometric stacks are stable under exponentials over tiny types.
\end{theorem}
\begin{proof}
	Let $P : D \to \GS$. 
%	$\prod_d P d$ has a geometric atlas. 
	We first prove the covering case by the $W$ induction principle of covering stacks: By choice of $D$ We may assume that $P : D \to W_n$. If $n = 0$ its fine by assumption on $\bT$.
	by choice of $D$ we can choose $W_{n-1}$ atlasses $p d : X d \to P d$ for $d : D$.
	Claim :  $\prod_{d: D} X(d) \to \prod_{d : D} P d$ is a geometric atlas.
	Proof : Indeed the fiber over $f$ is $\prod_{d : D} \fib_{p d} (f d)$ which is a dependent product over $W_{n-1}$ types, hence covering by induction. \\
	Hence $\prod_d P d$ is geometric.
	If, additionally, all the $P d$ are covering then we may choose the $X d$ to be covering affine. By assumption on $\bT$, $\prod_{d: D} X d$ belongs to $\bT$, hence $\prod_{d: D} P d$ is a covering stack.
\end{proof}
\begin{corollary}{\label{cor:GeomAtlExpStable}}
		For any $D$ tiny and $W_d \to X_d$ a family of geometric atlasses the map $\prod_{d : D} W_d \to \prod_{d: D} X_d$ is a geometric atlas.
\end{corollary}
\begin{corollary}
	Geometric stacks are stable under taking tangent spaces.
\end{corollary}
\begin{lemma}{\label{lemma:TinyStrongLocalChoice}}
 	For any $B : D \to \cU$ a type family we have
	\[
	\forall d:D \|B d\|_\bT \to \| \prod_d B d\|_\bT
	\]
	if $D$ has choice  such that $\bT$ is $\prod$-stable over $D$.  
%	 satisfies one of the following conditions:
%	\begin{enumerate}
%		\item Let $D$ has choice such that $\bT$ is $\prod$-stable over $D$. 
%		\item $D$ is a proposition
%		
%\end{enumerate}		
\end{lemma}
\begin{proof}
	\begin{enumerate}
		\item 
	By choice of $D$ we find $S_d \in \bT$ and $S_d \to \|B d\|$. by $\prod$-stability over $D$ on $\bT$ we have $\prod_d S_d \in \bT$,s  in particular $\| \prod_d S_d \|_\bT$. Hence we $\bT$-merely have $\prod_d \| B d\|.$ By choice of $D$ we $\bT$-merely get $\prod_d B d$. 
	\item 	If $D$ is a proposition just observe, that 
	\[\| \prod_d B d \|_\bT= \| D \to \sum_d B d \|_\bT = D \to \|\sum_d B d\|_\bT.\]
		\end{enumerate}
\end{proof}
\begin{lemma}
	Let $D$ be a finite wedge of infinitesimal varieties. Consider a family of smooth maps $f_d : W_d \to X_d$ and an element $w : W_0$.
	Consider $g : \prod_d X_d$ such that $p : g_0 = f_0 w$. Then we merely find some $h : \prod_{d: D} W_d$ such that $h_0 = w$ with $q_d : g_d = f_d (h_d) $.
%	and a commutative diagram
%	% https://q.uiver.app/#q=WzAsNCxbMCwwLCIxIl0sWzAsMSwiRCJdLFsxLDAsIlciXSxbMSwxLCJYIl0sWzAsMl0sWzEsM10sWzIsM10sWzAsMV0sWzEsMiwiIiwxLHsic3R5bGUiOnsiYm9keSI6eyJuYW1lIjoiZGFzaGVkIn19fV1d
%	\[\begin{tikzcd}
%		1 & W \\
%		D & X
%		\arrow["w",from=1-1, to=1-2]
%		\arrow["0",from=1-1, to=2-1]
%		\arrow[from=1-2, to=2-2]
%		\arrow[dashed, from=2-1, to=1-2]
%		\arrow["g",from=2-1, to=2-2]
%	\end{tikzcd}\]
%%	
	%we merely find a dashed lift.  %$\bT$-merely
	
\end{lemma}
\begin{proof}
	Let us first treat the special case where $D$ is tiny.
%	Using choice of $
	For any $d : D$ by smoothness of $W_d \to X_d$ we merely have a lift 
	% https://q.uiver.app/#q=WzAsMyxbMCwwLCJkPTAiXSxbMCwxLCIxIl0sWzEsMCwiXFxmaWJfZiBnIGQiXSxbMSwyLCJcXGV4aXN0cyIsMix7InN0eWxlIjp7ImJvZHkiOnsibmFtZSI6ImRhc2hlZCJ9fX1dLFswLDJdLFswLDFdXQ==
	\[\begin{tikzcd}
		{d=0} & {\fib_{f_d} g_d} \\
		1
		\arrow[from=1-1, to=1-2]
		\arrow[from=1-1, to=2-1]
		\arrow["\exists"', dashed, from=2-1, to=1-2]
	\end{tikzcd}\]
	where the above map is given by transport of $(w , p) : \fib_{f_0}(g_0)$.
	By choice of $D$ we can produce a term in 
	\[
	\prod_d (h_d : \fib_{f_d} g_d) \times ((r : d = 0) \to  h_d = \mathrm{tp}_r(w,p)) \simeq \left(\prod_d (h_d : \fib_{f_d} g_d) \right) \times h_0 = w 
	\]
	Which is the datum of a filler.	This concludes the special case of $D$ beeing tiny. \\
	If $D = \bigvee_{i=1}^n D^i$ we can produce by the special case sections $h^i : \prod_{d: D_i} W_d$ such that $h^i_0 = w$ with $q_d : g_d = f_d(h^i_d)$. As the $h_i$ agree on the basepoint, we get a dependent section $h : \prod_{d: D} W_d$ with $h_0 = w$and $g_d = f_d (h_d)$.
%	$\bT$-merely have point in the fiber of $f$ over $g d$ that moreover equals $w$, if $d = 0$.	
\end{proof}
\begin{lemma}
	Let $D$ be a finite wedge of infinitesimal varieties. Given a family of $\bT$-surjective smooth maps $W_d \to X_d$, the map $\prod_{d: D} W_d \to \prod_{d: D} X_d$ is $\bT$-surjective
\end{lemma}
\begin{proof}
	To apply the previous lemma, we just use that $W_0 \to X_0$ is $\bT$-surjective.
\end{proof}
\begin{prop}
	Let $j : D \to D'$ be a map between a finite wedge of infinitesimal varieties $D$ and a tiny type $D'$, that is local wrt to all affine schemes, i.e. $X ^ {D'} \to X ^D$ is an equivalence for any affine $X$. Then its local wrt to all geometric stacks.
	In particular, geometric stacks are infinitesimally linear, i.e. a geometric stack $X$ is local wrt $j : \bD(n_1) \lor \hdots \lor \bD(n_k) \to \bD(n_1+ \hdots + n_k)$ for any $n_1,\hdots,n_k : \bN$ .
\end{prop}
\begin{proof}
We prove more generally, that for any family $X :D' \to \GS$, the map
\[
\prod_{d:D'} X d \to \prod_{d:D} X (j d) \tag{$\star$}
\]
is an equivalence. Lets first check the special case where all the $X d$ are affine: We have a pullback
% https://q.uiver.app/#q=WzAsNCxbMCwxLCJcXGxlZnQgKFxcc3VtX3tkOlxcYkQobittKX1YIGkgZCBcXHJpZ2h0KV57XFxiRChuK20pfSJdLFsxLDEsIlxcbGVmdCAoXFxzdW1fe2Q6XFxiRChuK20pfSBYIGQgXFxyaWdodClee1xcYkQobikgXFxsb3IgXFxiRChtKX0iXSxbMCwwLCJcXHByb2Rfe2Q6IFxcYkQobittKX0gWCBkICJdLFsxLDAsIlxccHJvZF97ZDpcXGJEKG4pIFxcbG9yIFxcYkQobSl9IFggKGkgZCkgIl0sWzAsMV0sWzIsM10sWzMsMV0sWzIsMF0sWzIsMSwiIiwxLHsic3R5bGUiOnsibmFtZSI6ImNvcm5lci1pbnZlcnNlIn19XV0=
\[\begin{tikzcd}
	{\prod_{d:D'} X d } & {\prod_{d:D} X (j d) } \\
	{\left (\sum_{d:D'}X d \right)^{D'}} & {\left (\sum_{d:D} X j d \right)^{D}}
	\arrow[from=1-1, to=1-2]
	\arrow[from=1-1, to=2-1]
	\arrow["\ulcorner"{anchor=center, pos=0.125, rotate=45}, draw=none, from=1-1, to=2-2]
	\arrow[from=1-2, to=2-2]
	\arrow[from=2-1, to=2-2]
\end{tikzcd}\]

The lower map is an equivalence, as $\sum_{d :D'} X d$ is affine scheme, hence infinitesimally linear. So the above map is an equivalence as well. \\
We split the prove of equivalence up into $\bT$-surjectivity and beeing an embedding.
\begin{itemize}
	\item
%We need to show, that for any $(d : D) \to X (j d)$ we $\bT$-merely find an extension $(d :D') \to X d$. \\
%First observe, that the statement is a sheaf, as the $X d$ are stacks. 
By choice of $D'$ we find geometric atlasses $W d \to X d$ for $d :D'$.
Then by the special case and \ref{cor:GeomAtlExpStable} we can  the following commutative diagram

% https://q.uiver.app/#q=WzAsNCxbMSwxLCJcXHByb2Rfe2QgOiBcXGJEKG4pIFxcbG9yIFxcYkQobSl9IFgoamQpIl0sWzEsMCwiXFxwcm9kX3tkIDogXFxiRChuKSBcXGxvciBcXGJEKG0pfSBXKGpkKSJdLFswLDEsIlxccHJvZF97ZCA6IFxcYkQobittKX0gWCBkIl0sWzAsMCwiXFxwcm9kX3tkIDogXFxiRChuK20pfVcgZCJdLFszLDEsIlxcc2ltIl0sWzMsMl0sWzIsMF0sWzEsMCwiXFxiVCAtc3VyaiJdXQ==
\[\begin{tikzcd}
	{\prod_{d :D'}W d} & {\prod_{d : D} W(jd)} \\
	{\prod_{d :D'} X d} & {\prod_{d : D} X(jd)}
	\arrow["\sim", from=1-1, to=1-2]
	\arrow[from=1-1, to=2-1]
	\arrow["{\bT -surj}", from=1-2, to=2-2]
	\arrow[from=2-1, to=2-2]
\end{tikzcd}\]
\item
Now we need to show that the map is an embedding. Induction over the truncatedness of $X$. For $n=-2$ its fine.
For the induction step $n \mapsto n+1$, use function extensionality and observe that the identity types of $X$ are geometric $n$-stacks, so the map $(\star)$ where we replace $X d$ by its appropriate identity type, is an equivalence by induction.
\end{itemize}
\end{proof}





%\begin{lemma}[If closed dense props are tiny]
%	Let $X \to Y$ be a $\bT$-surjective map of stacks, where $X$ is formally smooth. Then $Y$ is formally smooth.
%\end{lemma}
%\begin{proof}
%	Consider a closed dense prop $P$ and a map $f : Y^P$. By \ref{lemma:TinyStrongLocalChoice} we $\bT$-merely have a lift (1.). As the goal $\fib_{Y \to Y^P}(f)$ is a stack, using that $Y$ is a stack, we may pick a lift (1.). As $X$ is formally smooth we obtain (2.). From this we get the required lift of $f$ by composition $1 \to X \to Y$ .
%	% https://q.uiver.app/#q=WzAsNCxbMCwwLCJQIl0sWzAsMSwiMSJdLFsxLDAsIlkiXSxbMSwxLCJYIl0sWzAsMV0sWzAsMywiMS4iLDEseyJzdHlsZSI6eyJib2R5Ijp7Im5hbWUiOiJkYXNoZWQifX19XSxbMCwyXSxbMywyXSxbMSwzLCIyLiIsMSx7InN0eWxlIjp7ImJvZHkiOnsibmFtZSI6ImRhc2hlZCJ9fX1dXQ==
%	\[\begin{tikzcd}
%		P & Y \\
%		1 & X
%		\arrow["f",from=1-1, to=1-2]
%		\arrow[from=1-1, to=2-1]
%		\arrow["{1.}"{description}, dashed, from=1-1, to=2-2]
%		\arrow["{2.}"{description}, dashed, from=2-1, to=2-2]
%		\arrow[from=2-2, to=1-2]
%	\end{tikzcd}\]
%\end{proof}
\begin{lemma}
	Let $A = R[X_1,\hdots,X_n] / (f_1,\hdots,f_m)$ be a finitely presented algebra. Let $\fp : \Spec A$ be a point. The tangent space of $\Spec A$ at $\fp : \Spec A \subset \bA^n$ is affine whose algebra is cut out by the polynomials $g_i = \sum_{k} \frac{\partial f_i}{\partial x_k}(\fp) x_k$:
\end{lemma} 
\begin{proof}
	I give two proofs
	\begin{itemize}
		\item 
	Write $V$ to the $A$-module $R$ obtained by $\fp$.
%	% https://q.uiver.app/#q=WzAsOCxbMCwwLCJcdFxcbWF0aHJte0Rlcn0oUltYXzEsXFxoZG90cyxYX25dLFYpICJdLFswLDEsIlxcbWF0aHJte0Rlcn0oQSxWKSAiXSxbMCwyLCJkICJdLFswLDMsIihoIFxcbWFwc3RvIFxcc3VtX2sgXFxmcmFje1xccGFydGlhbCBofXtcXHBhcnRpYWwgeF9rfShwKSBcXGNkb3Qgdl9rKSJdLFsxLDEsIlYoZ18xLFxcaGRvdHMsZ19uKSBcdCJdLFsxLDIsIihkeF8xLCBcXGhkb3RzLGQgeF9uKSJdLFsxLDMsInYiXSxbMSwwLCJcXFNwZWMgUltkIFhfMSxcXGhkb3RzLGQgWF9uXSAiXSxbNiwzLCIiLDAseyJzdHlsZSI6eyJ0YWlsIjp7Im5hbWUiOiJtYXBzIHRvIn19fV0sWzIsNSwiIiwwLHsic3R5bGUiOnsidGFpbCI6eyJuYW1lIjoibWFwcyB0byJ9fX1dLFsxLDQsIiIsMCx7InN0eWxlIjp7InRhaWwiOnsibmFtZSI6ImFycm93aGVhZCJ9LCJib2R5Ijp7Im5hbWUiOiJkYXNoZWQifX19XSxbMSwwLCIiLDIseyJzdHlsZSI6eyJ0YWlsIjp7Im5hbWUiOiJob29rIiwic2lkZSI6InRvcCJ9fX1dLFs0LDcsIiIsMCx7InN0eWxlIjp7InRhaWwiOnsibmFtZSI6Imhvb2siLCJzaWRlIjoidG9wIn19fV0sWzAsNywiXFxjb25nIiwyXV0=
%	\[\begin{tikzcd}
%		{	\mathrm{Der}(R[X_1,\hdots,X_n],V) } & {\Spec R[d X_1,\hdots,d X_n] } \\
%		{\mathrm{Der}(A,V) } & {V(g_1,\hdots,g_n) 	} \\
%		{d } & {(dx_1, \hdots,d x_n)} \\
%		{(h \mapsto \sum_k \frac{\partial h}{\partial x_k}(p) \cdot v_k)} & v
%		\arrow["\cong"', from=1-1, to=1-2]
%		\arrow[hook, from=2-1, to=1-1]
%		\arrow[dashed, tail reversed, from=2-1, to=2-2]
%		\arrow[hook, from=2-2, to=1-2]
%		\arrow[maps to, from=3-1, to=3-2]
%		\arrow[maps to, from=4-2, to=4-1]
%	\end{tikzcd}\]
% https://q.uiver.app/#q=WzAsNixbMCwwLCJcdFxcbWF0aHJte0Rlcn0oUltYXzEsXFxoZG90cyxYX25dLFYpICJdLFswLDEsImQgIl0sWzAsMiwiKGggXFxtYXBzdG8gXFxzdW1fayBcXGZyYWN7XFxwYXJ0aWFsIGh9e1xccGFydGlhbCB4X2t9KHApIFxcY2RvdCB2X2spIl0sWzEsMSwiKGR4XzEsIFxcaGRvdHMsZCB4X24pIl0sWzEsMiwidiJdLFsxLDAsIlxcU3BlYyBSW2QgWF8xLFxcaGRvdHMsZCBYX25dICJdLFs0LDIsIiIsMCx7InN0eWxlIjp7InRhaWwiOnsibmFtZSI6Im1hcHMgdG8ifX19XSxbMSwzLCIiLDAseyJzdHlsZSI6eyJ0YWlsIjp7Im5hbWUiOiJtYXBzIHRvIn19fV0sWzAsNSwiXFxjb25nIiwwLHsic3R5bGUiOnsidGFpbCI6eyJuYW1lIjoiYXJyb3doZWFkIn19fV1d
We have a bijection
\[\begin{tikzcd}
	{	\mathrm{Der}(R[X_1,\hdots,X_n],V) } & {\Spec R[Y_1,\hdots,Y_n] } \\
	{d } & {(dx_1, \hdots,d x_n)} \\
	{(h \mapsto \sum_k \frac{\partial h}{\partial x_k}(p) \cdot v_k)} & v
	\arrow["\cong", tail reversed, from=1-1, to=1-2]
	\arrow[maps to, from=2-1, to=2-2]
	\arrow[maps to, from=3-2, to=3-1]
\end{tikzcd}\]
This restricts to a bijection
$\mathrm{Der}(A,V) \cong V(g_1,\hdots,g_n) $
by construction of the $g_i$.
% https://q.uiver.app/#q=WzAsNixbMCwwLCJcdFxcbWF0aHJte0Rlcn0oUltYXzEsXFxoZG90cyxYX25dLFYpICJdLFswLDEsImQgIl0sWzAsMiwiKGggXFxtYXBzdG8gXFxzdW1fayBcXGZyYWN7XFxwYXJ0aWFsIGh9e1xccGFydGlhbCB4X2t9KHApIFxcY2RvdCB2X2spIl0sWzEsMSwiKGR4XzEsIFxcaGRvdHMsZCB4X24pIl0sWzEsMiwidiJdLFsxLDAsIlxcU3BlYyBSW2QgWF8xLFxcaGRvdHMsZCBYX25dICJdLFs0LDIsIiIsMCx7InN0eWxlIjp7InRhaWwiOnsibmFtZSI6Im1hcHMgdG8ifX19XSxbMSwzLCIiLDAseyJzdHlsZSI6eyJ0YWlsIjp7Im5hbWUiOiJtYXBzIHRvIn19fV0sWzAsNSwiXFxjb25nIiwwLHsic3R5bGUiOnsidGFpbCI6eyJuYW1lIjoiYXJyb3doZWFkIn19fV1d

%	Claim: For any derivation $d$ from $A$ to $V$, we have for any polynomial $h$ we have that
%	\[
%	d [h] =\sum_{k=1}^n  \frac{\partial h}{\partial x_k}(p) \cdot d x_k
%	\]
%	\begin{proof}
%		Any derivation from $A$ at $\fp$ induces a derivation of $\bA^n$ at $\fp$ and the statement is encoded in the fact, that the tangent space of $\bA^n$ at $\fp$ is $\bA^n$.
%	\end{proof}
%	 Let us show that the following maps are well-defined and mutually inverse.
\item We can write $\Spec A$ as a pullback
% https://q.uiver.app/#q=WzAsNCxbMCwwLCJcXFNwZWMgQSJdLFsxLDAsIlJebiJdLFsxLDEsIlJebSJdLFswLDEsIjEiXSxbMCwzXSxbMywyLCIwIl0sWzAsMV0sWzEsMl0sWzAsMiwiIiwxLHsic3R5bGUiOnsibmFtZSI6ImNvcm5lci1pbnZlcnNlIn19XV0=
\[\begin{tikzcd}
	{\Spec A} & {R^n} \\
	1 & {R^m}
	\arrow[from=1-1, to=1-2]
	\arrow[from=1-1, to=2-1]
	\arrow["\ulcorner"{anchor=center, pos=0.125}, draw=none, from=1-1, to=2-2]
	\arrow["F" , from=1-2, to=2-2]
	\arrow["0", from=2-1, to=2-2]
\end{tikzcd}\]
Now one can apply Tangentspaces to get a pullback again
% https://q.uiver.app/#q=WzAsNCxbMCwwLCJUX1xcZnAgXFxTcGVjIEEiXSxbMSwwLCJUX1xcZnAgUl5uID0gUl5uIl0sWzEsMSwiVF8wIFJebSA9IFJebSJdLFswLDEsIjEiXSxbMCwzXSxbMywyLCIwIl0sWzAsMV0sWzEsMiwiRCBGX1xcZnAiLDJdLFswLDIsIiIsMSx7InN0eWxlIjp7Im5hbWUiOiJjb3JuZXItaW52ZXJzZSJ9fV1d
\[\begin{tikzcd}
	{T_\fp \Spec A} & {T_\fp R^n = R^n} \\
	1 & {T_0 R^m = R^m}
	\arrow[from=1-1, to=1-2]
	\arrow[from=1-1, to=2-1]
	\arrow["\ulcorner"{anchor=center, pos=0.125}, draw=none, from=1-1, to=2-2]
	\arrow["{D F_\fp}"', from=1-2, to=2-2]
	\arrow["0", from=2-1, to=2-2]
\end{tikzcd}\]
	\end{itemize}
\end{proof}
\begin{lemma}
	Let $\varepsilon$ be nilpotent.
	If $R / \varepsilon$ is flat, then $\varepsilon = 0$.
\end{lemma}
\begin{proof}
	 Claim: $ R/ \varepsilon$ projective.
	 From $(\varepsilon) \to R \to R / \varepsilon$ we deduce a factorization $R \to G \to R / \varepsilon$ with $G$ free and $(\varepsilon) \to R \to G$ the zero map. From this we see that $G \to R / \varepsilon$ is split surjective.  \qed(Claim)
 	 As $R$ is local, $R / \varepsilon$ is free , but as $\lnot \varepsilon \neq 0$, it has to be free of rank 1, thus $\varepsilon = 0$.
\end{proof}
\begin{warning}{\label{warning:TangentSpacesNotFlat}}
	Tangent spaces of faithfully flat affines are not flat in general.
	Let $p \neq 0$ be prime. $R[X] / X^p$ is a faitfully flat algebra as $X^p$ is a monic polynomial \todocite. Then it is not the case that all tangent spaces are flat. 
	%This algebra is not flat! 
\end{warning}
\begin{proof}
	Assume this is the case, i.e.  that for any $\varepsilon : R$, such that $\varepsilon^p = 0$, 
	We have that
	\[T_\varepsilon \Spec R[X]/ (X^p) = \Spec R[Y] / (p \varepsilon^{p-1} Y) = \Spec R[Y] / (\varepsilon^{p-1} Y) \equiv \Spec A.\]
	is flat affine.
	We have an $R$-linear isomorphism
	\begin{align*}
		R \oplus R/\varepsilon^{p-1} [Y] &\to R[Y]/ (\varepsilon^{p-1} Y) \\
		(r , f) &\mapsto r + Y f
	\end{align*}
	As the RHS, the second factor of the LHS is flat over $R$. As $R / \varepsilon^{p-1} [Y]$ is a faithfully flat algebra over $R / \varepsilon^{p-1}$, we deduce that $R / \varepsilon^{p-1}$ is flat over $R$. By the lemma We conclude that $\varepsilon^{p-1} =0$ for all $\varepsilon : \Spec A$. But this means $\Spec R[Y] / (Y^{p-1}) = \Spec R[Y] / (Y^{p})$ by duality - a contradiction.
	
\end{proof}
\begin{rmk}
	Given a pointed geometric stack $(X,x)$, we can look at $Y = T_x X$ and the map $Y \to \|Y\|_0$, the fibers are deloopings of $\Omega Y$. Now there exists a $\Omega Y$-equivariant isomorphism $Y \cong \|Y\|_0 \times B \Omega Y$ over $\|Y\|_0$ iff the map $Y \to \|Y\|_0^\bT$ has a section.
\end{rmk}


\section{Questions // TODO}
\begin{theorem}[TODO]
	An Artin stack $X$ is Deligne Mumford iff one of the following conditions is satisfied:
	\begin{enumerate}
		\item There exists a geometric atlas $W \to X$
		\item The identity types of $X$ are $\bP$-seperated
	\end{enumerate}
\end{theorem}
\begin{proof}
	\begin{enumerate}
		\item [1. $\Rightarrow $2.] \ref{lemma:DMUnrIdTypes}
		\item [2. $\Rightarrow$ 1] Residual ??? [06MF]
		%As a stack beeing $\bP$-seperated is a sheaf and $%
	\end{enumerate}
\end{proof}
Prove \ref{lemma:StdEtSm}!!!
\begin{question}
	if $\bT \subset \bT'$ do we have that for each $X : \GS_\bT$ $L_{\bT'} X \in \GS_{\bT'}$?
\end{question}
\begin{theorem}[TODO]{\label{thm:FlatAffines}}
	The class of flat affines is stable under $\sum$. Moreover flatness can be defined fiberwise.
\end{theorem}
%\section{Todo}

\section{Not clear where to put that}
\begin{lemma}
	Let $\rho \neq 0$.
	$\Spec R[T] / (T^2 + 1)$  is compact.
\end{lemma}
\begin{proof}
	Let $U \subset \Spec C$ be open. Then we find $f_1,\hdots,f_n :C$, such that
	$U = D(f_1,\hdots,f_n)$.
	Choose representatives $f_i = a_i + b_iT \mod T^2 + \rho$.
	Then consider the following numbers 
	\[
	r_{ij} = \begin{cases} a_i b_j - a_j b_i  &, i \neq j \\ a_i^2 + \rho b_j^2 &, i = j \end{cases}
	\]
	We will show that $D((r_{ij})_{i,j}) \leftrightarrow (\Spec C \subset U)$. \\
	'$\rightarrow$' \\
	Assume $r_{ij} \neq 0$. If $i = j$, then $\Spec C \subset D(f_i) \subset U$. \\
	If $i \neq j$, then $\Spec C \subset D(f_i,f_j) \subset U$. \\
	'$\leftarrow$' \\
	Because this statement is a propositional sheaf, we may assume a term $x : \Spec C$. Choose $i,j$, s.th. $x \in D(f_i), -x \in D(f_j)$. In both cases $i = j$ and $i \neq j$, then $r_{ij} \neq 0$. 
\end{proof}
\begin{lemma}[Not needed]
	Open subtypes of $\bA^1$ are $\lnot \lnot$ principal open. %If the open is inhabited, $\lnot \lnot$ merely we can $\mu_{\ell}$-invariant write it as $D(p)$ with $p : R[X]$ regular .
\end{lemma}
\begin{proof}
	\begin{itemize}
		\item An open affine subscheme of $\bA^1$ is $\lnot \lnot$ principal open: Let $D(f_1,\hdots,f_n) \subset \bA^1$ be an arbitrary open subset. We may assume that each $f_i : R[X]$ is non constant (in particular non zero). By \todocite, $\lnot \lnot$-merely each $D(f_i) \subset R$ is cofinite. Thus $\lnot \lnot$-merely, the finite union $\bigcup_{i=1}^n D(f_i) \subset R$ is cofinite as well, hence principal open. % we find roots $a_{ij}$ of the $f_i$. $\bigcup_i D(f_i) = \bigcup_i R \setminus \{a_{ij}\}_j = R \setminus \bigcap_{i} \{a_{ij}\}$ for some finite subset. This is $\lnot \lnot$ a principal open. %By $\lnot \lnot $ stability 
	\end{itemize}
\end{proof}

\begin{prop}{\label{prop:htpyGroups}}
	Assume covering stacks are $\Omega$-stable.
	A truncated stack (e.g. geometric stack) is covering iff $\pi_0^\bT X := \|X\|_0^\bT$ and all higher homotopy groups 
	\[
	\pi_i^\bT(X,x) = \|\Omega^i (X , x)\|_0^\bT , i \ge 1
	\]
	are covering algebraic spaces.
\end{prop}
\begin{proof}
	Let $X$ be an $n$-stack.
	If $X$ is covering, then by $\Omega$-stability all the $\pi_i^\bT$ are covering \ref{prop:GerbeIffLooping} Now the converse.
	Consider the postnikov tower
	\[
	X = \|X\|_n^\bT \to \|X\|_{n-1}^\bT \to \hdots \to \|X\|_1^\bT \to \|X\|_0^\bT 
	\]
	As $\|X\|_0^\bT$ is covering, by quotient stability of covering stacks we may show that all the maps are geometric covers. 
	Let $1 \le k \le n$ and consider the map $f_k^X : \|X\|_{k}^\bT \to \|X\|_{k-1}^\bT$. By descent for covering stacks, we may only consider the fiber over $|x|$, as the $\eta_{k-1}^\bT$ is $\bT$-surjective. 
	It suffices to show, that the fiber is given by $B_\bT^k \pi^\bT_k(X,x)$ as deloopings of covering stacks are covering \ref{lemma:deloopingCS}.\\	
	We apply \ref{lemma:detectDelooping}.
	First observe that $\Omega^k (\fib(f_k^X) |x| = \fib (\Omega^k (f_k^X , x))$ is equivalent to the basefiber of
	
	\[
	\pi_k^\bT(X,x) \equiv \|\Omega^k X \|_0^\bT \simeq \Omega^k(\|X\|_k^\bT) \to \Omega^k \|X\|_{k-1}^\bT \simeq 1
	\]
	So it suffices to show by induction over $k$, that for all pointed stacks $(X,x)$, $\fib(f_k^X) |x|$ is $\bT$-$k$-connected.
	
	This is definitely $\bT$-connected by using that any term $(y, p) : \fib(f_k^X) |x| = \sum_{y : \|X\|_n^\bT} \|x = y\|_{n-1}^\bT$ yields a witness of $\|x - y\|^\bT$. Then $\Omega (\fib(f_k^X) |x| = \fib (\Omega (f_k^X , x)) = \fib (f_{k-1}^{\Omega (X , x)})$ which is $\bT$-$k-1$-connected by induction.
\end{proof}

\subsection{Remarks about weakly flat affines}

\begin{lemma}
	The proposition $\|X\|_\bT$ is geometric iff there exists a map from a weakly flat affine $\Spec B \to X$ such that $\|\Spec B\|_\bT \to \|X\|_\bT$ is an equivalence.
\end{lemma}
\begin{proof}
	'$\leftarrow$' is clear. \\
	'$\rightarrow$'. Choose $\Spec B'$ weakly flat such that $\|X\|_\bT = \|\Spec B'\|_\bT$. As the map $X \to \|X\|_\bT$ is $\bT$-surjective, by $\bT$-local choice we find a $\bT$-cover $\Spec B \to \Spec B'$ and a commutative diagram
	% https://q.uiver.app/#q=WzAsNCxbMSwwLCJcXFNwZWMgQSJdLFsxLDEsIlxcfFxcU3BlYyBBXFx8X1xcYlQiXSxbMCwxLCJcXFNwZWMgQiciXSxbMCwwLCJcXGV4aXN0cyBcXFNwZWMgQiJdLFszLDAsIiIsMCx7InN0eWxlIjp7ImJvZHkiOnsibmFtZSI6ImRhc2hlZCJ9fX1dLFszLDIsIiIsMix7InN0eWxlIjp7ImJvZHkiOnsibmFtZSI6ImRhc2hlZCJ9fX1dLFswLDFdLFsyLDFdXQ==
	\[\begin{tikzcd}
		{\exists \Spec B} & {X} \\
		{\Spec B'} & {\|X\|_\bT}
		\arrow[dashed, from=1-1, to=1-2]
		\arrow[dashed, from=1-1, to=2-1]
		\arrow[from=1-2, to=2-2]
		\arrow[from=2-1, to=2-2]
	\end{tikzcd}\]
	As $\Spec B'$ was weakly flat and the left vertical map is a $\bT$-cover, $\Spec B$ is weakly flat.
\end{proof}
\begin{lemma}[DM]
	If $\Spec A + \Spec B$ is weakly flat affine, then $\Spec A$ is weakly flat. 
\end{lemma}
\begin{proof}
	Indeed
	\[
	\|X\|_\bT \to \|X + Y\|_\bT \to X + Y \in \bT \to X \in \bP 
	\]
	but $\|X\|_\bT \land X \in \bP \to X \in \bT$. \\
\end{proof}
\begin{lemma}{\label{lemma:wfdesTCov}}
	if the topology is saturated Beeing weakly-flat descends along $\bT$-covers.
\end{lemma}
\begin{lemma}[DM]
	If $\|P + Q \|_\bT$ is a geometric prop, then TODO % $\|P\|_\bT$ and $\|Q\|_\bT$ are geometric props.
\end{lemma}
\begin{proof}
	
	By the previous two lemma and we find a map out of a weakly flat affine $\Spec B \to P + Q$ that induces an equivalence on $\bT$-truncations, but it splits into two map out of a  weakly affine $\Spec B_1 \to P, \Spec B_2 \to Q$.
\end{proof}
\begin{notation}
	For $P : (\varepsilon : \cN_\infty(0)) \to X \to \Prop$, let $\varepsilon : \cN_\infty(0) \vdash x : X$. We say $x$ is not$_\varepsilon$ $P$, if $\forall \varepsilon$ , $P_\varepsilon x \to \varepsilon = 0$. Observe, if $x$ is not$_\varepsilon$ $P$ for any $\varepsilon^2 = 0$, then $x$ is not $P$.
\end{notation}

\begin{rmk}
	If $2 \neq 0$. Let $\varepsilon , \varepsilon' : \cN_\infty(0)$. $\varepsilon = \varepsilon' + \varepsilon = -\varepsilon'$ is not$_\varepsilon$ weakly-flat
\end{rmk}
\begin{proof}
	We prove that once its $\bT$-merely inhabited, then its not$_\varepsilon$ covering, which is enough as $\lnot \lnot (\varepsilon = \varepsilon' + \varepsilon = -\varepsilon')$.% $\varepsilon = 0$.
	As the goal is a stack we may assume $\varepsilon = \varepsilon' + \varepsilon = -\varepsilon' $. wlog the first case. Then assume $1 + (\varepsilon = - \varepsilon) \simeq 1 + \varepsilon = 0$ is covering. Then $\varepsilon = 0$ is formally \etale, thus inhabited as a formally \etale closed dense proposition. %  for all $\varepsilon$.
\end{proof}
\begin{example}[Obsolete]
	The map $q : \bA^1 \to \bA^1 / \mu_\ell$ is not a geometric cover.
\end{example}
\begin{proof}
	The map factors through the geometric cover $\bA^1 \to \bA^1 // \mu_\ell$. Thus its enough to show that $\bA^1 // \mu_\ell$ is not a 0-gerbe, or that not every loop space is covering. 
	Let us show that, $\Omega(\bA^1 // \mu_\ell ,\varepsilon)$ is not$_\varepsilon$ covering. Assume it is covering for some $\varepsilon \in \cN_\infty(0)$.
	%Observe that we have a map
	%	\begin{align*}
	%		f_\varepsilon : \mu_\ell &\to \fib_q \varepsilon \\
	%		g &\mapsto g \varepsilon 
	%	\end{align*}
	%	Now $\fib_f \varepsilon$ is an \etale-flat geometric stack. \\%\bA^1 // \mu_\ell
	As $\mu_\ell$ has decidable equality,
	\begin{align*}
		\Omega(\bA^1 // \mu_\ell , \varepsilon) &= \left (\sum_{g : \mu_\ell} g \varepsilon = \varepsilon \right) \\
		&=  (\varepsilon = \varepsilon) + \sum_{g : \mu_\ell \setminus \{1\}} (g-1) \varepsilon = 0 \\ &= 1 + \mu_\ell \setminus \{1\} \times (\varepsilon = 0) 
	\end{align*}
	Thus $(\varepsilon = 0) \times (\mu_\ell \setminus \{1\})$ is an \etale-flat geometric stack. Moreover $(\mu_\ell \setminus \{1\})$ is a covering stack by \ref{lemma:CompOf1}. Thus $\varepsilon = 0$ is an affine \etale-flat geometric stack, thus formally \etale + flat affine by saturatedness of the \etale topology \ref{cor:EtaleSat}. So as a formally \etale + closed dense proposition, $\varepsilon = 0$ holds as desired. % for all $\varepsilon$ nilpotent. Contradiction to algebra.
\end{proof}
\printbibliography

\end{document}