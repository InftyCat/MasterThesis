	Consider a modality $L$, such that
\begin{enumerate}
	%	\item[L0]  $R$ is modal 
	\item[L1] Any $r : R$ invertible $L$-merely has a squareroot. 
	
	\item[L2] $\bot$ is modal 
	%\item T3. Schemes are sheaves.
\end{enumerate}

\ifsc
\section{Line with two origins}
\begin{lemma}{\label{lemma:clPropIsCnct}}
	Closed propositions are connected.
\end{lemma}
\begin{proof}
	Let $P = (I = 0)$ be a closed proposition, for $I \subset R$ an ideal. We wish to show, that the map $\Bool \to \Bool^{P}$ is surjective. Maps $\Spec A \to \Bool = \Spec (R \times R)$ biject with idempotents of $\Spec A$ (Indeed the image of $R \times R$ classifies idempotents by the image of $(1,0)$ in $A$). In our case $A = R/I$ has only the two trivial idempotents 0 and 1: Indeed: any idempotent of $A$ can be lifted to an idempotent of $A$, using locality of $R$. But then, using locality of $R$ again, we conclude. \\
\end{proof}
\else
\subsection{Line with two origins}
\fi

\begin{lemma}{\label{lemma:X1}}
	For any $x : R$: the map
	\begin{align*}
		f_x : \Susp(x \neq 0) &\to \left(\Bool^{x = 0} \right)\\
		N &\mapsto  \lambda \_ . true \\
		S &\mapsto \lambda \_ . false \\
	\end{align*}
	is an equivalence.\\
	In particular we have two equivalent models of the line with two origins
	\[
	\sum_{x : R} \Bool ^{x=0} \simeq \sum_{x : R} \Susp (x\neq0)
	\]
\end{lemma}
\begin{proof}
	It is well-defined: We have to check that $f_x(N) = f_x(S)$ if $x \neq 0$. But in this case the function type $\Bool^{x=0}$ is contractible. \\
	Is is surjective as closed propositions are connected \ref{lemma:clPropIsCnct}. 
	It is injective: As $\Bool \to \Susp(x \neq 0)$ is surjective, we may only study the points $N$ and $S$. By case-analysis we only need to show, that if $f_x(N) = f_x(S)$, then $x \neq 0$. But if $x = 0$, then $true =_{\Bool} false$, a contradiction. 
\end{proof}
\ifsc
\begin{lemma}[$0$ is a regular point of $R$]{\label{lemma:AlmostEverywhere}}
	If $0 \in U \subset R$ is an open neighbhorhood, then the restriction map $R^U \to R^{U \setminus \{0\}}$ is injective. 
\end{lemma}
\begin{proof}
	We may assume that $U = D(f)$ is a principal open neigbhorhood. Then, the element $X : R[X]_f$ is regular, as $X : R[X]$ is regular: Indeed if $X P f^n  =_{R[X]} 0$ for some $P : R[X]$, then $P f^n = 0$, thus $P =_{R[X]_f}  0$. \\
	In other words, the map $R[X]_f \to R[X]_{f X}$ is injective, which is a reformulation of the 	goal.
\end{proof}
\begin{rmk}
	One can define regularity of a point 0 in a scheme $X$ generally by asking that it admits a open affine neigbhorhood $0 \in \Spec A \subset X$ such that $\Spec A \setminus \{0\} = D(g_1,\hdots,g_n)$ for $A \to \prod_{i=1}^n A_{g_i}$ injective. This yields a well-behaved notion. \cite{geomstacks}
\end{rmk}
\fi
\begin{lemma}{\label{lemma:noAffNbhd}}
	There is no open affine subset of the line with two origins $L$ containing both points.
\end{lemma}
\begin{proof}
	Let us write $p : L \to R$ for the first projection.
	Assume there is an open affine subset of the line with two origins such that $\fib_p(0) \subset U \subset L$. Then $p(U) \subset R$ is an open neigbhorhood of 0, as 
	\[
	x \in p(U) \leftrightarrow (x,true) \in U \lor (x,false) \in U
	\]
	Claim: the map $R^{p(U)} \to R^U$ is an equivalence. If we have shown that: As $U$ is affine we conclude that the map
	\begin{align*}
		U &\to \Spec (R^{p(U)}) \\
		x &\mapsto \phi \mapsto (\phi(px))		
	\end{align*}
	is an equivalence, which is a contradiction to the assumption, that $U$ contains both (distinct!) origins. \\
	Proof of claim: 
	First the Proof idea: As $U$ is a subset of a quotient of $R + R$, the function $U \to R$ determines two (partially defined on open domain) functions to $R$ that coincide away from the origin, which is a regular point. Thus by \ref{lemma:AlmostEverywhere} they coincide everywhere. More precisely:\\
	Injectivity: If two maps $f , g : p(U) \to R$ coincide after precomposing with $U \to p(U)$, then they coincide away from $0$
	so conclude by \ref{lemma:AlmostEverywhere}. \\
	Surjectivity: Given a map $L \supset U \to R$, by pulling back along $f : R + R \to L$  we can view it as a map $R + R \supset f^{-1}(U) \to R$ defined at both origins, so in particular as a pair of maps to $R$ defined on some open neigbhorhood of 0 in $R$. They coincide away from 0 so by \ref{lemma:AlmostEverywhere} they are equal.
\end{proof}
\ifsc
\section{Twisted line with double origin}
\else
\subsection{Twisted line with double origin}
\fi


\begin{lemma}{\label{lemma:BoolSpecC}}
	\
	Let $2 \neq 0$. Let $r \neq 0 $. Denote $C_r = R[X] / (X^2 + r)$
	\begin{enumerate}
		\item Given $y,x : \Spec C_r$ we have $y = x$ or $y = -x$.
		\item 	Any embedding $\Bool \hookrightarrow \Spec C_r$ is already an equivalence 
		
		\item 	$\| \Spec C_r \| \leftrightarrow \|\Bool \simeq \Spec C_r\|$	\end{enumerate}
\end{lemma}
\begin{proof}
	\begin{enumerate}
		\item We have \[(x+y)(x-y) = x^2 - y^2 = x^2 + r = 0\]
		we know that one of the factors is invertible by locality ($y \neq -y$) so the other factor is zero. 
		\item 	Any embedding 
		\begin{align*}
			\Bool &\to \Spec C_r \\
			true &\mapsto i \\
			false &\mapsto i'
		\end{align*}
		is already an equivalence: We apply 1. twice: From $i' \neq i$ we get $i' = -i$ and the above map is surjective.
		\item 	\ \\ '$\leftarrow$' 
		Obvious \\
		'$\rightarrow$'  Because $i \neq -i$, this determines an embedding.
	\end{enumerate}
	
\end{proof}

\begin{lemma}{\label{lemma:heart}}
	Let $r : R^\times$. Denote 
		\[
		C_r = R[X] / (X^2 + r) 
		\]
	Consider an open subset  $U \subset \Spec C_r$, such that $\lnot (U =\Spec C_r)$. Then $U$ is an open proposition.
\end{lemma}
\begin{proof}
	Note, that $U$ is a proposition: If $x,x' : U$, then $x = x' \simeq \lnot \lnot (x = x')$ by decidable equality of $U$, using that $\Spec C_r$ is a formally \etale affine. But if $x \neq x'$, then $\{x,x'\} \hookrightarrow \Spec C_r$ is an embedding, so by \ref{lemma:BoolSpecC} an equivalence, but then $U = \Spec C_r$, contradiction. \\
	We first reduce to the case where $U$ is a principal open of $\Spec C$. By \cite{cherubini2023foundationsyntheticalgebraicgeometry} we find $f_1,\hdots,f_n : C_r$ such that $U = \bigcup_{i=1}^n D(f_i)$. As the left hand side is a proposition we have
	\[
	U \leftrightarrow \bigvee_{i=1}^n D(f_i)
	\]
	so we may show, that each $D(f_i) \subset \Spec C_r$ is an open proposition. \\
	Let $f : C_r = R[X] / (X^2 - r)$ such that $D(f)$ is a proposition. As $C_r$ as an $R$-module is free with basis $1 , X$, we may choose a representative $a + bX : R[X]$ with $a,b : R$.
	Let us show $(2a \neq 0) \leftrightarrow D(f)$, which is a modal proposition, as open propositions are $\lnot \lnot$-stable, thus modal. By (L1) we may assume $x : \Spec C_r$.
	Using that $D(f) \subset  \Spec C_r =  \{x,-x\}$ and that $D(f)$ is a proposition we have
	\begin{align}
		D(f) = (a+bx \neq 0) + (a-bx \neq 0) \overset{\sim}{\to} (a+bx \neq 0) \lor (a-bx \neq 0)
	\end{align}
	We may show both implications $2a \neq 0 \leftrightarrow (a+bx \neq 0) \lor (a-bx \neq 0) $. \\
	$'\rightarrow'$ $(a+bx) + (a-bx)$ is invertible, so by locality one of the summands is invertible. \\
	$'\leftarrow'$ by symmetry wlog $a + bx \neq 0$. Then by the first equation of (1) and the fact that $D(f)$ is a proposition, $\lnot \lnot (a - bx = 0)$. Thus $\lnot \lnot (a + a = a + bx \neq 0)$, hence $2 a \neq 0$. 	
\end{proof}
The rest of this section is devoted for the proof of \ref{prop}.
%$X_r$ is modal, as all affines are sheaves, using L0. \\
\begin{enumerate}
	\item [$\cB \rightarrow \cC$]
	we have $p : X_r \to R$ the first projection so we may use $Y_x :\equiv \fib_p x$ and the evident equivalence $\fib_p(0) \simeq \Spec C_r$.
	There is no open affine subset of $X_r$ containing $\fib_p(0)$: Indeed as the goal is $\lnot \lnot$-stable, it is modal by L2. So we may assume $X_r$ beeing the line with two origins, using (L1). So we can conclude by \ref{lemma:noAffNbhd}.
	\item [$\cA \rightarrow \cB$]
	if $\|\Spec C_r\|$, then $X_r$ is a scheme: \\
	$\sum_{x : R} \Bool^{x = 0}$ is the line with two origins by \ref{lemma:X1}, which is known to be a scheme. So by \ref{lemma:BoolSpecC}, $X_r \equiv \sum_{x : R} (\Spec C_r)^{x=0}$ is a scheme as well. 
	\item [$\cC \rightarrow \cA$]
	Let $p : X \to R$ be a map out of a scheme that comes with an equivalence $\fib_p(0) \simeq \Spec C_r$, such that $X$ does not admit an open affine neigbhorhood of $\fib_p(0)$. We wish to show $\|\Spec C\|$. Any finite open affine cover of $X$ can be restricted to a finite open affine cover $\Spec C_r= \bigcup_{j=0}^{n} U_j$ of the basefiber $\Spec C$ consisting of strictly smaller open subsets, using the assumption that $\fib_p(0)$ does not have an open affine neighborhood. 
	Then the goal is
	\[
	\| \Spec C_r\| = \| \bigcup_{j=0}^{n} U_j \| \leftrightarrow \| \sum_{j=0}^n U_j\| = \bigvee_j  U_j
	\]
	an open propoosition by \ref{lemma:heart}, thus $\lnot \lnot$-stable, hence modal by (L2). So it is inhabited, as $L \|\Spec C_r\|$ is inhabited (L1).
\end{enumerate}
\ifsc
\section{The type of schemes is not modal}
\else
\subsection{The type of schemes is not modal}
\fi
The key ingredient to prove that $\Sch$ is not modal, is the following:
\begin{prop}{\label{prop}}
	Let $2 \neq 0$. Let $r : R^\times$. Denote 
	\begin{align*}
		C_r &= R[X] / (X^2 + r) \\
		X_r &= \sum_{x : R} (\Spec C_r)^{x = 0}
	\end{align*}
	The following types (referred as $\cA$ $\cB$ $\cC$ )are logically equivalent, i.e.  we find functions
	% https://q.uiver.app/#q=WzAsMyxbMCwxLCJcXG1hdGhybXtpc1NjaGVtZX1cXGxlZnQgKCBYX3IgXFxyaWdodCkiXSxbMCwyLCJcXHN1bV97WSA6IFIgXFx0byBcXFNjaH0gWV8wID1fe1xcU2NofSBcXFNwZWMgQ19yXFx0aW1lcyAoXFxub3QgXFxleGlzdHMgXFx0ZXh0eyBhbiBvcGVuIGFmZmluZSBuZWlnYmhvcmhvb2Qgb2YgfSBZXzAgXFwgaW4gXFwgXFxzdW1fe3ggOiBSfSBZX3gpIl0sWzAsMCwiXFx8XFxTcGVjIENfclxcfCJdLFswLDFdLFsyLDBdLFsxLDIsIiIsMCx7Im9mZnNldCI6LTUsImN1cnZlIjotNX1dXQ==
	\[\begin{tikzcd}
		{\|\Spec C_r\|} \\
		{\mathrm{isScheme}\left ( X_r \right)} \\
		{\sum_{Y : R \to \Sch} Y_0 =_{\Sch} \Spec C_r\times (\not \exists \text{ an open affine neigbhorhood of } Y_0 \ in \ \sum_{x : R} Y_x)}
		\arrow[from=1-1, to=2-1]
		\arrow[from=2-1, to=3-1]
		\arrow[shift left=5, curve={height=-30pt}, from=3-1, to=1-1]
	\end{tikzcd}\]
\end{prop}
\begin{rmk}
	If $\Sch$ is a modal type, the advantage of $\cC$ is that it is modal, even if schemes are not assumed to be modal.
\end{rmk}

\begin{corollary}
	The type of Schemes $\Sch$ is not modal
\end{corollary}
\begin{proof}
	Assume $\Sch$ is modal. Lets call $C_r = R[X] / (X^2 + r)$.  By \cite{cherubini2023foundationsyntheticalgebraicgeometry} A . 0.3. its enough to show $\|\Spec C_r\|$ for all $r : R^\times$. \\
	Let $r : R^\times$. %In other words by the prop $T^2 + r$ merely has a root for $r : R \setminus \{0\}$ arbitrary, so we get a contradiction to \\ \\
	%	
	First I give a conceptual proof in the case where every scheme is modal only needing $\cA \leftrightarrow \cB$:
	\begin{align*}
		L1 &\rightarrow L \| \Spec C_r\| \\
		&\overset{\cA \rightarrow \cB}{\rightarrow} L (X_r \in \Sch) \\
		& \overset{*}{\leftrightarrow} (X_r \in \Sch ) \\
		&\overset{\cB \rightarrow \cA}{\rightarrow} \|\Spec C_r\| \\
	\end{align*}
	where at $(*)$ we used that $(X_r \in \Sch) \simeq \sum_{X: \Sch} (X = X_r)$ is modal: because both $X$ and $X_r$ are modal, the type of equivalences $X \simeq X_r$ is modal as well, so conclude by univalence \\
	
	Now we give a proof for the general case, where morally we replace $\cB$ in the previous proof by the modal type $\cC$. 
	If $\Sch$ is a modal type, then the type $\cC$ is modal:
	\begin{itemize}
		\item modal types are stable under $\sum$
		\item function types into modal types are modal,
		\item identity types in $\Sch$ are modal, 
		%		\item $R$ is modal (L0) , 
		\item $\bot$ is modal (L2).
	\end{itemize}
	Then conclude by
	\begin{align*}
		(L1) \to L \|\Spec C_r\| \overset{\cA \to \cC}{\to} L \cC \simeq \cC \to \cA
	\end{align*}
	%	We have a map $\|\Spec C_r\|  \to \cC$ from \ref{prop}. As $\cC$ is modal we find a map $L \|\Spec C_r\| \to \cC$, By (L1) a term in $\cC$. Conclude by $\cC \rightarrow \cA$
\end{proof}