
\subsection{Descent}
$\St$ a class of stacks in $\cV$, such that $\bT$ is contained in it and for any $\bT$-cover $X \to Y$ of stacks in $\cV$, $X \in \St$ iff $Y \in \St$. We call types in this class stacky.
\begin{lemma}{\label{lemma:stackificationHasTCover}}
	Let $\bT$ satisfy descent, i.e. beeing affine in the topology is a stack. If $Y$ admits a $\bT$-cover $f : X \to Y$ where $Y$ is seperated, then there is a $\bT$-cover $X \to L_\bT Y$.
\end{lemma}
\begin{proof}
	
	Consider $X \overset{f}{\to} Y \overset{\eta}{\to} L_\bT Y$. As beeing affine in $\bT$ is  a stack, we may just show that for all $y : Y$ , the fibers over $\eta y : L_\bT Y$ are in $\bT$. As $\eta$ is a monomorphism by \ref{lemma:sep} , $\eta$ restricts to an equivalence
	\[
	\fib_f y \to \fib_{ \eta f}(\eta y)
	\]
	
	But the left hand side is in $\bT$ by assumption. 
\end{proof}
\begin{lemma}
	Assume $\bT$ have descent.
	Let $X \in \St$ and $Y$ a type.	Let $f : X \twoheadrightarrow Y$ be fibered in $\bT$ and surjective. Then $L_\bT Y$ is stacky.
\end{lemma}
\begin{proof}
	As $X$ is stacky, it suffices to show, that $L_\bT Y$ admits a $\bT$-cover.
	We want to apply \ref{lemma:stackificationHasTCover}. So it remains to show, that $Y$ is seperated. By surjectivity of $f$ we may only show that for any $x : X, y : Y$, the type $f x =_Y y$ is a stack. If we define $U$ to be the fiber over $y$, it is in $\bT$ by assumption. But then $f x =_Y y$ is the outer pullback
	% https://q.uiver.app/#q=WzAsNixbMCwwLCJmIHggPSB5Il0sWzEsMCwiVSBcXGluXFxiVCJdLFswLDEsIjEiXSxbMSwxLCJYIl0sWzIsMCwiMSJdLFsyLDEsIlkiXSxbMyw1LCJmIl0sWzQsNSwieSIsMl0sWzIsMywieCJdLFsxLDNdLFsxLDRdLFsxLDUsIiIsMSx7InN0eWxlIjp7Im5hbWUiOiJjb3JuZXItaW52ZXJzZSJ9fV0sWzAsMl0sWzAsMV1d
	\[\begin{tikzcd}
		{f x = y} & {U} & 1 \\
		\arrow["\ulcorner"{anchor=center, pos=0.125}, draw=none, from=1-1, to=2-2]
		1 & X & Y
		\arrow[from=1-1, to=1-2]
		\arrow[from=1-1, to=2-1]
		\arrow[from=1-2, to=1-3]
		\arrow[from=1-2, to=2-2]
		\arrow["\ulcorner"{anchor=center, pos=0.125}, draw=none, from=1-2, to=2-3]
		\arrow["y"', from=1-3, to=2-3]
		\arrow["x", from=2-1, to=2-2]
		\arrow["f", from=2-2, to=2-3]
	\end{tikzcd}\]
	of stacky types, in particular stacks. \qed(Claim) \\\\
	
\end{proof}
\begin{theorem}
	Assume $\bT$ have descent. Then $\St$ is a stack.
\end{theorem}
\begin{proof}
	$\St$ is seperated: This follows from the embedding $\St$ into the seperated type of stacks \ref{lemma:stacksHaveDescent}. \\
	Let $U \in \bT$ and $P : \|U\| \to \St$. We want to construct a filler 
	% https://q.uiver.app/#q=WzAsMyxbMCwwLCJcXHwgVVxcfCJdLFswLDEsIjEiXSxbMSwwLCJcXFNtU3QiXSxbMCwyLCJQIl0sWzAsMV0sWzEsMiwiIiwyLHsic3R5bGUiOnsiYm9keSI6eyJuYW1lIjoiZGFzaGVkIn19fV1d
	\[\begin{tikzcd}
		{\| U\|} & \St \\
		1
		\arrow["P", from=1-1, to=1-2]
		\arrow[from=1-1, to=2-1]
		\arrow[dashed, from=2-1, to=1-2]
	\end{tikzcd}\]
	Claim: $L_\bT (\sum_{x: \|U\|} P x)$ is stacky.
	\begin{proof} of the claim. We want to apply the previous lemma to the surjection 
		\[\sum_{x : U} P | x | \to \sum_{x : \| U\|} P x \]
		The domain is in $\St$ by stability under $\sum$. The fibers are equivalent to $U \in \bT \subset \St$.				
	\end{proof}
	The claim provides the map $1 \to \St$. The diagram commutes: Assuming $x : \|\Spec A\|$ we wish to show $P x = \sum_{x: \|U\|} P x$. Using univalence, we may show that the maps 
	\[P x \to \sum_{x: \|U\|} P x \overset{\eta}{\to} L_\bT \sum_{x: \|U\|} P x\]
	are both equivalences.
	The first one is an equivalence as $\|U\|$ is contractible. Hence the middle term is a stack, thus the unit map is an equivalence as well. \\
	
	
	
\end{proof}
\begin{corollary}
	If $\bT$ has descent, (covering) geometric stacks satisfy descent.
\end{corollary}

\begin{corollary}
	If $\bT$ has descent. For all $n : \bN $, the class of (covering) ($n$-)stacks has descent.
\end{corollary}
\begin{proof}
	The class of (covering) geometric $n$-stacks is the intersection of (covering) geometric stacks and $n$-truncated stacks. Both have descent.
	%If $n-stack \ni X \to Y$ is fibered in $\bT$ where $Y$ is an $n$-type. Then its stackification is an $n$-type as well by the lemma.
\end{proof}

