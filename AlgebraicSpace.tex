



\subsection{Algebraic spaces}


%Let us also mention what we learned in the proof:
%\begin{lemma}[NECESSARY?@²]
%	A covering equivalence relation $R : S^2 \to \bT \Prop$ has values in geometric propositions.
%\end{lemma}

%\begin{corollary}
%	The identity types of algebraic spaces are geometric propositions.
%\end{corollary}
%\begin{proof}
%	By the previous lemma and \ref{lemma:geometricStacksClosedUnderId}
%\end{proof}
%
%\begin{lemma}{\label{lemma:detectGeomProp}}
%	Let $P$ be a sheaf and a proposition that admits a map $\Spec A \to P$ fibered in covering algebraic spaces. Then $P$ is a geometric proposition.
%\end{lemma}
%\begin{proof}
%	The fibers are covering algebraic spaces and affine, hence covering affine. By \ref{def:algprop} we conclude.
%\end{proof}
\begin{theorem}
	Let $X$ be a modal set. The following are equivalent:
	\begin{enumerate}
		\item $X$ is a (covering) geometric 0-stack
		\item $X$ is merely of the form $L_\bT (U / R)$ for some (covering) affine $U$ and  $R : U^2 \to \Prop_{\ci}$ a covering equivalence relation. 
		\item there exists some map $S \to X$ with $S$ (covering) affine whose fibers merely have $\bT$-catlasses.
	\end{enumerate}
	We call this class (covering) algebraic spaces.
\end{theorem}
\begin{proof}
\ 	\begin{enumerate}
	\item [2 $\leftrightarrow$ 3]
		This is \ref {lemma:fundamental-property-algebraic-spaces}
	\item [2 $\to$ 1]
	Choose a presentation $ R: U^2 \to \Prop$.
	It suffices to show, that the map $f : U \to L_\bT ( U / R)$ is a geometric (c)atlas. The map $f$ is $\bT$-surjective by the well-definedness of the bijection $\ref{quotient-by-equivalence-relation}$. By descent we may just show, that the fibers $\fib_f (f(s))$ for $s : U$ are covering 0-stacks. But by the bijection in \ref{quotient-by-equivalence-relation} those are equivalent to the fibers $R_s$, which are covering 0-stacks as the equivalence relation is covering. \\
	\item [1 $\to$ 2]
	This can be reformulated in the following way, using the recursion principle for (covering) geometric 0-stacks:
	Let $X$ be a sheaf of sets. Let $S$ be (covering-) affine and $f : S \to X$ be fibered in covering algebraic spaces. Then $X$ is a (covering) algebraic space.
%	The identity types of $X$ admit a map fibered in covering algebraic spaces (todo check stability under $\sum$) out of an affine by \ref{lemma:havingAbstractAtlasClosedUnderId}. by \ref{lemma:detectGeomProp} they are geometric propositions. 
This follows from the observation, that the equivalence relation determined by $f$ is covering \ref{def:coveringEqRel} , because the fibers of $f$ are covering 0-stacks.
	\end{enumerate}
\end{proof}
\begin{prop}
	For any $n \ge 1$, we have inclusions 
	\[W_{n} \subset \CS_{n-1} \subset W_{n+1}\]
\end{prop}
\begin{proof}
	Induction. $n = 1$ gives
	\[
	\mathsf{HasCatlas}_\bT \subset \CS_0 \subset \text{ types admitting a catlas fibered in } W_1
	\]
	the latter inclusion is the previous theorem. \\
	The induction step is obtained by \ref{prop:nstack}
\end{proof}


\subsection{Schemes are algebraic Spaces for the Zariski Topology}
\begin{definition}
 A proposition $U$ is open iff its merely of the form $f_1 \ inv \lor \hdots f_n inv$ for some $f_i : R$.
\end{definition}

\begin{lemma}
	Given $f_1, \hdots,f_n : R$ such that $\| D(f_1) + \hdots + D(f_n) \|$ then $\sum_{i=1}^n D(f_i) \in \Zar$.
\end{lemma}
\begin{prop}
	Every Zariski-merely-inhabited type that is merely of the form $U_1 + \hdots + U_n$ for open propositions $U_i$ admits a $\Zar$-catlas.
\end{prop}
\begin{proof}
	By definition of openness, We can choose a surjection $\coprod_{j=1}^{n_i} D(f_{ij}) \twoheadrightarrow U_i$ for any $i$. We want to show, that the map
	\[
	\coprod_{i , j} D(f_{ij}) \twoheadrightarrow U_1 + \hdots U_n
	\]
	is a $\Zar$-catlas. 
	\begin{itemize}
		\item Let us first show that the fibers are in $\Zar$. Assume $U_i$ holds. So we find a term in $\coprod_j D(f_{ij})$. In particular we have $\| \coprod_j D(f_{ij})\|_{\Zar}$. By the lemma we conclude, that the fiber $\sum_j D(f_{ij})$ belongs to $\Zar$.\\
		\item The total space is in $\Zar$: This follows as the surjection after propositional truncation becomes an equivalence. As we have $\| U_1 + \hdots + U_n\|$, we can conclude by the lemma.
	\end{itemize}
	
\end{proof}
\begin{warning}
	The converse does not hold! We want to apply \ref{lemma:stackificationHasTCover}, to the map
	\[\Zar \ni 1 + 1 \to \sum D(f) \]
	\begin{itemize}
		\item 	$\sum D(f)$ is seperated as $D(f)$ is a sheaf.
		\item 	All the fibers are equivalent to $1 + X$, hence they are in the Zariski topology.
	\end{itemize}	
\end{warning}
\begin{lemma}
	let $X$ be a scheme. There merely exists some affine $S$  map $S \to X$ whose fibers are merely inhabited finite sums of open propositions 
\end{lemma}

\begin{corollary}		
		Every scheme is an algebraic space.
\end{corollary}


\begin{lemma}
	If $X$ is an algebraic space, then the global sections embed via a $R$-algebra homomorphisms into a finitely presented $R$-algebra.
\end{lemma}
\begin{proof}
	Choose an atlas $S \to X$, in particular $\bT$-surjective. As $\bT$ is subcanonical the map $R^X \to R^S$ is an injection.
\end{proof}
\begin{question}
	Is it an open embedding of types?
\end{question}
%\begin{lemma}
%	Consider a morphism $X \to Y$ between algebraic spaces such that $R^F$ is finitely presented and the affinizations of the fibers $F \to \Spec R^F$ are open immersions. If $Y$ is a scheme, then $X$ is a scheme.
%\end{lemma}
%\begin{proof}
%	Consider the stein factorization
%	\[
%	X = \sum_{y : Y} \fib_f y \to \sum_{y : Y} \Spec R^{\fib_f y} \to Y
%	\]
%	The first map 
%\end{proof}
\section{Examples}
The goal of this subsection is to construct algebraic spaces. The first example actually gives us a scheme:
\begin{example}
	Let $p \neq 0$ be a prime. You can let $\mu_p := \Spec(R[X] / (X^p - 1))$ act on $\bA^\times$ via multiplication. Set $\bT= fppf$. Then the $p$.th power map
	
	\[
	pow : \| \bA^\times // \mu_p \|_0^\bT \to \bA^\times
	\]
	is an equivalence.
	\begin{itemize}
		\item 	 It is an embedding: 
		First note, that $\|\bA^\times // \mu_p \|_0$ is $\bT$-seperated:
		
		as $\mu_p$ act freely on $\bA^\times$, $\bA^\times // \mu_p$ is already a set. Meaning that the identity types of the set-quotient are $\sum_{g: \mu_p} g x =_{\bA^\times} y$ , hence sheaves. \\
		On the other hand the map $\|\bA^\times // \mu_p \|_0 \to \bA^\times$ is an embedding, as for any $x , y : \bA^\times$ the map $(\sum_{g : \mu_p} g x = y) \to (x^p = y^p)$ is an equivalence. 
		\item 	It is $\bT$-surjective, as for any $\lambda : \bA^\times$, we find $S = \Spec R [X] / (X^p - \lambda) \in \bT$ with 
		\[
		S \to \fib_{\bA^\times / \mu_p \to \bA^\times}(\lambda)
		\]
		hence 
		\[
		1 = \|S\|_\bT \to \|\fib_{pow}\|_0^\bT
		\]
	\end{itemize}
	
	
\end{example}
\begin{example}[TODO]
	The sheaf quotient of $\bA^1$ by the $\mu_\ell$ action is probably not an algebraic space.
\end{example}
\subsection{Non seperated Examples}
\begin{lemma}{\label{lemma:AlmostEverywhere}}
	Let $p : A$ be reguar. If $f : \Spec A \to R$ such that $f(x) = 0$ for all $x \in D(p)$, then $f(x) = 0$ for all $x : \Spec A$.
\end{lemma}
\begin{proof}
	$f$ is in the kernel of the diagonal map
% https://q.uiver.app/#q=WzAsNCxbMSwwLCJSXlIiXSxbMSwxLCJSXntSIFxcc2V0bWludXMgXFx7MFxcfX0iXSxbMCwwLCJSW1hdIl0sWzAsMSwiUltYXntcXHBtIDF9XSJdLFswLDFdLFsyLDMsIiIsMCx7InN0eWxlIjp7InRhaWwiOnsibmFtZSI6Imhvb2siLCJzaWRlIjoidG9wIn19fV0sWzAsMiwiIiwxLHsibGV2ZWwiOjIsInN0eWxlIjp7ImhlYWQiOnsibmFtZSI6Im5vbmUifX19XSxbMSwzLCIiLDEseyJsZXZlbCI6Miwic3R5bGUiOnsiaGVhZCI6eyJuYW1lIjoibm9uZSJ9fX1dXQ==
\[\begin{tikzcd}
	{A} & {R^{\Spec A}} \\
	{A_p} & {R^{D(p)}}
	\arrow[hook, from=1-1, to=2-1]
	\arrow[Rightarrow, no head, from=1-2, to=1-1]
	\arrow[from=1-2, to=2-2]
	\arrow[Rightarrow, no head, from=2-2, to=2-1]
\end{tikzcd}\]
	which is injective, as $p$ is regular in $A$. \\
	Thus $f = 0$ in $A$.
%	 Thus $f = 0 $ in $(R \setminus \{0\} \to R) = R[X^{\pm 1}]$ hence $f \cdot X^n = 0$ in $R[X]$ for some $n$, thus $f = 0$.
\end{proof}
\begin{question}
	What has this todo with separatedness?
\end{question}
Let $\ell \neq 0$ denote a prime. Consider $\mu_\ell = R[X] / (X^\ell - 1)$.
%\[\mu_\ell'  = \Spec R[X] / (X^{p-1} + \hdots + 1) = \mu_\ell \setminus\{1\}\]

%\begin{lemma}
%	Let $0 \in D(p)$. For a function $\phi : D(p) \to R$ TFAE % We say some $\phi: R[X]_p$ is $\ell$-even if one of the following equivalent conditions is satisfied:
%	\begin{enumerate}
%
%	\item The function $\phi : D(p) \to R$ is a $\mu_{\ell}$-invariant function, i.e. $\phi(x) = \phi(g x)$ for all $ x : D(p)$ and each $g : \mu_\ell$.
%	\item The function $\phi|_{D(p) \setminus 0} : D(p) \setminus \{0\} \to R$ is an $\mu_{\ell}$-invariant function
%%	\item[1']  There exists $f : R[X], n : \bN$ such that $\phi= f/ p^n$ and $f (g.p)^n (x) = (g.f) p^n (x)$ for each $x : R , g : \mu_\ell$.
%%	\item[2'] The same as 3. but only for $x : D(p) \setminus\{0\}$
%\end{enumerate}
%\end{lemma}
%\begin{proof}
%	 we can apply  \ref{lemma:AlmostEverywhere}, observing $\phi - g.\phi = 0$ on $D(X / 1) \subset \Spec R[X]_p$, where $X/1 : R[X]_p$ is regular, because $X$ is regular in $R[X]$. %$: 
%\end{proof}
\begin{prop}
	Let $G$ be a formally \etale flat affine group, such that $\lnot \lnot$ its finite with cardinality invertible in $R$. Let it act on an affine scheme $\Spec A$ with a fixpoint 0, such that the group action is free away from 0.
	Let $p : A^G$. Define $R: D(p)^2 \to \Prop$ as
	\[
	R(x,y) = (x = y) + (x \neq 0) \times \sum_{g : G \setminus \{1\}} g x = y
	\]
	Then the sheaf quotient $D(p) / G$ is an algebraic space.
\end{prop}
\begin{proof}
%	Define
%	\[
%	E(x,y) = (x = y) + (x \neq 0 \land \sum_{g : \mu_\ell \setminus \{1\}} gx = y)
%	\]
	This is a proposition: First note, that both summands are propositions because $G$ acts freely on $\bA^1 \setminus \{0\}$. If both summands are inhabited we get a contradiction, as $x = y$ and $gx = y$ implies $(g-1) x = y - x = 0$, but as $g-1$ is invertible $x = 0$. \\
	The relation is covering: 
	The propositions are affines, thus sheaves. Furthermore, for any $y : D(p)$ we have
	\[
	\sum_{x : D(p)}  (x = y) + (x \neq 0 \times \sum_{g : G \setminus \{1\}} gx = y) = 1 + (y \neq 0 \times G \setminus \{1\}) \in \bT
	\]
	as $G \setminus \{1\} = \sum_{g : G} g \neq 1$ is a $\sum$ of formally \etale + flat affines (recall that formally \etale affines have decidable equality).
	
\end{proof}
%\begin{prop}
%	Let $p : R[X]^{\mu_\ell}$ with $0 \in D(p)$ % be such that $0 \in D(p)$ and $x \in D(p)$ implies $gx \in D(p)$ for all $g : \mu_\ell$.
%	The sheaf-quotient of $D(p)$ by the relation which identifies $x$ and $gx$ when $x \neq 0, g : \mu_\ell \setminus \{1\}$ is not an affine scheme.
%\end{prop}
%\begin{proof}
%
%\end{proof}
\begin{lemma}
	Let $G$ be a finite group whose cardinality is invertible in $R$. Let $G$ act on an affine scheme equipped with a fixpoint $0$. Let $V$ be an invariant open neighborhood of 0. Then we find an invariant principal open neigbhorhood contained in $V$. Invariant means here that $g(U) = U$ for all $g : G$.%Let $A$ be a finitely presented algebra and $0 \in \Spec A$ a basepoint. 
\end{lemma}
\begin{proof}
	Let $U$ be an invariant open neighborhood. Choose a principal open neighborhood $0 \in D(p) \subset U$. $G$ acts on $R[X]$, via $(g.p)(x) = p(g x)$.Then 
	\[p' = \sum_{g : G} g . p : R[X]\]
	is a $G$-invariant polynomial, in particular $D(p)$ is $G$-invariant. Moreover $0 \in D(p')$ as
	\[
	p'(0) = \sum_{g : G} p(g(0)) = \sum_{g : G} p (0) = |G| \cdot p(0) 
	\]
	is invertible, as $|G|$ and $p(0)$ are both invertible. Furthermore, as $U$ was $G$ invariant and contained $D(p)$ it also has to contain $D(p')$.
\end{proof}
%\begin{prop}
%	Let $X$ be an algebraic space. Let $R : X^2 \to \Prop$ be a covering equivalence relation, such that $\sum_{x ,y :X} R(x,y) \to X^2$ is not a closed embedding.Then the sheaf quotient $X / R $ is not a scheme.
%\end{prop}
%\begin{proof}
%	Affine schemes are seperated.
%\end{proof}
\begin{lemma}
	Let $(\Spec B, 0)$ be a pointed affine scheme such that $R^{\Spec B} \to R^{\Spec B \setminus \{0\}}$ is injective. %that admits some regular $p : B$ with $p(0) = 0$ %, i.e. $\lnot \prod_x \lnot \lnot x = 0$. 
	Then the infinitesimal neigbhorhood of $0$ is not an open subtype.
\end{lemma}
\begin{proof}
	If it would, it would be principal open $D(g)$, as 0 admits a principal open neigbhorhood, which however already cotains the whole infinitesimal one. \\
	Then for any $x \neq 0$, we have $\lnot \lnot g(x) = 0$. As $\Spec B \setminus \{0\}$ is a scheme, it admits a boundedness principle, thus we find some $n$, such that $g^n (x) = 0$ for all $x \neq 0$. \\ %, in particular for all $x : D(p)$. By \ref{lemma:AlmostEverywhere} applied to the givenregular $p$,  
	By assumption we deduce $g^n =0$, hence $D(g) = D(g^n) = \varnothing$ contradiction.
%	 Let $I$ correspond to the closed subtype $\{0\} \subset \Spec B$.
%	 Let us show that $\sum_{x:\Spec B} \lnot \lnot x =0$ is not affine. We have by topology \todocite
%	\[
%	\left(\sum_{x:\Spec B} \lnot \lnot x =0 \right)= \bigcup_n \Spec B / I^n 
%	\]
%	the global sections of the right hand side are 
%	\[
%	\bigcap_n B /  I^n = B 
%	\]
%	but $\Spec B$ is not an infinitesimal variety.
\end{proof}
\begin{prop}
%	Let $0 : \Spec A$.
%	Let $Q : \Spec A \to \Prop$ be a predicate such that for all $p : A$, $0 \in D(p)$
%	\[\lnot \prod_{\substack{x : D(p)}} Q x.\]
%	Let $R$ be a covering equivalence relation on $\Spec A$ such that for all $x , y : \Spec A$
%	\[
%	R(x,y) \text { closed proposition} \to Q x
%	\]
	Let $G$ be a formally \etale + flat affine non-trivial group, such that $\lnot \lnot$ its finite, with cardinality invertible in $R$. 
	Let it act on an affine scheme $\Spec A$ with a fixpoint 0, such that  $R^{\Spec A} \to R^{\Spec A \setminus \{0\}}$ is injective and such that the group action is free away from 0. \\
Define $R: (\Spec A)^2\to \Prop$ as
\[
R_G(x,y) = (x = y) + (x \neq 0) \times \sum_{g : G \setminus \{1\}} g x = y
\]
	Then $(\Spec A) / R$ is an algebraic space that is not a scheme.
\end{prop}
\begin{proof}
	It is an algebraic space by putting $p \equiv 1 : R[X]$ in the previous prop. \\
	Assume the quotient is a scheme. 
	The preimage along the quotient map obtained from the relation induces a open neigbhorhood $V$ of $0$ in $\bA^1$. As we want to prove a contradiction we may assume that $\mu_\ell$ consists of $\ell$ many elements, where $\ell \neq 0 $ in $R$. We apply the previous lemma to $V$ to obtain an invariant principal open neigborhood $0 \in D(p) \subset V \subset \bA^1$. As its invariant, $p : \bA^1 \to R$ descends to $X \to R$. Restricting to $V$ yields a map $p' : V \to R$, such that setting $U \equiv D(p')$ yields such that $q^{-1}(V) =q^{-1}(D(p')) = D(p)$ . We are now in the following situation
	% https://q.uiver.app/#q=WzAsNCxbMCwwLCJEKHApIl0sWzEsMCwiXFxiQcK5Il0sWzEsMSwiWCJdLFswLDEsIlUiXSxbMCwxLCIiLDAseyJzdHlsZSI6eyJ0YWlsIjp7Im5hbWUiOiJob29rIiwic2lkZSI6InRvcCJ9fX1dLFswLDNdLFszLDIsIiIsMix7InN0eWxlIjp7InRhaWwiOnsibmFtZSI6Imhvb2siLCJzaWRlIjoidG9wIn19fV0sWzEsMl0sWzAsMiwiIiwxLHsic3R5bGUiOnsibmFtZSI6ImNvcm5lci1pbnZlcnNlIn19XV0=
	\[\begin{tikzcd}
		{D(p)} & {\bA^1} \\
		U & X
		\arrow[hook, from=1-1, to=1-2]
		\arrow[from=1-1, to=2-1]
		\arrow["\ulcorner"{anchor=center, pos=0.125}, draw=none, from=1-1, to=2-2]
		\arrow[from=1-2, to=2-2]
		\arrow[hook, from=2-1, to=2-2]
	\end{tikzcd}\]
	where $U$ is an open affine neighborhood of 0. \\
	Then we have $D(p) / \sim' \simeq U$ with restricted equivalence relation. \\
	Let us show that $D(p) / G = U$ is not seperated ( because then it cannot be affine). Assume that for each $x , y : D(p)$, $R(x,y)$ is a closed proposition. As we want to prove a contradiction, we may assume $g : G \setminus \{1\}$, using that $G$ is non trivial.
	is a closed embedding. Then 
	\[
	x\neq 0 \times \{g\} \hookrightarrow x \neq 0 \times \sum_{g \neq 1} g x = g x \hookrightarrow  R(x,gx) 
	\]
	is a composition of closed embeddings using that $G \setminus \{1\}$ is seperated, hence $x \neq 0$ is a closed proposition. Thus $\sum_{x: D(p)} \lnot \lnot x = 0$ is an open subtype of $D(p)$, thus of $\Spec A$ .\\
	Alternative approach: Through decidable subtypes.
%	But its also an open proposition, hence decidable. 
%	thus affine. Hence the right hand side \[
%	\sum_{x: D(p)} \lnot \lnot x = 0  = \sum_{x : \Spec A} \lnot \lnot x = 0\]
	 %is affine. 
	 Contradiction to the previous lemma. %, as $\Spec A$ is non-infinitesimal.
\end{proof}
\begin{example}[Not separated examples]
	Assume $\ell \neq 0$. Let $\mu_\ell$ act on $\Spec B$ in one of the following ways:
	\begin{enumerate}
		\item Let $\mu_\ell$ act on $\Spec B = \bA^1$. 
		\item Put $\ell= 2$. Let $\mu_2$ act on
		\[
		\Spec B \equiv \sum_{x , y : R} x y = 0
		\]
		via the swap.
	\end{enumerate}
	Then $\Spec B / R_{\mu_\ell}$ is an algebraic space that is not a scheme.
\end{example}
\begin{proof}
	\begin{enumerate}
		\item Free away from the origin. By \ref{lemma:AlmostEverywhere} %The infinitesimal neighborhood of 0 is not affine.
		\item  Again free away from the origin. Write $\Spec B = V(X) \cup V(Y)$, both components admit the property \ref{lemma:AlmostEverywhere}.
%		Moreover if $z : \bA^2$ is $\lnot \lnot z =0$, then $z \in \Spec B$.
	\end{enumerate}
\end{proof}
\subsection{Outdated}
\begin{prop}[Not needed]
	Let $\ell \neq 0$ be prime. Let $\mu_\ell$ act on $\Spec B$ . Let $V$ be an infinitesimal neigborhood of $0$, i.e. a subtype   $0 \in V \subset \Spec B$ such that $\lnot \lnot x =0$ for every $x : V$. Assume
	\begin{enumerate}
		\item[Strong freeness]  We find some $0 \in V' \subsetneq V$ for any $\varepsilon : \Spec B, g\neq 1$, $g \varepsilon = \varepsilon$ implies $\varepsilon \in V'$

	\item[checking away from 0] For any $p : B$ and any $\phi: R^{D(p)} $ such that $\phi|_{D(p) \setminus\{0\}} = 0$, we have that $\phi|_V = 0$.
		\end{enumerate} 
	The sheaf quotient of $\Spec B$ by the relation as above is an algebraic space but not an affine scheme.
\end{prop}
\begin{proof}
	First note, that strong freeness implies that $\mu_\ell$ acts free away from zero: Indeed TODO. let $x \neq 0$
	Let $p : B$ be as above. We have to show that the quotient of $D(p)$ is not affine. \\
	The conditions on $p$ give $p(0) \neq 0$ and $p(x) \neq 0 \to p(gx) \neq 0$ for all $g : \mu_\ell$.
	
	Lets call this quotient $X$.
	
	Define 
	\[
	A = \{\phi : R^{D(p)} \ | \ \phi|_{D(p) \setminus \{0\}} \text{  is $\mu_{\ell}$-invariant }\}
	\]
	This is an $R$-subalgebra: for any $r : R$, $r : B_p$ is $\mu_{\ell}$-invariant. $\mu_{\ell}$-invariant functions are stable under addition and multiplication . \\
	
	Claim: The affinization map of $X$ is the induced dashed map $f : X \to \Spec A$ in
	
	% https://q.uiver.app/#q=WzAsNCxbMCwwLCJEKHApIl0sWzEsMCwiXFxTcGVjIFJbWF1fcCJdLFswLDEsIlgiXSxbMSwxLCJcXFNwZWMgQSJdLFswLDIsInEiXSxbMiwzLCJcXGV4aXN0ISBmIiwwLHsic3R5bGUiOnsiYm9keSI6eyJuYW1lIjoiZGFzaGVkIn19fV0sWzEsM10sWzAsMSwiIiwyLHsibGV2ZWwiOjIsInN0eWxlIjp7ImhlYWQiOnsibmFtZSI6Im5vbmUifX19XV0=
	\[\begin{tikzcd}
		{D(p)} & {\Spec R^{D(p)}} \\
		X & {\Spec A}
		\arrow[Rightarrow, no head, from=1-1, to=1-2]
		\arrow["q", from=1-1, to=2-1]
		\arrow["q'",from=1-2, to=2-2]
		\arrow["{\exists! f}", dashed, from=2-1, to=2-2]
	\end{tikzcd}\]
	Proof: A function $\phi : D(p) \to R$ factors through $q$ iff $\phi|_{D(p) \setminus\{0\}}$ is $\mu_{\ell}$-invariant. Thus the embedding (using that $R$ is a sheaf) $R^X \hookrightarrow R^{D(p)}$ has image $A$ $\qed$(Claim). 	\\ \\ %respective $\phi : R[X]_p$ satisfies $	\phi (x) = \phi(-x)  $, i.e. (1) if $\phi$ is $\mu_{\ell}$-invariant. 
	Proof that $X$ is not an affine:	Assume that $X$ were affine. Then the map $f$ would be in particular an embedding. 
	We may assume a term $g : \mu_\ell \setminus \{1\}$: Indeed, as we want to prove a contradiction we may assume a term in $g : \Spec R[X] / (\sum_{i=0}^{\ell-1} X^i)$. But this type is equivalent to $\mu_\ell \setminus \{1\}$, using that $\sum_{i=0}^{\ell-1} X^i | X^\ell -1 $ and $\ell \neq 0$. \\
	 %$\cN_{\lnot \lnot}(0) = \{x : \bA^1 \ | \ \lnot \lnot x = 0\}$ be be a non-contractible subtype that is $\lnot \lnot$-connected. 
	The given infinitesimal neigbhorhood $V$ satisfies $ V \subset D(p)$ , using that invertibility is $\lnot \lnot$ stable.
	Then for any $\varepsilon : V$ we have
	\begin{align*}
		(q\varepsilon =_X q (g \varepsilon)) \overset{\ref{quotient-by-equivalence-relation}}{=} (\varepsilon = g\varepsilon) + (\varepsilon \neq 0 \land \sum_{h \neq 1} \varepsilon = h g \varepsilon) = (\varepsilon = g\varepsilon) = (\varepsilon \in V')
	\end{align*}
	where the last step comes from strong freeness.
	But we have 
	\[(q' \varepsilon =_{\Spec A} q' (g \varepsilon)) = \left (\prod_{\phi : A} \phi(q' \varepsilon) = \phi(q' (g \varepsilon)) \right)= \prod_{\substack{\phi : R^{D(p)} \\ \phi \in A}} \phi (\varepsilon) = \phi(g \varepsilon),\] 
	%as for any $\phi : A$ we have $\phi(\varepsilon) = \phi(g \varepsilon)$ 
	The right hand side is inhabited: For any $\phi: D(p) \to R$ such that $\psi := \phi - g . \phi$ satisfies $\psi|_{D(p) \setminus \{0\}} =0$  we have $\psi|_V =0$ by 'checking away from 0', inparticular $\psi(\varepsilon) =0$ . 	So we conclude the the embedding $V' \hookrightarrow V$ is an equivalence.  But we asked $V' \subsetneq V$ to be a proper subset.% I claim that the right hand side is inhabited:

\end{proof}

\begin{example}
	Let $\mu_\ell$ act on $\Spec B = \bA^1$.
\end{example}
\begin{proof}
	\begin{enumerate}
		\item Put $ V:\equiv\Spec R[X] / X^n$ for some $n >1$.
		\item As $(g-1)$ is invertible, $ ((g-1)\varepsilon = 0) $  gives us $\varepsilon \in \{0\} \equiv V' \subsetneq V$. Note that indeed $V$ is non contractible, because $R[X] / X^n \to R[X] / X$ is not an algebra isomorphism
		\item 	We have to show, that then $\phi$ is $\mu_{\ell}$ invariant. We can apply  \ref{lemma:AlmostEverywhere}, observing $\phi - g.\phi = 0$ on $D(X / 1) \subset \Spec B_p$, where $X/1 : B_p$ is regular, because $X$ is regular in $B$. TODO as each $\phi$ satisfies the cond. \qed(Claim)\\
	
	\end{enumerate}
\end{proof}
\begin{example}
	Assume $2 \neq 0$. Let $\mu_2$ act on
	\[
	\Spec B \equiv \sum_{x , y : R} x y = 0
	\]
	via the swap. Then $\Spec B / R$ is an algebraic space that is not a scheme.
%	sheaf-quotiented by the relation that identifies $(x,0)$ and $(0,x)$ if $x \neq 0$ is an algebraic space.
\end{example}
\begin{proof}
	\begin{enumerate}
		\item Put $V = \Spec R[X] / X^k \subset \Spec B$, $k > 2$.
		\item If $(x,y) = (y,x)$ but $x y = 0$ we get $x \in V' \equiv \Spec R[X] / X^2$.
		\item Let $\phi: D(p) \to R$ be 0 everywhere except near the origin. Then we get a restricted map $\phi' : D(p') \to R$ where $D(p') \subset V(X)$ is given by the intersection $D(p) \cap V(X) $ . Indeed : Put $p' : R[X]$ the image of $p : R[X,Y] / (XY)$ und the map induced by evaluating $Y$ at 0. \\
		Here we can apply \ref{lemma:AlmostEverywhere}, getting that $\phi'$ is 0 everywhere in particular in $V \subset V(X)$.
	\end{enumerate}
%	The equivalence relation is given by
%	\[
%	E((x,y) , (x',y')) = (x = x' \land y = y') + (x \neq 0 \land x = y' \land x' = 0)
%	\]
%	as $x \neq 0$ implies $y = 0$ as $x y =0$. This is a covering relation, as for any $x' y' = 0$ we have
%	\[
%	\sum_{x,y : R} x y = 0 \land E((x,y) , (x' , y')) = 1 + (y' \neq 0) \in \Zar \subset \bT
%	\]		
\end{proof}
\begin{lemma}
	Given a map $P : \Susp(Q) \to \Prop$, such that $P(N)$ and $P(S)$ hold, then $\prod_{t: \Susp(Q)} P(t)$
\end{lemma}
\begin{example}
	Let $L = \sum_{x : \bA^1} \Susp(x \neq 0)$ be the line with two origins. By the evident $\mu_2$ action on any suspension we obtain a $\mu_2$ action on $X$, such that the first projection $L \to \bA^1$ is $\mu_2$-invariant. Then the map $L / \mu_2 \to \bA^1$ is an equivalence
\end{example}
\begin{proof}
	By $\sum$-stability of algebraic spaces we may show, that the fiber of $L / \mu_2 \to \bA^1$ over $x : \bA^1$ is contractible. The fiber is $\Susp(x \neq 0) / \mu_2$ as 
	\[
	X / \mu_2 = \sum_{x : \bA^1} \Susp(x \neq 0)/\mu_2
	\]
	 But for $X$ a proposition the sheaf quotient $\Susp(X) / \mu_2$ is contractible.	
\end{proof}

\begin{example}[TODO]
	Let $L = \sum_{x : \bA^1} \Susp(x \neq 0)$ be the line with two origins. By the evident $\mu_2$ action on any suspension we obtain a $\mu_2$ action on $X$, such that the first projection $L \to \bA^1$ is $\mu_2$-invariant. Then the homotopy quotient $L // \mu_2$ is an algebraic space.	
\end{example}
\begin{proof}
	By $\sum$-stability of algebraic spaces we may show, that the fiber $\Susp(x \neq 0) // \mu_2$ of $L //´ \mu_2 \to \bA^1$ over $x : \bA^1$ is an algebraic space. As $\Susp(x \neq 0)$ is a scheme \todocite, it is an algebraic space, so it remains to show, that the equivalence relation is covering. \\
	Let us show that all the homotopy orbits of the $\mu_2$-action of some $s : \Susp(x \neq 0)$ are covering. Since this is a proposition depedent on $s : \Susp(x \neq 0)$,by the previous lemma we only have to consider orbits of points $N : \Susp(x\neq0)$ and $S : \Susp(x \neq 0)$ . Lets stick to $N$. The homotopy orbit is equivalent to 
	\[
	\sum_{t : \Susp(x \neq 0)}  (t = N) + (t = S) % \simeq \mathrm{im}(2 \overset{(N,S)}{\to} \Susp(x \neq 0)) = \Susp(x \neq 0)
	\]
	As $\Susp(x \neq 0)$ is a covering algebraic space, by $\sum$-stability we may show, that for any $t$, $(t = N) + (t = S)$ is covering. Again by the lemma, we may check this for $t \equiv N$ and $t \equiv S$. In either case the type is equivalent to $1 + x \neq 0 \in \Zar \subset \bT$. So we can conclude.
\end{proof}


\section{Group quotients}
For this section let $G$ denote a group that is a covering 0-stack. Let $X$ be a sheaf equipped with a $G$ action.
\begin{lemma}
	 $\mu_p = \Spec R[X] / (X^p - 1)$ is covering for $p \neq 0$ prime.
\end{lemma}
\begin{proof}
	It is fppf + \etale as $X^p - 1$ is monic seperable. TODO
\end{proof}
\begin{definition}
	A $G$ action on $X$ is free, if for all $x , y : X$ the type 
	\[
	\sum_{g: G} g x = y
	\]
	is a proposition. 
\end{definition}
\begin{lemma}
	Let $G$ act freely on a sheaf $X$. Then the relation
	\[
	x , y\mapsto \sum_{g : G} g x = y
	\]
	is a covering equivalence relation on $X$
\end{lemma}
\begin{proof}
	All those propositions are modal as $X$ and $G$ are sheaves. For all $x : X$ , the fiber
	\[
	\sum_{y : X} \sum_{g : G} g x = y \simeq \sum_{g : G} \sum_{y: X} g x = y \simeq G
	\]
	is a covering 0-stack by assumption.
\end{proof}
\begin{lemma}{\label{lemma:algSpacesStabFreeQuots}}
	%For $n \ge 0$, geometric $(n)$-stacks 
	Algebraic spaces are stable by free quotients of covering group 0-stacks.
\end{lemma}
\begin{proof}
	The map $ X \to L_T (X / G)$ is fibered in covering 0-stacks, so in particular covering $0$-stacks. As $X$ is a geometric $0$-stack, the quotient is a geometric $0$-stack as well, This follows by the description in \label{prop:nstack}, choosing a geometric atlas of $X$ and postcomposing this to get a geometric atlas of the quotient.
\end{proof}

%\begin{lemma}
%	Let $X$ be a geometric stack, whose identity types are covering stacks. Let $G$ be a finite group acting on $X$. Then $L_\bT (X / G)$ is a geometric stack.
%\end{lemma}
%\begin{proof}
%	Consider for $x , y : X$ , $R(x,y) \equiv \| \sum_{g : G} g x = y\|_\bT$ which is indeed modal. We have to check that the relation is covering, i.e. that for all $x : X$, 
%	\[
%	\sum_{y: X} \|\sum_{g: G} g x = y\|_\bT
%	\]
%	is a covering stack. \\
%	To prove this, as covering stacks are stable under quotients, it suffices to show, that the map
%	\[
%	G \simeq \left (\sum_{y : X} \sum_{g : G} g x = y \right) \to \sum_{y: X} \|\sum_{g: G} g x = y\|_\bT
%	\]
%	is a geometric cover. But the fibers look like $\sum_{g : G} gx  = y$ which is a finite sum of identity types in $X$, which were assumed to be covering stacks. By \ref{lemma:geomStackPlusStable} the fibers are covering stacks.
%	
%\end{proof}
