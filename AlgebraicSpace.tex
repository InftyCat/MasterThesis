



\subsection{Algebraic spaces}


%Let us also mention what we learned in the proof:
%\begin{lemma}[NECESSARY?@²]
%	A covering equivalence relation $R : S^2 \to \bT \Prop$ has values in geometric propositions.
%\end{lemma}

%\begin{corollary}
%	The identity types of algebraic spaces are geometric propositions.
%\end{corollary}
%\begin{proof}
%	By the previous lemma and \ref{lemma:geometricStacksClosedUnderId}
%\end{proof}
%
%\begin{lemma}{\label{lemma:detectGeomProp}}
%	Let $P$ be a sheaf and a proposition that admits a map $\Spec A \to P$ fibered in covering algebraic spaces. Then $P$ is a geometric proposition.
%\end{lemma}
%\begin{proof}
%	The fibers are covering algebraic spaces and affine, hence covering affine. By \ref{def:algprop} we conclude.
%\end{proof}
\begin{theorem}
	Let $X$ be a modal set. The following are equivalent:
	\begin{enumerate}
		\item $X$ is a (covering) geometric 0-stack
		\item $X$ is merely of the form $L_\bT (U / R)$ for some (covering) affine $U$ and  $R : U^2 \to \Prop$ a covering equivalence relation. 
		\item there exists some map $S \to X$ with $S$ (covering) affine whose fibers merely have $\bT$-catlasses.
	\end{enumerate}
	We call this class (covering) algebraic spaces.
\end{theorem}
\begin{proof}
\ 	\begin{enumerate}
	\item [2 $\leftrightarrow$ 3]
		This is \ref {lemma:fundamental-property-algebraic-spaces}
	\item [2 $\to$ 1]
	Choose a presentation $ R: U^2 \to \Prop$.
	It suffices to show, that the map $f : U \to L_\bT ( U / R)$ is a geometric (c)atlas. The map $f$ is $\bT$-surjective by the well-definedness of the bijection $\ref{quotient-by-equivalence-relation}$. By descent we may just show, that the fibers $\fib_f (f(s))$ for $s : U$ are covering 0-stacks. But by the bijection in \ref{quotient-by-equivalence-relation} those are equivalent to the fibers $R_s$, which are covering 0-stacks as the equivalence relation is covering. \\
	\item [1 $\to$ 2]
	This can be reformulated in the following way, using the recursion principle for (covering) geometric 0-stacks:
	Let $X$ be a sheaf of sets. Let $S$ be (covering-) affine and $f : S \to X$ be fibered in covering algebraic spaces. Then $X$ is a (covering) algebraic space.
%	The identity types of $X$ admit a map fibered in covering algebraic spaces (todo check stability under $\sum$) out of an affine by \ref{lemma:havingAbstractAtlasClosedUnderId}. by \ref{lemma:detectGeomProp} they are geometric propositions. 
This follows from the observation, that the equivalence relation determined by $f$ is covering \ref{def:coveringEqRel} , because the fibers of $f$ are covering 0-stacks.
	\end{enumerate}
\end{proof}
\subsection{Schemes are algebraic Spaces for the Zariski Topology}
\begin{definition}
 A proposition $U$ is open iff its merely of the form $f_1 \ inv \lor \hdots f_n inv$ for some $f_i : R$.
\end{definition}

\begin{lemma}
	Given $f_1, \hdots,f_n : R$ such that $\| D(f_1) + \hdots + D(f_n) \|$ then $\sum_{i=1}^n D(f_i) \in \Zar$.
\end{lemma}
\begin{prop}
	Every Zariski-merely-inhabited type that is merely of the form $U_1 + \hdots + U_n$ for open propositions $U_i$ admits a $\Zar$-catlas.
\end{prop}
\begin{proof}
	By definition of openness, We can choose a surjection $\coprod_{j=1}^{n_i} D(f_{ij}) \twoheadrightarrow U_i$ for any $i$. We want to show, that the map
	\[
	\coprod_{i , j} D(f_{ij}) \twoheadrightarrow U_1 + \hdots U_n
	\]
	is a $\Zar$-catlas. 
	\begin{itemize}
		\item Let us first show that the fibers are in $\Zar$. Assume $U_i$ holds. So we find a term in $\coprod_j D(f_{ij})$. In particular we have $\| \coprod_j D(f_{ij})\|_{\Zar}$. By the lemma we conclude, that the fiber $\sum_j D(f_{ij})$ belongs to $\Zar$.\\
		\item The total space is in $\Zar$: This follows as the surjection after propositional truncation becomes an equivalence. As we have $\| U_1 + \hdots + U_n\|$, we can conclude by the lemma.
	\end{itemize}
	
\end{proof}
\begin{warning}
	The converse does not hold! We want to apply \ref{lemma:stackificationHasTCover}, to the map
	\[\Zar \ni 1 + 1 \to \sum D(f) \]
	\begin{itemize}
		\item 	$\sum D(f)$ is seperated as $D(f)$ is a sheaf.
		\item 	All the fibers are equivalent to $1 + X$, hence they are in the Zariski topology.
	\end{itemize}	
\end{warning}
\begin{lemma}
	let $X$ be a scheme. There merely exists some affine $S$  map $S \to X$ whose fibers are Zariski-merely inhabited finite sums of open propositions 
\end{lemma}

\begin{corollary}
		Every scheme is an algebraic space for the Zariski topology.
\end{corollary}
\begin{question}
	Is every algebraic space for the zariski topology a scheme?
\end{question}
