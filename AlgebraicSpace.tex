
\section{Talk: Algebraic Spaces: (Counter-)examples}

\subsection{Recap: Sheaftheory}
\begin{notation}
	\[\bT = \sum_{X : \Aff} X \text{ formally \etale } \land \underbrace{\text{faithfully flat}}_{=\text{ flat }+ \lnot \lnot \text{ inhabited } }\]
\end{notation}
$\bT$ contains $1$ and is stable under $\sum$.
A $\bT$-cover is a map fibered in $\bT$.
\begin{definition}
	A type $X$ is an etale-stack, if for any $\Spec A : \bT$ and any map $\|\Spec A \| \to X$, there exists a unique filler
	% https://q.uiver.app/#q=WzAsMyxbMCwwLCJcXHxcXFNwZWNBIFxcfCJdLFsxLDAsIlgiXSxbMCwxLCIxIl0sWzAsMSwiXFxmb3JhbGwiXSxbMCwyXSxbMiwxLCJcXGV4aXN0cyEiLDIseyJzdHlsZSI6eyJib2R5Ijp7Im5hbWUiOiJkYXNoZWQifX19XV0=
	\[\begin{tikzcd}
		{\|\Spec A \|} & X \\
		1
		\arrow["\forall", from=1-1, to=1-2]
		\arrow[from=1-1, to=2-1]
		\arrow["{\exists!}"', dashed, from=2-1, to=1-2]
	\end{tikzcd}\]
	An \etale-sheaf is a 0-type that is an \etale-stack.
\end{definition}

Beeing an \etale-stack determines a lex-modality, that we write $L_\bT$. \\
Given an equivalence relation $R$ on a sheaf $X$ valued in propositional sheaves, we can form the sheaf-quotient $p : X \to X / R :\equiv L_\bT \| X // R \|_0$ that is uniquely characterized by the following properties
\begin{itemize}
	\item for any $x, y : X$, $p(x) =_{X/R} p(y) \leftrightarrow R(x,y)$
	\item The map $p$ is $\bT$-surjective, i.e. for any $z : X / R$, $\|\fib_p (0)\|_\bT$ ('$\bT$-merely the proposition $\|\fib_p (0)\|$ holds').
\end{itemize}
\subsection{Algebraic Spaces}
We assume today, that schemes are \etale sheaves.
\begin{definition}
	A naive algebraic space is an \etale-sheaf $X$, that merely
	admits a $\bT$-cover $\Spec B \to F$. \\
	We call it covering, if we can choose $\Spec B \in \bT$.
\end{definition}
This is not good enough, because we can NOT prove the following
\begin{itemize}
	\item all schemes are naive algebraic spaces
	\item naive algebraic spaces having descent, i.e. the type of them is an \etale-stack. \\
\end{itemize}
Instead we have to repeat the process asking for an atlas twice. 
\begin{definition}
	An algebraic space is an \etale-sheaf $X$, such that one of the following equivalent conditions holds:
	\begin{itemize}
		\item 	 merely we find a map $\Spec A \to X$ such that each fiber is a covering naive algebraic space. 
		\item We merely can express it as the sheaf-quotient of an affine $\Spec A$ by an equivalence relation $R$ that is covering, i.e. for each $x : \Spec A$, $\sum_{y : \Spec A} R(x,y)$ is a covering naive algebraic space.  \\	
	\end{itemize}
	We call $X$ covering, if we can choose $\Spec A$ to lie in $\bT$.
\end{definition}
This class is stable under $\sum$ and under quotients by covering equivalence relations
\begin{theorem}[DESCENT]
	The type of (covering) algebraic spaces is an \etale-stack.
\end{theorem}
\begin{example}
	Schemes are algebraic spaces!
\end{example}
\begin{proof}
	\begin{enumerate}
		\item 	Merely inhabited finite sums of principal open propositions belong to $\bT$. 
		\item open propositions are naive algebraic spaces. 
		\item	If $X$ is a scheme, merely we find a map $\coprod_i \Spec A_i \to X$ whose fibers are merely inhabited finite sums of open propositions, which are covering.
	\end{enumerate}
\end{proof}
\begin{question}
	Can we find algebraic spaces that are not schemes? \\
	Can we prove with them, that Schemes do not have descent? \\	
\end{question}
\subsection{Quotients by Group actions}
Let $\ell \neq 0$ denote a prime. Consider $\mu_\ell = R[X] / (X^\ell - 1)$.
\begin{example}[Non-free action]
	If $2 \neq 0$, the sheaf-quotient of $\bA^1$ by the non-free $\mu_2$ action is not an algebraic space!
\end{example}
This suggests that we need a free action. On the other hand, classically, the quotient of a quasi-projective scheme by a finite free group action is a scheme.
%\begin{example}
%	The quotient of $\bA^{\times}$ by the free $\mu_\ell$ action gives a scheme.
%\end{example}
%Having a free action on the whole space might be not good enough to cook up examples of algebraic spaces that are not schemes.
\begin{construction}
	Given a formally \etale + flat affine (e.g. $\mu_\ell$ or finite) group $G$ that acts on an affine $\Spec A$. Assume $G$ acts free on some open subset $U \subset \Spec A$.  % that belongs to $\bT$ (more generally a covering algebraic space)
	
	Then we construct a covering equivalence relation $R$ on $\Spec A$, such that
	\begin{itemize}
		\item On $U$ we are just taking the quotient by the $G$-action: for any $x : U$  and $y : \Spec A$
		\[R(x,y) \leftrightarrow \sum_{g : G} g x  = y. \]
		\item On the complement $U^c \equiv \Spec A \setminus U$ we do nothing: for some $x : U^c , y : \Spec A $, we have $R(x,y) \leftrightarrow x = y$.
	\end{itemize}
	We write $\Spec A /_U G \equiv \Spec A / R$ and call it the quotient of $\Spec A$ by the $G$-action on $U$ .
\end{construction}
\begin{proof}
	\[R_{G}(x,y) \equiv (x = y) + (x \in U) \times \sum_{g : G \setminus \{1\}} g x = y\]
	This is covering: For any $x : \Spec A$ we have
	\[\sum_{y : X} x = y + (x \neq 0) \times \sum_{g : G \setminus \{1\}} g x = y \simeq 1 + (x \in U) \times G \setminus \{1\}\]
	$G \setminus \{1\} = \sum_{g : G} g \neq 1$ is a $\sum$ of formally \etale + flat affines (recall that formally \etale affines have decidable equality). \\
	Indeed, the two conditions hold, using that $G$ has decidable equality.
\end{proof}

\begin{example}[Free action]
	Let $G$ act freely on the whole space $U \equiv \Spec A$. Then this construction yields the actual group quotient: $\Spec A  /_{\Spec A} G = \Spec A / G$.
\end{example}
\begin{proof}
	Indeed, the equivalence relation is the same, using that $G$ has decidable equality.
\end{proof}
%\begin{notation}
%	If $U = \Spec A \setminus Z$ the complement of a closed subset we write \[U \equiv Z^c\]
%\end{notation}

\begin{example}[Quotient of the Line]
	If $\ell \neq 0$ is prime, we have $R /_{0^c} \mu_\ell$ is an algebraic space
\end{example}
\begin{example}[Quotient of the Cross]
	Let $\mu_\ell$ act on $X = \Spec R[X,Y] / X^\ell - Y^\ell$ via multiplication on the second component. Then
	\[
	X /_{\{0,0\}^c} \mu_\ell
	\]
	is an algebraic space.
\end{example}
Are those schemes?
\subsection{Not a scheme?}

\begin{prop}	
	Let $0 : \Spec A$ be a regular point, i.e. we can write $\Spec B \setminus \{0\} = D(p_1,\hdots,p_n)$ for some $p_1,\hdots,p_n : B$ jointly-reguar, i.e. $B \to \prod_{j=1}^n B_{p_j}$ is injective. \\
	Let $G$ be a non-trivial formally \etale flat affine group acting on $\Spec A$, such that 
	\begin{itemize}
		\item 0 is a fixpoint
		\item if $g x = x$ for some $g \neq 1$, then $x = 0$. \\
	\end{itemize}
	Then $\Spec A /_{0^c} G$ from \ref{ex:GAction} is an algebraic spaces that is not a scheme.
\end{prop}
\begin{proof}
	As $\lnot \lnot (G \setminus \{1\})$ and our goal is $\lnot \lnot$-stable, we may choose some $g : G \setminus \{1\}$. Then for all $y : \Spec A$
		\[
	R_{\sharp}(y,gy) \simeq (y = gy) + (y \neq 0) \land \sum_{h \neq 1} h y = g y  {\simeq} (y = 0) + (y \neq 0) \simeq (y = 0) + (y \neq 0)
	\]	
	Regularity of $0$ gives us that $\{0\} + D(0) \subset \Spec A$ is not a locally closed subtype, i.e a closed subset of an open subset.
	But the identity types of a scheme are locally closed propositions (subsets of the point).
\end{proof}

\begin{example}
	Assume $\ell \neq 0$ prime. Let $\mu_\ell$ act on $\Spec B$ in one of the following ways:
	\begin{enumerate}
		\item Let $\mu_\ell$ act on $\Spec B = \bA^1$. 
		\item Let $\mu_\ell$ act on
		\[
		\Spec B \equiv \sum_{x , y : R} x^\ell = y^\ell
		\]
		via $g (x,y) = (x,g y)$
	\end{enumerate}
	Then $\Spec B /_{0^c} \mu_\ell$ is an algebraic space that is not a scheme.
\end{example}
\begin{proof}
	$\lnot \lnot$ merely, $\mu_\ell$ is finite (\todocite) and $\mu_\ell \setminus \{1\}$ is inhabited by \ref{lemma:CompOf1}. 
	\begin{enumerate}
		\item Pointed-Free action is clear. $0 : \bA^1$ is a regular point by first projection.
		\item  Pointed-Free action is clear. The cross is regular pointed, witnessed by  the first projection: It is regular vanishing at $(0,0)$ And for any point $(0,y) : \Spec B$ we deduce $y^\ell = -0^\ell = 0$, hence $\lnot \lnot (x,y) = (0,0)$.
	\end{enumerate}
\end{proof}
\subsection{Fiber Collapse!}
An alternative approach to construct algebraic spaces is the fiber collapse away from the origin.

\begin{definition}
	Given a sheaf proposition $P$, there is a closed modality $\mathsf{Cl}_P$ where a type $X$ is modal, if it is a sheaf and $P \to \isContr(X)$.
	We have to stackify to belong to the sheaf topos:
	\begin{align*}
	\cU &\to \cU \\
	X &\mapsto P \star X :\equiv L_\bT(P \sqcup_{P \times X} X)
\end{align*}
	where we need to stackify the pushout. This determines a lex modality.
\end{definition}
\begin{definition}
	Let $Y_\bullet : R \to \Aff$ be a dependent family of affines %$(Y \in \EF)^{x \neq 0}$ . 
	The fiber collapse of $Y_\bullet$ away from the origin $\emdash Y_\bullet \emdash $ is the following type over $R$
	\[
	\sum_{x : R} (x \neq 0) \star Y_x \to R %\sum_{x : R} Y_x \to 
	\]
	%	: \prod_{x : R} \sum_{Y:\Aff} (Y \in \EF)^{x \neq 0}$ %
\end{definition}
\begin{itemize}
	\item The \emph{infinitesimal fibers} over $\varepsilon : \cN_\infty$ are $\fib_p(\varepsilon) = Y_\varepsilon$. 
	\item In particular the \emph{basefiber} $\fib_p (0)$ is equivalent to $Y_0$, 
	\item The \emph{other} fibers $\fib_p (x)$, $x \neq 0$, are contractible. 
\end{itemize}
So ---$Y_\bullet$--- is obtained by keeping only the infinitesimal fibers and collapsing all the other fibers.
This space over $R$ looks exactly like the line away from the origin. % and over an infinitesimal 
\begin{lemma}{\label{lemma:FiberCollapsAS}}
	Assume that if $x \neq 0$, then $Y_x \in \bT$. Then $\emdash Y \emdash$ is an algebraic space.
\end{lemma}
\begin{proof}
	Let $x : R$. Let $Y : \Aff$ such that $x \neq 0$ implies that $Y$ is formally \etale + flat. We will show that $\eta : Y \to (x \neq 0) \star Y$ is the sheaf-quotient map of the relation on $Y$ given by $y \sim y' \equiv (y = y') + (x \neq 0) \times y \neq y'¸$, which is enough by \ref{ex:MapOverR}. We apply  \ref{lemma:quotient-by-equivalence-relation} \\
	\begin{itemize}

			\item The map is $\bT$-surjective: We have a $\bT$-surjection $(x \neq 0) + Y \to (x \neq 0) \star Y$. In case $x \neq 0$, the map of interest is $Y \to 1$, which is $\bT$-surjective, as then $Y \in \bT$. 
		\item Given $y,y' : Y$, we have
		\begin{align*}
			\eta(y') = \eta(y) &\simeq (x \neq 0) \star (y = y')&& \ | \  \text{ closed modality is lex (\cite{Modalities} Example 3.1.4).} \\
			&\simeq L_\bT\left((y = y') \lor (x \neq 0) \right) && \ | \ (x \neq 0) \to \mathsf{HasDecEq}(Y) \\
			&\simeq (y = y') + (x \neq 0) \times y \neq y'¸
		\end{align*}
		\end{itemize}
\end{proof}
\begin{example}
	$\emdash Bool \emdash$ is the line with two origins. \\
	---$\Spec R[X] / (X^2 + 1)$--- is the twisted line with two origins, i.e. over the origin we have the roots of $-1$. \\
	---$\Spec R[Y] / (Y^2 - \bullet)$--- is $\bA^1 /_{0^c} \mu_2$ which looks like $\bD(1)$ over $0$. \\
	---$\Spec R[Y] / (Y^2 - \bullet^2)$--- is the quotient of the cross that looks like $\bD(1)$ over every $\varepsilon : \bD(1)$. \\

	
	---$\Spec R[Y] / (\bullet Y)$--- is the affine plus.
\end{example}
\begin{prop}
	Let $G$ be a formally + flat affine group.
	Let $p : \tilde Y \to R$ such that the pullback to $R^\times$ can be enhanced to a $G$ torsor over $R^\times$. Write $Y_x \equiv \fib_p x$. Then there is a canonical equivalence
	% https://q.uiver.app/#q=WzAsNCxbMSwwLCJcXHRpbGRlIFkiXSxbMCwxLCJcXHRpbGRlIFkgLyBHfF97KFxcZmliX3AgMCleY30iXSxbMiwxLCJcXGVtZGFzaCBZIFxcZW1kYXNoIl0sWzEsMiwiUiJdLFswLDFdLFswLDJdLFsxLDIsIlxcc2ltZXEiXSxbMSwzXSxbMiwzXV0=
	\[\begin{tikzcd}
		& {\tilde Y} \\
		{\tilde Y /_{Y_0^c}} G && {\emdash Y \emdash} \\
		& R
		\arrow[from=1-2, to=2-1]
		\arrow[from=1-2, to=2-3]
		\arrow["\simeq", from=2-1, to=2-3]
		\arrow[from=2-1, to=3-2]
		\arrow[from=2-3, to=3-2]
	\end{tikzcd}\]
\end{prop}
\begin{proof}
	As every \emph{other} fiber is merely equivalent to $G$, its formally \etale + flat.
	In between you can put for $U_x \equiv x \neq 0 \times Y_x$
	\[\tilde Y \to \sum_{x : R} Y_x /_{U_x} G\]
	As all three maps defined on $\tilde Y$ are $\bT$-surjective, by \ref{lemma:quotiquotient-by-equivalence-relation} we may only check that the identity types coincide. For any $y,y' : \tilde Y$. Using that if $py \neq 0$ then the $G$ action on $Y_{py}$ is a $G$-torsor, We have an equivalence
	
	\begin{align*}
		(y = y') + y \not \in Y_0 \land \underbrace{\sum_{g \neq 1} g y = y' }_{\simeq (py = py') \land y \neq y'} &\simeq  (py = py') \land ((y = y') + py \neq 0 \land y \neq y' ) \\
	\end{align*}
	If we fix $x : R$ and put $y,y' : Y_x$ we have, writing $\eta : Y_x \to (x \neq 0) \star Y_x$
	\[
	(y = y') + py \neq 0 \land y \neq y'  \simeq \eta y = \eta y'
	\]
	by the proof of \ref{lemma:FiberCollapsAS}.
\end{proof}
\subsection{Schemes do not have descent}
\begin{rmk}
	Whenever we want to show a proposition that is an \etale-sheaf, we may assume a term in $\Spec C$, thus an equivalence $\Spec C = Bool$.
\end{rmk}
\begin{prop}
	let $\rho : R \setminus \{0\}$ (e.g. $\rho = 1$). Set $C = R[T] / (T^2 + \rho)$. If ---$\Spec C$--- is a scheme, then $X^2 + \rho$ merely has a root.
\end{prop}
\begin{proof}
	Let $p : \emdash \Spec C \emdash \to R$ be the first projection.
	We proceed as follows
	\begin{enumerate}
		\item  There is no open affine subset of $\emdash \Spec C \emdash$ containing $\fib_p(0)$.
		\item Any inhabited cover of $\Spec C$ by open subsets strictly smaller than $\Spec C$ yields a root.
	\end{enumerate}
	Any finite open affine cover of ---$\Spec C$--- can be restricted to a finite open affine cover of the basefiber $\Spec C$, that satisfies the condition in 2.
	Proofs:
	\begin{enumerate}
		\item Because we want to show $\bot$, we may assume ---$\Spec C \emdash = \emdash Bool \emdash$. Assume there is an open affine subset $\fib_p(0) \subset U \subset \emdash Bool \emdash$. Then $p(U) \subset R$ is an open neigbhorhood of 0, as 
		\[
		x \in p(U) \leftrightarrow (x,N) \in U \lor (x,S) \in U
		\]
		Claim: the map $R^{p(U)} \to R^U$ is an equivalence. If we have shown that: As $U$ is affine we conclude that the map
		\begin{align*}
			U &\to \Spec (R^{p(U)}) \\
			x &\mapsto \phi \mapsto (\phi(px))		
		\end{align*}
		is an equivalence, which is a contradiction to the assumption, that $U$ contains both origins. \\
		Proof of claim:
		Injectivity: If two maps $f , g : p(U) \to R$ coincide after precomposing with $U \to p(U)$, then they coincide away from $0$
		so conclude by \ref{lemma:AlmostEverywhere}. \\
		Surjectivity: Given a map $U \to R$, by pulling back along $p : R + R \to \emdash Bool \emdash$  we can view it as a map $R + R \supset U' \to R$ defined at both origins, so in particular as a pair of maps to $R$ defined on some open neigbhorhood of 0 of $R$. They coincide away from 0 so by \ref{lemma:AlmostEverywhere} they are equal.
		\item Let $\Spec C = \bigcup_{j=1}^n U_j$ be an open cover of strictly smaller subsets of $\Spec C$. \\
		Define 
		\begin{align*}
			A : \Spec C &\to \Prop^{\Fin(n)}  \\
			x &\mapsto \{ j : x \in U_j\}
		\end{align*}

		Observe 
		\begin{enumerate}
			\item for any $x,x' : \Spec C$, 
			\[\| A x \cap  A x' \| \to \lnot \lnot (x = x') \overset{\mathsf{DecEq}}{\to} x = x'\] where the first implication follows like this : if $x \neq x'$ $j \in A_x \cap A_{x'}$, then using that any embedding $Bool \to \Spec C$ is an equivalence, $\Spec C \overset{\sim}{\hookleftarrow} \{x , x'\} \subset U_j$, . Then we have a contradiction to the first point, 
			\item For any $x : \Spec C$ , $\| A x\|$.
		\end{enumerate}
		As $\Spec C$ is an \etale sheaf, its enough to show to construct a function $\|\Spec C\| \to \Spec C$. 
		Assume $\|\Spec C\|$. \\
		Lets try to construct a term that maximizes the composite
		\[
	 	\sum_{x : \Spec C} A x \hookrightarrow \Spec C \times \Fin(n) \to \Fin(n) \\
		\]
		For this we may assume $\Spec C = Bool = \{+,-\}$, as there is at most one maximum:  Let $(x_1,j), (x_2,j')$ be maximizing, then as $\Fin n $ is totally ordered, $j = j'$ so  $j : A_{x_1} \cap A_{x_2}$ , thus $x_1 = x_2$ by (a). \\
%		, we can set first $x' \equiv x_2$ and then $x' \equiv x_1$ respectively and then (by (b)) choosing $j : A_{x_1},j' : A_{x_2}$ gives $j \le j' \le j$ so that
		I will explain an algorithm to do the following:
		Given $n \ge 2$, and a pair of disjoint (a) and merely inhabited (b) subsets $A_+$ and $A_-$ of $\Fin(n)$, we can decide in which of the two we find the bigger number of $\Fin(n)$ and which number it is. \\
		Induction over $n$. If $n = 2$, then $\|A_+\|$, so we find a term in the proposition $(0 \in A_+) + (1 \in A_+)$ . In the left case return $(-,1)$, in the right case $(+,1)$. \\
		For $n \mapsto n+1$, by (b) we may choose $k_+, k_- : \Fin n$ such that $k_+ \in A_+$ and $k_- \in A_-$. Now, exploiting decidable equality in $\Fin n$: If $n = k_+$, return $(+,n)$. If $n = k_-$, return $(-,n)$. Otherwise, the subsets $A_+ \cap \Fin (n)$ and $A_- \cap \Fin (n)$ of $\Fin n$ still satisfy both conditions (a) (b), so conclude by induction.  
	\end{enumerate}
\end{proof}
\begin{corollary}
	Schemes do not have descent.
\end{corollary}
\begin{proof}
	If Schemes have descent, then ---$\Spec R[T]/(T^2 + \rho) \emdash \in \mathsf{Sch}$ is a sheaf. As ---$\Spec R[T]/(T^2 + \rho) \emdash$ is $\bT$-merely a scheme, it is a scheme, so by the previous lemma $T^2 + \rho$ has a root. As $\rho : R \setminus \{0\}$ was arbitrary, we get a contradiction to \cite{cherubini2023foundationsyntheticalgebraicgeometry} A . 0.3. \\
\end{proof}
\section{Algebraic spaces}


%Let us also mention what we learned in the proof:
%\begin{lemma}[NECESSARY?@²]
%	A covering equivalence relation $R : S^2 \to \bT \Prop$ has values in geometric propositions.
%\end{lemma}

%\begin{corollary}
%	The identity types of algebraic spaces are geometric propositions.
%\end{corollary}
%\begin{proof}
%	By the previous lemma and \ref{lemma:geometricStacksClosedUnderId}
%\end{proof}
%
%\begin{lemma}{\label{lemma:detectGeomProp}}
%	Let $P$ be a sheaf and a proposition that admits a map $\Spec A \to P$ fibered in covering algebraic spaces. Then $P$ is a geometric proposition.
%\end{lemma}
%\begin{proof}
%	The fibers are covering algebraic spaces and affine, hence covering affine. By \ref{def:algprop} we conclude.
%\end{proof}
\begin{theorem}
	Let $X$ be a modal set. The following are equivalent:
	\begin{enumerate}
		\item $X$ is a (covering) geometric 0-stack
		\item $X$ is merely of the form $L_\bT (U / R)$ for some (covering) affine $U$ and  $R : U^2 \to \Prop_{\ci}$ a covering equivalence relation. 
		\item there exists some map $S \to X$ with $S$ (covering) affine whose fibers merely have $\bT$-catlasses.
	\end{enumerate}
	We call this class (covering) algebraic spaces.
\end{theorem}
\begin{proof}
\ 	\begin{enumerate}
	\item [2 $\leftrightarrow$ 3]
		This is \ref {lemma:fundamental-property-algebraic-spaces}
	\item [2 $\to$ 1]
	Choose a presentation $ R: U^2 \to \Prop$.
	It suffices to show, that the map $f : U \to L_\bT ( U / R)$ is a geometric (c)atlas. The map $f$ is $\bT$-surjective by the well-definedness of the bijection $\ref{quotient-by-equivalence-relation}$. By descent we may just show, that the fibers $\fib_f (f(s))$ for $s : U$ are covering 0-stacks. But by the bijection in \ref{quotient-by-equivalence-relation} those are equivalent to the fibers $R_s$, which are covering 0-stacks as the equivalence relation is covering. \\
	\item [1 $\to$ 2]
	This can be reformulated in the following way, using the recursion principle for (covering) geometric 0-stacks:
	Let $X$ be a sheaf of sets. Let $S$ be (covering-) affine and $f : S \to X$ be fibered in covering algebraic spaces. Then $X$ is a (covering) algebraic space.
%	The identity types of $X$ admit a map fibered in covering algebraic spaces (todo check stability under $\sum$) out of an affine by \ref{lemma:havingAbstractAtlasClosedUnderId}. by \ref{lemma:detectGeomProp} they are geometric propositions. 
This follows from the observation, that the equivalence relation determined by $f$ is covering \ref{def:coveringEqRel} , because the fibers of $f$ are covering 0-stacks.f
	\end{enumerate}
\end{proof}
\begin{prop}
	For any $n \ge 1$, we have inclusions 
	\[W_{n} \subset \CS_{n-1} \subset W_{n+1}\]
\end{prop}
\begin{proof}
	Induction. $n = 1$ gives
	\[
	\mathsf{HasCatlas}_\bT \subset \CS_0 \subset \text{ types admitting a catlas fibered in } W_1
	\]
	the latter inclusion is the previous theorem. \\
	The induction step is obtained by \ref{prop:nstack}
\end{proof}


\subsection{Schemes are algebraic Spaces for the Zariski Topology}
\begin{definition}
 A proposition $U$ is open iff its merely of the form $f_1 \ inv \lor \hdots f_n inv$ for some $f_i : R$.
\end{definition}

\begin{lemma}
	Given $f_1, \hdots,f_n : R$ such that $\| D(f_1) + \hdots + D(f_n) \|$ then $\sum_{i=1}^n D(f_i) \in \Zar$.
\end{lemma}
\begin{prop}
	Every Zariski-merely-inhabited type that is merely of the form $U_1 + \hdots + U_n$ for open propositions $U_i$ admits a $\Zar$-catlas.
\end{prop}
\begin{proof}
	By definition of openness, We can choose a surjection $\coprod_{j=1}^{n_i} D(f_{ij}) \twoheadrightarrow U_i$ for any $i$. We want to show, that the map
	\[
	\coprod_{i , j} D(f_{ij}) \twoheadrightarrow U_1 + \hdots U_n
	\]
	is a $\Zar$-catlas. 
	\begin{itemize}
		\item Let us first show that the fibers are in $\Zar$. Assume $U_i$ holds. So we find a term in $\coprod_j D(f_{ij})$. In particular we have $\| \coprod_j D(f_{ij})\|_{\Zar}$. By the lemma we conclude, that the fiber $\sum_j D(f_{ij})$ belongs to $\Zar$.\\
		\item The total space is in $\Zar$: This follows as the surjection after propositional truncation becomes an equivalence. As we have $\| U_1 + \hdots + U_n\|$, we can conclude by the lemma.
	\end{itemize}
	
\end{proof}
\begin{warning}
	The converse does not hold! We want to apply \ref{lemma:stackificationHasTCover}, to the map
	\[\Zar \ni 1 + 1 \to \sum D(f) \]
	\begin{itemize}
		\item 	$\sum D(f)$ is seperated as $D(f)$ is a sheaf.
		\item 	All the fibers are equivalent to $1 + X$, hence they are in the Zariski topology.
	\end{itemize}	
\end{warning}
\begin{lemma}
	let $X$ be a scheme. There merely exists some affine $S$  map $S \to X$ whose fibers are merely inhabited finite sums of open propositions 
\end{lemma}

\begin{corollary}		
		Every scheme is an algebraic space.
\end{corollary}


\begin{lemma}
	If $X$ is an algebraic space, then the global sections embed via a $R$-algebra homomorphisms into a finitely presented $R$-algebra.
\end{lemma}
\begin{proof}
	Choose an atlas $S \to X$, in particular $\bT$-surjective. As $\bT$ is subcanonical the map $R^X \to R^S$ is an injection.
\end{proof}
\begin{question}
	Is it an open embedding of types?
\end{question}
%\begin{lemma}
%	Consider a morphism $X \to Y$ between algebraic spaces such that $R^F$ is finitely presented and the affinizations of the fibers $F \to \Spec R^F$ are open immersions. If $Y$ is a scheme, then $X$ is a scheme.
%\end{lemma}
%\begin{proof}
%	Consider the stein factorization
%	\[
%	X = \sum_{y : Y} \fib_f y \to \sum_{y : Y} \Spec R^{\fib_f y} \to Y
%	\]
%	The first map 
%\end{proof}

%\begin{lemma}
%	Let $X$ be a geometric stack, whose identity types are covering stacks. Let $G$ be a finite group acting on $X$. Then $L_\bT (X / G)$ is a geometric stack.
%\end{lemma}
%\begin{proof}
%	Consider for $x , y : X$ , $R(x,y) \equiv \| \sum_{g : G} g x = y\|_\bT$ which is indeed modal. We have to check that the relation is covering, i.e. that for all $x : X$, 
%	\[
%	\sum_{y: X} \|\sum_{g: G} g x = y\|_\bT
%	\]
%	is a covering stack. \\
%	To prove this, as covering stacks are stable under quotients, it suffices to show, that the map
%	\[
%	G \simeq \left (\sum_{y : X} \sum_{g : G} g x = y \right) \to \sum_{y: X} \|\sum_{g: G} g x = y\|_\bT
%	\]
%	is a geometric cover. But the fibers look like $\sum_{g : G} gx  = y$ which is a finite sum of identity types in $X$, which were assumed to be covering stacks. By \ref{lemma:geomStackPlusStable} the fibers are covering stacks.
%	
%\end{proof}

%\section{Examples}
The goal of this subsection is to construct algebraic spaces. The first example actually gives us a scheme:
\begin{example}
	Let $p \neq 0$ be a prime. You can let $\mu_p := \Spec(R[X] / (X^p - 1))$ act on $\bA^\times$ via multiplication. Set $\bT= fppf$. Then the $p$.th power map
	
	\[
	pow : \| \bA^\times // \mu_p \|_0^\bT \to \bA^\times
	\]
	is an equivalence.
	\begin{itemize}
		\item 	 It is an embedding: 
		First note, that $\|\bA^\times // \mu_p \|_0$ is $\bT$-seperated:
		
		as $\mu_p$ act freely on $\bA^\times$, $\bA^\times // \mu_p$ is already a set. Meaning that the identity types of the set-quotient are $\sum_{g: \mu_p} g x =_{\bA^\times} y$ , hence sheaves. \\
		On the other hand the map $\|\bA^\times // \mu_p \|_0 \to \bA^\times$ is an embedding, as for any $x , y : \bA^\times$ the map $(\sum_{g : \mu_p} g x = y) \to (x^p = y^p)$ is an equivalence. 
		\item 	It is $\bT$-surjective, as for any $\lambda : \bA^\times$, we find $S = \Spec R [X] / (X^p - \lambda) \in \bT$ with 
		\[
		S \to \fib_{\bA^\times / \mu_p \to \bA^\times}(\lambda)
		\]
		hence 
		\[
		1 = \|S\|_\bT \to \|\fib_{pow}\|_0^\bT
		\]
	\end{itemize}
	
	
\end{example}
\begin{example}
	Let $P$ be the open proposition $x \neq 0$ for some $x : \bA^1$. Then $H = 1 + P$ is an open subgroup of $\bZ/2$. The sheaf quotient $G / H$ is the scheme $\Susp(x \neq 0)$.
\end{example}
Let $\ell \neq 0$ denote a prime. Consider $\mu_\ell = R[X] / (X^\ell - 1)$.
\begin{lemma}{\label{lemma:CompOf1}}
	Let $(G,1)$ be a pointed formally \etale flat affine type. Then $(G \setminus \{1\})$ is formally \etale + flat affine. \\
	In particular $\mu_\ell \setminus\{1\}$ is a covering stack.
\end{lemma}
\begin{proof}
	$G \setminus \{1\} = \sum_{g : G} g \neq 1$ is a $\sum$ of formally \etale + flat affines (recall that formally \etale affines have decidable equality). \\
	To show, that $\mu_\ell \setminus \{1\}$ is a covering stack, by \ref{lemma:covOfEF}, we need to show it is $\lnot \lnot$-inhabited. Indeed as we want to prove a contradiction we may assume a term in $g : \Spec R[X] / (\sum_{i=0}^{\ell-1} X^i)$. But this type is equivalent to $\mu_\ell \setminus \{1\}$, using that $\sum_{i=0}^{\ell-1} X^i | X^\ell -1 $ and $\ell \neq 0$. \\
\end{proof}
\begin{lemma}
	Given a modal equivalence relation $R$ on a sheaf $X$ and a 1-stack $T$ and a map $f : X \to T$ and term $p : \prod_{x , y : X} R(x,y) \to f x = f y$ such that $p (x , y) \cdot p (y , z) = p (x,z)$, where the witnesses for $R$ are left implicit. Then $f$ factors through the quotient.
\end{lemma}
\begin{lemma}
	Put $\ell = 2$
		If $\ell \neq 0$, the sheaf quotient of $\bA^1$ by the $\mu_2$ action is not an algebraic space.
\end{lemma}
\begin{proof}
		Assume this it is an algebraic space. \\
		Set $\bD(1) = \Spec R[X] / X^\ell$.
		Then $\sum_{x : \bA^1 / \mu_\ell } x^\ell =_{\bA^1} 0 \simeq \bD(1)/\mu_\ell$ is an algebraic space by $\sum$-stability. \\
		Then we can choose a geometric atlas $p : \Spec A \to \bD(1) / \mu_\ell$. 
		We proceed in the following steps
		\begin{enumerate}
		
			\item %The bundle of fibers $\bD(1) / \mu_\ell \to \CS$ is constant. More precisely 
			There is an equivalence $\Spec A \simeq \fib_p 0 \times \bD(1) / \mu_\ell$.
			\item The fiber over 0 is affine
			\item $\bD(1) / \mu_\ell$ is $\lnot \lnot$ affine
			\item $\bD(1) / \mu_\ell$ is $\lnot$ affine%The quotient of the $\mu_\ell$ action on $\Spec B$ does not give us an affine scheme for $\ell \neq 0$.
		\end{enumerate}
		Proofs
		\begin{enumerate}
			\item Let us denote $F : \bD(1) / \mu_2 \to \CS_0$ the bundle of fibers of $f$, where we note that the fibers are indeed sets. As $\CS_0$ is formally \etale (\todocite), we have terms
			\[
			\phi : \prod_{x: \bD(1)} F[x] = F[0]		, \phi^- : \prod_{x: \bD(1)} F[-x] = F(0)	
			\]
			that both evaluate at $x = 0$ to $\refl_{F[0]}$. \\
			The goal is to produce a term in 
			\[
			\prod_{x: \bD(1) / \mu_2} F x = F [0]
			\]
			By the previous lemma, using that $\CS_0$ is a 1-stack, we need to show, that under the path $p_x : [x] = [-x]$ in the quotient we have
			\[
			\ap_{p_x} F \cdot \phi^- x = \phi x
			\]                                     
			This proposition is formally \etale as $\CS_0$ is formally \etale. Thus we may assume the closed dense proposition $x = 0$. Then $p_x = \refl_{[0]}$ and $\phi^- 0 = \refl = \phi 0$ by assumption. 
			
%			 if we precompose with the $\bT$-surjection the map  is constant, i.e. $\prod_{x : \bD(1)} \fib_f{[x]} \simeq F$. As covering stacks have descent, the type $\CS$ is in particular $\bT$-seperated, thus $\bD(1) / \mu_\ell \to \CS$ is constant. 
			\item 
			Let us first show, that We may assume that our geometric cover factors through the $\bT$-surjection $\Spec A \overset{f}{\to} \bD(1) \to \bD(1) / \mu_\ell$. 
			Proof: By $\bT$-local choice applied to the $\bT$-surjection $\bD(1) \to \bD(1) / \mu_\ell$, we find a $\bT$-cover $\Spec B \to \Spec A$ and a factorization % https://q.uiver.app/#q=WzAsNCxbMCwwLCJcXGV4aXN0cyBcXFNwZWMgQiJdLFswLDEsIlxcU3BlYyBBIl0sWzEsMSwiXFxiQV4xIC8gXFxtdV9cXGVsbCJdLFsxLDAsIlxcYkFeMSJdLFswLDEsIiIsMCx7InN0eWxlIjp7ImJvZHkiOnsibmFtZSI6ImRhc2hlZCJ9fX1dLFswLDMsIiIsMix7InN0eWxlIjp7ImJvZHkiOnsibmFtZSI6ImRhc2hlZCJ9fX1dLFsxLDJdLFszLDIsIiIsMix7InN0eWxlIjp7ImhlYWQiOnsibmFtZSI6ImVwaSJ9fX1dXQ==
			\[\begin{tikzcd}
				{\exists \Spec B} & {\bD(1)} \\
				{\Spec A} & {\bD(1) / \mu_\ell}
				\arrow[dashed, from=1-1, to=1-2]
				\arrow[dashed, from=1-1, to=2-1]
				\arrow[two heads, from=1-2, to=2-2]
				\arrow[from=2-1, to=2-2]
			\end{tikzcd}\]
			\qed(Claim)  \\
			Its enough to see that the map $\fib_f 0 \to F$ is an equivalence. That follows because $0$ is a fix point of the $\mu_\ell$ action on $\bD(1)$. %if $x : \bD(1)$ such that $\|\sum_g g x = 0 \|_\bT$, then $x = 0$.
			\item $F$ is a covering stack, hence $\lnot \lnot$-inhabited. As the goal is $\lnot \lnot$-modal, we may pick a map $1 \to F$. Then, by step 1			\[\bD(1) / \mu_\ell = 1 \times_F (F \times \bD(1) / \mu_\ell) = 1 \times_F \Spec A\]
			is a fiber product of affines, hence affine. 
			\item Here we need that $\ell = 2$ . The affinization map would be induced by % https://q.uiver.app/#q=WzAsMyxbMCwwLCJcXGJBXjEiXSxbMSwxLCJcXGJBXjEiXSxbMCwxLCJcXGJBwrkvXFxtdV9cXGVsbCJdLFswLDEsInogXFxtYXBzdG8gel5cXGVsbCJdLFswLDJdLFsyLDEsIiIsMix7InN0eWxlIjp7ImJvZHkiOnsibmFtZSI6ImRhc2hlZCJ9fX1dXQ==
			\[\begin{tikzcd}
				{\bD(1)} \\
				{\bD(1)/\mu_\ell} & {\bD(1)}
				\arrow[from=1-1, to=2-1]
				\arrow["{z \mapsto z^\ell}", from=1-1, to=2-2]
				\arrow[dashed, from=2-1, to=2-2]
			\end{tikzcd}\]
			But the map is not an embedding: For any $\varepsilon : \Spec R[X] / X^\ell$, we have $\varepsilon^\ell = 0^\ell$ but $\varepsilon =_{\bD(1)/\mu_\ell} 0$ iff there $\bT$-merely exists some $g : \mu_\ell$ with $g \varepsilon = 0$, but as $g$ is invertible this is equivalent to $\varepsilon = 0$.
		\end{enumerate}

		
		
%		Then the identity types $\|x = y + x = -y\|_\bT$ are geometric propositions. Let $\varepsilon, \varepsilon'$ in $\cN_\infty(0)$. We find $f : \Spec A \to \varepsilon = \varepsilon'+ \varepsilon = -\varepsilon' \equiv P$  with weakly flat domain which induces an equivalence on $\bT$-truncations. \\
%		$f$ is not$_\varepsilon$ a $\bT$-cover, indeed: By \ref{lemma:wfdesTCov} and the codomain beeing not$_\varepsilon$ weakly flat . %If its a $\bT$-cover then its codomain is also weakly-flat by a lemma, thus $\varepsilon = 0$. \\
		%But the fiber of  $f$ over some $p : P$ is not$_\varepsilon$ not in $\bT$. Note that $\Spec A \in \bT$, as by assumption it is weakly flat and $\bT$-inhabited. So if $\fib_f p \hookrightarrow \Spec A$ is not an equivalence, meaning $\varepsilon = \varepsilon'$ and $\varepsilon = -\varepsilon'$. Thus $\varepsilon =0$.   % $\Spec A \to \varepsilon = \varepsilon'$ , $\Spec B \to \varepsilon = -\varepsilon'$ such that the map out of the sum  \\
%		We have $\lnot \lnot (\Spec A + \Spec B)$. As we want to prove a contradiction, we may assume $\Spec A + \Spec B$. wlog assume a point in $\Spec A$. 

		%If $f$ is not $\bT$-surjective, then $\varepsilon = 0$: 
		
		%Let $q : \bA^1 \to \bA^1 / \mu_\ell$ the projection map.  TODO
\end{proof}

\subsection{Non locally-seperated Examples}

%\begin{proof}
%	%Subgroups always act free, and $\bZ/2$ is an algebraic space, thus by \ref{lemma:algSpacesStabFreeQuots} the sheaf quotient $G / H$ is an algebraic space. \\
%	%We can calculate it: I Claim that
%\end{proof}

%\[\mu_\ell'  = \Spec R[X] / (X^{p-1} + \hdots + 1) = \mu_\ell \setminus\{1\}\]

%\begin{lemma}
%	Let $0 \in D(p)$. For a function $\phi : D(p) \to R$ TFAE % We say some $\phi: R[X]_p$ is $\ell$-even if one of the following equivalent conditions is satisfied:
%	\begin{enumerate}
%
%	\item The function $\phi : D(p) \to R$ is a $\mu_{\ell}$-invariant function, i.e. $\phi(x) = \phi(g x)$ for all $ x : D(p)$ and each $g : \mu_\ell$.
%	\item The function $\phi|_{D(p) \setminus 0} : D(p) \setminus \{0\} \to R$ is an $\mu_{\ell}$-invariant function
%%	\item[1']  There exists $f : R[X], n : \bN$ such that $\phi= f/ p^n$ and $f (g.p)^n (x) = (g.f) p^n (x)$ for each $x : R , g : \mu_\ell$.
%%	\item[2'] The same as 3. but only for $x : D(p) \setminus\{0\}$
%\end{enumerate}
%\end{lemma}
%\begin{proof}
%	 we can apply  \ref{lemma:AlmostEverywhere}, observing $\phi - g.\phi = 0$ on $D(X / 1) \subset \Spec R[X]_p$, where $X/1 : R[X]_p$ is regular, because $X$ is regular in $R[X]$. %$: 
%\end{proof}
\begin{prop}{\label{prop:QuotientByIrrefAndP}}
	Consider an affine $S$ and an open subset $U \subset S$. Consider a $\bT$-flat irreflexive relation $\sharp$ on $U$ , i.e.
	\begin{enumerate}
	%	\item  its valued in propositional sheaves,	 	$\lnot (x \sharp y)$ 
		\item Irreflexivity:  $\lnot (x \sharp x)$

		\item $\bT$-flatness. For all $x : U$ , the fiber $\sum_{y : S} x \sharp y$ is $\bT$-flat. 
\end{enumerate}
 Define a relation on $S$ as
	%Let $p : A^G$. Define $R: D(p)^2 \to \Prop$ as
	\[
	R_{\sharp}(x,y) = (x = y) + (x \in U \land y \in U) \times (x \sharp y) %\sum_{g : G \setminus \{1\}} g x = y
	\]
	(Abuse of notation: where the $\times$ is secretly a $\sum$)
	Then the sheaf quotient $S / R_{\sharp}$ is an algebraic space. \\
\end{prop}
\begin{proof}
%	Define
%	\[
%	E(x,y) = (x = y) + (x \neq 0 \land \sum_{g : \mu_\ell \setminus \{1\}} gx = y)
%	\]
\begin{itemize}
	\item 	This is a proposition: First note, that both summands are propositions and if both summands are inhabited we get a contradiction. 
%	\item 	The propositions are sheaves.
%	\item Its an equivalence relation valued in modal propositions: By assumption on $\sharp$ and the compatbility with $P$.
	\item The relation is covering: 
	 Furthermore, for any $x : S$ we have
	\[
	\sum_{y : S}  (x = y) + (x,y \in U \times x \sharp y) = 1 + (x,y \in U \times \sum_{y} x \sharp y) \in \bT
	\]
	as $\sharp$ was assumed to be $\bT$-flat on $U$: we can write the binary product as $\sum_{p: x \in U} \sum_{y} x \sharp y$, a $\sum$ of of $\bT$-flat geometric stacks.
\end{itemize}
\end{proof}


\subsection{Group quotients}
For this section let $G$ denote a group that is a covering 0-stack. Let $X$ be a sheaf equipped with a $G$ action.
\begin{lemma}
	$\mu_p = \Spec R[X] / (X^p - 1)$ is covering for $p \neq 0$ prime.
\end{lemma}
\begin{proof}
	It is fppf + \etale as $X^p - 1$ is monic seperable. TODO
\end{proof}

%\begin{lemma}
%	Let $G$ act freely on a sheaf $X$. Then the relation
%	\[
%	x , y\mapsto \sum_{g : G} g x = y
%	\]
%	is a covering equivalence relation on $X$
%\end{lemma}
%\begin{proof}
%	For all $x : X$ , the fiber
%	\[
%	\sum_{y : X} \sum_{g : G} g x = y \simeq \sum_{g : G} \sum_{y: X} g x = y \simeq G
%	\]
%	is a covering 0-stack by assumption.
%\end{proof}

\begin{definition}
	A $G$ action on $X$ is free, if for all $x , y : X$ the type 
	\[
	\sum_{g: G} g x = y
	\]
	is a proposition. 
\end{definition}
\begin{example}{\label{ex:GAction}}
	Given a formally \etale + flat affine (e.g. $\mu_\ell$ or finite) group that acts on an affine $\Spec A$. Assume $G$ acts free on some open subset $U$.  % that belongs to $\bT$ (more generally a covering algebraic space)
	
	Then put $x \sharp y = \sum_{g : G \setminus \{1\}} gx = y$. \\	
	This provides a covering equivalence relation $R_{G,U}$ on $\Spec A$, such that
	\begin{itemize}
		\item  for any $x : U$ 
		\[[x] = \sum_{y : \Spec A} \sum_{g : G} g y  = x. \]
		\item for some $y \not \in U$, we have $R(x,y) \leftrightarrow x = y$.
	\end{itemize}
	By abuse of notation we write $\Spec A /_U G \equiv \Spec A / R_{G}$ and call it the quotient of $\Spec A$ by the $G$-action .
	
\end{example}
	\begin{proof}
	\begin{itemize}
		%		\item $x \sharp y$ is a modal proposition. 
		\item It is irreflexive: If $x : U$ then $g x \neq x$ , by freeness. % $x = 0$. %as $x = y$ and $gx = y$ implies $(g-1) x = y - x = 0$, but as $g-1$ is invertible $x = 0$. 
		
		\item We have $G \setminus \{1\}$ is flat affine using \ref{lemma:CompOf1}. \\
	\end{itemize}
\end{proof}
\begin{notation}
	If $U = \Spec A \setminus Z$ the complement of a closed subset we write \[U \equiv Z^c\]
\end{notation}

\begin{example}[Free action]
	Set $U = \Spec A$. Then this construction yields the actual group quotient.
	The quotient of $\bA^{\times}$ by the free $\mu_\ell$ action gives a scheme.
\end{example}
\begin{lemma}{\label{lemma:algSpacesStabFreeQuots}}
	%For $n \ge 0$, geometric $(n)$-stacks 
	Algebraic spaces are stable by free quotients of covering group 0-stacks.
\end{lemma}
\begin{proof}
	The map $ X \to L_T (X / G)$ is fibered in covering 0-stacks, so in particular covering $0$-stacks. As $X$ is a geometric $0$-stack, the quotient is a geometric $0$-stack as well, This follows by the description in \label{prop:nstack}, choosing a geometric atlas of $X$ and postcomposing this to get a geometric atlas of the quotient.
\end{proof}

\begin{example}{\label{ex:MapOverR}}
 If $p : \sum_{r : R} S_r \to R$ be a map between formally \etale + flat affine into $R$ whose fibers, except possibly the fiber over $0$, are formally \etale + flat. Define $U = (x \neq 0) \times S_x \subset S_x$. %Then for any $x : R$, 
 $y \sharp y' \equiv y \neq y'$ is an irreflexive $\bT$-flat relation on $S_x$. From this we obtain the algebraic space 
 \[\sum_{x : R} Y_x / R_{\sharp} \]
 which we will later recognize as a fiber collapse.
\end{example}
\begin{proof}
	$\sharp$ is a modal irreflexive relation. By assumption we have given $\bT$-flatness of $S_x$ if $x \neq 0$.
\end{proof}
%\begin{prop}
%	Let $p : R[X]^{\mu_\ell}$ with $0 \in D(p)$ % be such that $0 \in D(p)$ and $x \in D(p)$ implies $gx \in D(p)$ for all $g : \mu_\ell$.
%	The sheaf-quotient of $D(p)$ by the relation which identifies $x$ and $gx$ when $x \neq 0, g : \mu_\ell \setminus \{1\}$ is not an affine scheme.
%\end{prop}
%\begin{proof}
%
%\end{proof}

%\begin{prop}
%	Let $X$ be an algebraic space. Let $R : X^2 \to \Prop$ be a covering equivalence relation, such that $\sum_{x ,y :X} R(x,y) \to X^2$ is not a closed embedding.Then the sheaf quotient $X / R $ is not a scheme.
%\end{prop}
%\begin{proof}
%	Affine schemes are seperated.
%\end{proof}

%\begin{lemma}
%	Let $X$ be an algebraic spac presented by $U  / R$ with $R$ valued in locally closed props.
%	Then we find some formally unramified map of algebraic spaces $U' \to X \to Y$ with $U' \to Y$ seperated.
%\end{lemma}
\begin{lemma}[Not needed]
	Let $Y$ be affine.
	Let $X \hookrightarrow Y$ be a map fibered in locally closed propositions. Then its factors as the composite of a closed and then an open embedding
\end{lemma}
\begin{proof}
	By zariski local choice we find $Y = \bigcup Y_i$ and factorizations of the basechanges $X_i \to Z_i \to Y_i$. Then $\bigcup X_i \to \bigcup Z_i \to \bigcup Y_i = Y$ is a global factorization.
\end{proof}



\begin{prop}
   Let $S$ be affine with a regular point $*$.
  Assume we have function $g : S \to S$ such that $*$ is the unique fixpoint $*$ (e.g. if $(S,*)$ admits a pointed-free action of a nontrivial group) 
%	 \begin{itemize}
%	 	\item  it has a unique fixpoint $*$, that is a good point. 
%%	 	\item forall $y : S$, we have 	\[
%%	 	y \neq * \to g y \neq *
%%	 	\]	 	 	
%	 \end{itemize}
 	Let $\sharp$ be an irreflexive $\bT$-flat relation on $U \equiv \sum_{x : S}  x \neq *$, such that for all $y : U$, we have $gy : U$ and $y \sharp g y$.
%	 \begin{enumerate}^
%%	 	\item $x = gx \leftrightarrow x = *$
%	 	\item 
%	 \end{enumerate}
 Then the algebraic space $S / R_{\sharp}$ is non locally seperated, in particular not a scheme.
\end{prop}
\begin{proof}
	It is an algebraic space by the previous prop. \\
		
	A pointed-free action of a non-trivial group yields such a map $g$: If $\lnot (G = \{1\})$, then $\lnot \lnot (G \setminus \{1\})$ by decidable equality of $G$. As we want to prove a contradiction, we may assume $g : G \setminus \{1\}$, this yields a map $S \to S$ such that
	\begin{itemize}
		\item $*$ is the unique fixpoint by the pointed-freeness
		\item If $y \neq *$, then $gy \neq *$ and $y \sharp gy$ 
	\end{itemize} 

	
	%putting $p \equiv 1 : R[X]$ in the previous prop. \\
	
	We have that every scheme $X$ is locally-seperated, i.e. its identity types are locally closed. Indeed, this follows from the proof of Foundations Prop 5.5.2 . \\
	Let us show that $R$ is not valued in locally closed propositions.
%	Note from $y \in U  = y \neq g y$ we deduce
	Recall
	\begin{align}
		y \in U \to y \sharp g y \\
		y \in U \leftrightarrow y \neq *
	\end{align}
	We have for any $y : S$
	\[
	 R_{\sharp}(y,gy) \simeq (y = gy) + (y \in U) \land y \sharp g y  \overset{(1)}{\simeq} (y = *) + (y \in U) \overset{(2)}{\simeq} (y = *) + (y \neq *)
	\]	
	
	But if this is locally closed for all $y : S$, we have a contradiction to \ref{prop:NotLocClosed}.
	
	%Another approach: 	 Through decidable subtypes. %, as $\Spec A$ is non-infinitesimal.
\end{proof}


\begin{corollary}
	Let $Y : R \to \Aff$ be formally \etale + flat affine away from the origin , such that $p : \tilde Y = \sum_{x : R} Y_x \to R$ is regular and $Y_0$ is infinitesimal.
	% such that $\Spec A \setminus \{0\} = D(g)$ for some regular $g : B$. %the infinitesimal neigbhorhood is not open. %  $R^{\Spec A} \to R^{\Spec A \setminus \{0\}}$ is injective.
	If you find a map $g : \tilde Y \to \tilde Y$ over $p$ with a unique fixpoint, which lies over $0$, then  the algebraic space---$Y_\bullet$--- is non-locally-seperated, In particular not a scheme.
\end{corollary}
\begin{proof}
	Lets call the unique fix point $*$, i.e. we have
	\[g y = y  \leftrightarrow y = *\]
	Note that $*  : \tilde Y$ is a regular point , as $p : \tilde Y \to R$ is a regular section with $Y_0$ infinitesimal.
\end{proof}
\begin{definition}
	A pointed-free action of $G$ on a pointed type $(X,0)$ is a $G$-action with fixpoint 0, such that if $g \varepsilon = \varepsilon$ for some $g \neq 1$, then $\varepsilon = 0$. \\
\end{definition}
\begin{lemma}{\label{lemma:StrongFree}}
	Let $G$ be a group with decidable equality acting pointed free on a pointed type $(X,0)$.
	Then $G$ acts free away from zero.
\end{lemma}

\begin{proof}
	let $x , y\neq 0$. We need to show, that $\sum_g g x = y$ is a proposition. Let $g, g' : G$ such that $g x = y$. as $G$ has decidable equality, we may show $\lnot \lnot (g = g')$. If $g^{-1} g' \neq 1$, then by pointed-freeness applied to $g^{-1} g' x = x$, we have $x = 0$. Contradiction.
\end{proof}
\begin{corollary}

	Let $0 : \Spec A$ be a regular point. % such that $\Spec A \setminus \{0\} = D(g)$ for some regular $g : B$. %the infinitesimal neigbhorhood is not open. %  $R^{\Spec A} \to R^{\Spec A \setminus \{0\}}$ is injective.
	Let $G$ be a nontrivial formally \etale flat affine group acting pointed- freely on the pointed affine $(\Spec A,0)$.
	Then the pointed-free quotient of $\Spec A$ by $G$ from \ref{ex:GAction} is non-locally-seperated, In particular not a scheme.
\end{corollary}

%\begin{proof}
%		
%		
%		%Another approach: 	 Through decidable subtypes. %, as $\Spec A$ is non-infinitesimal.
%
%\end{proof}
\begin{example}[Non locally-separated examples]
	Assume $\ell \neq 0$ prime. Let $\mu_\ell$ act on $(\Spec B,0)$ in one of the following ways:
	\begin{enumerate}
		\item Let $\mu_\ell$ act on $\Spec B = \bA^1$
		\item Let $\mu_\ell$ act on
		\[
		\Spec B \equiv \sum_{x , y : R} x^\ell = y^\ell
		\]
		via $g (x,y) = (x,g y)$
	\end{enumerate}
	Then $\Spec B /_{0^c} \mu_\ell$ is an algebraic space that is not a scheme.
\end{example}
\begin{proof}
	$\lnot \lnot$ merely, $\mu_\ell$ is finite (\todocite) and $\mu_\ell \setminus \{1\}$ is inhabited by \ref{lemma:CompOf1}. 
	\begin{enumerate}
		\item Pointed-Free action is clear. $0 : \bA^1$ is regular by first projection.
		\item  Pointed-Free action is clear. The cross middlepoint  regular, witnessed by  the first projection: It is regular vanishing at $(0,0)$ And for any point $(0,y) : \Spec B$ we deduce $y^\ell = -0^\ell = 0$, hence $\lnot \lnot (x,y) = (0,0)$.
	\end{enumerate}
\end{proof}
\begin{question}
	If $\mu_\ell$ acts on $Y$ some affine, does every $\mu_\ell$-invariant $\phi : Y \to R$ is invariant on a $\ell$-neigborhood?
\end{question}
\subsection{Obsolete}
\begin{prop}
	Let $Y : R \to \Aff$ be formally \etale + flat affine away from the origin %, such that $p : \tilde Y = \sum_{x : R} Y_x \to R$ is regular and  (i.e. each two points are $\lnot \lnot$-equal). 
	% such that $\Spec A \setminus \{0\} = D(g)$ for some regular $g : B$. %the infinitesimal neigbhorhood is not open. %  $R^{\Spec A} \to R^{\Spec A \setminus \{0\}}$ is injective.
	If you find two sections $y,y' : \prod_{x: R} Y_x$ such that $y_x = y'_x \leftrightarrow x = 0$, then then the algebraic space---$Y_\bullet$--- is non-locally-seperated, In particular not a scheme.
\end{prop}
\begin{proof}
	It is an algebraic space by the previous prop. \\
	
	%putting $p \equiv 1 : R[X]$ in the previous prop. \\
	
	We have that every scheme $X$ is locally-seperated, i.e. its identity types are locally closed. Indeed, this follows from the proof of Foundations Prop 5.5.2 . \\
	Let us show that $R$ is not valued in locally closed propositions.
	We have 
	\[
	\prod_{x : R} \eta y_x = \eta y'_x \simeq \prod_{x : R} y_x = y_x' + (x \neq 0) \times y_x \neq y'_x \simeq (x = 0) + (x \neq 0)
	\]	
	but if this is locally closed for all $x : R$, we have a contradiction to \ref{prop:NotLocClosed}.
	
	%Another approach: 	 Through decidable subtypes. %, as $\Spec A$ is non-infinitesimal.
\end{proof}
\begin{lemma}[Not needed]
	For an algebraic space $X$, we have implications $1 \Rightarrow 2 \Rightarrow 3$
	\begin{enumerate}
		
		\item $X$ admits an seperated open cover.
		\item For any covering equivalence relation $R : U^2 \to \Prop$ on an affine $U$ such that $ X = U / R$, $F$ is valued in locally closed propositions
		\item We find such a presentation  such that $R$ is valued in locally closed propositions.
	\end{enumerate}
\end{lemma}
\begin{proof}
	\begin{enumerate}
		\item [1 $\Rightarrow$ 2]
		Let $X' \to X$ be a map fibered in merely inhabited finite sums of open propositions with $X'$ a seperated algebraic space. Then any geometric atlas $U \to X'$ will be fibered in closed subtypes of $U$. We need to show, that the fibers of $U \to X' \to X$ are locally closed subtypes of $U$. Let $x : X$. the fiber in $X'$ is of the form $U_1 + \hdots + U_n$. Thus the fiber in $U$ is a finite sums of $\sum$ of $U_i \to (U \to \mathsf{ClosedProp})$, which is enough.
		\item [3 $\Rightarrow$ 1] Let $x : X$. 
		
	\end{enumerate}
\end{proof}
\begin{lemma}[Not needed]
	Let $char \neq 2$.
	Let $p : R[X]$ be such that $0 \in D(p)$ and $x \in D(p)$ implies $-x \in D(p)$. 
	If $f : R[X]$ is a polynomial such that $f(x) = f(-x)$ for all $x : D(p) \setminus \{0\}$, then $f$ is even i.e. in the image of $R[X^2] \hookrightarrow R[X]$.
\end{lemma}
\begin{proof}
	We splitting $f$ into $f_1 + X f_2$ for $f_i : R[X^2] \subset R[X]$. I claim, that $f_2 = 0$ in $R[X]$. realizing that $(X f_2)(x) = (X f_2)(-x)$ implies $2 f_2(x) x  = 0$, thus $f_2(x) x = 0$ for all $x : D(p) \setminus 0 = D(pX)$, thus by the previous lemma $X \cdot f_2 =0$ in $R[X]$, hence $f_2 = 0$. 
\end{proof}
\begin{lemma}
	Let $G$ be a finite group whose cardinality is invertible in $R$. Let $G$ act on an affine scheme equipped with a fixpoint $0$. Let $U$ be an open neighborhood of 0, such that $g(U) = V$ for all $g : G$. Then we find some $G$-invariant $p$ such that $0 \in D(p) \subset V$. %Let $A$ be a finitely presented algebra and $0 \in \Spec A$ a basepoint. 
\end{lemma}
\begin{proof}
%	Let $U$ be an invariant open neighborhood. 
Choose a principal open neighborhood $0 \in D(p) \subset U$. $G$ acts on $R[X]$, via $(g.p)(x) = p(g x)$.Then 
	\[p' = \sum_{g : G} g . p : R[X]\]
	is a $G$-invariant polynomial, in particular $D(p)$ is $G$-invariant. Moreover $0 \in D(p')$ as
	\[
	p'(0) = \sum_{g : G} p(g(0)) = \sum_{g : G} p (0) = |G| \cdot p(0) 
	\]
	is invertible, as $|G|$ and $p(0)$ are both invertible. Furthermore, as $U$ was $G$ invariant and contained $D(p)$ it also has to contain $D(p')$: Indeed
	\[
	D(p') \subset \bigcup_g D(g . p) \subset U
	\]
\end{proof}
\begin{lemma}
	Let $G$ be a formally \etale + flat affine  group, such that $\lnot \lnot$ its finite, with cardinality invertible in $R$ and $G \setminus \{1\}$ inhabited. 
	Let it act on an affine scheme $\Spec A$ with a fixpoint 0. Let $R$ be a relation on $\Spec A$ such that
	\begin{itemize}
		\item $R(x,y)$ implies that there merely is some $g$ with $y = gx$. 
		\item $\lnot \lnot R(x,gx)$
	\end{itemize} Assume that for all $p : A^G$ with $0 \in D(p)$, $D(p) / R$ is not an affine scheme. Then $\Spec A / R$ is not a scheme. \\
	
\end{lemma}

\begin{proof}
	Assume $0$ admits a open affine neibhorhood $U$ in $\Spec A / R$.  The preimage along the quotient map obtained from the relation induces a open neigbhorhood $V$ of $0$ in $\Spec A$. As we want to prove a contradiction we may assume that $\mu_\ell$ consists of $\ell$ many elements, where $\ell \neq 0 $ in $R$. Note that $V$ is $G$-invariant: For any $x \in V, g :G$, the goal $g x \in V$ as an open proposition is $\lnot \lnot$-stable , thus we may assume $R(x,gx)$. \\
	We apply the previous lemma to $V$ to obtain an invariant principal open neigborhood $0 \in D(p) \subset V \subset \Spec A$. As $p$ is $G$-invariant, $p : \Spec A \to R$ descends to $X \to R$. Restricting to $U'$ yields a map $p' : U \to R$, such that setting $U' \equiv D(p')$ yields $q^{-1}(U') =q^{-1}(D(p')) = D(p' \circ q) =  D(p)$ . We are now in the following situation
% https://q.uiver.app/#q=WzAsNixbMCwwLCJEKHApIl0sWzIsMCwiXFxTcGVjIEEiXSxbMiwxLCJYIl0sWzAsMSwiVSciXSxbMSwwLCJWIl0sWzEsMSwiVSJdLFswLDNdLFsxLDIsInEiXSxbMCwyLCIiLDEseyJzdHlsZSI6eyJuYW1lIjoiY29ybmVyLWludmVyc2UifX1dLFswLDQsIiIsMCx7InN0eWxlIjp7InRhaWwiOnsibmFtZSI6Imhvb2siLCJzaWRlIjoidG9wIn19fV0sWzQsMSwiIiwwLHsic3R5bGUiOnsidGFpbCI6eyJuYW1lIjoiaG9vayIsInNpZGUiOiJ0b3AifX19XSxbMyw1LCIiLDIseyJzdHlsZSI6eyJ0YWlsIjp7Im5hbWUiOiJob29rIiwic2lkZSI6InRvcCJ9fX1dLFs0LDVdLFs1LDIsIiIsMCx7InN0eWxlIjp7InRhaWwiOnsibmFtZSI6Imhvb2siLCJzaWRlIjoidG9wIn19fV0sWzQsMiwiIiwwLHsic3R5bGUiOnsibmFtZSI6ImNvcm5lci1pbnZlcnNlIn19XV0=
\[\begin{tikzcd}
	{D(p)} & V & {\Spec A} \\
	{U'} & U & X
	\arrow[hook, from=1-1, to=1-2]
	\arrow[from=1-1, to=2-1]
	\arrow["\ulcorner"{anchor=center, pos=0.125}, draw=none, from=1-1, to=2-3]
	\arrow[hook, from=1-2, to=1-3]
	\arrow[from=1-2, to=2-2]
	\arrow["\ulcorner"{anchor=center, pos=0.125}, draw=none, from=1-2, to=2-3]
	\arrow["q", from=1-3, to=2-3]
	\arrow[hook, from=2-1, to=2-2]
	\arrow[hook, from=2-2, to=2-3]
\end{tikzcd}\]
	where $U'$ is an open affine neighborhood of 0. \\
	By assumption $U = D(p) / \sim'$ cannot be affine. Contradiction. 
\end{proof}
%\begin{example}
%	Let $\mu_\ell$ act on $\bA^1$ by multiplication with $\ell \neq 0$. Then $\bA^1 / \mu_\ell$ is not a scheme. We dont know whether it is an algebraic space
%\end{example}
%\begin{proof}
%
%\end{proof}

\begin{prop}[Not needed]
	Let $\ell \neq 0$ be prime. Let $\mu_\ell$ act on $\Spec B$ with fixpoint $0$. . Let $V$ be an infinitesimal neigborhood of $0$, i.e. a subtype   $0 \in V \subset \Spec B$ such that $\lnot \lnot x =0$ for every $x : V$. Assume
	\begin{enumerate}
		\item[Strong freeness]  We find some $0 \in V' \subsetneq V$ for any $\varepsilon : \Spec B, g\neq 1$, $g \varepsilon = \varepsilon$ implies $\varepsilon \in V'$

	\item[checking away from 0] For any $p : B$ and any $\phi: R^{D(p)} $ such that $\phi|_{D(p) \setminus\{0\}} = 0$, we have that $\phi|_V = 0$.
		\end{enumerate} 
	The sheaf quotient of $\Spec B$ by the relation as above is an algebraic space but not an affine scheme.
\end{prop}

\begin{proof}
	\begin{itemize}	
\item 	Let us check the conditions on the relation
	\begin{itemize}
		\item If $R(x,y)$ then either $x = y$ putting $g = 1$ or in the second case we get some $g$ such that $g x =y $
		\item Let $x : X, g : G$. Assume $\lnot R(x,gx)$, i.e. $x \neq g x$ and $\lnot \lnot x=0$.But $0$ was assumed to be a fixpoint, hence $\lnot \lnot g x = x$.
	\end{itemize}
\item 	Let $p : B$ be as above. We have to show that the quotient of $D(p)$ is not affine. \\
	The conditions on $p$ give $p(0) \neq 0$ and $p(x) \neq 0 \to p(gx) \neq 0$ for all $g : \mu_\ell$.
	
	Lets call this quotient $X$.
	
	Define 
	\[
	A = \{\phi : R^{D(p)} \ | \ \phi|_{D(p) \setminus \{0\}} \text{  is $\mu_{\ell}$-invariant }\}
	\]
	This is an $R$-subalgebra: for any $r : R$, $r : B_p$ is $\mu_{\ell}$-invariant. $\mu_{\ell}$-invariant functions are stable under addition and multiplication . \\
	
	Claim: The affinization map of $X$ is the induced dashed map $f : X \to \Spec A$ in
	
	% https://q.uiver.app/#q=WzAsNCxbMCwwLCJEKHApIl0sWzEsMCwiXFxTcGVjIFJbWF1fcCJdLFswLDEsIlgiXSxbMSwxLCJcXFNwZWMgQSJdLFswLDIsInEiXSxbMiwzLCJcXGV4aXN0ISBmIiwwLHsic3R5bGUiOnsiYm9keSI6eyJuYW1lIjoiZGFzaGVkIn19fV0sWzEsM10sWzAsMSwiIiwyLHsibGV2ZWwiOjIsInN0eWxlIjp7ImhlYWQiOnsibmFtZSI6Im5vbmUifX19XV0=
	\[\begin{tikzcd}
		{D(p)} & {\Spec R^{D(p)}} \\
		X & {\Spec A}
		\arrow[Rightarrow, no head, from=1-1, to=1-2]
		\arrow["q", from=1-1, to=2-1]
		\arrow["q'",from=1-2, to=2-2]
		\arrow["{\exists! f}", dashed, from=2-1, to=2-2]
	\end{tikzcd}\]
	Proof: A function $\phi : D(p) \to R$ factors through $q$ iff $\phi|_{D(p) \setminus\{0\}}$ is $\mu_{\ell}$-invariant. Thus the embedding (using that $R$ is a sheaf) $R^X \hookrightarrow R^{D(p)}$ has image $A$ $\qed$(Claim). 	\\ \\ %respective $\phi : R[X]_p$ satisfies $	\phi (x) = \phi(-x)  $, i.e. (1) if $\phi$ is $\mu_{\ell}$-invariant. 
	Proof that $X$ is not an affine:	Assume that $X$ were affine. Then the map $f$ would be in particular an embedding. 
	We may assume a term $g : \mu_\ell \setminus \{1\}$: Indeed, as we want to prove a contradiction we may assume a term in $g : \Spec R[X] / (\sum_{i=0}^{\ell-1} X^i)$. But this type is equivalent to $\mu_\ell \setminus \{1\}$, using that $\sum_{i=0}^{\ell-1} X^i | X^\ell -1 $ and $\ell \neq 0$. \\
	 %$\cN_{\lnot \lnot}(0) = \{x : \bA^1 \ | \ \lnot \lnot x = 0\}$ be be a non-contractible subtype that is $\lnot \lnot$-connected. 
	The given infinitesimal neigbhorhood $V$ satisfies $ V \subset D(p)$ , using that invertibility is $\lnot \lnot$ stable.
	Then for any $\varepsilon : V$ we have
	\begin{align*}
		(q\varepsilon =_X q (g \varepsilon)) \overset{\ref{quotient-by-equivalence-relation}}{=} (\varepsilon = g\varepsilon) + (\varepsilon \neq 0 \land \sum_{h \neq 1} \varepsilon = h g \varepsilon) = (\varepsilon = g\varepsilon) = (\varepsilon \in V')
	\end{align*}
	where the last step comes from pointed-freeness.
	But we have 
	\[(q' \varepsilon =_{\Spec A} q' (g \varepsilon)) = \left (\prod_{\phi : A} \phi(q' \varepsilon) = \phi(q' (g \varepsilon)) \right)= \prod_{\substack{\phi : R^{D(p)} \\ \phi \in A}} \phi (\varepsilon) = \phi(g \varepsilon),\] 
	%as for any $\phi : A$ we have $\phi(\varepsilon) = \phi(g \varepsilon)$ 
	The right hand side is inhabited: For any $\phi: D(p) \to R$ such that $\psi := \phi - g . \phi$ satisfies $\psi|_{D(p) \setminus \{0\}} =0$  we have $\psi|_V =0$ by 'checking away from 0', inparticular $\psi(\varepsilon) =0$ . 	So we conclude the the embedding $V' \hookrightarrow V$ is an equivalence.  But we asked $V' \subsetneq V$ to be a proper subset.% I claim that the right hand side is inhabited:
	\end{itemize}
\end{proof}

\begin{example}
	Let $\mu_\ell$ act on $\Spec B = \bA^1$.
\end{example}
\begin{proof}
	\begin{enumerate}
		\item Put $ V:\equiv\Spec R[X] / X^n$ for some $n >1$.
		\item As $(g-1)$ is invertible, $ ((g-1)\varepsilon = 0) $  gives us $\varepsilon \in \{0\} \equiv V' \subsetneq V$. Note that indeed $V$ is non contractible, because $R[X] / X^n \to R[X] / X$ is not an algebra isomorphism
		\item 	We have to show, that then $\phi$ is $\mu_{\ell}$ invariant. We can apply  \ref{lemma:AlmostEverywhere}, observing $\phi - g.\phi = 0$ on $D(X / 1) \subset \Spec B_p$, where $X/1 : B_p$ is regular, because $X$ is regular in $B$. TODO as each $\phi$ satisfies the cond. \qed(Claim)\\
	
	\end{enumerate}
\end{proof}
\begin{example}
	Assume $2 \neq 0$. Let $\mu_2$ act on
	\[
	\Spec B \equiv \sum_{x , y : R} x y = 0
	\]
	via the swap. Then $\Spec B / R$ is an algebraic space but not a scheme.
%	sheaf-quotiented by the relation that identifies $(x,0)$ and $(0,x)$ if $x \neq 0$ is an algebraic space.
\end{example}
\begin{proof}
	\begin{enumerate}
		\item Put $V = \Spec R[X] / X^k \subset \Spec B$, $k > 2$.
		\item If $(x,y) = (y,x)$ but $x y = 0$ we get $x \in V' \equiv \Spec R[X] / X^2$.
		\item Let $\phi: D(p) \to R$ be 0 everywhere except near the origin. Then we get a restricted map $\phi' : D(p') \to R$ where $D(p') \subset V(X)$ is given by the intersection $D(p) \cap V(X) $ . Indeed : Put $p' : R[X]$ the image of $p : R[X,Y] / (XY)$ und the map induced by evaluating $Y$ at 0. \\
		Here we can apply \ref{lemma:AlmostEverywhere}, getting that $\phi'$ is 0 everywhere in particular in $V \subset V(X)$.
	\end{enumerate}
%	The equivalence relation is given by
%	\[
%	E((x,y) , (x',y')) = (x = x' \land y = y') + (x \neq 0 \land x = y' \land x' = 0)
%	\]
%	as $x \neq 0$ implies $y = 0$ as $x y =0$. This is a covering relation, as for any $x' y' = 0$ we have
%	\[
%	\sum_{x,y : R} x y = 0 \land E((x,y) , (x' , y')) = 1 + (y' \neq 0) \in \Zar \subset \bT
%	\]		
\end{proof}
\subsection{Locally seperated examples}
\begin{lemma}[not needed]
	Given a map $P : \Susp(Q) \to \Prop$, such that $P(N)$ and $P(S)$ hold, then $\prod_{t: \Susp(Q)} P(t)$
\end{lemma}
\begin{lemma}[not needed]
	Assume $2 \neq 0$. For any $x : R$, the map
	\begin{align*}
		\Susp(x \neq 0) &\to  \sum_{y: R / x} y^2 = 1  \\
		N &\mapsto 1 & \\
		S &\mapsto -1 & 
	\end{align*}
	is well-defined and an equivalence.
\end{lemma}
\begin{proof}
	The following maps are mutually inverse
	\begin{align*}
		\sum_{y: R / x} y^2 = 1 &\simeq \sum_{e : R / x} e^2 = e \\
		y &\mapsto (y-1) / 2  \\
		2e - 1 &\mapsfrom e
	\end{align*}
	
	So it remains to show  that the map
	\begin{align*}
		f : \Susp(x \neq 0) &\to \sum_{e : R / x} e^2 = e \\
		N &\mapsto 1 & \\
		S &\mapsto 0 & 
	\end{align*}
	is a bijection. \\
	\begin{itemize}
		\item It is injective, i.e. for all $p , q : \Susp(x \neq 0)$, if $f(p) = f q$, then $ p =q$. As the latter is a proposition, we may assume $p , q$ beeing combinations of north and south poles. The interesting case is if wlog $p = N, q = S$. Then assuming $0 =_{R/x} 1$ means $R / x = 0$, i.e. $x \neq 0$, thus $N = S$ in $\Susp(x \neq 0)$.
		\item It is surjective: Choose $e : R$, such that $e^2 = e$ in $R / x$.  By locality in $R$,  $e$ or $1 - e$ is invertible in $R$, thus in $R / x$. By $e^2 = e$ we deduce $e = 0$ or $e = 1$ in $R / x$, both lie in the image of $f$.
	\end{itemize}
\end{proof}
\begin{example}[Not needed]
	Let $L = \sum_{x : \bA^1} \Susp(x \neq 0) = \sum_{x : \bA^1} \sum_{y : R / x} y^2 =_{R/x} 1$ be the line with two origins. 	
\end{example}
\begin{lemma}[Not needed]
	Let $2 \neq 0$. Let $y , y' : A$ be two elements of an fp-algebra, whose squares coincide and such that $y$ is invertible. Then $y =_A y'$ is formally \etale
\end{lemma}
\begin{proof}
	We may assume that $A = R$, as equality in $A$ can be checked pointwise and formally \etale is a modality.
	We may show its $\lnot \lnot$-stable. Assume $\lnot \lnot (y =_{R} y')$, i.e. $y - y'$ beeing nilpotent in $A$. So pick $n$ large enough, such that $(y-y')^{2^n} =0$. Proof by induction over $n$\\
	If $n =0$, then its fine. 
	Induction step $n \mapsto n+1$. %Assume now, that for any such $y,y'$ we have that $(y-y')^{2^n} =0$ implies $y = y'$.
	Let $(y-y')^{2^{n+1}} =_{R} 0$, then $(2y^2 - 2yy')^{2^n} = 0$, or $(y(y - y'))^{2^n} = 0$, as $y$ is invertible, $(y - y')^{2^n} = 0$, so by induction hypothesis $y = y'$. \\
\end{proof}
\subsection{FiberCollaps away from the origin}
OUTDATED!
%\begin{definition}
%	Given a sheaf proposition $P$, the closed modality in the sheaf topos associated to $P$ is given by
%	\begin{align*}
%		\cU &\to \cU \\
%		X &\mapsto P \star X :\equiv L_\bT(P \sqcup_{P \times X} X)
%	\end{align*}
%	where we need to stackify the pushout. A type $X$ is modal, if it is a sheaf and $P \to \isContr(X)$.
%\end{definition}
%\begin{definition}
%	Let $Y : R \to \Aff$ be a dependent family of affines, such that $(Y \in \bT)^{x \neq 0}$ . 
%	The fiber collapse of $Y$ away from the origin $\emdash Y \emdash $ is the space over $R$
%	\[
%	\sum_{x : R} (x \neq 0) \star Y_x \to R %\sum_{x : R} Y_x \to 
%	\]
%%	: \prod_{x : R} \sum_{Y:\Aff} (Y \in \EF)^{x \neq 0}$ %
%\end{definition}
%This space over $R$ looks exactly like the line away from the origin and over an infinitesimal $\varepsilon$ the fiber is $Y_\varepsilon$.
%\begin{lemma}
%	$\emdash Y \emdash$ is an algebraic space.
%\end{lemma}
%\begin{proof}
%	Let $x : R$. Let $Y : \Aff$ such that $x \neq 0$ implies that $Y$ is formally \etale + flat. We will show that $\eta : Y \to (x \neq 0) \star Y$ is the sheaf-quotient map of the relation on $Y$ given by $y \sim y' \equiv (y = y') + (x \neq 0) \times y \neq y'¸$, which is enough by \ref{ex:MapOverR}. We apply  \ref{lemma:quotient-by-equivalence-relation} \\
%	\begin{itemize}
%		\item The map is $\bT$-surjective: We have a $\bT$-surjection $(x \neq 0) + Y \to (x \neq 0) \star Y$. In case $x \neq 0$, the map of interest is $Y \to 1$, which is $\bT$-surjective, as then $Y \in \bT$. 
%	\item Given $y,y' : Y$, we have
%	\begin{align*}
%		\eta(y') = \eta(y) &\simeq (x \neq 0) \star (y = y')&& \ | \  \text{ closed modality is lex (\cite{Modalities} Example 3.1.4).} \\
%		&\simeq L_\bT\left((y = y') \lor (x \neq 0) \right) && \ | \ (x \neq 0) \to \mathsf{HasDecEq}(Y) \\
%		&\simeq (y = y') + (x \neq 0) \times y \neq y'¸
%	\end{align*}
%	\end{itemize}
%%JUMP
%%	Hence the fiber over $y$ is 
%%	\[
%%	1 + (x \neq 0) \times Y \setminus \{y\} = 1 + \sum_{p : x \neq 0}Y \setminus \{y\}
%%	\]
%%	which is inhabited affine and formally \etale + flat. %, because the right summand is a $\sum$ of formally \etale + flat affines: Indeed, whenever $x \neq 0$ then $Y \setminus \{y\}$ is formally \etale + flat.
%	%	 As $y = y'$ is open by assumption the fiber is merely inhabited and a $\sum$ of $\bT$-flat geometric stacks, thus covering. 
%	%	Let us show, that the fiber over $\Delta(y)$ is $1 + (Y - y) \times (x \neq 0)$ which is formally \etale + flat and inhabited affine.	Here we use, that $Y$ has decidable equality and if $y \neq y'$, then % we have $inl(y) = inl(y') \simeq (x \neq 0) \star (y = y') = (x \neq 0)$. \\
%\end{proof}
\begin{example}
	$\emdash Bool \emdash$ is the line with two origins. \\
	---$\Spec R[X] / (X^2 + 1))$--- is the twisted line with two origins, i.e. over the origin we have the roots of $-1$. \\
	---$\Spec R[Y] / (Y^2 - \bullet^2)$--- is the quotient of the cross,  that looks like $\bD(1)$ over the origin. \\\
	---$\Spec R[Y] / (\bullet Y)$--- is the affine Plus.
\end{example}
\subsection{Schemes do not have descent}
For this section, let $\rho : R \setminus \{0\}$ denote a term, e.g. $\rho = 1$. Set $C = R[T] / (T^2 + \rho)$.

\begin{prop}[OUTDATED! Copy from talk!]
	If ---$\Spec C$--- is a scheme, then $X^2 + \rho$ has a root.
\end{prop}
%\begin{proof}
%	Let $p : \emdash \Spec C \emdash \to R$ be the first projection.
%	We proceed as follows
%	\begin{enumerate}
%		\item  There is no open affine subset of $L(\Spec C)$ containing $\fib_p(0)$.
%		\item Any cover of $\Spec C$ by open subsets strictly smaller than $\Spec C$ yields a root.
%	\end{enumerate}
%	Proves:
%	\begin{enumerate}
%		\item Because we want to show a sheaf, we may assume $L(\Spec C) = L(2)$. Assume there is an open affine subset $\fib_p(0) \subset U \subset L (2)$. Then $p(U) \subset R$ is an open neigbhorhood of 0, as 
%		\[
%		x \in p(U) \leftrightarrow (x,N) \in U \lor (x,S) \in U
%		\]
%		Claim: the map $R^{p(U)} \to R^U$ is an equivalence. If we have shown that: As $U$ is affine we conclude that the map
%		\begin{align*}
%			U &\to \Spec (R^{p(U)}) \\
%			x &\mapsto \phi \mapsto (\phi(px))		
%		\end{align*}
%		is an equivalence, which is a contradiction to the assumption, that $U$ contains both origins. \\
%		Proof of claim:
%		Injectivity: If two maps $f , g : p(U) \to R$ coincide after precomposing with $U \to p(U)$, then they coincide away from $0$
%		so conclude by \ref{lemma:AlmostEverywhere}. \\
%		Surjectivity: Given a map $U \to R$, by pulling back along $p : R + R \to L(2)$ we can view it as a map $R + R \supset U' \rightsquigarrow R$ defined at both origins, so in particular as a pair of maps to $R$ defined on some open neigbhorhood of 0 of $R$. They coincide away from 0 so by \ref{lemma:AlmostEverywhere} they are equal.
%		%$U \setminus \fib_p(0)$ is an open subset of $L(2) \setminus \fib_p(0) \simeq R^{\times}$. 
%		\item As $\Spec C$ is an \etale sheaf, its enough to show to construct a function $\|\Spec C\| \to \Spec C$. \\
%		Define 
%		\begin{align*}
%			A : \Spec C &\to \Prop^{\Fin(n)}  \\
%			x &\mapsto \{ j : x \in U_j\}
%		\end{align*}
%		where we transport $x : \Spec C$ along $\Spec C \overset{\simeq}{\to} \fib_p 0$.
%		Observe 
%		\begin{enumerate}
%			\item for any $x,x' : \Spec C$, 
%			\[\| A x \cap  A x' \| \to \lnot \lnot (x = x') \overset{DecEq(\Spec C)}{\to} x = x'\] where the first implication follows like this : if $x \neq x'$ and $\fib_p 0 = \{x , x'\} \in U_j$ then we have a contradiction to the first point 
%			\item For any $x : \Spec C$ , $\| A x\|$.
%		\end{enumerate}
%		Assume $\|\Spec C\|$.
%		Lets try to construct a term of the following type
%		\[
%		\sum_{x : \Spec C} \forall x' : \Spec C , j : A x, j' : A x' \to j \le j' 
%		\]
%		For this we may assume $\Spec C = Bool \equiv \{N,S\}$, as the above type is a proposition: If we have given two such maximal $x_1, x_2$, we can set first $x' \equiv x_2$ and then $x' \equiv x_1$ respectively and then (by (b)) choosing $j : A_{x_1},j' : A_{x_2}$ gives $j \le j' \le j$ so that $j : A_{x_1} \cap A_{x_2}$ such that $x_1 = x_2$ by (a). \\
%		I will explain an algorithm to do the following:
%		Given $n \ge 2$, and a pair of  merely inhabited disjoints subsets $A_N$ and $A_S$ of $\Fin(n)$, we can decide in which of the two we find the smaller number of $\Fin(n)$. \\
%		Induction over $n$. If $n = 2$, then $\|A_N\|$, so we find a term in the proposition $(0 \in A_N) + (1 \in A_N)$ . In the left case return $S$, in the right case $N$. \\
%		For $n \mapsto n+1$, if $n \in A_N$ return $S$. If $n \in A_S$, return $N$. Otherwise both $A_N$ and $A_S$ are subset of $\Fin(n)$, so conclude by induction.  
%	\end{enumerate}
%\end{proof}
\begin{corollary}
	Schemes do not have descent.
\end{corollary}
\begin{proof}
	If Schemes have descent, then ---$\Spec R[T]/(T^2 + \rho) \emdash \in \mathsf{Sch}$ is a sheaf. As ---$\Spec R[T]/(T^2 + \rho) \emdash$ is $\bT$-merely a scheme, it is a scheme, so by the previous lemma $T^2 + \rho$ has a root. Contradiction to \cite{cherubini2023foundationsyntheticalgebraicgeometry} A . 0.3. \\
\end{proof}
\subsection{Gluing in an affine on the line}

\begin{definition}
	Let $Y$ be an affine. The $n$.th order gluing of $Y$ on the line is given by the sheaf\[
	L_n(X) = \sum_{x : R} Y^{x^n = 0}
	\]
\end{definition}
\begin{lemma}
	If $Y = \Spec R[T]/f$, we have
	\[
	L_n(X) = \sum_{x : R} \sum_{y : R / x^n} f(y) =_{R/x^n} 0
	\]
	
\end{lemma}
\begin{proof}
	For any $R$-algebra $A$ (e.g. $R / x^n$) we have by the universal property of $R[T] / f$
	\[
	\sum_{y : A} f(y) =_A 0 = \Hom_R(R[T] / f , A) = Y^{\Spec A} %  \Hom_A(A \otimes R[T] / f , A) =
	\]
\end{proof}
\begin{lemma}
	If $Y$ is formally \etale, then the map over $R$
	% https://q.uiver.app/#q=WzAsMyxbMCwwLCJSIFxcdGltZXMgWSJdLFsyLDAsIkxfbihZKSJdLFsxLDEsIlIiXSxbMCwyXSxbMCwxXSxbMSwyXV0=
	\[\begin{tikzcd}
		{R \times Y} && {L_n(Y)} \\
		& R
		\arrow[from=1-1, to=1-3]
		\arrow[from=1-1, to=2-2]
		\arrow[from=1-3, to=2-2]
	\end{tikzcd}\]
	pulls back to an equivalence over $\cN_\infty(0)$. \\
	If $Y$ is formally unramified, then $L_n(x)$ is locally seperated.
\end{lemma}
\begin{proof}
	Indeed, the diagonal map 
	\[
	Y \to Y^{x^n = 0}
	\]
	is an equivalence, as for any $\lnot \lnot x = 0$, $x^n = 0$ is a closed dense proposition and $Y$ is formally \etale. \\
	If $Y$ is formally unramified, then the identity types look like
	\[
	(x,y) =_{L_n(Y)} (x',y') \simeq (x = x') \times (x^n = 0 \to Q)
	\]
	where $Q$ is an open proposition such that for any $p : x^n = 0$ we have $Q \equiv  y p = y' p$. Indeed by the proof of \ref{lemma:OpenIsSmooth} we can find a filler of $y_\bullet = y'_\bullet : P \to \mathsf{Open}$. By \cite{cherubini2023foundationsyntheticalgebraicgeometry}(4.2.11) this proposition is locally closed.	
\end{proof}
\begin{question}
	Is the map $\sum_{y : R / x^3} y^2 = 0 \to \sum_{y : R / x^2} y^2 =_0$ surjective? This is how i understand \href{http://www.madore.org/~david/weblog/d.2013-09-21.2160.definition-schema.html#d.2013-09-21.2160.droite-origine-doublee}{David Madore}.
\end{question}
\begin{lemma}
	For $\varepsilon : \cN_\infty(0)$, the affine $Ann(\varepsilon) = \{x : R \ | \  x \varepsilon = 0 \}$ is not$_\varepsilon$ formally smooth. In particular $R \to R / \varepsilon$ is not$_\varepsilon$ a geometric cover.
\end{lemma}
\begin{proof}
	We have the map $1 : (\varepsilon = 0) \to \Ann(\varepsilon)$. Assume there is a filler $x : \Ann(\varepsilon)$, i.e. $(\varepsilon = 0) \to x = 1$. Then not not, $x = 1$, i.e. $(x-1)^n = 0$ for $n$ large enough. Hence
	\[
	0 = \varepsilon (x-1)^n = \varepsilon x (\hdots) + (-1)^n \varepsilon = (-1)^n \varepsilon
	\]
	as desired.
\end{proof}


\begin{lemma}[TODO]
	If $Y$ is formally \etale + flat affine, then $L_1(Y)$ is an algebraic space.
\end{lemma}
\begin{proof}
	Recall the closed modality associated to a proposition $P$, given by $P \star \_$.
	We can define a map
	\begin{align*}
		f : (x \neq 0) \star Y &\to Y^{x = 0} \\
		y &\mapsto \Delta (y) \\
	\end{align*}
	where we check, that if $x \neq 0$ holds, then indeed $Y^{x = 0}$ is contractible. \\

	$f$ is a bijection: \\
	\begin{itemize}
		\item injectivity: Given two terms of the domain, as the map out of $Y$ is $\bT$-surjective (and the goal is a sheaf), we may assume that they are of the form $inl(y) , inl(y')$ for $y,y' : Y$. Then if $\Delta(y) = \Delta(y')$ we have $(x = 0) \to (y = y')$. As $y = y'$ is open, we have $(x \neq 0) \lor (y = y')$. If $x \neq 0$, then $inl(y) = inl(y')$ by the construction of the join.
		\item surjectivity: TODO
	\end{itemize}

\end{proof}
\begin{question}
	Is $L_2(\bD(1))$ an algebraic space or fppf-geometric 0-stack? For this:
	Is 
	\begin{align*}
		(\Spec R[X,Y] / X^2 - Y^2)/\sim &\to L_2(\bD(1)) = \sum_{x : R} \sum_{y : R /x^2} y^2 = 0 \\
		(x,y) &\mapsto (x , [y])
	\end{align*}
	an equivalence? Here we mod out the relation generated by $(x,-x) \sim (x,x) \forall x \neq 0$. \\
	
	This is equivalent to : For any $x : R$, is the map
	\[
	(x \neq 0) \star \Spec R[Y] / (Y^2 - x^2) \to \bD(1)^{x^2 = 0}
	\]
	an equivalence?
\end{question}
\begin{example}
	I suggest a new definition of fppf topology: We take the topology generated by the Zariski topology and algebras of the form $R[X] / f$ where one of coefficients of $f$ is invertible (non necessarily the leading coefficient). This is still a free module hence fppf. \\
	%then one can show, that every solution $e^2 -e
\end{example}
%\begin{example}
%	What happens for\[\sum_{x: R} \sum_{y: R / x^2} y (x-y) =_{R/x^2} 0 \]
%	where we apparently glue in $\bD(1)$ at the origin and leave the line invariant away from the origin?
%\end{example}
