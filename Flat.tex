\section{Flat}


\begin{definition}
	Denote $\Top$ the topologies containing $\mathsf{Bool}$, e.g. finer than the Zariski-topology.
	Let $\mathsf{FLAT}$ consists of all the classes of affines $\bP$ containing $1, \bot$ stable under $\sum$. \\
	Given $\bP: \mathsf{FLAT} , \bT: \Top$ we say $\bP$ flattens $\bT$ iff ($\bT \subset \bP$ and)
	\[
		\bT= \{X : \bP\ | \ \|X\|_\bT \}
	\]
\end{definition}
The goal of this section is to prove the following theorem
\begin{theorem}{\label{thm:Flat}}
 \
	\begin{enumerate}
		\item There is at most one $\bP$ that flattens a topology. Then we say, the topology is flatten.
		\item A topology can be idempotently flattened without changing the stacks
		\item For any $\bP: \mathsf{FLAT}$ and any Lavwere Tierney Operator $j$, $\{ X: \bP \ | \ \|X\|_j \}$ is flattened by $\bP$.
	\end{enumerate}
\end{theorem}
We first want to show the power of this theorem.
\begin{example}
	finite sums of principal opens flatten the Zariski topology. 
\end{example}
\begin{example}
	flat affines flatten the fppf topology. %Beeing flat if fppf-local.
\end{example}
\begin{proof}
	Indeed we can either put $j = \lnot \lnot$ or $j$ the fppf sheafification, because TFAE
	\begin{enumerate}
		\item  $X$ is flat and fppf-merely inhabited
		\item $X$ is flat  and $\lnot\lnot$-inhabited
		\item $X$ is fppf
	\end{enumerate}
	
	\begin{proof}
		$1 \Rightarrow 2 \Rightarrow 3 \Rightarrow 1$ \todocite
	\end{proof}	 
	
\end{proof}

\begin{example}
	 formally \etale + flat affines flatten the \etale topology.
	For the etale topology $=$ formally \etale $+$ fppf, we can put $\bP$ = formally \etale + flats. %$\bP$ is local.
\end{example}
\begin{proof}
	By the same argument as above. \\
%	The locality: 
\end{proof}
%\begin{lemma}{\label{lemma:coveringDMstacks}}
%	Any formally \etale, \etale-merely inhabited Deligne-Mumford-stack is covering.
%\end{lemma}
%\begin{proof}
%	Todo
%\end{proof}

\begin{lemma}{\label{lemma:TruncOfP}}
	Assume $\bT$ is flatten.
	If $X$ is $\bT$-flat geometric stack, then $\|X\|_{\bT}$ is a geometric prop.
\end{lemma}
\begin{proof}
	
	If $\Spec A$ is $\bT$-flat, then $\Spec A$ is weakly-flat, i.e $\| \Spec A\|_{\bT}$ is a geometric prop.
\end{proof}
\begin{lemma}{\label{lemma:detectCovering}}
	Assume $\bT$ is flatten.
	A stack is covering iff it it a $\bT$-flat geometric stack and $\bT$-merely inhabited.
\end{lemma}

\begin{lemma}
	Assume $\bT$ is flatten.
	If $X$ is a covering stack and $Y$ a $\bT$-flat geometric stack, then $X + Y$ is covering
\end{lemma}
\begin{proof}
	Let $\bP$ flatten $\bT$.
	Let ${\bP} \ni \tilde X \to X, \tilde Y \to Y$ be geometric atlasses. Then $\tilde X+ \tilde Y$ is $\bP$ and ${\bT}$-merely inhabited, hence in the topology.
\end{proof}

\subsection{Lex flatten Topologies}


\begin{definition}
	A saturated topology $\bT$ is lex-flatten, if its flattened by some lex $\bP$. % there exists a lex subclass $\bP$ of affines such that $\bT \subset \bP$ and $\cT_{\bP}^{\|\cdot\|_\bT} \equiv \{X \in \bP \ | \ \|X\|_\bT\} \subset \CS_\bT$.
\end{definition}
Note that $\bot = (left =_{1+1} right) \in \bP$ is automatic as $\bP$ is lex.
%If $\bT$ is saturated, equivalentely $\cT_{\bP}^{\|\cdot\|_\bT}  = \bT$.

\begin{example}
	The \etale topology is lex-flatten:
	formally \etale $+$ flat affines are stable under identity types , as formally \etale seperated schemes have decidable equality.  %\ref{lemma:PImpliesPsep}
\end{example}

\begin{prop}{\label{prop:LexflattenOmegaStable}}
	Let $\bT$ be lex-flatten. Then covering stacks are $\Omega$-stable. %In particular:
	%	Let $\bP$ be stable under $\id$-types and $j$ any Lavwere Tierney Operator. Then $\cT^j_{\bP}$-covering stacks are $\Omega$-stable
\end{prop}
\begin{proof}
	If $X$ is a covering stack then $\Omega X$ is a $\bT$-flat geometric stack \ref{lemma:PImpliesPsep} and $\bT$-merely inhabited. Conclude by  \ref{lemma:detectCovering}.
\end{proof}
\begin{lemma}{\label{lemma:TflatIs0-gerbe}}
	Assume that $\bT$ is lex-flattened. Then any $\bT$-flat geometric stack is a 0-gerbe.
\end{lemma}
\begin{proof}
	I give a second proof of this at \ref{prop:GerbeIffLooping} without using the language of lex-flattened.
	By descent, we may only show that the fiber $\sum_{y: X} \|x = y\|_{-1}^\bT$ of $\eta_0^\bT$ over $|x|$ is a covering stack. Note that $x = y$ has $\bP$ by id-stability of $\bP$ \ref{lemma:PImpliesPsep}. The $\bT$-truncation of a $\bP$-geometric stack is a $\bP$ geometric stack \ref{lemma:TruncOfP}. by $\sum$-stability of $\bP$ the fiber is $\bP$, but its $\bT$-merely inhabited. by \ref{lemma:detectCovering} its covering.
\end{proof}

\subsection{Proof of the theorem}
 Observe that if $X + Y$ is affine, then $X$ is affine, as $X \to X + Y$ is an affine map.

Let $\bT$ be a topology containing $2$.% finer than the Zariski-topology.
\begin{definition}
	$\cP_\bT$ is the smallest topology containing $\bT \cup \{\bot\}$
\end{definition}
\begin{lemma}{\label{lemma:SummandStable}}
	Let $\bP$ be $\sum$ stable containing $1 , \bot$. Then its stable under decidable subtypes, i.e. If $X + Y \in \bP$ then $X \in \bP$.
\end{lemma}
\begin{proof}
	Given $X + Y \in \bP$, we can define $(1, \bot) : X + Y \to \bP$ Its $\sum$ will be $X$. \\
\end{proof}

\begin{prop}
	Assume that $\bT$ is saturated. 	
	\[
	\cP_\bT= \{X  \ | \ \exists Y , X + Y \in \bT\}
	\]
\end{prop}
\begin{proof}
	By \ref{lemma:SummandStable} and as $\bT \subset \cP_\bT$, we have $'\supset'$. So it remains to show that the RHS, lets call it $\bP$, is a topology containing $\bT, \bot$.
%\begin{lemma}{\label{lemma:flatBasics}}
%
\begin{enumerate}
	
	\item $\cP_\bT \subset \Aff$.
	\item $\bot \in \bP$
	\item $\bT \subset \bP$
	\item Assume $\bT$ is saturated. Whenever $\bP \ni S \to X \in \Aff$ is a $\bT$-cover, then $X \in \bP$. Indeed : choose $S + Y \in \bT$, Then $\bT \ni S + Y \to X + Y$ is a $\bT$-cover, hence by saturatedness $X + Y \in \bT$. Thus $X \in \bP$.
	\item If $\bT$ is saturated, then $\bP$ is stable under $\sum$. Proof:
%\begin{proof}
	Let $\bP \ni X \overset{B}{\to} \bP$. Lets first handle the special case, where $B x \in \bT$ for any $x : X$. Choose $Y$ such that $X + Y \in \bT$ . Then $\sum_{x: X} B x + \sum_{y:Y} 1$ can be expressed as $\sum_{x : X + Y} (B + \mathrm{cnst}_1) x$, which belongs to $\bT$. \\	 
	Now the general case. By Zariski local choice we find a Zariski cover $p : X' \to X$ with 
	\[
	\prod_{x' : X'} \sum_{Y_{x'}} B (p x) + Y_{x'} \in \bT
	\]
	Then $\sum_{x' : X'} Y_{x'} + \sum_{x' : X'} B (p x) \in \bP$, hence by \ref{lemma:SummandStable} $\sum_{x': X'} B (p x) \in \bP$. As $\sum_{x': X'} B (p x) \to \sum_{x: X} B x \in \Aff$ is a $\bT$-cover, we conclude by (4.)
\end{enumerate}
\end{proof}

\begin{definition}
	$\bT$ is decompostable if for any type $X$
	\[ \left(\|X\|_\bT \land \exists Y , X + Y \in \bT  \right) \to X \in \bT. \]\\ 

\end{definition}

\begin{prop}
	Let $\bT$ be saturated. There exists a smallest decompostable topology $\tilde \bT$ containing $\bT$. Moreover the stacks coincide.	
\end{prop}
\begin{proof}
	Define
\begin{align*}
	\Top &\to \Top \\
	\bT &\mapsto \tilde \bT \equiv \{X \ | \ \|X\|_{\bT} \land \exists Y , X + Y \in \bT \} %\equiv \cT_{\cP_\bT}^{\|\cdot\|_\bT}%
\end{align*}

	We apply \ref{prop:TopologyMonad}.
	\begin{itemize}
		\item 	The class is stable under $\sum$  as $\cP_\bT$ and $\bT$-merely inhabited types are both  $\sum$-stable. \\
		\item Monotonicity clear.
		\item Inflationarity clear
		\item stack-preservation is clear by construction.
	%Let us first show, that the sheaves coincide.Obviously we have $\bT \subset \tilde \bT$, so it remains to show that every $\bT$-sheaf is a $\tilde \bT$-sheaf. Let $X \in \tilde \bT$. Then as in particular $\|X\|_\bT$ , every $\bT$-sheaf will be local wrt $\|X\|$. \\
	\item idempotency: %	Now we will prove that $\tilde \bT$ is decompostable. 
Let $X$ be a type such that $\|X\|_{\tilde \bT}$ and there exists a $Y$  with $X + Y \in \tilde \bT$. By the first assumption, we have $\|X\|_\bT$ as the stacks coincide by \ref{prop:TopologyMonad}. \\
The latter means in particular that we find $Z$ with $X + Y + Z \in \bT$. But this witnesses that $X  \in \tilde \bT$.		
\end{itemize}

\end{proof}



\begin{lemma}
	Let $\bT$ be a topology, such that any $X : \cP_\bT$ is $(\bT-1)$-seperated, i.e. that the identity types of $X$ belong to $\bT-1  \equiv \{X \ | \ X + 1 \in \bT\}.$. Then we have for all $X$
	\[
	(\exists Y : \bT - 1 , X + Y \in \bT) = (X \in (\bT - 1)) \to (\|X\|_\bT \to X \in CS)
	\]
\end{lemma}
\begin{proof}
	For the first equality notice that $X + Y \to X + 1$ is a $\bT$-cover. For the last implication, by descent for covering stacks we may choose a map $1 \to X$. Then $\bT \ni X + 1 \to X$ is a $\bT$-cover by assumption.
\end{proof}
\begin{warning}
	In general, the $\tilde \cdot$ -construction is presumably not covering-stack preserving: In the above lemma we would need 
	\[
	X \in \bP \to (\|X\|_\bT \to X \in CS)
	\]
\end{warning}

\begin{example}
	If any type in $\bP$ has decidable equality, then any type in $\cP$ is $(\bT-1)$seperated.
\end{example}
\begin{prop}{\label{prop:detectDecompostable}}
Let $\bT$ be saturated. TFAE
	\begin{enumerate}
		\item  $\bT$ is decompostable, i.e. for any $X \in \cP_\bT$ we have $\|X\|_\bT \to X \in \bT$.
		\item $\cP_\bT$ flattens $\bT$, i.e. $\bT = \{X : \cP_\bT \ | \ \|X\|_\bT \}$ % \}\cT_{\cP_\bT}^{\|\cdot\|_\bT}$
	\end{enumerate}
	In this case we have  $3 . \cP_\bT = \bT - 1$. If $\cP_\bT \subset (\bT-1)-$seperated and $\bT$ is saturated.  , then the converse holds.
\end{prop}
\begin{proof}
	\ \begin{itemize}
			\item [1 $\Leftrightarrow$ 2]
		We have
		\[
		\{X \in \cP_\bT \ | \ \|X \|_\bT\} = \{ X \ | \ \|X\|_\bT \land \exists Y , X + Y \in \bT \}
		\] % \cT_{\cP_\bT}^{\|\cdot\|_\bT} \equiv 
		which coincides with $\bT$ iff $\bT$ is decompostable.
	\item [1 $\Rightarrow$ 3]
	For the second observe $\bT - 1 \subset \cP_\bT$. Then If $X + Y \in \bT$, then $1 + X + Y \in \bT$ as $\bT$ is stable under $+$. By decompostability $1 + X \in \bT$. Hence $X \in \bT - 1$. %$\Zar \in \bT$.
	\item [3 $\Rightarrow$ 1]
	By the above lemma and saturatedness of the topology.

\end{itemize}	
\end{proof}

%\begin{theorem}{\label{thm:ModSec}}
%	 Denote $\tilde \Top$ for decompostable topologies. Consider the map 
%	\begin{align*}
%		\cP_\bullet : \tilde \Top &\to \mathsf{FLAT} \\ \bT &\mapsto \cP_\bT. 
%		\end{align*}
%	Then any Lavwere Tierney Operator $j$ induces a section of $\cP_\bullet$ mapping onto the class of topologies $\bT$ such that forall $X$
%	\[
%\ \|X\|_j \land (\exists Y, X + Y \in \bT) \to X \in \bT  %	\{\bT  : \Top \ | \forall X,  
%	\]
%	For any decompostable topology $\bT$ we find such a section whose image contains $\bT$.
%%	Moreover, any such topology will be decompostable.
%%	Then we have a bijection
%%	% https://q.uiver.app/#q=WzAsNixbMCwwLCJcXG1hdGhzZntGTEFUfSJdLFsxLDAsIlxcVG9wIl0sWzEsMSwiXFxiVCJdLFswLDEsIlxcY1BfXFxiVCJdLFswLDIsIlxcYlAiXSxbMSwyLCJcXGNUX3tcXGJQfV57XFxsbm90XFxsbm90fSJdLFswLDEsIlxcc2ltZXEiLDAseyJzdHlsZSI6eyJ0YWlsIjp7Im5hbWUiOiJhcnJvd2hlYWQifX19XSxbMiwzLCIiLDAseyJzdHlsZSI6eyJ0YWlsIjp7Im5hbWUiOiJtYXBzIHRvIn19fV0sWzQsNSwiIiwwLHsic3R5bGUiOnsidGFpbCI6eyJuYW1lIjoibWFwcyB0byJ9fX1dXQ==
%%	\[\begin{tikzcd}
%%		{\mathsf{FLAT}} & \{\bT  : \Top \ | \forall X,  \ \|X\|_j \land (\exists Y, X + Y \in \bT) \to X \in \bT \} \\ % _{decompostable} \\
%%		{\cP_\bT} & \bT \\
%%		\bP & {\cT_{\bP}^{j}}
%%		\arrow["\simeq", tail reversed, from=1-1, to=1-2]
%%		\arrow[maps to, from=2-2, to=2-1]
%%		\arrow[maps to, from=3-1, to=3-2]
%%	\end{tikzcd}\]
%\end{theorem}
%\begin{proof}
%	The map $\cP_\bullet$ is well-defined by \ref{lemma:flatBasics}. \\
%	
%	Given any $\sum$-stable class of affines $\bP$ containing 1 and a  Lawvere-Tierney operator $j$ , i.e. a reflective subuniverse of $\Prop$, we will construct a decompostable topology $\cT^j_{\bP}$ such that \[
%	\cT^j_{\bP} = \{ X \in \bP \ | \ \| X \|_{\cT_\bP^{j}} \}
%	\]
%	\begin{construction}
%		
%		Just set 
%	
%		Decompostability: From $X + Y \in \cT_{\bP}^{j}$ we get $X + Y\in \bP$, thus we deduce $X \in \bP$ by Summand-stability and $\|X\|_{\cT_{\bP}^j} = \|X\|_j$ by construction of $\cT^j$. This ends the construction \\		
%	\end{construction}	
%%	Given $j$, the section $\cT_\bullet^j$ is well-defined, i.e. has image in decompostable topologies: . % and $\|X\|_{\bP}^{\lnot \lnot}$ from the former we . From the latter we see $\lnot \lnot X$. Hence $X \in \cT_{\bP}^{\lnot \lnot}$. \\
%	We have 
%	\[
%	\cT_{\cP^{j}_\bT} = \bT
%	\]
%	by assumption on $\bT$.
%
%%	It remains to show that $\cT_\bP^j$ is decompostable. If $X + Y \in \cT_\bP^j$, then by Summand stability 
%\end{proof}
\begin{lemma}{\label{lemma:AtMostOneFlat}}
		For any $\bP : \mathsf{FLAT}$ and any Lavwere Tierney operator $j$,
	\[\cT^j_{\bP} := \{ X \in \bP \ | \ j \|X \| \}	 \]
	
	is flattened by $\bP$. Furthermore
	\[\bP  = \cP_{\cT^{j}_{\bP}}.\]
\end{lemma}
\begin{proof}

				This is indeed a topology as $\bP$ and $j$ are $\sum$-stable %and $j$ is $\sum$-closed.
			We need to show, that for any $X \in \bP$, we have $\|X\|_{\cT^j_{\bP}} = j\|X\|$.
			Note
			\[
			\|X\|_{\cT^j_{\bP}} = \exists Y \in \cT^j_{\bP} : \|Y\| \to \|X\|
			\]
			If $j\|X\|$, then put $Y := X$. Conversely, given $Y \in \cT^j_{\bP}$ such that $\|Y\| \to \|X\|$, we deduce from $j \|Y\|$ that $j\|X\|$. \\
			Furthermore,
		\[
	\{X \ | \ \exists Y , X + Y \in \bP \land j \|X + Y\| \} = \{X \ | \ X \in \bP\}
	\]
	by Summand-stability on $\bP$ we have $'\subset'$.
	if $X \in \bP$, then use $Y:=1$: $X + 1 \in \bP$ and $j \| X + 1\|$. \\
\end{proof}
Proof of theorem \ref{thm:Flat}:
\begin{enumerate}
	\item [1. and 2.] 	Assume that $\bP : \mathsf{FLAT}$ flattens $\bT$, i.e. $\cT_{\bP}^{\|\cdot\|_\bT}  = \bT$. We want to show that then $\bT$ is decompostable and $\bP= \cP_\cT$. 
	First observe that $\cP_\bT \subset \bP$ as $\{\bot\} \cup \bT \subset \bP$ %and $\bP$ is stable by decidable subtypes by \ref{lemma:SummandStable}. \\
	For decompostability we apply \ref{prop:detectDecompostable}.
	Observe
	\[\cT_{\cP_\bT}^{\|\cdot\|_\bT} \subset \cT_{\bP}^{\|\cdot\|_\bT} = \bT \]
	The other inclusion is automatic. This shows decompostability.
	Note
	\[
	\cP_\bT = \cP_{\cT_{\bP}^{\|\cdot\|_\bT}} \overset{\ref{lemma:AtMostOneFlat}}{=} \bP
	\]
	\item [3.] By the first point and \ref{lemma:AtMostOneFlat}.
\end{enumerate}
%\begin{lemma}
%	Let $L$ be a lex subcanonical modality. Assume $\bT$ is flattened. Assume for $X$ affine, $X \in \bT$ is an $L$-sheaf.
%	Then we have 
%	\[\{X : \cP_\bT \ | \ L \| X \|\} \subset \bT\]
%\end{lemma}
%\begin{proof}
%	$'\supset'$ is clear. 
%	We prove the other inclusion by induction over $\cP_\bT$ for the predicate $Q X := L \| X \| \to X \in \bT$. \\
%	It holds for $X \in \bT$ and $X = \bot$. Now suppose $ X: \cP_\bT$ with $Q \ni X \overset{B}{\to} Q$ is a type family, we want to show $\sum_{x: X} B x \in Q$. So assume $L \left \| \sum_{x: X} Bx \right \|$, we get $L \| X \|$, so as $X \in Q$ , we have $X \in \bT$. We may show $\|X\|_\bT$. Then as the goal is an $L$-sheaf by assumption, we may assume $(x , t) : \sum_{x: X} Bx $ . Hence we have a $\bT$-merely 
%\end{proof}
\begin{question}
	If $\bT$ is flattened, what is the difference between $\Omega$-stability for covering stacks and lex $\bP$? \\
	Are 0-gerbes $\bT$-flat ?
\end{question}


%\begin{definition}
%	Let $\bP$ contain 1 and be stable under $ \sum$ and under summands. $\bP$ flattens $\bT$ iff there exists a modality $j$ such that $\bT = \cT_\bP^j$.
%	%	Let $L$ be a lex modality.
%	%	$\bP$ flattens $L$ if for a type to be a $L$-sheaf its enough to check that its $\|X\|$-local for all $X \in \bP_L$. Here we view $L$ as a Lavwere Tierney Operator.
%	
%	%if there exists a Lavwere Tierney operator $j$, such that $L$ sheaves are exactly the types that are $\|X\|$-local for all $X \in \bP_L$. 
%\end{definition}

%Consider a lex modality $L$.
%\begin{definition}
%	A subset of affines $\bP$ \emph{flattens} $L$ if
%	\begin{itemize}
%		\item $1 \in \bP$
%		\item $\bP$ is $\sum$-stable
%		\item The topology ${\bP_L} := \{ X \in \bP \ | \ L \| X \| \}$ induces the lex modality $L$.
%
%	%	\item If $\bP \ni X \to Y \in \Aff$ is a ${\bP_L}$-cover, then $Y \in \bP$.
%%		\item $\bP$ is ${\bP_L}$-local property of affines. local satisfies descent
%	\end{itemize}
%	
%	%	\begin{itemize}
%	%		\item Either $X$ is contractible
%	%		\item Or Whenever we find $\Spec B \in {\bP_L}$ such that $\Spec B \to L \|X\|$
%	%	\end{itemize}
%	%	generates $L$.
%\end{definition}

%Things in $\bP$ we call 'flat'.
